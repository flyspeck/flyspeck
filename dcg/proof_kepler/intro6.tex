%% SPVI intro

This paper is the last in the series of paper devoted to the proof
of the Kepler conjecture.  The first several sections prove a result
that asserts that ``all contravening graphs are tame.''  A
contravening graph is one that is attached to a potential
counterexample to the Kepler conjecture.  Contravening graphs by
nature are elusive and are studied by indirect methods. In contrast,
the defining properties of tame graphs lend themselves to direct
examination.  (By definition, tame graphs are planar graphs such
that the degree of every vertex is at least $2$ and at most $6$, the
length of every face is at least $3$ and at most $8$, and such that
other similar explicit properties hold true.)

It is no coincidence that contravening graphs all turn out to be
tame.  The definition of tame graph has been tailored to suit the
situation at hand.  We set out to prove explicit properties of
contravening graphs, and when we are satisfied with what we have
proved, we brand a graph with these properties a tame graph.

The first section of this paper gives the definition of tame graph.
The second section gives the classification of all tame graphs.
There are several thousand such graphs.  The classification was
carried out by computer.  This classification is one of the main
uses of a computer in the proof of the Kepler conjecture.  A
detailed description of the algorithm that is used to find all tame
graphs is presented in this section.

The third section of this paper gives a review of results from
earlier parts of the paper that are relevant to the study of tame
plane graphs.  In the abridged version of the proof \cite{KC}, the
results cited in this section are treated as axioms. This section
thus serves as a guide to the results that are proved in this
volume, but not in the abridged version of the proof.

This section also contains a careful definition of what it means to
be a contravening plane graph.  The first approximation to the
definition is that it is the combinatorial plane graph associated
with the net of edges on the unit sphere bounding the standard
regions of a contravening decomposition star. The precise definition
is somewhat more subtle because we wish ensure that every face of a
contravening plane graph is a simple polygon. To guarantee that this
property holds, we simplify the net of edges on the unit sphere
whenever necessary.


The fourth and fifth sections of this paper contain the proof that
all contravening plane graphs are tame.  These sections complete the
first half of this paper.

The second half of this paper is about linear programming.  Linear
programs are used to prove that with the exception of three tame
graphs (those attached to the face-centered cubic packing, the
hexagonal-close-packing, and the pentahedral prism), a tame graph
cannot be a contravening graph.  This result reduces the proof of
the Kepler conjecture to a close examination of three graphs.
Pentahedral prism graphs are treated in Paper~\ref{part:ferguson}.
The face-centered cubic and hexagonal-close packing graphs are
treated in Section~\ref{sec:local-opt} of Paper~\ref{part:iii}. The
linear programming results together with these earlier results
complete the proof of the Kepler conjecture.

The sixth section of this paper describes how to attach a linear
program to a tame plane graph.  The output from this linear program
is an upper bound on the score of all decomposition stars associated
with the given tame plane graph. The seventh section of this paper
shows how to use linear programs to eliminate what are called the
{\it aggregate} tame plane graphs.  The {\it aggregates\/} are those
cases where the net of edges formed by the edges of standard regions
was simplified to ensure that every face of a contravening plane
graph is a polygon.  By the end of this section, we have a proof
that every standard region in a contravening decomposition star is
bounded by a simple polygon.

The final section of this paper gives a long list of special
strategies that are used when the output from the linear program in
the sixth section does not give conclusive results. The general
strategy is to partition the original linear program into a
collection of refined linear programs with the property that the
score is no greater than the maximum of the outputs from the linear
programs in the collection. These branch and bound strategies are
described in this final section.  Linear programming shows that
every decomposition star with a tame plane graph (other than the
three mentioned above) has a score less than that of the
decomposition stars attached to the face-centered cubic packing.
This and earlier results imply the Kepler conjecture.
