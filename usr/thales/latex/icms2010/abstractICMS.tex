
%



\subsection*{Newton-Gregory Problem}  

In a famous discussion, Isaac Newton claimed that at most twelve
nonoverlapping congruent balls in Euclidean three space can touch one
further ball at the center of them all.  Gregory thought that it might
be possible for thirteen balls to touch the one at the center.  It was
only in 1953 that Newton was finally proved correct.  Earlier this
year, Musin and Tarasov announced that they have finally determined
the optimal arrangement of thirteen balls~\cite{Musin-Tarasov}. These thirteen balls
do not touch the one at the center, but they come as close as possible.  Their
proof involves an analysis of more than $94$ million planar graphs, which
have been generated with the program {\it plantri}~\cite{plantri}.  Linear
programming methods are used to exclude all but the one optimal graph.


\tikzfig{svdw}
{Musin and Tarasov recently proved that this arrangement of thirteen congruent is
optimal.
Each node of the graph represents one of the thirteen balls and each edge represents a pair
of touching balls.  The node at the center of the graph
corresponds to the uppermost ball in the second frame.}
{
\begin{scope}[scale=0.004]
%Set the coordinates of the points:
%\tikzstyle{every node}=[draw,shape=circle];
\path (45:400) coordinate (P0) ;
\path (135:400)  coordinate (P1) ;
\path (225:400) coordinate (P2) ;
\path (315:400) coordinate (P3) ;
\path (0:200) coordinate (P4) ;
\path (90:200) coordinate (P5) ;
\path (180:200) coordinate (P6) ;
\path (270:200) coordinate (P7) ;
\path(45:150) coordinate (P8) ;
\path (135:150) coordinate (P9) ;
\path (225:150) coordinate (P10) ;
\path (315:150) coordinate (P11) ; 
\path (0,0) coordinate (P12) ;
\foreach \i in {0,...,12}
{
  \fill (P\i) circle (15);
}
%Draw edges:
\draw
  (P12) -- (P8)
  (P12) -- (P9)
  (P12) -- (P10)
  (P12) -- (P11)
  (P8) -- (P4)
  (P4) -- (P11)
  (P11) -- (P7)
  (P7) -- (P10)
  (P10) -- (P6)
  (P6) -- (P9)
  (P9) -- (P5)
  (P5) -- (P8)
%
  (P0) -- (P1)
  (P1) -- (P2)
  (P2) -- (P3)
  (P3) -- (P0)
%
  (P0) -- (P5)
  (P5) -- (P1)
  (P1) -- (P6)
  (P6) -- (P2)
  (P2) -- (P7)
  (P7) -- (P3)
  (P3) -- (P4)
  (P4) -- (P0);
\end{scope}
%
\begin{scope}[scale=0.5,xshift=8cm]
\def\rr#1#2{\shade[ball color=gray](#1,#2) circle (1);  }
\rr{-0.504725}{0.79793}
\rr{0.987379}{-0.530059}
\rr{-0.406371}{-1.76776}
\rr{-1.8337}{-0.370827}
\rr{1.68242}{1.01951}
\rr{0.}{2.0538}
\rr{1.35457}{-1.58937}
\rr{0.}{0.}
\rr{-1.68242}{1.20943}
\rr{-1.35457}{-1.43645}
\rr{1.8337}{0.000711695}
\rr{0.504725}{1.431}
\rr{0.406371}{-1.25805}
\rr{-0.987379}{0.159943}
\end{scope}
%\shade[ball color=blue] (2,2) circle (1); % color = gray
%\shade[ball color=blue] (2.5,2) circle (1); % color = gray
}


\subsection*{Hilbert's Eighteenth Problem}


In 1900, in his famous list of problems, Hilbert asked,
``How can one arrange most densely in
space an infinite number of equal solids of given form, e.\,g., spheres
with given radii or regular tetrahedra with given edges (or in
prescribed position), that is, how can one so fit them together that
the ratio of the filled to the unfilled space may be as great as
possible''~\cite{Hilbert}?


\subsubsection*{Dense Sphere Packings}

The solution to the sphere-packing problem was published
in~\cite{Hales:2006:DCG}. It is now the subject of a large scale
formal-proof project, Flyspeck, in the HOL Light proof assistant.
The talk will describe the current status of this project.


\subsubsection*{Tetrahedra}

Aristotle erroneously believed that the regular tetrahedron tiles
  three dimensional space: ``It is agreed that there are only three
  plane figures which can fill a space, the triangle, the square, and
  the hexagon, and only two solids, the pyramid and the
  cube''~\cite{Aristotle}.  In fact, the tetrahedron cannot tile
  because its dihedral angle is about $70.5^\circ$, which falls short
  of the angle $72=360/5$ that would be required of a space-filling
  tile.

Attention has turned to the
tetrahedron-packing problem, which has come under intensive investigation
over the past few years~\cite{Chen-2010},~\cite{Torquato-2010}.  As a
result of Monte Carlo simulations by Chen et al.,  the optimal packing is 
now given by an explicit conjecture.

\subsection*{Dense Lattice Packings of Spheres in High Dimensions}

Lagrange found that the denset packing of congruent disks in the
plane, among all lattice packings, is the hexagonal
packing~\cite{Lagrange}.  See Figure~\ref{fig:lagrange}.
Gauss solved the analogous problem in three
dimensions~\cite{Gau31}.  During the early decades of the twentieth
century, the problem of determining the densest lattice packing of
balls was solved in dimensions up to eight.  Cohn and Kumar, in a
computer assisted proof, have solved the problem in dimension
$24$~\cite{Cohn-Kumar}.  Their proof relies on a variety of
computations and mathematical methods, including the Poisson summation
formula and spherical harmonics. 

\tikzfig{lagrange}
{Lagrange proved that this is the densest of all lattice packings in two dimensions.}
{
[scale=0.3]
\def\rr#1#2{\draw[color=gray](#1,#2) circle (1);  }
\rr{-12.}{-3.4641}
\rr{-11.}{-1.73205}
\rr{-10.}{0.}
\rr{-9.}{1.73205}
\rr{-10.}{-3.4641}
\rr{-9.}{-1.73205}
\rr{-8.}{0.}
\rr{-7.}{1.73205}
\rr{-8.}{-3.4641}
\rr{-7.}{-1.73205}
\rr{-6.}{0.}
\rr{-5.}{1.73205}
\rr{-6.}{-3.4641}
\rr{-5.}{-1.73205}
\rr{-4.}{0.}
\rr{-3.}{1.73205}
\rr{-4.}{-3.4641}
\rr{-3.}{-1.73205}
\rr{-2.}{0.}
\rr{-1.}{1.73205}
\rr{-2.}{-3.4641}
\rr{-1.}{-1.73205}
\rr{0.}{0.}
\rr{1.}{1.73205}
\rr{0.}{-3.4641}
\rr{1.}{-1.73205}
\rr{2.}{0.}
\rr{3.}{1.73205}
\rr{2.}{-3.4641}
\rr{3.}{-1.73205}
\rr{4.}{0.}
\rr{5.}{1.73205}
\rr{4.}{-3.4641}
\rr{5.}{-1.73205}
\rr{6.}{0.}
\rr{7.}{1.73205}
\rr{6.}{-3.4641}
\rr{7.}{-1.73205}
\rr{8.}{0.}
%\rr{9.}{1.73205}
\rr{8.}{-3.4641}
\rr{10.}{-3.4641}
\rr{9.}{-1.73205}
%\rr{10.}{0.}
%\rr{11.}{1.73205}
}


\subsection*{Other problems}

This abstract has mentioned just a few of a large number of problems
in discrete geometry that have been or that are apt to be solved by
computer.  Others include Fejes Toth's contact conjecture, the Kelvin
problem, circle packing problems, the strong dodecahedral conjecture,
the Reinhardt conjecture, and the covering problem.  Discrete geometry
depends on the development of software to assist in the solution to
these problems.
