% File added Nov 20, 2012.

%\chapter{Supplementary Notes}\label{sec:supplement}

\def\bve{{\underline {\v}}_{\mathbf e}}
\def\ke{{ {\mathbf k}}_{\mathbf e}}
\def\pe{{ {\mathbf p}}_{\mathbf e}}
%\def\sgn{\op{sgn}}
\def\cX{{\mathcal X}}
\def\sig{\sigma}

\newpage
\section{Appendix on OXLZLEZ}\label{sec:sup-local-fan}

This appendix was written in Nov 2012 to give details of the formalization of Lemma~{\tt OXLZLEZ}.

\begin{definition}[leaf] \guid{NIPHFIE}
Let $V$ be a saturated packing.  A \newterm{leaf} of $V$ is an element $\bu \in \bV(2)$ such
that $h(\bu) < \sqrt2$.
\end{definition}





\begin{lemma}\guid{GBEWYFX} Let $V$ be a saturated packing, and let $\bu = [\u_0;\u_1;\u_2]$ be a leaf of $V$.
Then $S=\{\u_0,\u_1,\u_2\}$ is not collinear.
\end{lemma}

\begin{proof}  Part 1 of MHFTTZN states that $S$ has affine dimension $2$, hence the set is not
collinear.
\end{proof}

(Move down.)  We have the following variant of GRUTOTI.


\begin{lemma}\guid{REUHADY}
  Let $V$ be a saturated packing.  Assume that $\u_0,\u_1\in V$
  satisfy $\norm{\u_0}{\u_1}<2\nsqrt2$.  Set $\ee=\{\u_0,\u_1\}$.  
Let $[\u_0;\u_1;\w]$ and $[\u_0;\u_1;\w']$ be two leaves (not necessarily distinct) such
that for every cell $X$ with edge $\{\u_0,\u_1\}$ we have
\[
X\subset \op{wedge}(\u_0,\u_1,\w,\w')\quad\text{ or } \quad
X\subset \op{wedge}(\u_0,\u_1,\w',\w).
\] 
Then 
\[
\sum_{X\in\cX} \op{dih}(X,\ee) = \azim(\u_0,\u_1,\w,\w').
\]
The sum runs over the set $\cX$ of
 cells $X$ such that $\ee\in E(X)$ and $X \subset \op{wedge}(\u_0,\u_1,\w,\w') $.
\end{lemma}

\begin{proof} 
  Consider the sets
\[
C=B(\u_0,r)\cap \op{rcone}^0(\u_0,\u_1,a),\text{ and } C' = C\cap \op{wedge}(\u_0,\u_1,\w,\w') ,
\]
 where
  $r$ and $a$ are small positive real numbers.  From the definition of
  $k$-cells, it follows that we can choose $r$ and $a$ sufficiently
  small so that if $X$ is a $k$ cell that meets $C'$ in a set of
  positive measure, then $k\ge 2$ and there exists $\bu\in \bV(3)$
  such that $X=\cell(\bu,k)$ and $d_1\bu=[\u_0;\u_1]$.  Moreover,
\[
C'\cap X = C\cap X = C\cap A, \quad A=\op{aff}_+(\{\u_0,\u_1\},\{\v,\w\}),
\]
where $A$ is the lune of Definition~\ref{def:lune} and $\v$, $\w$ are
chosen as in Definition~\ref{def:dihX}.  By
Lemma~\ref{lemma:wedge-sol} and Definition~\ref{def:dihX}, the volume
of this intersection is
\[
\op{vol}(C\cap A) = \op{vol}(C)\,
 {\op{dih}_V(\{\u_0,\u_1\},\{\v,\w\}) }/{(2\pi)} =
  \op{vol}(C)\, {\dih(X,\ee)}/{(2\pi)}.
\]
The set of cells meeting $C'$ in a set of positive measure gives a 
partition of $C'$ into finitely many measurable sets.
This gives
\begin{align*}
\op{vol}(C) \azim(\u_0,\u_1,\w,\w')/(2\pi) &= 
\op{vol}(C') \\
&= \sum_{X\in\cX} \op{vol}(C\cap X)  \\
&= \op{vol}(C)\sum_{X\in\cX} \dih(X,\ee)/(2\pi).
\end{align*}
The calculation of volumes in Chapter~\ref{chapter:volume} gives
$\op{vol}(C)>0$.  The conclusion follows by canceling $\op{vol}(C)$
from both sides of the equation.
\end{proof}




\begin{lemma}\guid{NWVRFMF}\label{lemma:facetv}  Let $V$ be a saturated packing, and let $\bu$ be a leaf of $V$.  Let
$\p\in\ring{R}^3$ be such that $\{\p\}$ is a facet of $\Omega(V,\bu)$.  Then
there exists $\bv\in \bV(3)$ such that $d_2\bv = \bu$ and $\omega_3(\v)=\p$.
\end{lemma}

\begin{proof} This follows directly from Lemma IDBEZAL and $\bu\in \bV(2)$.
\end{proof}

\begin{lemma}\guid{YBZFUPO}\label{lemma:p1p2} 
Let $V$ be a saturated packing with leaf $\bu$.  Then there exist distinct $\p_1$ and $\p_2$ such
that $\Omega(V,\bu)$ is the convex hull of $\{\p_1,\p_2\}$ and such that
$F$ is a facet of $\Omega(V,\bu)$ if and only if $F\in \{\{\p_1\},\{\p_2\}\}$.
\end{lemma}

\begin{proof}  By the definition of $\bV(2)$,  we have that $\bu\in \bV(2)$ implies that
the affine dimension of $\Omega(V,\bu)$ is one.  This is a bounded polyhedron of dimension one,
hence a segment given as a convex hull of distinct points $\p_1$ and $\p_2$.  The facets of a segment
are its extreme points as given.
\end{proof}

\begin{lemma}\guid{FUZBZGI}  Let $V$ be a saturated packing with leaf $\bu$.  let $\q$ be the circumcenter of $\bu$.
Then $\q\in\Omega(V,\bu)$, but is not an extreme point of $\Omega(V,\bu)$.
\end{lemma}

\begin{proof} The third part of Lemma MHFTTZN gives that $\q\in \Omega(V,\bu)$. 

Assume for a contradiction that
$\q$ is an extreme point, then Lemma~\ref{lemma:facetv} gives $\bv\in \bV(3)$ such that
$d_2\bv = \bu$ and $\omega_3(\v)=\q$.  The set $\Omega(V,\bv)$ is convex of affine dimension $0$,
and is therefore a singleton $\omega_3(\bv)$.   By Lemma MHFTTZN applied to $\bv$, we have that
$\q=\omega_3(\bv)$ is the circumcenter of $\bv$.  This contradicts the strict inequality of Lemma XYOFCGX.
\end{proof}

\begin{definition}[$\chi$]\guid{MSBKFLD} For any list $\bu = [\u_0;\u_1;\u_2]$ of  elements in $\ring{R}^3$, define 
$\chi(\bu,\p) = \det(\u_1-\u_0;\u_2-\u_0;\p-\u_0)$.
\end{definition}

\begin{lemma}\guid{JDHAWAY} Let $V$ be a saturated packing with leaf $\bu$.  Let $\p_1$ and $\p_2$ be the
distinct points constructed in Lemma~\ref{lemma:p1p2}.  Then $\chi(\bu,\p_i)$ is not zero,
and $\chi(\bu,\p_1)$ and $\chi(\bu,\p_2)$ have opposite signs.
\end{lemma}

\begin{proof}
Let $\q$ be the circumcenter of $\bu$.
If $\chi(\bu,\p_i) = 0$, then $\p_i$ lies in the affine hull of $\bu$.   By MHFTTZN, this implies that
$\q = \p_i$, which is impossible by the previous lemma.  Hence $\chi(\bu,\p_i)\ne 0$.

By the previous lemma, the circumcenter $\q$ of $\bu$ has the form $\q=\p_1 t_1 + \p_2 t_2$,
for some $t_i$ such that $t_1+t_2=1$ and $t_i>0$.  Since $\q$ lies in the affine hull of $\bu$, we have
\[
0 = \chi(\bu,\q) = t_1 \chi(\bu,\p_1) + t_2 \chi(\bu,\p_2)
\]
Since $t_i>0$, this implies that $\chi(\bu,\p_1)$ and $\chi(\bu,\p_2)$ have opposite signs.
\end{proof}

\begin{lemma}\guid{RIJRIED}\label{lemma:em2}  Let $V$ be a saturated packing with leaf $\bu=[\u_0;\u_1;\u_2]$. Let $X$ be a
$2$-cell with edge $\{ \u_0,\u_1 \}$.  Then $X$ does not meet $A_+^0=\op{aff}_+^0(\{\u_0,\u_1\},\u_2\}$.
\end{lemma}

\begin{proof} Suppose for a contradiction that the $2$-cell $\op{cell}(\bv,2)$ 
meets $A_+^0$ at $\p$.  Using the description
of points in the $2$-cell and $A_+^0$, we obtain
\[
\p = t_0 \u_0 + t_1 \u_1 + t_2 \p' = t_0' \u_0 + t_1' \u_1 + t_2' \u_2
\]
where $t_0 + t_1 + t_2 = 1$, $t_2\ge 0$, $t_0'+t_1'+t_2'=1$, $t_2'>0$ and $\p'$ lies in the convex
hull of $\xi(\bv)$ and $\omega(\bv)$, for some $\bv$ with $d_2(\bv) = d_2(\bu)$.
BY MHFTTZN, the affine hull of $\bu$ is two dimensional.  Since $t_2'>0$, we find that $\p$ is not
in the affine hull of $\{\u_0,\u_1\}$.  Thus, $t_2 > 0$.  We find the $\p$ lies in the
half-plane  $B^0_+ = \op{aff}_+^0(\{\u_0,\u_1\},\p')$.  In fact, from the description of $X$, we have that
\[
C = \op{rcone}(\u_0,\u_1,a) \cap \op{rcone}(\u_1,\u_0,a) \cap B^0_+
\]
is a subset of $X$.  The set $C$ contains a unique point $\q$ at distance $\sqrt2$ from both $\u_0$ and
$\u_1$.  However, since $h(\bu)<\sqrt2$, we see that  $\q$ is closer to $\u_2$ than to $\u_0$.
This contradicts Lemma QZKSYKG.
\end{proof}

\begin{lemma}\guid{ZWVCBMN} Assume that $S=\{\u_0,\ldots,\u_3\}\subset\ring{R}^3$ is not coplanar.  Then
the convex hull of $S$ has positive measure.
\end{lemma}

\begin{proof} The volume of a tetrahedron is has the form $\sqrt(\Delta)/12$, and the condition
of planarity is $\Delta=0$.
\end{proof}

\begin{lemma}\guid{ASVAYEW}\label{lemma:em34} 
Let $V$ be a saturated packing and let $X$ be a nonempty $3$ or $4$-cell.  Then
$X$ has positive measure.
\end{lemma}

\begin{proof} Write the cell as $op{cell}(\bu,k)$. 
A nonempty $3$ or $4$-cell has the form of a convex hull of four points,
 $\{\u_0,\u_1,\u_2,\p\}$, where $\p=\u_3$ in the case of a $4$-cell ($h(\bu)<\sqrt2$) or
$\p=\xi$ in the case of a $3$-cell ($h_2 <\sqrt2 \le h(\bu)$).  
By the previous lemma, it is enough to show that this set of four points is not coplanar.
In the case of a $4$-cell
the points are not coplanar by Lemma MHFTTZN.  

In the case of a $3$-cell, we have that the affine hull of $\{\u_0,\u_1,\u_2\}$ has dimension $2$
by Lemma MHFTTZN.  It is enough to show that $\xi$ is not in this affine hull.   
The point $\xi$
is equdistant from $\u_0,\u_1$ and  $\u_2$.  If $\xi$ is in the affine hull, then it is the circumcenter,
and we arrive at a contradiction
\[
\sqrt2 = \norm{\xi}{\u_0} = h(d_2\bu) < \sqrt2.
\]
\end{proof}



\begin{definition}[$\pe$, $\bve$,$\ke$,$c$]\guid{AQEQEDX}  Let $V$ be a saturated packing with leaf $\bu$.
Let $\sig = \pm$ be a sign.  Define 
We define $\pe$, $\bve$, and $\ke$ as functions of $\bu$ and $\sig$ as follows.
By the previous lemma, there exists a unique extreme point $\pe$ of $\Omega(V,\bu)$ such that
$\chi(\bu,\pe) \sig > 0$.  By an earlier lemma, there exists $\bve$ (choose one) such that
$\bve\in \bV(3)$ such that $d_2(\bve) = \bu$ and $\omega_3(\bve) = \pe$.  Finally, let $\ke\in\{3,4\}$
be given by 
\[
\ke = \begin{cases} 4, &h(\bve) < \sqrt2\\
    3,&\text{otherwise}.
\end{cases}
\]
We abbreviate $c(\bu,\sig)=\op{cell}(\bve(\bu,\sig),\ke(\bu,\sig)$.
\end{definition}


\begin{lemma}\guid{MOFYNXM} Assume that $\{\u_0,\u_1,\u_2,\u_3\}\subset\ring{R}^3$ is not coplanar.  Assume
that $\det(\u_1 - \u_0,\u_2-\u_0, \u_3 - \u_0) > 0$.  Then for every $\epsilon >0$,
there exists $\delta>0$ such that  for all  $0 < t <\delta$,
we have 
\[
0 < \azim (\u_0,\u_1,\u_2, (1-t)\u_2 + t \p) < \epsilon.
\]
\end{lemma}

\begin{proof}  Let $\e_1,\e_2,\e_3$ be an oriented orthonormal basis adapted to $(\u_0,\u_1,\u_2,\u_3)$.
By orthogonal projection
to the plane through $\u_0$ with normal $\e_1$, the determinant is positive precisely when
the $\e_3$-coordinate of $\u_3-\u_0$ is positive.  Again, by orthonormal projection,
the azimuth angle corresponds to the complex argument.  The statement corresponds to a positive
complex argument in the upper half plane.
\end{proof}


\begin{lemma}\guid{CFFONNL}\label{lemma:meet-halfplane}  
Let $V$ be a saturated packing with leaf $\bu = [\u_0;\u_1;\u_2]$.  
Let $X$ be a cell of $V$, and let $\{ \u_0,\u_1 \}\in E(X)$ be
an edge.  Suppose that 
\[
X \cap  A_+^0 \ne\emptyset, \text{ where } A_+^0 := \op{aff}_+^0(\{\u_0,\u_1\},\{\u_2\}).
\]
Then there exists a sign $\sig$ such that
\[
X = c(\bu,\sig).
\]
Moreover, the sign is positive iff there exists $\p\in X$ with $\chi(\bu,\p)>0$.
\end{lemma}

\begin{proof} Since $X$ has an edge, it is at least a $2$-cell.  By Lemma~\ref{lemma:em2}, it is
not a $2$-cell.  Hence $X$ is a $3$ or $4$-cell.  By Lemma~\ref{lemma:em34}, the cell $X$ has
positive measure.  Therefore it is not a subset of $A=\op{aff}\{\u_0,\u_1,\u_2\}$.  
Pick $\q\in X\setminus A$, and let $\sig$ be the sign of $\chi(\bu,\q)\ne0$.  Let 
$\bv=\bve(\bu,\sig)\in \bV(3)$ and $k = \ke(\bu,\sig)$.

Let $X'=\op{cell}(\bv,k)$.  By construction, it is nonempty, and contains the two-dimensional set
$R(\bu)$.  We find that $X\cap X'$ has positive measure.  Hence by Lemma AJRIPQN, we have
$X = X'$.  This gives the result.
\end{proof}

\begin{lemma}\guid{FUEIMOV}  Suppose that $V$ is a saturated packing with leaves $\bu$ and $\bu'$ and
signs $\sig$ and $\sig'$.  
Suppose that $d_1(\bu) = d_1(\bu')$.  Writing $\bve=\bve(\bu,\sig)$, $\ke=\ke(\bu,\sig)$
and similarly for primed quantities,  assume that
$(\bve,\sig)\ne (\bve',\sig')$.
Then the following are equivalent
\begin{enumerate}
\item
\[
\op{cell}(\bve,\ke) = 
\op{cell}(\bve',\ke') 
\]
\end{enumerate}
and this is a nonempty set.
\item $\ke=\ke'=4$, and $\sig \ne \sig'$, and $\bve' = [\v_0;\v_1;\v_3;\v_2]$
where $\bve = [\v_0;\v_1;\v_2;\v_3]$.
\end{lemma}

\begin{proof}
$\Leftarrow):$  This implication follows from RVFXZBU and the nonemptiness condition on $4$-cells.

$(\Rightarrow):$ If the cells are equal, then $\ke=\ke'\in \{3,4\}$ by the definiton of $\ke$ and Lemma AJRIPQN.

We consider two cases depending on the value of $\ke$.
Suppose that $\ke=4$.  The definition of $\ke$ gives $h(\bve) < \sqrt2$.  The cell is the convex hull
of its extreme points, and $\op{cell}(\bv,4)$ determines the parameter $\bv$ up to rearrangement.
The elements $d_2(\bv)$ are fixed, and so the only possibilities are $\bve'=\bve$ or a transposition
in the last two places.  If $\bv'=\bv$, then opposite signs $\sig\ne\sig'$ gives two cells separated
by the affine hull of $\bu$.  Hence $\bve'$ and $\bve$ differ by a transposition.  The result now
follows easily in this case.

Suppose that $\ke=3$.  The definition of $\ke$ gives $h(d_2\bve) < \sqrt2 \le h(\bve)$.  The
cell is the convex hull of its extreme points, three of which are elements of the vertex set $V(X)$.
This determines $d_2\bve$.  No nontrivial rearrangement fixing $d_2\bve$ is possible, so $\bve=\bve'$.
If the signs are opposite, then we obtain two cells separated by the affine hull of $\bu$.  Hence equality
of cells forces equality of parameters $(\bve,\sig)$.
\end{proof}

\begin{lemma}\guid{RBUTTCS}  Let $V$ be a saturated packing.  Let $X$ be a cell with
parameter $3$ or $4$ with edge $\{\u_0,\u_1\}$.  Then there exists a leaf $\bu$ such that
$d_2\bu = [\u_0;\u_1]$ and a sign $\sig$, such that $X=c(\bu,\sig)$.
\end{lemma}

\begin{proof}  Let $X = \op{cell}(\bv,k)$, where $k\in\{3,4\}$. 
Since $\{\u_0,\u_1\}$ is an edge, by replacing $\bv$ with a $k$-rearrangement,
we may assume that $\bv = [\u_0;\u_1;\v_2;\v_3]$, for some $\v_2$ and $\v_3$.  
Set $\bu = d_2\bv$.
We have $h(\bu)<\sqrt2$, so that $\bu$ is a leaf.
The cell meets
$\op{aff}_+^0(\{\u_0,\u_1\},\u2)$ in the set $R(d_2\bv)$.  Lemma~\ref{lemma:meet-halfplane} gives
 that $X = c(\bu,\sig)$ for some choice of sign $\sig$.
\end{proof}

\begin{lemma}\guid{EWYBJUA} Let $V$ be a saturated packing.  Let $X$ be any cell with edge $\{\u_0,\u_1\}$.
Let $\bu=[\u_0;\u_1;\u_2]$ and $\bu'=[\u_0;\u_1;\u_2']$ be distinct leaves of $V$. Then
\[
X \subset \op{wedge}(\u_0,\u_1,\u_2,\u_2') \text{ or } X \subset\op{wedge}(\u_0,\u_1,\u_2',\u_2).
\]
\end{lemma}

\begin{proof}  If $X$ is a null set then by Lemma~\ref{lemma:em34}, its parameter is $k=2$.
In this case, by Lemma~\ref{lemma:em2}, $X\subset \op{aff}_+^0(\{\u_0,\u_1\},\p)$ for some $\p$
and any such half-plane lies in one of the wedges.   Hence we may assume without loss of generality
that $X$ is not a null set.

Pick $\p\in X\cap \op{wedge}^0 (\u_0,\u_1,\u_2,\u_2')$ and $\q\in X\cap \op{wedge}^0(\u_0,\u_1,\u_2',\u_2)$.
Since $X$ is not a null set, we may assume that $\{\u_0,\u_1,\p,\q\}$ are not coplanar.
Let $\q_t = (1-t) \p + t \q$, for $0\le t\le1$.  We then have that $\{u_0,\u_1,\q_t\}$ are not collinear.
Exchanging $\u_0$ and $\u_1$ if necessary, 
the image of  $t\mapsto\op{azim}(\u_0,\u_1,\p,\q_t)$ on $[0,1]$ is an interval $[0,x]$ for some $x$.

We claim that $x\ge \op{azim}(\u_0,\u_1,\p,\u_2')$.  Otherwise,
\[
\op{azim}(\u_0,\u_1,\p,\q) < \azim(\u_0,\u_1,\p,\u_2'),
\]
which implies the contradiction that $\q\in \op{wedge}(\u_0,\u_1,\u_2,\u_2')$  (because the two wedges
are disjoint).  This gives the claim.

By the claim, there exists $t$ such that $\azim(\u_0,\u_1,\p,\q_t) = \azim(\u_0,\u_1,\p,u_2')$.
This gives $\q_t\in X\cap \op{aff}_+(\{\u_0,\u_1\},\u_2')$.  By an earlier lemma,
$X = c(\bu,\sig)$ for some sign $\sig$.

Reversing the roles of $\bu$ and $\bu'$, we obtain similarly that
$X = \op{cell}(\bve(\bu',\sig'),\ke(\bu',\sig'))$, for some $\sig'$.
By an earlier lemma, $\bve=[\u_0;\u_1;\u_2;\u_2']$ and $\ke=4$.
The cell is then the convex hull of $\{\u_0,\u_1,\u_2,\u_2'\}$, which is contained in one of the two wedges.
\end{proof}


\begin{lemma}\guid{NUNRRDS}  Let $V$ be a saturated packing with distinct leaves $\bu$, $\bv$ with $d_2\bu=d_2\bv$.
The cell $c(\bu,\sig)$ is contained in $\op{wedge}(\u_0,\u_1,\u_2,\v_2)$ or $\op{wedge}(\u_0,\u_1,\v_2,\u_2)$
depending on the sign $\sig = +$ or $-$ respectively.
\end{lemma}

\begin{proof} By an earlier lemma,  the cell belongs to one of the two wedges.  It is enough
to produce a point $\p \in \op{wedge}^0\cap X$ for an appropriate choice of wedge.
We work the case $\sig=+$, leaving $\sig=-$ to the reader.
Recall that $\chi(\bu,\pe(\bu,+)) > 0$.  Let $\q$ be the circumcenter of $\bu$.  For small positive $t$,
we have 
\[
\p_t := (1-t)\q  + t \pe(\bu,+) \in c(\bu,\sig).
\]
Also, $\chi(\bu,\p_t) = t \chi(\bu,\pe(\bu,+)) > 0$.  By an earlier lemma,
\[
0 < \azim(\u_0,\u_1,\u_2,\p_t) < \sig,
\]
so that $\p_t$ lies in $\op{wedge}(\u_0,\u_1,\u_2,\v_2)\cap c(\bu,\sig)$.
\end{proof}

Let $V$ be a saturated packing and let $\{\u_0,\u_1\}$ be a critical edge (of some cell).
We may order the set 
\[
\op{Leaf} = \{ \v\in V \mid [\u_0;\u_1;\v] \text{ is a leaf } \}
\]
(and the corresponding leaves)
by arbitrarily fixing one element $\v_0\in L$ and then ordering them by increasing
azimuth angle $\azim(\u_0,\u_1,\v_0,\v)$, for $\v\in \op{Leaf}$.  This partitions $\ring{R}^3$ into
finitely many wedges delimited by the leaves.  Each cell with edge $\{u_0,\u_1\}$ lies in one of these
wedges.    

If  $\bu$ is one of the leaves and $\bu'$ is the following leaf, then let $\bv = [\u_0;\u_1;\u_2;\u_2']$.
$h(\bv) < \sqrt2$, then $c(\bu,+)=c(\bu',-)$ is a $4$-cell in the wedge,
 and this is the only cell (of positive measure)
contained in the wedge.

If $h(\bv)\ge\sqrt2$, then $c(\bu,+)$ and $c(\bu',-)$ are distinct $3$-cells in the wedge,
and these are the only $3$ cells in the wedge.  The wedge may also include a finite number of 
$2$-cells, but it has no $4$-cells.  The $2$-cells within a given wedge are combined into a 
total $\gamma$ and total azimuth angle.

We have a number of nonlinear inequalities bounding the value of 
\[
\gamma(X,L) \op{wt}(X) + \beta(\{\u_0,\u_1\},X)
\]
 as a function
of its azimuth angle.  These inequalities are based on the partition of cells according to wedges.
Based on these inequalities,
we run linear programs giving lower bounds for $\sum \gamma(X,L)$, subject to the constraint
that the azimuth angles sum to $\pi$.  In every case, we find that $\Gamma\ge0$ for each cluster.
We run a separate linear program depending on the number of leaves.  A generic case handles
the case of five or more leaves.


