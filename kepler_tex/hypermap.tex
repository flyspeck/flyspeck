%% HYPERMAPS



\section{Definitions}



\begin{definition}  A hypermap is a finite set $D$, together with
three functions $e,n,f:D\to D$ that satisfy the identity
    $$e\circ n\circ f = I.$$
The elements of $D$ are called {\it darts}.  The functions $e,n,f$
are called the {\it edge maps}, the {\it node map}, and the {\it
face map}, respectively.
\end{definition}



\begin{remark*} We represent hypermaps graphically as follows.  A
dart is drawn as a small black dart.
\end{remark*}

\begin{figure}[htb]
  \centering
  \myincludegraphics{\ps/dart.eps}
  \caption{A dart}
  \label{fig:dart}
\end{figure}

\begin{remark*}\label{rem:hypermap} A hypermap is an abstraction of
the combinatorial properties of planar graphs.  The following
example shows how the abstraction was made.  Let $G$ be a planar
graph.  Place a dart at each angle.    Let $f$ be the function that
cycles counterclockwise through the angles of each face.  Let $n$ be
the function that moves counterclockwise through the angles at each
node.  Finally, let $e$ be the function that pairs each angle with
the angle at the opposite end of the opposite side of each edge.  A
hypermap is the abstraction that forgets everything about the planar
graph except for the data $(D,e,n,f)$.
\end{remark*}

\begin{figure}[htb]
  \centering
  \myincludegraphics{\ps/hypermap_ex.eps}
  \caption{Darts mark each angle of a planar graph. A hypermap comes by
  permuting darts around faces, nodes, and edges.}
  \label{fig:hypermap_ex}
\end{figure}

By symmetry, a hypermap also satisfies $n\circ f\circ e = f\circ
e\circ n = I$.  The symmetry between $n$ and $f$ is an abstraction
of the duality between nodes and faces in a planar graph. Because of
the symmetry in the definition, there will be three versions of many
of the theorems, all obtained from one proof by symmetry.

Each of the functions $e,n,f$ is invertible.  We write $\#h$ for the
number of orbits of a function $h$ on $D$, and $\#\tangle{e,n,f}$
for the number of orbits of the combined action of functions $e,n,f$
on $D$.  We say that the hypermap is connected if $\#\tangle{e,n,f}
= 1$.

\begin{definition}  A node of a hypermap is an orbit of darts under
$n$.  A face of a hypermap is an  orbit of darts under $f$.  An edge
of a hypermap is an orbit of darts under $e$.
\end{definition}

\begin{definition} A hypermap is {\it plain} (note the spelling!) if
$e$ is an involution on $D$ (that is, $e\circ e = I$).  A hypermap
is {\it planar} (note the spelling!) if
    $$\# n + \# e + \# f = \# D + 2 \#\tangle{e,n,f}.$$
\end{definition}

\begin{definition} A dart in a hypermap is degenerate if it is a
fixed point of one of the maps $e,n,f$.  It is nondegenerate
otherwise.
\end{definition}

\begin{example*}  A connected $2$-connected
planar graph $G$ is known to satisfy the Euler relation
    $$ V - E + F = 2$$
where $V$ is the number of vertices, $E$ is the number of edges, and
$F$ is the number of faces on $G$ (including the unbounded face). If
we create a hypermap $(D,e,n,f)$ from $G$ along the lines of
Remark~\ref{rem:hypermap}, then the function $e$ is an involution,
so the hypermap is plain. Moreover,
    $$\begin{array}{lll}
    V &= \# n\\
    E &= \# e\\
    F &= \# f\\
    2E &= \# D\\
    1 &= \#\tangle{e,n,f}\\
    \# n + \#e + \# f &= V + E + F = 2 E + 2 = \# D + 2 \#\tangle{e,n,f}.
    \end{array}
    $$
Thus, the hypermap is also planar.  The identity defining a planar
hypermap should be viewed as a transcription of the Euler relation
in terms of hypermaps, and the definition of planarity is simply
that the Euler relation is satisfied.
\end{example*}

\begin{lemma}  Let $H$ be a plain planar hypermap that is connected.
Let $f_i$ be the number of faces with $i$ darts.  Then
    $$2 \# n - 2 =  f_3 + 2 f_4 + 3 f_5 +\cdots$$
\end{lemma}

\begin{proof}  We have
    $$
    \begin{array}{lll}
     \# D &= 2 \# e = 3 f_3 + 4 f_4 + \cdots\\
    \# f &= f_3 + f_4 + \cdots.
    \end{array}
    $$
Use these equations to eliminate $\#e$, $\#f$, $\#D$ from the Euler
relation.  The result follows.
\end{proof}



\section{Walkup transformations}

When we focus our attention on a particular dart $x$ in a
hypermap, it is sometimes useful to represent the six darts $e x$,
$n x$, $f x$, $e^{-1} x$, $n^{-1} x$, $f^{-1} x$ as a hexagon
around the center $x$ as in Figure~\ref{fig:dart+}.  These $7$
darts are not necessarily distinct.   When $x$ is fixed under one
of the transformations $e$, $n$, or $f$, then the figure takes one
of the degenerate forms of Figure~\ref{fig:dart-fix}.

\begin{figure}[htb]
  \centering
  \myincludegraphics{\ps/dart.eps}
  \caption{A dart $x$ and its entourage}
  \label{fig:dart+}
\end{figure}

\begin{figure}[htb]
  \centering
  \myincludegraphics{\ps/dart.eps}
  \caption{A dart fixed under a face map.}
  \label{fig:dart-fix}
\end{figure}

Walkup transformations are certain transformations that modify a
hypermap by deleting one of its darts.  The result of a walkup
transformation is a new hypermap with one fewer dart.  Walkup
transformations come in three flavors: edge walkups, face walkups,
and node walkups.

\begin{definition}
Let $x$ be a dart in a hypermap.  The edge walkup transformation
$W_e$ at $x$ of the the hypermap is the hypermap
$(D\setminus\{x\},e',n',f')$ obtained by deleting the dart $x$ and
modifying the permutations $f,n$ to skip past $x$ as follows:
    $$
    \begin{array}{lll}
    f' y &= \text{ if } (y = f^{-1} x) \text{ then } f x \text{ else
    } f y\\
    n' y &= \text{ if } (y = n^{-1} x) \text{ then } n x \text{ else
    } n y\\
    e' = (n'\circ f')^{-1}
    \end{array}
    $$
\end{definition}

The effect of the edge walkup transformation on the hexagon at $x$
is shown in Figure~\ref{XX}.  The face walkup $W_f$ and node walkup
$W_n$ transformations are defined by symmetry.  The effect of these
transformations on the hexagon at $x$ is shown in Figure~\ref{XX}.

We say that a walkup transformation is degenerate if the dart $x$ it
uses is degenerate.   If $x$ is a degenerate dart, all three walkup
transformations at $x$ become equal: $W=W_e=W_n=W_f$. This
degenerate walkup transformation is shown in Figure~\ref{XX}.

\subsection{walkups and planarity}

\begin{definition} Define the planar index of a hypermap to be
$$\# f + \# e + \# n - \# D - \# \tangle{e,n,f}.$$
(Thus, a planar hypermap is a hypermap whose planar index is $0$.)
\end{definition}

\begin{lemma} Let $x$ be the dart of a hypermap $(D,e,n,f)$ (not necessarily
planar). Let $(D',e',n',f')$ be the result of the edge walkup at
$x$.  The effect on the various constants in the planar index is as
follows.
    $$
    \begin{array}{lll}
    \text{\bf Non-degenerate dart $x$: }&\\
    \# f &= \# f'\\
    \# e &= \# e' \pm 1\\
    \# n &= \# n'\\
    \# D &= \# D' + 1\\
    \# \tangle{e,n,f} &= \#\tangle{e',n',f'} + (0 \text{
    or 1})\\
    \text{\bf Degenerate dart:}&\\
    \# f &= \# f' +1\\
    \# e &= \# e' + 1\\
    \# n &= \# n' + 1\\
    \# D &= \# D' + 1\\
    \#\tangle{e,n,f} &= \#\tangle{e',n',f'} + 1
    \end{array}
    $$
With $\# e = e' \pm 1$, the sign is positive when deleting $x$
splits the cycle of $e$ through $x$ and is negative otherwise.  In
$\#\tangle{e,n,f}$ we pick $1$ when deleting $x$ splits the orbit of
$\tangle{e,n,f}$ through $x$, and $0$ otherwise.
\end{lemma}

\begin{proof} This is evident from the Figures.
\end{proof}

\begin{definition} Let $h=n,e$, or $f$.
A walkup transformation $W_h$ at $x$ is a merge,
if the orbit of $h$ through $x$ is combined with another orbit by
the walkup transformation.  It is a split transformation, if the
orbit at $x$ is split into two orbits by the walkup transformation.
\end{definition}

\begin{lemma} Every nondegenerate walkup transformation is a merge or a split.
The walkup $W_f$ at $x$ is a merge if and only if $x$ and $e x$  lie
in distinct $h$-orbits.  (The same result holds with $(f,e)$
replaced by $(f,n^{-1})$ or other situations obtained by
$e,n,f$-symmetry.)
\end{lemma}

\begin{proof} The walkup splits if and only if $f x$ (or equivalently $x$)
and $e x$ are in the same $f$-orbit before the split. This is clear
from the figures.
\end{proof}

\begin{lemma}  Let $x$ be the dart of a hypermap $(D,e,n,f)$ (not necessarily
planar). The effect of a walkup transformation on the planar index
is as follows.  The index is preserved if $x$ is degenerate (a fixed
point of $f$, $e$, or $n$). The index is preserved by a merge
transformation. If $x$ is non-degenerate and and the walkup is
split, the planar index is preserved iff the orbit of
$\tangle{e,n,f}$ through $x$ splits. When the walkup transformation
does not preserve the planar index, it increases the index by $2$.
\end{lemma}

\begin{proof}  This is evident from the preceding lemma together
with the observation that if the $\tangle{e,n,f}$ orbit splits then
the $e$ orbit also splits.
\end{proof}



\begin{lemma}  The planar index
of a hypermap is never positive.
\end{lemma}

\begin{proof}  An edge walkup never drops the index.  By a sequence
of edge walkups we eventually reach the empty hypermap, which has
index zero.
\end{proof}


\begin{lemma} Walkup transformations take planar hypermaps to planar
hypermaps.
\end{lemma}

\begin{proof}  By symmetry, it is enough to consider the case of edge walkup
transformations.  A planar hypermap has index zero.  The walkup
transformation can only increase the index, but can never increase
it above zero.  Thus, the index remains at zero.
\end{proof}


\section{Double Walkup Transformations}

Double walkup transformations are the composite of one walkup
transformation followed by another of the same type.  For a double
walkup transformation, we need to choose two darts.  We will always
choose both darts to be the two members of an orbit of cardinality
two under $n$, $e$, or $f$.  The first walkup transformation will be
chosen so that it is a merge transformation.  The second walkup will
be chosen to be a degenerate walkup transformation.

If we chose the type of the walkups to be different from the type of
the orbit of cardinality two, then the second walkup will
automatically be degenerate.  (The first walkup reduces the
cardinality of the orbit to $1$, which means that it is a
fixed-point.)

We point out a few interesting cases of this construction. In each
case, we assume the target darts of the walkup form orbits of
cardinality $2$.
\begin{itemize}
    \item A double $W_n$ applied to an edge has the effect on
    the hypermap of deleting the edge and merging the two nodes into
    one.
    \item A double $W_f$ applied to an edge has the effect of
    deleting the edge and merging the two faces along the node into
    one.
    \item A double $W_e$ applied to a node has the effect of
    deleting the node and merging the two edges along the node into
    one.
\end{itemize}

\begin{lemma}  The three preceding double walkups transform plain
hypermaps into plain hypermaps.
\end{lemma}

\begin{proof} The transformations $W_n$ and $W_f$ preserve the orbit
structure of edges, except for dropping one dart.  By dropping both
darts from the same edge, one edge is lost and all others edges
remain unchanged.

For the double $W_e$, XX ??.
\end{proof}

The following is a useful criterion for detecting merge
transformations.

\begin{lemma}  Suppose that each face of a hypermap intersects each node in at
most one dart.  Suppose that the edge $\{x,y\}$ through $x$ has
cardinality two, with both darts nondegenerate.  Then the walkup
transformation $W_f$ (resp. $W_n$) is a merge.
\end{lemma}

\begin{proof} It is a merge if and only if $f x$ and $e x$ are in
distinct faces (by Lemma REFXX).  Assume they are in the same face.
We have $n (f x) = e^{-1} x = e x$. So $f x$ and $e x$ are at the
same node. By the first hypothesis in the lemma, we have $f x = e x
= y$. Then $$n y = n e x = n f x = n f e y = y,$$ so $y$ is a fixed
point of $n$, hence degenerate, contrary to assumption.  Thus, $f x$
and $e x$ are in different faces, and the walkup is a merge.
\end{proof}




\section{Contours}

\begin{definition}  A dart path is a function $p:\{0,\ldots,k\}\to D$
for some $k$.  A contour path is a dart path such that $p_{i+1} =
n^{-1} p_i$ or $f p_i$ for each $i<k$.  (That is, the only allowed
steps in the path are clockwise steps around the darts in a node and
counterclockwise steps around the darts in a face.)  A dart loop is
a path $p:\{0,\ldots,k\}$ that is injective on $\{0,\ldots,k-1\}$
and such that $p_0 = p_k$.  A contour loop is a dart loop that is
also a contour path.
\end{definition}

\begin{definition} A M\"obius contour is an
injective contour path $p$ that satisfies
    \begin{equation}
    \label{eqn:mobius}
    p_j = n p_0\quad p_k = n p_i
    \end{equation}
for some $0 < i\le j< k$.
\end{definition}

\begin{remark*}
This definition takes some getting used to.  G. Gonthier was led to
this definition while abstracting the Jordan curve property: a
simple closed curve in the plane separates the plane into an
interior and an exterior.  The term {\it M\"obius} comes by way of
analogy with the M\"obius strip, which has the property that the
simple closed curve through the center of the strip fails to
separate the strip into two halves.
\end{remark*}



\begin{lemma}  Planar hypermaps have no M\"obius contours.
\end{lemma}

\begin{proof} Assume for a contradiction that there exist planar
hypermaps with M\"obius contours.  Among all counterexamples,
consider one that has the fewest darts.  The edge walkup transforms
planar hypermaps into planar hypermaps.  If we apply an edge walkup
transformation on a dart that is not on the M\"obius contour, the
M\"obius contour appears in the transformed hypermap.  By the
minimality of our counterexample, we may assume that it contains no
darts except those on the M\"obius contour.

In the M\"obius Condition~\ref{eqn:mobius}, if $1<i$, $i<j$, or
$j+1<k$, we may apply an edge walkup transformation to  delete the
darts along the contour that are not at index positions $0$, $i$,
$j$, $k$.  By minimality of our counterexample, we may now assume
that $i=1$, $j=i$, and $k=2$.

This is a three darted hypermap.  The M\"obius condition, the
definition of contours, together with $e\circ n\circ f=I$ force
$e=n=f$, all permutations of order three. We reach the contradiction
that this hypermap is not planar:
    $$3 = \# e + \# n + \# f,\quad 5 = 3+2 = \# D + 2 \#\tangle{e,n,f}.$$
\end{proof}

\begin{lemma}  Suppose that a hypermap has no M\"obius contours.
Let $L$ be a contour loop.  Let $P$ be any injective contour path
that starts and ends on $L$, but visits no other darts of $L$ in
between.  Then the first and last steps of $P$ are both of the same
type $n^{-1}$ or $f$.
\end{lemma}

\begin{proof}  Suppose $P$ is $n x,f n x,\ldots,n y,y$.   Form
contour path starting at $x$, then $n^{-1}$ steps to $L$, then
follow $L$ to $y$, and on to $n x$.  Follow $P$ back to $n y$.  This
is a M\"obius contour.

Suppose $P$ is $n x,x,\ldots,f^{-1} y,y$.  Form a contour path
starting at $x$, then along $P$ to $y$, along $L$ to $n x$, and
continuing on $L$ to $n y$.  This is a M\"obius contour.
\end{proof}

\begin{definition}  Let $L$ be a contour loop on a hypermap.  We say that a dart
lies in the interior (resp. exterior) of the loop $L$, if there is a
path $P$ as in the previous lemma starting and ending with $f$ steps
(resp. $n^{-1}$ steps).
\end{definition}

\begin{lemma}  Let $L$ be a contour loop on a plain hypermap without
M\"obius contours.  Assume a dart $x$ lies in the interior (resp.
exterior) of the loop $L$. Then every dart in its $f$-orbit lies in
the interior (resp. exterior) of the loop.  Moreover, if the dart
$x$ does not lie on the same node as any dart in $L$, then every
dart in the $n$-orbit of $x$ lies in the interior (resp. exterior)
of $L$.
\end{lemma}

\begin{proof} First consider the $f$-orbit.
For a contradiction, we may assume that $x$ is in the interior but
not $f^{-1} x$.  If we have a contour path $P = \ldots,n  x,
x,\ldots$, we can modify it by stepping around the node of $f^{-1}
x$:
  $$P' = \ldots,n  x,f n  x = n^{-1}f^{-1} x,n^{-2}
f^{-1} x,\ldots,f^{-1} x,f x,\ldots.
    $$
This works when the new darts added to the path are not already on
$P$ and not on $L$.  If one of the new darts is on $L$, a subpath of
$P'$ starts and ends on $L$ in different types of steps, which is
prohibited by Lemma XX.  If a new dart is on $P$, we just eliminate
a segment of $P'$ if the new dart appears on the initial segment of
$P$.  If it appears on the final segment of $P$, there is a M\"obius
loop on $P$, which is contrary to hypothesis.  [XX finish.]


For the $n$-orbit, assume that $x$ but not $n^{-1} x$ is in the path
$P$.  We take a path $P = \ldots,x,f x,\ldots$ and modify it by
stepping around the face of $n^{-1} x$
    $$
    P' = \ldots,x,n^{-1}x,f n^{-1}x,f^2 n^{-1}x\ldots,f^{-1} n^{-1}
    x = n f x, f x,\ldots
    $$
[Can you get everything by symmetry? XX] [XX finish]
\end{proof}

\begin{definition}  A face or a node is interior (resp. exterior) to a
loop in a hypermap if all of its darts are interior (resp.
exterior).
\end{definition}

\begin{lemma} Suppose that in a nonempty hypermap without M\"obius contours,
there is a face that coincides with an $\tangle{e,n,f}$-orbit of
darts.  Then that face contains a dart that is fixed under $n$.
\end{lemma}

\begin{proof}  If the face contains a single dart, then it is
obviously fixed by $n$.  Assume that the face contains at least two
darts.  For a contradiction, assume that none of the darts is fixed
by $n$.  Thus, every node contains at least two darts.

We will use the face path $z,f z,f^2 z,\ldots$ to construct a
M\"obius contour. Since the set of darts is finite, the face path
must eventually revisit a node already encountered.  Thus, we can
find a subpath $z',f z',\ldots,f^{k+1} z'$ such that the first $k$
darts lie on distinct nodes, but $f^{k+1} z'$ and $z'$ lie in the
same node $A$.

If we had $f^{k+1} z' = z'$, then we have the full $f$-orbit in this
subpath, and hence also the full $\tangle{e,n,f}$-orbit.  We have
$k>0$, so $f z'$ is then a dart that has no other darts in its node,
and is hence a fixed-point.  Hence $f^{k+1} z'\ne z'$.

Continue the path further, so that $f^{r+1} z'$ is at the same node
as some $f^p z'$ (with $p < r$), at a different node than
$z',\ldots,f^r z'$, but so that the only repeated node among
$z',\ldots,f^r z'$ is the one at $z'$.  We have $0 < p$.

If $f^{r+1} z' = f^p z'$, then $f^{r+1-p} z' = z'$ so all darts are
in the segment $z',f z',\ldots,f^{r-p} z'$ and the only node with
more than one dart is the one at $z'$.  Thus, we have a fixed point.
So $f^{r+1} z'\ne f^p z'$.

Let $0 < k_1 < k_2 < \cdots < k_m < r+1$ be the indices at which
$f^{k_i} z'$ is at $A$.  Let $x = n^{-1} f^{k_m} z'$. If the segment
$x,n^{-1} x,\ldots,z'$ contains some $f^{k_i} z'$, we obtain a
M\"obius contour.  [XX DETAIL.]  Take $n^{-1}$ steps from $x$ to
$z'$. Then take $f$-steps to $y = f^p z'$, on to $n x = f^{k_m} z'$,
on to $f^{r+1} z'$, then $n^{-1}$ steps to $n y$.  This is a
M\"obius contour.
\end{proof}

\begin{lemma}[Jordan curve theorem for hypermaps]  If a plain hypermap
has no M\"obius contours then it is planar.
\end{lemma}

\begin{proof}  Face double walkups along edges preserve the planar
index.  By repeated application, we reduce to the case where every
connected component of $\tangle{e,n,f}$ contains a single face. In
other words, $f$ acts transitively on the darts in a given connected
component.

If there are any darts that are fixed by all three maps $e,n,f$,
then the walkup at that dart eliminates the dart while preserving
the planar index.  Thus, we may assume there are no such darts.

If there are any darts that are fixed points under $n$, then the
dart is degenerate.  The double walkup (of any type) along the edge
that meets that dart eliminates the dart while preserving the planar
index.  Thus, we may assume that there are no such darts.

By the previous lemma, the plane hypermap must be empty.  Thus, our
planar-index preserving transformations have transformed an
arbitrary plane hypermap without M\"obius contours into the empty
hypermap, which is clearly planar.  The result follows.
\end{proof}

%%%%%%%%%%%%%%%%%%%%
\section{Quotient Hypermaps}


\begin{definition} Two hypermaps $(D,e,n,f)$ and $(D',e',n',f')$ are
isomorphic, if there is a bijection $F:D\to D'$ such that
    $$h'\circ F = F\circ h$$
for $h=e,n,f$ (and $h'=e',n',f'$, respectively).
\end{definition}


\begin{definition}
Let $H$ be a hypermap and let $\cal L$ be a collection of contour
loops.  We say that $\cal L$ is a normal collection if the following
conditions hold. \begin{itemize}
 \item No dart is visited by two different loops in $\cal
L$.
 \item There are no fixed points under $e$ (so that we can distinguish $f x$
from $n^{-1} x$ at each dart).
 \item Every loop in the
collection visits darts in at least two nodes.
 \item If a loop in the
 collection visits a dart at a node, then every dart at that node is
 visited by some loop in the collection.
\end{itemize}
\end{definition}

From a normal collection we can create a new hypermap.   The darts
$D$ are maximal sequences $x,n^{-1} x, n^{-2} x,\ldots,n^{-k} x$
    appearing in some loop in $\cal L$.
The map $f$ takes the sequence
    $x,\ldots,n^{-k}x$ to the sequence (in the same contour loop) starting $f (n^{-k}
    x),\ldots$.
The map $n^{-1}$ takes the sequence
    $x,\ldots,n^{-k}x$ to the sequence (in some other contour loop)
starting $n^{-k-1} x$. Equivalently, $n$ takes the sequence
$x,\ldots,n^{-k}x$ to the sequence ending $n x$. The map $e$ is
defined by $e\circ n\circ f = I$.

\begin{definition}  The hypermap constructed from the normal collection
is called the quotient of $H$ by $\cal L$, and is denoted $H/{\cal
L}$.  $H$ is said to be a cover of $H/{\cal L}$.
\end{definition}

\begin{example} Assume that $H$ is a hypermap with no fixed points under $e$.
Assume that it satisfies the conditions above. Then the set of faces
defines a collection of contour loops (follow $f$ around each face:
$x,f x,\ldots$).  Each dart of the quotient is then just a singleton
set consisting of a single dart of $H$, and the quotient is
isomorphic to $H$ itself.
\end{example}

\begin{example} Assume that $H$ is a hypermap with no fixed points
under $e$.  Let $F = (x,f x,\ldots)$ be a face.  Let $\cal L$ be the
collection with two contour loops:  $(x,f x,\ldots)$ and its
``complement''
$$(n^{-1} x,
n^{-2} x,\ldots,n x,f n x = y,n^{-1} y, n^{-2} y,\ldots, n y, f n
y,\ldots)
$$
(See Figure XX.) The quotient hypermap has two faces $F$ and a
backside $F'$ of the same cardinality.
\end{example}

\begin{example} In the previous example, the quotient hypermap can
be described directly.  Let $2k$ be the number of darts.  We have a
hypermap $H_{2k}$, whose darts are arranged in two faces
    $$\{\pm x,\pm f x,\ldots, \pm f^{k-1} x\},$$
(one positive face, one negative face) whose node map is
    $$\pm y \mapsto \mp y,$$
and whose edge map is given by $e\circ n\circ f$.
\end{example}

\begin{lemma} Let $H$ be a plain hypermap, and let $\cal L$ be a
normal family.  Then $H/{\cal L}$ is a plain hypermap.
\end{lemma}

\begin{proof} Let $e'$, $f'$, and $n'$ be the edge, face, and node maps on the
quotient hypermap.  Write $\ldots x$ for the node in the quotient
ending in dart $x$ in $H$ and $x\ldots$ for the node in the quotient
starting with dart $x$ in $H$.  We have $e^2 x = x$, so that for any
dart $\ldots x$ in the quotient:
    $$\begin{array}{lll}
    {e'}^{-2} (\ldots x) &= n' f' n' f' (\ldots x) = n f n (f x \ldots) \\&=
    n f (\ldots n f x) = n (f n f \ldots) = \ldots n f n f x\\ &=
    \ldots e^{-2} x = \ldots x.
    \end{array}$$
Thus, ${e'}^{-1}$ and $e'$ have order $2$ on the quotient.
\end{proof}

\begin{lemma} Let $H$ be a plain planar hypermap, and let $\cal L$
be a normal family.  Then $H/{\cal L}$ is a plain planar hypermap.
\end{lemma}

\begin{proof} Suppose $H/{\cal L}$ is not planar.
Let $P$ be a M\"obius contour on $H/{\cal L}$.  It lifts uniquely to
a contour on $H$ with the property that the darts visited on $H$ are
precisely the darts that belong to a dart in the quotient.  This is
compatible with the node map $n$.  So the contour path lifts to a
M\"obius contour on $H$.  Thus, $H$ is not planar.
\end{proof}

\section{Flags}

\begin{definition}  If we have a function from the set of faces of a hypermap
to the set $\{\op{true},\op{false}\}$, we say that a face is true or
false, according to the value of the function.  A flag on a hypermap
is a function from its set of faces to the set
$\{\op{true},\op{false}\}$ that satisfies the following two
constraints.
\begin{itemize}
    \item If darts $x,y$ belong to true faces,
    then there is a contour path from $x$ to $y$ consisting of darts
    in true faces.
    \item Every edge of a false face is shared with a true face.
    \end{itemize}
An isomorphism of hypermaps with flags is an isomorphism of
hypermaps that respects the flags.
\end{definition}

\begin{example} Let $H/{\cal L}$ be a quotient hypermap.  We define
a natural $\phi$ from faces to $\{\op{true},\op{false}\}$ by setting
the value of a face $F$ to be $\op{true}$ exactly when every dart in
the face is a singleton of $H$.
\end{example}

\begin{example} The natural boolean map on the faces of the
quotient of example (REFXX 2 faces) is a flag.  If we identify this
quotient with the example $H_k$, then it is natural to define a flag
on $H_k$ with one true face and one false face.
\end{example}

\begin{example} If $H$ is a connected plain hypermap and $e$ has no fixed points,
and $\cal L$ is the example of (REF XX all faces), then the natural
map takes value $\op{true}$ on every face.  This is in fact a flag.
To say that $H$ is connected is to say $\#\tangle{e,n,f}=1$. Then
$x$ and $y$ are joined by a sequence where the step at every stage
is $z\mapsto h z$, for $h=e,n$, or $f$.  Using the relation $e\circ
n\circ f = I$, we eliminate the $e$-steps. Using the fact that $n$
and $f$ have finite order, we replace $z \mapsto n z$ by a sequence
of $n^{-1}$ steps.  This gives the desired path.
\end{example}

\begin{lemma}  Let $H$ be a hypermap with normal family ${\cal L}$.
If the natural boolean function on $H/{\cal L}$ has at least as many
true values as there are faces of $H$, then $\cal L$ is the family
in Example REFXX (faces).
\end{lemma}

\begin{proof}  If a face is true in $H/{\cal L}$ its darts are
singletons, and the darts in the face are naturally identified with
the darts of a face in $H$.  If there are at least as many true
values as there are faces of $H$, all the darts of $H$ are accounted
for in the true faces of the quotient.  Thus, the quotient has no
false faces.  The result follows.
\end{proof}


\section{Face insertion}


In this section, we describe an inductive construction of all
connected, plain, planar hypermaps that satisfy certain constraints.

We define a set ${\cal H}$ of tuples $(H,\phi,\{x,e x\},N,\lambda)$.
The triples are assumed to satisfy the following conditions:
\begin{itemize}
    \item $H$ is a hypermap.
    \item $\phi$ is a flag on $H$.
    \item $\{x,e x\}$ is an edge of cardinality two, such that one
    of its darts lies on a true face and the other on a false face.
    \item $N$ is the cardinality of the false face meeting $\{x,e
    x\}$.
    \item $\lambda$ is a partition of natural numbers:
        $$
        (\lambda_1,\lambda_2,\ldots,\lambda_r)
        $$
    arranged in increasing order $\lambda_1\le \cdots \le
    \lambda_r$.
    \item $\lambda_1 = 1$ and $\lambda_r \le N-1$.
\end{itemize}

We define a function from $\cal H$ to the set of hypermaps with
flags.  We call it face insertion, because the function creates a
hypermap with one more true face than its source hypermap.

In the ordered pair $\{x,e x\}$, we write $x_0$ for the dart in the
false face and $e x_0$ for the dart in the true face.  Write $f^i
x_0 = x_i$ for the darts in the face.

In the partition $\lambda$ replace each $\lambda_j$ except for the
first occurrence of a given natural number by a dummy symbol $*$.
Insert $N$ at the end of the list. For example, if $N=7$,
    $$(1,1,1,2,3,3,3,5,5)$$ becomes
    $$(1,*,*,2,3,*,*,5,*,7).$$
Then break at each natural number (except the first and the last)
creating shorter sequences, each starting and ending with a natural
number, and padded in between with dummy symbols. So
    $$(1,*,*,2,3,*,*,5,*,7)$$ becomes
$$(1,*,*,2),\ (2,3),\ (3,*,*,5),\ (5,*,7).$$
Then delete any sequence $(j,j+1)$ containing no dummy symbols and
consecutive numbers. Our example becomes
  $$
  (1,*,*,2),\  (3,*,*,5),\ (5,*,7).
  $$
This becomes an instruction for how to draw new edges to create a
new hypermap. A fragment $(j,*,*,\ldots,*,k)$ with $r$ dummy symbols
is an instruction to insert an edge between darts $x_j$ and $x_k$
and then to insert $r$ new nodes along this edge.  (Set $x_N =
x_0$.) Expressed in terms of double walkups, we first apply (in
reverse) the double walkup $W_f$, and then apply $r$ times (in
reverse) the double walkup $W_n$.  Each fragment creates $r$ new
nodes, one new face, and $r+1$ new edges.  The net result is the
value of the face insertion function on $(H,\phi,\{x,e
x\},\lambda)$.

For example, Figure REFXX shows the effect of the fragments
$$(1,*,*,2),\ (3,*,*,5),\ (5,*,7).
  $$ on a face with seven darts.


We mark the face containing the dart $x_0$ as true, and the other
faces created by the construction as false.   All other faces are
marked true or false as in the originating hypermap.  By
construction, the new hypermap contains exactly one more true face
than the originating hypermap.

\begin{lemma} The boolean function on faces constructed in this way is a
flag.
\end{lemma}

\begin{proof}  We show first that every edge of a false face is shared
with a true face.  If the false face is one of the newly created
faces, its edges are those along the new true face or the old false
face. Either way they are shared with a true face.   If it is a
false face from the originating hypermap, it shares edges with the
same true faces as before.

Next we show that every pair of darts $x,y$ in true faces can be
joined by a contour path.  If neither of the darts is in the
$f$-orbit of $x_0$, we can use the contour path that was used in the
originating hypermap.  If both are in the $f$-orbit of $x_0$, then
we can join $x$ to $y$ by a sequence of $f$-steps.

If $x$ is in the $f$-orbit of $x_0$, but not $y$, we first join $x$
to $x_0$ by $f$-steps, then to $n^{-1} x_0 = f e x_0$, which is in
the $f$-orbit of $e x_0$, which is a true face by construction.
From $n^{-1} x_0$, we can move to $y$.

Similarly, if $y$ is in the $f$-orbit of $x_0$ but not $x$; we first
take a contour path from $x$ to $e x_0$ then to $n^{-1} e x_0 = f
x_0$ then to $y$.
\end{proof}

\section{Inductive Hypermaps}

We claim that we can generate all sufficiently nice hypermaps by a
sequence of face-insertions.  The precise claim is as follows.

\begin{lemma}  Let $H$ be a hypermap with the following properties:
    \begin{enumerate}
        \item It is plain.
        \item It is planar.
        \item It is connected.
        \item The edge map $e$ has no fixed points.
        \item The node map $n$ has no fixed points.
        \item Every orbit of $f$ has cardinality at least $3$.
        \item There are at least $2$ faces.  (XX All hypoth. Needed?)
        \item Every face meets every node in at most one
        dart.
    \end{enumerate}
Let $\cal L$ be a normal family of contour paths in $H$. Assume that
the natural boolean function $\phi$ on $H/{\cal L}$ is a flag. Let
$\{x,e' x\}$ be an edge of $H/{\cal L}$ that meets both a true face
and a false face. Let $N$ be the cardinality of that false face. Let
$M$ be a constant such that every face of $H$ has cardinality at
most $M$. Then there is exists a partition $\lambda$ with at most
$M$ parts satisfying the conditions of XX, and a normal family
${\cal L}'$ such that $H/{\cal L}'$ with its natural boolean
function is isomorphic to the image of the face insertion map on
$(H/{\cal L},\phi,\{x,e' x\},N,\lambda)$. Moreover, $H/{\cal L}'$
has one more true face than $H/{\cal L}$ has.
\end{lemma}

\begin{remark} The lemma says that sufficiently nice hypermaps can be recovered
from their quotients by face insertions.  (Here {\it `nice'} means
that it satisfies the conditions of the lemma.)   (By Lemma XX, as
soon as a quotient has sufficiently many true faces, it is
isomorphic to the the original hypermap, and the process
terminates.) We can start with a standard quotient such as $H_k$
from example XX that every nice hypermap is known to have. Then we
generate all face insertions (as we vary over relevant partitions)
and we reconstruct a list of all nice hypermaps up to isomorphism.
Note that we get to choose the edge at which we apply the face
insertion, but we cannot choose the partition.  We must run through
all partitions.  To make the entire process finite, we can bound the
number of partitions that must be considered, say by placing a
priori bounds $M$ on the cardinalities of the faces.
\end{remark}

\begin{proof}
Write $f,e,n$ for the maps on $H$ and $f',e',n'$ for the maps on the
quotient hypermap.   The construction picks out an edge of $H/{\cal
L}$ that borders a true face and a false face. Let the edge be given
by $\{y,e' y\}$, where $y$ lies on the true face and is hence a
singleton set. It can be identified with a dart $y$ in $H$.  Let
$y_0 = e y$, and let $y_k = {f}^k y_0$ be the darts of $H$ on the
face of $y_0$.  Let $r$ be the cardinality of the face. (XX Fix
notation to be consistent with construction XX.) Let $*$ be an
object distinct from the darts of $H/{\cal L}$.  We have a map
$\phi$ from $\{y_i\}$ to the union of $\{*\}$ with the darts of
$H/{\cal L}$ sending $y_i$ to $\{*\}$ if $y_i$ is not visited by any
of the loops in ${\cal L}$, and to the appropriate dart otherwise.
Since every face meets every node in at most one dart, no two darts
$y_k$ map to the same dart in $H/{\cal L}$.

Let $x_0 = e' \{y\}$ and let $x_k = {f'}^k x_0$.  Let $r'$ be the
cardinality of the $f'$-orbit through $x_0$.  The darts $x_k$ are
the darts of $H/{\cal L}$ used in the construction of the map
$X\mapsto X'$. We claim that $\phi(y_i)$ is either $*$ or one of the
nodes $x_k$.  For those that give darts $x_k$, write $\phi(y_i) =
x_{\phi' i}$, for some $0\le i < r'$. We claim further that if
$i<j\le n$, and both $\phi(y_i)$ and $\phi(y_j)$ are darts, then
$\phi' i < \phi' j$.

Accepting these two claims, we complete the proof.  Set $\phi' i =
*$, if $\phi(y_i)$ is not a dart.  We form a sequence
    $$(\phi' 1,\phi' 2,\ldots,\phi' r).$$
In this sequence, replace each subsequence $i,*,*,\ldots,*$ by a
sequence of the same length consisting of straight $i$s. Thus,
    $$1,*,*,2,*,4,*,*,5$$
becomes
    $$1,1,1,2,2,4,4,4,5.$$
This gives the partition $\lambda$ used in the construction.

Applying the face-insertion for $\lambda$ to the hypermap $H/{\cal
L}$ corresponds to constructing some $H/{\cal L}'$.  We describe
${\cal L}'$.  It is the collection of contour loops obtained from
${\cal L}$ after deleting the loop $L$ that gives the $f'$-orbit
$\{x_k\}$, and adding the loop corresponding to the $f$-orbit
$\{y_k\}$ as well as the loops corresponding to the false faces on
the new hypermap.

More specifically, we show that each false face of the face-inserted
hypermap corresponds to some contour loop in $H$. These are the
contour loops that are added to ${\cal L}\setminus\{L\}$ to obtain
${\cal L'}$.  There is a contour loop for each sequence
$j,*,\ldots,j'$ with $j< j'$ and one for each sequence $j,j'$ with
$j+1< j'$.  A contour loop in $H$ is determined by the darts it
visits.  We will describe the darts.

Take the case $j,j'$, with $j+1<j'$ In this case we take the darts
in $x_j$ in the consecutive sequence starting $n^{-1} y_j$, and the
darts in $x_{j'}$ in the consecutive sequence starting $n y_{j'}$.
(Note: it can be shown that $j+1 < j'$ implies that both $n^{-1}
y_j$ and $n y_{j'}$ are in $x_j$ and $x_{j'}$ respectively.) We also
have it visit all the darts of $x_k$ for $j<k<j'$.

In the case $j,*,\ldots,j'$ we form consecutive sequences of darts
starting at $n^{-1} y_j$ and $n y_{j'}$ as in the previous case, and
all the darts of $x_k$ for $j<k<j'$, but we also have the loop visit
all the darts of the nodes $*$, except for the ones in $\{y_k\}$.

This construction gives a normal family of loops that has all the
required properties.

The proof is now complete except for two earlier claims.  We claim
that $\phi(y_i)$ is either $*$ or one of the darts $x_k$.  Suppose
to the contrary that we arrive at a dart $z$ at a node that is not
one of the darts $x_k$.  We take $n x = y_1$.  Define $y$ to be such
that $\{y,\ldots,y_0\} = x_0$. Start at $x$ follow $L$ all the away
around the loop to $n x$, passing through $y$ on the way.  Then from
$y_1$ take $f$ steps to $z$, step around that node to a true face,
along true faces to $n y$.  This is a M\"obius contour.

We claim further that if $i<j\le n$, and both $\phi(y_i)$ and
$\phi(y_j)$ are darts, then $\phi' i < \phi' j$.  Assume to the
contrary that $\phi' j \le \phi' i$.  Then we take a contour loop
$L'$ from $y_1$ along the darts $y_j$ until reaching $x_{\phi' i}$,
then continuing for the rest of the path of $L$ back to $y_1$.
Giving the identical argument as in the preceding paragraph, but
using $L'$ instead of $L$, we again get a M\"obius contour.  Since
M\"obius contours do not exist on planar graphs, we get the desired
contradiction that completes the proof.
\end{proof}

\chapter{Hypermap and Geometry}

In this section we show how a hypermap can be attached to certain
graphs whose nodes are vectors in $\ring{R}^3$. This hypermap will
encode the combinatorial properties of the vectors.

\begin{definition}  Let $v\in\ring{R}^3$ and $W \subset
\ring{R}^3$.  We say that $\sigma:W\to W$ is an azimuth cycle on $W$
coming from $v$, if there is a orthonormal $2$-frame $P=(0,e_1,e_2)$
with $e_1 \times e_2 = v/|v|$, and a cycle is $\sigma:W\to W$ with
respect to $P$. (By REFXX, an azimuth cycle is unique, but may not
exist.)
\end{definition}

%[XX change the following definition, so that edges are triples
%$(v,w,u)$, where $u$ is a unit vector orthogonal to $v$ and $w$.
%Assume that if $(v,w,u)\in E$ then $(w,v,-u)\in E$.  We then ask for
%an azimuth cycle on the vectors $u\times v$ rather than on the
%vectors $w$.  The third element $u$ allows for the case that the $v$
%and $w$ are antipodal, or the long end of a great circle, which is
%convenient for some of the proofs. In standard situations, we can
%just take $u$ to be the unit length vector in the direction $v\times
%w$.  This change ripples through the text.  For instance, the proof
%that  $\#c = \#f$ for linear graphs reduces all the way down to the
%case of a single plane.]

If $e=\{v,v'\}$ is a set of vectors of cardinality two,  set
  $$
  \begin{array}{lll}
  C_e &= \{t v + t' v' \mid t \ge 0,\ t'\ge 0\}\\
  C_e^0 &= \{t v + t' v' \mid t > 0,\ t' > 0\}.
  \end{array}
  $$

[XX problem of base point in $C_e$.  Is it $0$ or $v_0$?]


\begin{definition}  Let $(V,E)$ be a pair consisting of a set of
vectors and a subset of the powerset of $V$.  The pair is said to be
a pre-planar graph if the following conditions hold.
    \begin{itemize}
    \item $V$ is finite and nonempty.
    \item Each element of $V$ is a nonzero vector in $\ring{R}^3$.
    \item Each element of $E$ is finite of cardinality two.
    \item For each $v\in V$, there exists an azimuth cycle $\sigma_v$ coming from $v$ on
        $$
        E_v = \{w\in V\mid \{v,w\}\in E\}
        $$
        coming from $v$.
    \item For sets $e,e'\in E$,   we have
        $$C^0_e \cap C_{e'}^0\ne\emptyset\ \Rightarrow e = e'.$$
    \end{itemize}
\end{definition}

\begin{remark*}
The pair $(V,E)$ is a graph with nodes $V$ and edges $E$.  The set
$E_v$ is the set of edges around a fixed node $v$.
\end{remark*}

Note that $w\in E_v$ if and only if $v\in E_w$.   The condition on
$E_v$ is a nondegeneracy condition that says that if we project all
the vectors to the plane through the origin perpendicular to $v$,
the vectors sufficiently close to $v$ have different angles.

The final condition states that the cones generated by edges do not
meet, except when they are forced to meet for obvious reasons.  It
is this condition that will give us a construction of planar
hypermaps.

\begin{remark} The hypothesis of the existence of an azimuth cycle
prevents $\{0,v,v'\}$ from being a collinear set, when $\{v,v'\}\in
E$.
\end{remark}

\begin{remark}
For each $v\in V$, we have an azimuth cycle $\sigma_v:E_v\to E_v$.
We specifically allow the situation where $E_v$ is finite of
cardinality one.  (See example XX-sum of azim.) In this case,
$\sigma_v$ is the identity map.
\end{remark}

\section{Construction of hypermaps}

Let $(V,E)$ be pre-planar.  We define a set of darts by
    $$D = \{(v,w,w')\mid v\in V,\ w\in E_v,\ w' = \sigma_v w\}.$$
We define a permutation $n$ on $D$ by
    $$n(v,w,w') = (v,w',\sigma_v w').$$
We define a permutation $f$ on $D$ by
    $$
    f (v,w,w') = (w,\sigma_w^{-1} v,v).
    $$
Define a permutation $e$ on $D$ by
    $$
    e (v,w,w') = (w,v,\sigma_w v).
    $$
Write $\op{hyper}(V,E)=(D,e,n,f)$.

Note that if $u\in V$ is a vector that is not an element of any
edge, then it is not an entry in any of the darts $(v,w,w')$ and
does not enter into the construction of $\op{hyper}(V,E)$  in any
way.

\begin{lemma} Let $(V,E)$ be pre-planar.  Then
    \begin{itemize}
    \item $\op{hyper}(V,E)$ is a plain hypermap.
    \item The edge map $e$ has no fixed
points.
    \item The face map $f$ has no fixed points.
    \item For every pair of distinct nodes, there is at most one
    edge meeting both.
    \item Each dart of an edge lies on a different node.
    \end{itemize}
\end{lemma}

\begin{lemma}  We compute
    $$e(n(f(v,w,w'))) = e(n(w,\sigma_w^{-1} v,v))) =
        e(w,v,\sigma_w v) = (v,w,\sigma_w v) = (v,w,w').$$
So it is a hypermap. We compute
    $$e(e(v,w,w')) = e(w,v,\sigma_w v) = (v,w,w').$$
So it is plain. A fixed point under $e$ would force $v = w\in E_v$,
but by construction $v\not\in E_v$.  The argument that $f$ has no
fixed points is similar.

    Next we show that for every pair of distinct nodes, there is at
most one edge meeting both.  That is,
        $$(n^k e x = e n^\ell x)\Rightarrow n^\ell x = x.$$
Let $x = (v,w,w')$.  Then
    $$
    \begin{array}{lll}
    n^\ell x &= (v,\sigma^\ell w,\sigma^{\ell+1}w)\\
    e n^\ell x &= (\sigma^\ell w,*,*)\\
    e x &= (w,*,*)\\
    n^k e x &= (w,*,*)\\
    n^k e x &= e n^\ell x \Rightarrow w = \sigma^\ell w \Rightarrow
    n^\ell x &= (v,w,\sigma w) = x
    \end{array}
    $$

Finally, we show that each dart of an edge lies on a different node.
That is, $e x \ne n^k x$.  We have
    $$
    \begin{array}{lll}
        e(v,w,w') &= (w,*,*),\quad w\in E_v\\
        n^k(v,w,w') &= (v,*,*),\quad v\not\in E_v.
    \end{array}
    $$
The result follows.
\end{lemma}

\section{Topology of pre-planar graphs}

Let $(V,E)$ be a pre-planar.  Let $X=X(V,E)$ be the union of the
cones
    $$C_e$$
as $e$ ranges over $E$.

Let $$S^2 = \{ v \mid | v | = 1\}$$ be the unit sphere in
$\ring{R}^3$.  The set $\ring{R}^3$ is a metric space under the
Euclidean distance function $d(v,w) = |v-w|$.  Subsets of
$\ring{R}^3$ are a metric space under the restriction of the metric
$d$ to the subset. Subsets carry the metric space topology.  We
investigate the connected components of $S^2\setminus X$.  If two
points in $S^2$ can be joined by a continuous path that avoids $X$,
then they lie in the same connected component of $S^2\setminus X$.
If we produce a family of nonempty connected open sets in
$S^2\setminus X$, whose union is all of $S^2\setminus X$, then the
members of the family are the connected components of $S^2\setminus
X$.

\begin{remark} Warning: the term `connected' is now being used in
two different senses: as in connected hypermaps and as in
topologically connected sets.  The first is a purely combinatorial
property of hypermaps, the second is purely topological.
\end{remark}

If $e=\{v,v'\}\in E$, then $0,v,v'$ are not collinear, so that $C_e$
does not lie in a line, and does not contain any points $u,-u\in
C_e$ with $u\ne 0$.  The intersection of $C_e$ with $S^2$ is an arc
of a great circle that does not contain any antipodal points.



\section{Components and Darts}

In the following, we assume that $2$-frames are chosen as in the
definition of pre-planar graphs.  Each dart $x=(v,w,w')$ carries
with it a $2$-frame $P$, and we have the wedges
    $$W_x = \{u \in\ring{R}^3 \mid \hc(P, u) = P(r\cos\theta,r\sin\theta)\
       r > 0\ \op{azim}_P w < \theta < \op{azim}_P w'\}.$$
These are the vectors lying in one of the four sectors defined by
two planes through the origin, the plane through $0,v,w$ and the
plane through $0,v,w'$.

We can restrict further by putting conditions on the zenith angle.
Let $P'$ be the $3$-frame $P= (0,e_1,e_2,e_1\times e_2)$.
    $$W_x(\phi_0) =
    \{P'(r\cos\theta\sin\phi,r\sin\theta\sin\phi,r\cos\phi)\in W_x\mid
    \phi < \phi_0\}$$

\begin{lemma} For each $x$, and $\phi$ sufficiently small and positive,
$W_x(\phi)\cap S^2$ is nonempty and lies in a single connected
component of $S^2\setminus X$.
\end{lemma}

\begin{proof}  First we show that $W_x(\phi)$ lies in $S^2\setminus X$,
for $\phi$ small.  Let $x=(v,w,w')$.  By making $\phi$ small enough,
the sets avoid the compact sets $C_e\cap S^2$ when $v\not\in e$.
Each set $C_e$ with $v\in e$ has constant azimuth around $v$.  By
the azimuth cycle on vectors, none has azimuth between that of $w$
and $w'$, so these sets avoid $W_x$.

The paths along coordinate axes in spherical coordinates are
continuous:
    $$
    \theta \mapsto
    P'(\cos\theta\sin\phi_0,\sin\theta\sin\phi_0,\cos\phi_0),\
    \phi \mapsto
    P'(\cos\theta_0\sin\phi,\sin\theta_0\sin\phi,\cos_0\phi),\
    $$
and can be used to connect any two points in $S^2\cap W_x(\phi)$.
Thus, $S^2\cap W_x(\phi)$ is connected, and lies in a single
connected component of $S^2\setminus X$.
\end{proof}

\begin{definition} For each dart, there is then a well-defined connected
component of $S^2\setminus X$ that contains $W_x(\phi)$ (for all
sufficiently small $\phi$). We say that the dart {\it leads into}
that component.
\end{definition}

\begin{lemma}
For each connected component of $S^2\setminus X$, there exists a
dart $x$ that leads into that component.
\end{lemma}

\begin{proof} This follows from the following stronger claim, which constructs
a continuous curve from any point in $S^2\setminus X$ to a wedge
$W_x(\phi)$.
\end{proof}

\begin{lemma}  Let $(V,E)$ be a pre-planar graph.  Assume that $E\ne\emptyset$.
Let $X=X(V,E)$.
Let $v\in S^2\setminus X$.  Then there exists a dart $x$ of
$\op{hyper}(V,E)$ such that there exists $\theta_v,\phi_v$ with
    $$p = P'(\cos\theta_v\sin\phi_v,\sin\theta_v\sin\phi_v,\cos\phi_v)\in
    W_x$$
and such that the curve
    $$\phi\mapsto
    P'(\cos\theta_v\sin\phi,\sin\theta_v\sin\phi,\cos\phi)
    $$
lies in $S^2\setminus X$,  for $0 < \phi \le \phi_v$.
\end{lemma}

\begin{proof}  Suppose that no such dart exists.  Let $P$ be an
orthonormal $2$-frame $(0,e_1,e_2)$ such that $e_1\times e_2 =
v/|v|$.  [XX We use a fact of spherical geometry that needs proof:
if we take a arc from $\phi=\phi_0$ to $\phi=\phi_1$ fixing
$\theta,r$ in spherical coordinates with respect to $P$, and if $v'$
is a point along that arc, then taking spherical coordinates with
respect to $P'$ the plane orthogonal to $v'$ again gives an arc
along a longitudinal line.]  We define a function $\psi$ from
$[0,2\pi]$ to the set of edges $E$ augmented by a symbol $\{*\}$ by
sending $\theta$ to the edge corresponding to the first cone $C_e$
encountered along the great circle starting out along constant
azimuth $\theta$ from $v$ (and returning back to $v$ along the
azimuth $-\theta$). (Assign value $*$ if this great circle does not
meet $X$.)   Using this fact about spherical geometry and the
assumption that $(V,E)$ is pre-planar, the assumption that no such
dart exists implies that this function is well-defined.  (This is
where we are finally using the pre-planar assumption.)

The preimage of each value is an open set.  These sets are disjoint
and cover $[0,2\pi]$.  By connectivity of $[0,2\pi]$, the function
is constant on $[0,2\pi]$.  Since $E$ is nonempty, and the great
circles through $x$ cover the sphere [XX another fact to prove],
some great circle through $v$ meets an edge, so the value of $\psi$
is not identically $*$.

[XX We need another fact from spherical geometry.  If $\rho$ is an
arc of a great circle that does not contain antipodal points, and if
$v$ is a point on the sphere that does not meet $\rho$, then there
is a great circle through $v$ that does not meet $\rho$.]  This fact
shows that $\psi$ cannot be constant.  This completes the proof.
\end{proof}

\section{Components and Faces}

Let $(V,E)$ be a pre-planar graph.  Let $X=X(V,E)$ as above. For
every dart $x=(v,w,w')$, we define a region $U_x(\epsilon)$ along
the edge $e = \{v,w\}$ as follows. Let $P$ be the orthogonal
$2$-frame for $v$ and $P'$ the orthogonal $2$-frame for $w$.  Then
take the intersection of wedges:
    $$\begin{array}{lll}
    U_x(\epsilon) &= \{u \in\ring{R}^3 \mid \hc(P, u) = P(r\cos\theta,r\sin\theta)\
       r > 0\ \op{azim}_P w < \theta < \op{azim}_P w + \epsilon\}
       \cap\\
       &\{u \in\ring{R}^3 \mid \hc(P', u) = P'(r\cos\theta,r\sin\theta)\
       r > 0\ \op{azim}_{P'} v -\epsilon < \theta < \op{azim}_{P'} v\}.
    \end{array}$$
[XX check]

\begin{lemma}
For $0<\epsilon$ sufficiently small, the set $S^2\cap U_x(\epsilon)$
is non-empty, and is a subset of some connected component of
$S^2\setminus X$.
\end{lemma}

\begin{proof} This is similar to the proof  for $W_x(\phi)$.  This
proof uses the assumption that $(V,E)$ is pre-planar and that
$C_e\cap S^2$ is compact to bound the sets $C_e\cap S^2$ away from
$U_x(\epsilon)$ ($\epsilon$ small) when $e\cap \{v,w\} = \emptyset$.
\end{proof}

\begin{lemma} Let $(V,E)$ be pre-planar.
If $x$ and $y$ are darts in the same face of the hypermap
$\op{hyper}(V,E)$, then they lead into the same connected component.
\end{lemma}

\begin{proof}  For small $\epsilon$, $\phi$, $\phi'$,
$S^2 \cap U_x(\epsilon)$ meets $S^2\cap W_x(\phi)$ and $W_{f
x}(\phi')$ so all lie in the same connected component of
$S^2\setminus X$.  So $x$ and $f x$ lead into the same connected
component.  By induction, $x$ and $f^i x$ lead into the same
connected component for all $i$.
\end{proof}

This implies that the number of connected components is no more than
the number of faces.  We can do even better.  First we need a little
lemma.

\begin{lemma} Let $(V,E)$ be a pre-planar graph.  Suppose that
deleting an edge of the graph is given combinatorially as a split
double face walkup.  Then the number of connected components of
$S^2\setminus X(V,E)$ is preserved by the edge deletion.  All
components are left as before, except one that differs only by the
presence of the edge.
\end{lemma}

\begin{proof}
The components other those along the edge are unaffected. Suppose
that there are two connected components $U,U'$ in $H'$ that become
connected after the edge deletion.  If sets $U_x(\epsilon)$ and
$U_{e x}(\epsilon)$  ($\epsilon$ small) were both to belong to the
same component $U$, then $U\cup\{\text{the open-ended edge}\}$ XX
and $U$ are disjoint open after the deletion, contrary to the
assumption they become connected.  So $U_x(\epsilon) \subset U$ and
$U_{e x}(\epsilon) \subset U'$.  Thus, $x$ and $e x$ lie on
different faces of $H'$ and this implies that the double walkup is a
merge, contradicting the hypothesis that it is split. Thus, the
number of connected components remains constant.
\end{proof}

\begin{lemma} Let $(V,E)$ be a pre-planar graph.  Suppose that
deleting an edge of the graph is given combinatorially as a merge
double face walkup.  Let $x$ be a dart along the given edge.  Assume
that $x$ and $e x$ lead into distinct components. Then the number of
connected components of $S^2\setminus X(V,E)$ is decreased by one by
the edge deletion. All components except the components of $x$ and
$e x$ (along the given edge) are left as before.  The components of
$x$ and $e x$ are joined.
\end{lemma}

\begin{proof} The components other than those of $x$ and $e x$ are
unaffected.  The darts $f x$ and $e x$ lead into different
components before the deletion, but the wedge $W_{f x}$ and $W_{e
x}$ are combined into $W_{f' x}$ after the deletion.  So the two
components are combined.
\end{proof}

\begin{lemma}  Fix an orbit $A$ of $\tangle{e,n,f}$.  Let $X_1$ be
the union of cones $C_{\{v,w\}}$ such that there is a dart $x =
(v,w,w')$ in $A$.  Let $X_2$ be the union of cones coming in the
same way, but for darts not in $A$.  Then $S^2\cap X_1$ is disjoint
from $S^2\cap X_2$.
\end{lemma}

\begin{proof} For a fixed $v$, the darts of
the form $(v,\cdot,\cdot)$ or $(\cdot,v,\cdot)$ are in the same
orbit of $\tangle{e,n,f}$ so the sets are disjoint. [XX details?]
\end{proof}



\section{Linear Graphs}


\begin{definition}
  $(V,E)$ is a linear graph if the following hold:
  \begin{itemize}
  \item $(V,E)$ is a pre-planar graph.
  \item For each $\{v,w\}\in E$ and each $u\ne0$ orthogonal to both
  $v$ and $w$, we have an azimuth cycle $\tau_u$ on
    $$V_u = \{ v' \in V \mid v'\cdot u = 0\}$$
    coming from $u$.  Moreover, if $v'\in V_u$, then $\{v',\tau_u
    v'\}\in E$, and
    $$\cup_{i} C_{\{\tau_u^i v',\tau_u^{i+1} v'\}} = \{x \in \ring{R}^3\mid
    x\cdot u = 0\}.$$
  \end{itemize}
\end{definition}

[XX notation is too heavy.]

Intuitively, this says that once there is one edge with $C_e$ in a
plane, then the cones continue in a cyclic way to fill the entire
plane. The intersection of the cones $C_e$ with $S^2$ will then be a
great circle.  A linear graph is one that can be viewed as
constructed from a finite number of great circles on the unit
sphere, whose nodes include all the points of intersection of the
great circles.

Let $(V,E)$ be a linear graph.   Let $F$ be a finite set of linear
functions whose zero sets define the planes $P$ entering the
construction. For every choice of signs $\epsilon: F\to \{\pm 1\}$,
we have a cone
    $$R_\epsilon = \{v\in\ring{R}^3 \mid \forall f\in F.\ \epsilon_f f(v) > 0\}.$$
The sets are clearly disjoint for different sign choices $\epsilon$.
The union of the sets $R_\epsilon$ fills all of $\ring{R}^3$ except
for the points in the planes $\{v\mid f(v)=0\}$.


\begin{lemma} Let $(V,E)$ be a linear graph.  Let $X=X(V,E)$.
The sets $S^2\cap
R_\epsilon$ lie in $S^2\setminus X$.  Each $S^2\cap R_\epsilon$ is
empty or a connected component of $S^2\setminus X$.
\end{lemma}

\begin{proof}  The set $X$ lies in the union of the planes $\{v\mid
f(v)\}$, so $R_\epsilon$ is disjoint from $X$.  This proves the
first claim.  Each $R_\epsilon$ is open, and its intersection with
$S^2$ is open.  Each $R_\epsilon$ is convex (an intersection of
half-spaces).  The line segment between two points $v,w$ of
$R_\epsilon$ projects to an arc in $S^2\cap R_\epsilon$, showing
that any two points $v,w\in S^2\cap R_\epsilon$ can be joined by a
continuous curve.  Thus, $S^2\cap R_\epsilon$ is connected.

Every point of $S^2\setminus X$ lies in some $R_\epsilon$.  The sets
are open, connected, disjoint, and fill $S^2\setminus X$. Thus, the
nonempty ones are the connected components.
\end{proof}

\begin{lemma} Let $(V,E)$ be a linear graph.  Each face of
$\op{hyper}(V,E)$ has cardinality at least $3$.
\end{lemma}

\begin{proof} We have proved in Lemma XX that $f$ has no fixed
points.  If $f^2 (v,w,w')$ has order two, then
    $$(v,w,w') = f^2(v,w,w') = f(w,*,v) = (*,*,w).$$
So $w=w'=\sigma_v(w)$.  Thus, the set $E_v$ of edges at $v$ is a
singleton set $E_v = \{\{v,w\}\}$.  Let $e=\{v,w\}$.  Let $P$ be the
plane of $C_e$.

Let $u\ne0$ be orthogonal to $v$ and $w$ and let $\tau=\tau_u$ be
the azimuth cycle on $P\cap V$.  Since $\{\tau^{-1}v,v\}$ and $\{tau
v,v\}$ are in $E$, we have $\tau^{-1}v = \tau v = w$.  So $\tau$ has
order $2$.  Then by the definition of linear graph,
  $$C_{\{v,w\}} = \cup_{i} C_{\{\tau_u^i v',\tau_u^{i+1} v'\}} = P.$$
This contradicts the fact that in a pre-planar graph, no $C_e$
contains antipodal points.
\end{proof}

\begin{lemma}  Let $(V,E)$ be a linear graph.  Let $C$ be a
connected component of $S^2\setminus X$.  Then every dart that leads
into $C$ belongs to the same face of $\op{hyper}(V,E)$.
\end{lemma}

\begin{proof}  Let $\#c$ be the number of connected components.  The
map from faces to onto the set of connected components gives
        $$\#c \le \#f.$$
If we prove the reverse inclusion $\#f \le \#c$, then we have a
bijection between connected components and faces, and the result
follows.


[XX fix proof. We now allow antipodal points for endpoints of an
edge.  Thus, we can reduce to a single plane.] Suppose first that
there is a line that contains all the intersections of the planes.
Suppose there are $k$ planes. There are two nodes (the north and
south poles of the sphere). There are $k$ darts at the north pole.
Every face contains a dart at the north pole, so $\#f \le 2 k$. The
wedges $W_x$ as $x$ runs over darts at the north pole give nonempty
disjoint open connected sets, so $2 k \le \#c$.  This proves the
result in this case.

If there is no line that contains all the pairwise intersections of
the planes, pick a plane $P$ and delete all the nodes that do not
lie on the intersection with another plane, and then delete all the
edges such that $C_e\subset P$.  This is again a linear graph. This
operation decreases $\#c - \#f$, because the number of components
decreases by at least $1$ for each edge deleted and the number of
faces increases by at most $1$ for each edge deleted.

By repeatedly eliminating planes, we find that eventually there is a
line that contains all the intersections of the planes.  So $\#c -
\#f$ has been decreased to zero, and hence it must have been
non-negative in the beginning.
\end{proof}

Thus, we have a 1-1 correspondence between connected components and
faces of the hypermap.  Each dart $x=(v,w,w')$ has associated with
it the azimuth angle of $(0,v,w,w')$ [XX bad notation].  Write it
$\op{azim}(x)$.

\begin{lemma} Let $(V,E)$ be a linear graph.  Let $C$ be a connected
component of $S^2\setminus X$, associated with the face $F$ of the
hypermap $\op{hyper}(V,E)$.    Then $C$ is spherically measurable,
and its solid angle  is
    $$2\pi + \sum_{x\in F} (\op{azim}(x)-\pi).$$
\end{lemma}

\begin{proof} If $\card(F)=3$, then this is the Formula~REFXX for the solid
angle of a spherical triangle.  In general, proceed by complete
induction on $k$, with base case $3$.  [XX either add an edge to
break the triangle, or add an external triangle that decreases the
number of sides by one.  Add detail.]
\end{proof}

\begin{lemma} Let $(V,E)$ be a linear graph.  Then $\op{hyper}(V,E)$
is connected.
\end{lemma}

\begin{proof}  Let $x$ and $y$ be two darts.  Pick any edge $e$ at
$x$ and $e'$ at $y$.  Construct the planes $P_e$ and $P_{e'}$
containing $C_e$ and $C_{e'}$.  These planes, if not equal, meet in
a line.  Pick a ray from the origin in this line.  This ray meets
two different cones, one on $P_e$ and one on $P_{e'}$.  By
nondegeneracy, this ray contains a vertex $v\in V$.  Pick a dart $z
= (v,\ldots)$.  It is enough to construct a path from $x$ to $z$
then $z$ to $y$.  Thus, we can now assume that $x$ and $y$ are darts
on the same plane.  Nodes on the same plane are linked by an azimuth
cycle.  So it enough to assume that $x$ and $y$ are darts at $v$ and
$\tau v$.  But then $\{v,\tau v\}\in E$, so connectivity is clear.
\end{proof}

\begin{lemma} Let $(V,E)$ be a linear graph.  Then $\op{hyper}(V,E)$
is planar.
\end{lemma}

\begin{proof}  Since every edge has order $2$, we have $\# D = 2\#
e$.  Since the graph is connected $\#\tangle{e,n,f} = 1$.  The Euler
relation takes the form
    $$
    \# n - \# e + \# f = 2.
    $$
Let $k(F)$ be the number of darts on face $F$.  We have $\sum_F k(F)
= \# D = 2 \# e$.  The solid angle of a sphere is $4\pi$, which is
the same as the sum of the solid angles of the connected components.
We have by Lemma XX and Lemma XX,
    \begin{equation}\begin{array}{lll}
    4\pi &= - \sum_F (k(F) - 2)\pi + \sum_F \sum_{x\in
    F}\op{azim}(x)\\
        &= - 2 \pi\# e  + 2 \pi\# f  + 2\pi\# n.
    \end{array}
    \end{equation}
Dividing by $2\pi$, we get the Euler relation.  Hence the hypergraph
is planar.
\end{proof}

\section{Planarity}



\begin{lemma}  Let $(V,E)$ be a pre-planar graph.  Let $\#c$ be the number
of connected components of $S^2\setminus X(V,E)$.  We have
    \begin{itemize}
    \item The hypergraph $\op{hyper}(V,E)$ is planar.
    \item No two faces in a given $\tangle{e,n,f}$ orbit lead
    into the same connected component.
    \item $\#c = 1 + \#f - \#\tangle{e,n,f}$
    \item Let $C$ be a connected component. Let $F\mapsto C$ mean
    that the face leads into $C$.  Then $C$ is spherically measurable and
        $$\op{sol}(C) = 4\pi + \sum_{F\mapsto C}(-2\pi + \sum_{x\in F}
        (\op{azim}(x)-\pi))$$
    \end{itemize}
\end{lemma}

\begin{proof}  These formulas have all been established for linear
graphs.  We show that every pre-planar graph can be obtained from a
linear graph by deleting nodes and edges (double walkup
transformations).  We then keep track of the effect of these
deletions on the truth of the the statements of the lemma.

For each edge of $(V,E)$, construct the plane containing $C_e$. Add
elements to $V$ to get $V'$ for each point of intersection of $S^2$
with two planes (unless there is already a point of $V$ on the same
ray). For each plane, divide it into cones by drawing the rays
through the vectors, and use these cones to construct edges $E'$.
This is a linear graph.

We recover $(V,E)$ and its hypermap by deleting vertices and edges.
These operations can be described combinatorially on the hypergraph,
but also geometrically in terms of $V$, $E$, and so forth.

The walkup transformations send planar hypermaps to planar
hypermaps.  Thus, $\op{hyper}(V,E)$ and the intermediate hypermaps
are all planar.  For planar maps, the planar index is preserved by
walkups.

Consider a split form of the double face walkup.  A face walkup that
splits the face also increases the number of orbits of
$\tangle{e,n,f}$ by one, and preserves the number of connected
components (by LemmaXX).  The faces that lead into the connected
component are as before, together with the two faces created by the
split.  The solid angle formula decreases by $2\pi$ for the new
face, and increases by $2\pi$ for the two fewer darts.  So both
sides of the solid angle formula are unchanged.  It is now clear
that all of the statements are preserved in the split case.

Consider the merge form of the double face walkup.  The number of
connected components drops by $1$.  The number of faces drops by
$1$.  The darts lead into the same connected components as before,
except the merged face leads into the merged component.  The orbits
under $\tangle{e,n,f}$ are unchanged.  It is easily seen that the
area formula is compatible with the merge $$\op{sol}(C_1) +
\op{sol}(C_2) = \op{sol}(C_1\cup C_2).$$ It is now clear that all of
the statements are preserved in the merge case.

We now consider a double edge walkup at a node of cardinality two.
This does not affect any of the terms in any of the formulas.  All
of the statements are preserved.

By a combination of these transformations we arrive at $(V,E)$.  The
conclusion follows.
\end{proof}


\begin{lemma*} (Jordan curve theorem)  Let $(V,E)$ be a pre-planar
graph.   If $\op{hyper}(V,E)$ is a combinatorial polygon (a
connected hypermap such that every node has cardinality two), then
$S^2\setminus X(V,E)$ has exactly two connected components.
\end{lemma*}

\begin{proof} By the preceding lemma, $\# c = \#f$, and the
hypermap is planar.  Since every node and every edge has order two,
we have $\#D = 2\#n = 2\# e = \#n +\#e$.  Since it is connected,
$\#\tangle{e,n,f} = 1$.  By the preceding lemma, the hypermap is
planar. Hence, the Euler relation gives:
    $$
    \#c = \#f = (\#D - \#n - \#e) + 2\#\tangle{e,n,f} =2.
    $$
\end{proof}

\begin{remark*}   Rather than deducing the Euler relation from the Jordan
curve theorem, we have worked in the opposite direction and proved
the Jordan curve theorem from the Euler relation.  For us, the
genesis of the Euler relation is fact that the solid angles of a
triangulation of a sphere must sum to the area of a sphere.
\end{remark*}



%% XX \section{All about Polygons}

\section{Hypermaps}

[There is repetition here that requires editing.]

The combinatorial structure of the Kepler conjecture is expressed
through various planar graphs.  Following Gonthier's work on the
four-color theorem, we reformulate the combinatorial structures as
hypermaps.

\subsection{Basic Definitions}

\begin{definition}  A {\it hypermap} $H=(D,e,n,f)$ is a finite set $D$
together with three permutations $e,n,f:D\to D$ satisfying the
identity $e\circ n\circ f = I$.  The elements of $D$ are called
{\it darts}.  The permutations $e,n,f$ are called the {\it edge},
{\it node}, and {\it face} permutations, respectively.
\end{definition}

\begin{definition}  A {\it face} of a hypermap $H=(D,e,n,f)$ is an orbit of $D$
under $f$.  We write $\op{face}(\alpha)$ for the face of
$\alpha\in D$.  We write $H/f$ for the set of faces.
\end{definition}

\begin{definition}  An {\it edge} of a hypermap $H=(D,e,n,f)$ is an orbit of $D$
under $e$.  We write $\op{edge}(\alpha)$ for the edge of
$\alpha\in D$.  We write $H/e$ for the set of edges.
\end{definition}

\begin{definition}  A {\it node} of a hypermap $H=(D,e,n,f)$ is an orbit of $D$
under $n$.  We write $\op{node}(\alpha)$ for the node of
$\alpha\in D$.  We write $H/n$ for the set of nodes.
\end{definition}

\begin{definition} A hypermap is {\it plain} (note the spelling!)
if $e\circ e=I$.  It is {\it simple} if for every two darts
$\alpha,\beta\in D$, we have
    $$\card{(\op{face}(\alpha) \cap \op{node}(\beta))} \le 1.$$
\end{definition}

\begin{definition} A hypermap is {\it planar} (note the spelling!)
if Euler's relation holds:
    $$
    \card{H/e} + \card{H/n} + \card{H/f} = \card(D) + \card{H/\langle
    n,e,f\rangle},
    $$
where $H/\langle n,e,f\rangle$ is the number of orbits of $H$
under the combined action of $n,e,f$.
\end{definition}

\subsection{Patching}

Let $A$ and $B$ be simple hypermaps with face and node
permutations
    $f_A,f_B$, and $n_A,n_B$, respectively; and darts $D_A$ and
    $D_B$.  Assume that the darts of $A$ and the darts of $B$ are
    disjoint sets.

Let $\phi:F_A\to F_B$ be a bijection between a face of $A$ and a
face of $B$ such that
    $$
    \phi(f_A \alpha) = f_B^{-1}\phi(\alpha),\quad \forall
    \alpha\in F_A.
    $$

\begin{definition} Let $\op{patch}(A,B,\phi)$ be the following hypermap
$(D,f,n,e)$:
    $$D = D_A \cup D_B \setminus (F_A\cup F_B)$$
    $$f\alpha = f_A\alpha \text{if } \alpha\in D_A, \ f_B\alpha
    \text{otherwise}.
    $$
    $$n\alpha = \begin{cases}
    n_A\alpha &
        \text{if } \alpha\in D_A \wedge n_A\alpha\not\in F_A\\
    n_B[\phi(n_A\alpha)] &
        \text{if } \alpha\in D_A \wedge n_A\alpha\in F_A\\
    n_B\alpha &
        \text{if } \alpha\in D_B \wedge n_B\alpha\not\in F_B\\
    n_A[\phi^{-1}(n_B \alpha)] &
        \text{if } \alpha\in D_B \wedge n_B\alpha\in F_B\\
    \end{cases}
    $$
Let $e$ be defined by the relation $e\circ n\circ f = 1$.
\end{definition}

To justify this construction, the following lemma shows that the
maps $\phi,\phi^{-1}$ are applied only to darts in their domains.

\begin{lemma} With context as above (simple hypermaps, etc.),
if $\alpha\in F_B$ then $n_B\alpha\not\in F_B$.  If $\alpha\in
F_A$, then $n_A\alpha\not\in F_A$.
\end{lemma}

\begin{proof} Elementary.
\end{proof}

\begin{lemma} If $n_A$ and $n_B$ have no fixed points, then
    $\op{patch}(A,B,\phi)$ does not either.
\end{lemma}

\begin{proof}
\end{proof}

\begin{lemma} If $A$ and $B$ are simple, then
$\op{patch}(A,B,\phi)$ is simple.
\end{lemma}

\begin{lemma} If $A$ and $B$ are plain, then
$\op{patch}(A,B,\phi)$ is plain.
\end{lemma}

\begin{lemma} If $A$ and $B$ are planar and plain, then
    $\op{patch}(A,B,\phi)$ is planar.
\end{lemma}
