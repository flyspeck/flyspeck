%% SPV Intro

This paper is the fifth in a series of papers devoted to the proof
of the Kepler conjecture, which asserts that no packing of congruent
balls in three dimensions has density greater than the face-centered
cubic packing.

In this paper, we prove that decomposition stars associated with the
plane graph of arrangements we term pentahedral prisms do not
contravene.  Recall that a contravening decomposition star is a
potential counterexample to the Kepler conjecture. We use interval
arithmetic methods to prove particular linear relations on
components of any such contravening decomposition star.  These
relations are then combined to prove that no such contravening stars
exist.

Pentahedral prisms come remarkably close to achieving the optimal
score of $8 \myscorept$, that achieved by the decomposition stars of
the face-centered cubic lattice packing. In this sense, we consider
pentahedral prisms to be ``worst case" decomposition stars.

Pentahedral prisms constituted a counterexample to an early version
of Hales's approach to a proof of the Kepler conjecture, and have
always been a somewhat thorny obstacle to the proof of the
conjecture. Relations required to treat pentahedral prisms are
delicate in contrast to the more general bounds which suffice  to
treat other decomposition stars.

This paper is a revised version of the author's PhD thesis at the
University of Michigan.  The author wishes to thank Tom Hales, Jeff
Lagarias and the referees for their many contributions to this
revision.
