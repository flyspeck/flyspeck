
\chapter{Packings with Full Contact}

\begin{note}%XX 
The results sketched in this chapter are still preliminary.  A number of the estimates that have been stated have not yet been rigorously proved by computer. 
\end{note}

We call a nonempty packing $\Lambda$ in $\ring{R}^3$ in which every vertex has distance $2$ from  $12$ other vertices a {\it packing with full contact}. L Fejes T\'oth has made the following conjecture.

\begin{conjecture}  Let $\Lambda$ be a packing with full contact.  Then for every vertex $\lambda\in\Lambda$, the arrangement of $12$ around that vertex is the kissing configuration of the face-centered cubic or hexagonal-close packing. 
\end{conjecture}



\begin{lemma}  Conjecture~\label{conj:m2} implies that for every vertex $\lambda\in\Lambda$ in a packing with full contact, $\norm{\lambda'}{\lambda}\ge 2.52$, whenever $\norm{\lambda'}{\lambda}> 2$.
\end{lemma}

\begin{proof} Let $\lam_1,\ldots,\lam_{12}$ be the twelve kissing vertices.  The conjecture gives
$$
12 + M_2(h(\lam,\lam')) = \sum_{i=1}^{12} M_2(h(\lam,\lam_i)) + M_2(h(\lam,\lam')) \le 12.
$$
This imples that $M_2(h(\lam,\lam'))\le 0$, so $\norm{\lam}{\lam'}\ge 2.52$.
\end{proof}

Thus, the contact fan is the same as the standard fan (using a cutoff for edges that is strictly less than $2.52$) for any vertex in a packing with full contact.  We consider the hypermap attached to the fan.


\section{hypermaps with tame contact}

We modify the notion of tameness to cover hypermaps that arise as the standard fan of centered packing with full contact.  

\begin{definition}[b]
  Define $b:\ring{N}\times \ring{N}\to \ring{R}$ by $b\pqr{(p,q,0)}=1.541$,
  except for the values in the following table
  (with  $\op{tgt}=1.541$):
  {
  \def\tx{\op{tgt}}
  $$\begin{matrix}  &q=0&1&2&3\\
           p=0&\tx&\tx&\tx&0.618\\
           1&\tx&\tx&0.412&0.618\\
           2&\tx&\tx&0.412&\tx\\
    \end{matrix}
   $$
   }
\end{definition}

\begin{definition}[d]
    Define $d:\ring{N}\to \ring{R}$ by
  $$d(n) = \begin{cases}
    0 & n=3, \\
    0.206 & n=4, \\
    0.483 & n=5, \\
    0.760 & n=6, \\
    1.037 & n=7, \\
    1.314 & n=8,\\
    \op{tgt}=1.541 & \text{otherwise}.
  \end{cases}
  $$
(In particular, $d(n) = 0.206 + 0.277 (n-4)$, for $n=4,\ldots,8$.)
\end{definition}

\begin{definition}[weight~assignment]
%
A {\it weight assignment\/} of a hypermap $H$ is a function $\tau$ on
the set of faces of $H$, taking values in the set of non-negative
real numbers. A weight assignment is a {\it contact} weight assignment if the
following properties hold:
%
 \indy{Index}{weight assignment}
 \indy{Index}{contact (weight assignment)}
\begin{enumerate}
  \item If the face $F$ has cardinality $n$, then
        $\tau(F) \ge d(n)$
  \item If a node $v$ has type $(p,q,0)$, then
        $$\sum_{F:\,v\cap F\ne\emptyset} \tau(F) \ge b{\pqr{(p,q,0)}}.$$
        \label{admissible:b}
        \label{definition:admissible:excess}
\end{enumerate}
The sum $\sum_F \tau(F)$ is called the {\it total weight} of $\tau$.
\indy{Index}{total weight}
\end{definition}



We say that a hypermap has {\it tame contact\/} if it satisfies the following
conditions.
%
 \indy{Index}{tame}

\begin{enumerate}
    \label{definition:tame}
    %1
    \item {\bf (planar)} The hypermap is plain, planar.
    \item {\bf (biconnected)} The hypermap is connected and biconnected.  In particular, every face meets every node in at most one dart.
    \item {\bf (nondegenerate)} The edge map $e$ has no fixed points.
    \item {\bf (no loops)} The two darts of each edge lie in different nodes.
    \item {\bf (no double joins)} The set of edges meeting any two given nodes has cardinality at most $1$.
    \label{definition:tame:40}

    \item {\bf (triangles)} If $L$ is a contour loop with $3$ face steps, and if there exists a node in
    the exterior of $L$, then $L$ is a face of the hypermap.
    \label{definition:tame:3-circuit}

    \item {\bf (quadrilaterals)} If $L$ is a $4$-step contour loop, and there is at least one nodes
    in the exterior of $L$, then the interior of $L$ takes one of the forms
    illustrated in Figure
    \ref{fig:fourcircuit-FT}.
    \label{definition:tame:4-circuit-FT}
    \begin{figure}[htb]
        \centering
        \myincludegraphics{\ps/tame4circuitFT.eps}
        \caption{Tame $4$-circuits}
        \label{fig:fourcircuit}
    \end{figure}
  \item {\bf (face)} There are at least two faces.
    \item {\bf (face)} The cardinality of each face is at least $3$ and at most $8$.
    \label{definition:tame:length}
    \item {\bf (node)} There are $12$ nodes.
    \item {\bf (node)} The cardinality of every node is at least $2$ and at most    $4$.
    \label{definition:tame:degree}
    \item {\bf (node)} {\tt NO CONDITION}
    \label{definition:tame:degreeE}
    \item {\bf (weights)} There exists a contact weight assignment
        of total weight less than the target, $\op{tgt}=1.541$.
    \label{definition:tame:squander}
\end{enumerate}
%

The set of all hypermaps with tame contact have been classified (up to isomorphism, possibly reversing orientation).  There are $25$ such hypermaps.  They have been classified by the same process described in Section~\ref{sec:proof-classification}.


\section{aggregate fans}

We may create aggregated fans in the same way as in the 1998 proof of the Kepler conjecture so that each face is simple.  We review the construction. 

Extend the length of the edges in the fan to anything less than  $\sqrt8$ to create a new fan.  The blades of the new fan do not meet and give a new planar hypermap.  We call the newly added edges {\it cut edges}.  By moving different connected components closer together (without creating any new edges to the stadard fan), we may assume that the new hypermap is connected.  Similarly, we may assume it is biconnected.  We call this the aggregate fan, aggregate hypermap, and so on.  Each face is now simple with a certain number $r$ of standard edges (length exactly $2$) and a number $s$ of cut edges (length at least $2.52$ and less than $\sqrt8$), for a total of $r+s$ edges.  
We have $0\le s$ and $3-s \le r$.

The following is the analogue for packings with full contact of the main estimate:

\begin{theorem}  If a face $F$ of the aggregate hypermap has $r$ standard edges and $s$ cut edges, then 
$$\tau(F) \ge \min(d(r,s),\op{trgt})$$
where 
$$
d(r,s) = 0.103 (2-s) + 0.277 (r+2s-4).
$$
\end{theorem}

\begin{proof} We may imitate the proof of the main estimate from the 1998 proof.  We recall the method.

It is enough to consider a single simple face of the aggregate hypermap, so without generality, we may assume that the vertices $V$ are all nodes that meet the face.  Let $U\subset Y(\orgn,V,E)$ be the connected component corresponding to the face $F$.

Additional internal blades $C^0(\orgn,e)\subset U$ may be added to the fan of length at most $3.2$ as long as the blades do not cross.  By the additivity of the constants $d(r,s)$ of (\ref{eqn:drs-add}), we may consider a counterexample that minimizes $r+s$.  Such a counterexample will not have any blades of length at most $3.2$ and none will be created as the example is deformed to decrease $\tau(F)$.  When $r$ and $s$ are fixed, deformations that decrease the solid angle $\sol(F)$, decrease $\tau(F)$.

We call a dart $x\in F$ concave or convex, according to whether $\op{azim}(x)\ge\pi$ or $\le\pi$.  Edges may be stretched (to decrease solid angle) at a concave dart until both edges at that vertex have length $3.2$. Assume this.

The half-disk of arc-radius $\arc(2,2,3.2)/2$ and area
$$
s_0 = \pi (1- \sqrt{1-0.4^2}) = 2.87\ldots.
$$
at each concave vertex lies entirely in $U$.  So 
$$
\sol(F) \ge s_0 k,
$$
where $k$ is the number of 
\end{proof}


Take the set $D'$ of all darts leading into a given connected component of $Y(\orgn,V,E)$.  Pick $x\in D'$ so that the smallest $n>0$ such that $f^n x$ and $x$ lie in the same node is as large as possible.  There is a contour loop $L = [x;f x;f^n x;n^{-1} f^n x;\cdots,x]$.  (If the face is already simple, then $L$ is simply the contour loop around the face.)  Do this for each connected component.  The resulting 

\section{contravention gives tame contact}

\begin{lemma} The hypermap does not contain any nodes of degree five or greater.
\end{lemma}

\begin{proof} Let $\alpha_0 = \op{azim}(2,2,2,2,2,2)$.  We have $5\alpha_0 < 2\pi < 6\alpha_0$.  Thus, at least one of the darts at a the node is not in a triangular face.  The angle at such a dart is at least $\alpha_1=\op{azim}(2,2,2,2.52,2,2)$ and $4\alpha_0+\alpha_1 > 2\pi$.
\end{proof}

\begin{lemma} The hypermap does not contain a node of degre four at which no face is nontriangular.
\end{lemma}

\begin{proof} $4\alpha_1 > 2\pi$.
\end{proof}

The function $\tau(F)$, when restricted to kissing configurations, takes the following form:
$$
\tau(F) = \sol(F) + (2-n(F)) \Delta_0.
$$

\begin{lemma} $$\sum_{F} \tau(F) = 4\pi - 20\Delta_0.$$
\end{lemma}

\begin{proof} This is equation~(\ref{eqn:delta0}).
\end{proof}



\section{linear programs and conclusion}

\begin{lemma}\label{lemma:kiss-fcc} Let $H$ be the hypermap of the face-centered cubic or hexagonal-close packing.   Assume that it occurs as the aggregated hypermap of a centered packing with full contact.  Then the kissing configuration of the centered packing is congruent to that of the face-centered cubic or hexagonal close packing.
\end{lemma}

\begin{proof} Every face of the hypermap is a triangle or quadrilateral.  No aggregate face is a triangle or quadrilateral.  Thus, the aggregated hypermap is the same as the contact (and standard) hypermap.  The contact hypermap of the face-centered cubic and hexagonal-close packings fixes the eight regular triangles in the kissing arrangement.  The eight regular triangles fix the kissing arrangement up to congruence.
\end{proof}

\begin{theorem}\label{lemma:fcc} Let $H$ be a hypermap with tame contact.  Assume that it occurs as the aggregate fan of a centered packing with full contact.  Then $H$ is the contact hypermap of the face-centered cubic or hexagonal-close packing.
\end{theorem}

\begin{theorem}[packings with full contact]  
Fejes T\'oth's conjecture on packings with full contact holds.
\end{theorem}

\begin{proof} The aggregated hypermap of a centered packing with full contact has tame contact.  By Theorem~\label{lemma:fcc}, the aggregated hypermap is that of the fcc or hcp.  By Lemma~\ref{lemma:kiss-fcc}, the kissing configuration of the centered packing is congruent to the fcc or hcp.  As the center of the packing may be chosen at an arbitrary vertex, every vertex is congruent to one of these two arrangements.  The result follows.
\end{proof}