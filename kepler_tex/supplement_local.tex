% File added May 15, 2012

%\chapter{Supplementary Notes}\label{sec:supplement}

\newpage
\section{Appendix on the main estimate}\label{sec:sup-local-fan}

This appendix was written in May 2012 in response to Hoang Le Truong's
observation that the dependent type $\BB_s\subset \ring{R}^k$, with
variable $k$, is unpleasant to deal with in HOL Light.  Lemmas have
been reworded to involve a single $\BB_s$ rather than a collection of
them, whenever possible.

It also grew out of an Objective CAML program that checks the
completeness of the arguments in Section~\ref{sec:weight}, at an
informal level.  This involved translating metric properties of local
fans into combinatorial properties of stable constraint systems,
and then checking the combinatorial relations by computer.  As a
result, the verification of the main estimate can now largely be
viewed as a combinatorial proof that $S_{init}\Ra S_{term}$.  This
appendix documents the program \verb!check_completeness.hl!.

One other improvement in exposition has been to replace the (inexplicit)
function $\ell$, which was used to guarantee finite termination of algorithms,
with a relation $(\Ra)$, whose essential features are captured by
finitely
many iterations of a few basic operations.

\subsection{statement of results}\label{sec:statement'}

This appendix can be read immediately before subsection~\ref{sec:statement}
and all material after that point.
For completeness, we repeat a few definitions and results.



\begin{definition}[$\hm$,~$\tau$,~$\dih_i$]\guid{CUFCNHB}\label{def:tau}
\formaldef{$\rho_0$}{rho\_fun}
\formaldef{$\tau$}{tau\_fun}
\formaldef{$\sol_0$}{sol0}
\formaldef{$\azim$}{azim\_in\_fan}
\formaldef{$\tau_{tri}$}{taum}
Let $(V,E,F)$ be a nonreflexive local fan.  Recall that $\hm = 1.26$ and
$L(h) = ({h_0-h})/({h_0-1})$, when $h \le h_0$.
  Set
\begin{align*}
\rho_0(y) &= 1 + \dfrac{\sol_0}{\pi} \cdot
    \dfrac{y-2}{2\hm-2} = 1 + \dfrac{\sol_0}{\pi}(1 - L(y/2)),\\
  \tau(V,E,F) &=\sum_{x\in F}^{\phantom{!}} \rho_0(\normo{\nd(x)})\op{azim}(x)
+ \left(\pi+{\sol_0}\right) (2- k(F)),
\end{align*}
where $\sol_0=3\arccos(1/3)-\pi\approx0.551$ is the solid angle of a
spherical equilateral triangle of side $\pi/3$, and $k(F)$ is the
cardinality of $F$.  
Let 
\begin{equation}\label{eqn:tautri}
  \tau_{tri}(y_1,y_2,y_3,y_4,y_5,y_6) =
  \sum_{i=1}^3 \rho_0(y_i) \dih_i(y_1,\ldots,y_6)
- \left(\pi+{\sol_0}\right),
\end{equation}
where
\begin{align}\label{eqn:dihi}
\dih_1(y_1,y_2,y_3,y_4,y_5,y_6) &= \dih(y_1,y_2,y_3,y_4,y_5,y_6),\notag\\
\dih_2(y_1,y_2,y_3,y_4,y_5,y_6) &= \dih(y_2,y_3,y_1,y_5,y_6,y_4),\textand\notag \\
\dih_3(y_1,y_2,y_3,y_4,y_5,y_6) &= \dih(y_3,y_1,y_2,y_6,y_4,y_5).
\end{align}
\indy{Notation}{h@$h_0 = 1.26$}%
\indy{Notation}{zzt@$\tau_{tri}$}%
\indy{Notation}{zzt@$\tau$ (weight assignment)}%
\indy{Notation}{zzrho1@$\rho_0$ (real-valued function)}%
\indy{Notation}{sol0@$\sol_0 = 3\arccos(1/3)-\pi$}%
\indy{Notation}{dih@$\dih_i$}%
\indy{Notation}{L@$L$ (linear function)}%
\end{definition}


\begin{definition}[standard,~protracted,~diagonal]\guid{KRACSCQ} 
\formaldef{standard}{standard}
\formaldef{protracted}{protracted}
\formaldef{diagonal}{diagonal0}
Let $(V,E)$ be a fan.  
We write $\normo{\ee}$ for $\norm{\v}{\w}$, when $\ee=\{\v,\w\}\subset V$.
We say that  $\ee$ is \newterm{standard} if
\[
2\le \normo{\ee}\le2\hm.
\]
We say that  $\ee$ is \newterm{protracted} if
\[
2\hm\le \normo{\ee}\le\sqrt{8}.
\]
If $\v,\w\in V$ are distinct, and $\ee=\{\v,\w\}$ is not an edge in $E$, then
we call $\ee$ a \newterm{diagonal} of the fan.
\end{definition}
%\indy{Notation}{4@$\norm{\wild}$ (norm of fan edge)}% % doesn't parse


\begin{theorem}[main~estimate]\guid{JEJTVGB}\label{lemma:empty-d}
\formaldef{main estimate}{main\_estimate}
\formaldef{\case{annulus}}{ball\_annulus}
\formaldef{\case{diagonal}}{diagonal1}
%\label{theorem:main}
Let $(V,E,F)$ be a nonreflexive local fan (Definition~\ref{def:convex-local}).
We make the following additional
assumptions on $(V,E,F)$.
\begin{enumerate}
\item \case{packing} $V$ is a packing.  That is, for every $\v,\w\in
V$, if $\norm{\v}{\w}<2$, then $\v=\w$.
\item \case{annulus} $V\subset \BB$.
\item \case{diagonal} For all distinct elements $\v,\w\in V$, if
$\{\v,\w\}\not\in E$, then 
\[ 
\norm{\v}{\w}\ge 2\hm.
\] 
\item \case{card} 
Let   $k=\card(E)=\card(F)$.  Then $3\le k \le 6$.
\end{enumerate}
In this context, we have the following conclusions.
\begin{enumerate}
\item Assume $k\ge 4$.  If  every edge of $E$ is standard, then
\[ 
\tau(V,E,F) \ge d (k), \text{ where } d(k) =
\begin{cases}
  0.206,&\text{if }k=4,\\
  0.4819,&\text{if }k=5,\\
  0.712,&\text{if }k=6.
\end{cases}
\] 
\item Assume $k=5$.  Assume that every edge of $E$ is standard.
Assume that every diagonal $\ee$ of the fan satisfies $\normo{\ee}\ge\sqrt{8}$.
Then 
\[
\tau(V,E,F)\ge 0.616.
\]
\item Assume $k=5$.  Assume there exists some protracted edge in $E$ 
and that the other four are standard.  Then 
\[
\tau(V,E,F)\ge 0.616.
\]
\item Finally, assume that $k=4$.  Assume that there exists some protracted
 edge in $E$ and that the other three are standard.  
Assume that both diagonals $\ee$ of the fan satisfy $\normo{\ee}\ge\sqrt{8}$.
Then
\[
\tau(V,E,F)\ge 0.477.
\]
%% XX condition on diags >= sqrt8 added 2012-5-20.
%% based on the semantics of glpk/tame_archive/head.mod.
\end{enumerate}
\end{theorem}

There are two related inequalities that we will prove separately. For that reason,
we state them as a separate lemma.

\begin{lemma}\guid{HGDRXAN}\label{lemma:tau3}
Let $(V,E,F)$ be a nonreflexive local fan.
Under the same hypotheses on $(V,E,F)$ as in Theorem~\ref{lemma:empty-d}, 
\begin{enumerate}
\item Assume $k=3$. Then
\[\tau(V,E,F)\ge 0.\]
\item Assume $k=4$.  Assume that every edge of $E$ is standard.
Assume that both diagonals $\ee$ of the fan satisfy $\normo{\ee}\ge3$.
Then
\[
\tau(V,E,F)\ge 0.467.
\]
\end{enumerate}
\end{lemma}


\subsection{Definitions}


\begin{definition}[torsor,~adjacent]
%\formaldef{$\cstab$}{cstab}
\formaldef{torsor}{torsor}
  Let $k>1$ be an integer.  A \newterm{torsor} is a set $I$ with a
  given simply transitive action of $\ring{Z}/k\ring{Z}$ on $I$.  We
  write the application of $j\in\ring{Z}/k\ring{Z}$ to $i\in I$ as
  $j+i$ or $i+j$.  We also write $j+i$
  for the application of the image of $j\in\ring{Z}$ in $\ring{Z}/k\ring{Z}$ to
  $i\in I$.  Note that each choice of base point $i_0\in I$ gives a
  bijection $i\mapsto i+i_0$ between $\ring{Z}/k\ring{Z}$ and $I$.  
  We say that $i$ and $j$ are not \newterm{adjacent} if $i\ne j\pm 1$.
  %If $i,j\in I$, write 
  %\[
  %|i-j|_0 = \min\left \{ |m-n| \mid m + i = n +j,\ m,n\in\ring{Z}\right \}.
  %\]
  An
  \fullterm{isomorphism of torsors}{isomorphism!torsor} 
 is a bijection that respects the action.
\indy{Notation}{I@$I$ (torsor)}%
\end{definition}

\begin{example} If $H=(D,e,n,f)$ is a hypermap with face $F$,
then $F$ is a torsor under the action $x \mapsto f x$.  If $H$ is
isomorphic to $\op{Dih}_{2k}$ and has vertex set $V$, then $V$ is a torsor
under the action of $x\mapsto \rho x$, where $\rho$ is constructed
analogously to $\rho_{(V,E,F)}$.  We may mostly restrict our attention to
these cases.
\end{example}

% XX stab. May 17, 2012.  Dropped the constraint a_{ij}\le \stab.
% It doesn't hold for one of the quad terminal constraint systems:
% 1637868761 has diag at least 3.41.

\begin{definition}[constraint~system]\guid{ZGFHNKX}
\formaldef{constraint system}{constraint\_system}
A \newterm{constraint system} $s$ consists of the following data:
\begin{enumerate}
\item a natural number $k\in \{3,4,5,6\}$,
\item a $\ring{Z}/k\ring{Z}$-torsor $I$,
\item a real number $d$,
\item real constants $a_{ij}$, $b_{ij}$, $\alpha_{ij}$, $\beta_{ij}$ satisfying
   $a_{ij} = a_{ji}$, $b_{ij}=b_{ji}$, $\alpha_{ij}=\alpha_{ji}$,
$\beta_{ij}= \beta_{ji}$,  
\[
a_{ij}\le \alpha_{ij}\le \beta_{ij}\le b_{ij},
\]
 for $i,j\in I$,
\item a subset $J\subset \{ \{i,1+i\} \mid i\in I\}$, 
such that $\card(J)+k\le 6$.
\item subsets $I_{str}$, $I_{str} \subset I$.
\end{enumerate}
\indy{Notation}{J@$J$}%
\end{definition}

For each constraint system $s$, we write $k(s)$,
$d(s)$, $I(s)$, $a_{ij}(s)$, and so forth for the associated
parameters.  

(We merge the definitions of stable,  tri-stable, and augmented
(augmented) from earlier versions.)
% renamed augmented -> substandard, then mreged with stable.

\begin{definition}\guid{ZBJCZHI} \guid{RIUGHVX} 
\formaldef{stable constraint system}{stable\_system}
\formaldef{$(k,a,b,d,I,J,+1)$}{(k\_sy,a\_sy,b\_sy,d\_sy,I\_sy,J\_sy,f\_sy)}
\formaldef{tri-stable}{tri\_stable}
\formaldef{$(a,b,d,k,I,J,+1)$}{$(a\_ts,b\_ts,d\_ts,k\_ts,I\_ts,J\_ts,f\_ts)$}
\formaldef{substandard (stable)}{augmented\_constraint\_system1}
\formaldef{substandard (tri-stable)}{augmented\_constraint\_system2}
We say that a constraint system $s$ is \newterm{stable} if the following
additional properties hold.
\begin{enumerate}
\item 
\[
0 = a_{ii}\text{ and } 2\le a_{ij} \text{ for all }  i,j\in I \text{ such that } i\ne j.
\]  
\item
  Also, 
\[
  b_{i,i+1} ~\quad \begin{cases}
    < 4, & k = 3\\
    \le \stab, & k > 3.
    \end{cases}
\]  
\item
If $\{i,j\}\in J$, then $\leftclosed
  a_{ij},b_{ij}\rightclosed=\leftclosed\sqrt{8},\stab\rightclosed$.
\item $d < 0.9$,
\item  $m+k(s)\le 6$, where 
$m=m(s)$ is the number of edges $\{i,i+1\}\subset I(s)$ such that
$b_{i ,i+1}(s)> 2h_0$ or $a_{i ,i+1}(s)>2$. 
\end{enumerate}
\indy{Notation}{m@$m(s)$ (number of nonstandard edges)}
\end{definition}

\begin{definition}[ear]  
\formaldef{$a_{ij}$}{a\_ear0}
\formaldef{$b_{ij}$}{b\_ear0}
\formaldef{ear}{ear\_sy}
We have a stable constraint system $s$ given by
$k=\card(I)=3$, $d=0.11$, $J$ a singleton, 
and
\[
\leftclosed a_{ij},b_{ij}\rightclosed=
\begin{cases}
 \leftclosed0,0\rightclosed,
 &\text{~~if } i=j,\\
 \leftclosed\sqrt8,\stab\rightclosed,
 &\text{~~if } \{i,j\} \in J,\\
 \leftclosed 2,2\hm\rightclosed,
 &\text{~~otherwise. }
\end{cases}
\]
We call $s$ an \newterm{ear} (by analogy with an ear
in a triangulation of a polygon, which is a triangle that has two of
its edges in common with the polygon).
\end{definition}


Next we associate a set $\BB_s$ with each constraint system $s$.
\indy{Notation}{BBs@$\BB_s$}%


\begin{definition}[$\BB_s$]\guid{KTFVGXF}
  For every constraint system $s$, and every function
  $\v:I(s)\to \BB$, let $V_\v\subset \BB$ be the image of
  $\v$.  Let $E_\v$ be the image of $i\mapsto \{\v_i,\v_{i+1}\}$.  Let
   $F_\v$ be the image of $i\mapsto (\v_i,\v_{i+1})$.
 Let $\BB_s$ be
  the set of all functions $\v$ that have the following properties.
\begin{enumerate}
\item $a_{ij}(s)\le\norm{\v_i}{\v_j}\le b_{ij}(s)$, for all $i,j\in I(s)$.
\item if $k(s)>3$, then $(V_\v,E_\v,F_\v)$ is a nonreflexive local fan.
\end{enumerate}
\end{definition}






The set $J$ is used to make a small correction $d(s,\v)$ to the
constant $d(s)$.  

\begin{definition}[$d(s,\v)$]\guid{TPLCZFL}
\formaldef{$\sigma$}{sigma\_sy}
Set $\sigma(s) =1$ when $s$ is an ear;  $\sigma =
-1$, otherwise.  Let $V=\{\v_i\mid i\in I(s)\}$ 
be a set of points in $\ring{R}^3$.
Write
\begin{equation}
d(s,\v) = d(s) +  0.1\, \sigma(s)\,\sum_{\{i,j\}\in J(s)} (\stab - \norm{\v_i}{\v_j}).
\end{equation}
\end{definition}
\indy{Notation}{zzs@$\sigma=\pm1$}%
\indy{Notation}{d@$d(s,\v)$}%

This correction to $d(s)$  makes it a bit easier to prove inequalities when
$\sigma(s)=-1$, at the cost of slightly more difficult inequalities for ears.

When
$k(s)=3$,
the set $\{\v_i\mid i\in I\}$ may degenerate to planar
configurations. By the
stability
constraint $b_{ij}(s)<4$, we have well-defined dihedral angles, so
that $\tau^*(s,\wild)$ may be defined in general as follows.

\begin{definition}[$\tau^*$]\guid{BGCEUKP}\label{def:tau-star}
\indy{Notation}{zzt@$\tau^*$}%
Let $s$ be a stable constraint system.  Define 
\[
\tau^*:\{(s,\v)\mid \ \v\in \BB_s\} \to \ring{R}
\] 
by
\[ 
(s,\v) \mapsto \tau(V_\v,E_\v,F_\v)-d(s,\v).
\] 
\indy{Notation}{zzt@$\tau^*$}%
\end{definition}


\begin{definition}[unadorned]\guid{SDJTENL}
We say that a constraint system $s$ is \newterm{unadorned} if the following additional
properties hold (with established notation):
\begin{enumerate}
\item For all $i,j\in I(s)$,  $a_{ij}=\alpha_{ij}$ and $b_{ij}=\beta_{ij}$.
\item $I_{lo}=I_{str}=\emptyset$.
\end{enumerate}
\end{definition}


\begin{example}\label{ex:extend-cs} 
We may always transform a stable constraint system into another that is unadorned by
setting $\alpha=a$, $\beta=b$, $I_{lo}=\emptyset$, $I_{str}=\emptyset$, and
keeping the rest of the data the same.
\end{example}

\begin{definition}[$\smain$]\label{ex:main} 
The constants in the conclusions of the main estimate
  (Theorem~\ref{lemma:empty-d}) can be packaged into unadorned stable constraint
  systems.  For example, the standard main estimate for $k=6$ gives
  the constraint system $d=0.712$, $J=\emptyset$, $I$ an indexing set
  of cardinality six, and
\[
a_{ij} = \begin{cases} 0, & i=j,\\
  2, & j= i\pm1,\\
  2 \hm, & \text{otherwise},
  \end{cases}
\qquad
b_{ij}=\begin{cases}
 0, & i=j,\\
 2\hm, & j= i\pm1,\\
 4h_0^+, & otherwise,
  \end{cases}
\]
where $h_0^+$ is any constant greater than $\hm$.
The upper bound $4\hm$ on any diagonal comes from the triangle
inequality: $\norm{\v_i}{\v_j} \le \normo{\v_i}+\normo{\v_j} \le
4\hm$.   
% XX added $\hm^+$ May 26, 2012 to avoid binding constraints.
We write $\smain$ for the set of stable constraint systems $s$, with a
fixed choice of torsor for each $k$, for all cases of the main
estimate.
\end{definition}
\indy{Notation}{Smain@$\smain$ (main estimate constraint systems)}%

The set $J(s)$ is empty for $s\in \smain$, so this correction does not
directly affect the main estimates:
\[
d(s,\v) = d(s), \text{ for all } s \in \smain.
\]

\begin{definition}[$S_{init}$] \guid{BTKIQGE}
For each of the cases of Theorem~\ref{lemma:empty-d} and \ref{lemma:tau3}, we fix an
stable constraint system that encodes its parameters, as described in Example~\ref{ex:main}. 
Let $S_{init}$ be this set of  stable constraint systems.
\end{definition}




\subsection{minimization}


\begin{lemma}\guid{PCRTTID} \label{lemma:aug-compact}
Let $s$ be a stable constraint system. Then
$\BB_s$ is compact (as a subset of
$\BB^k \subset \ring{R}^{3k}$).
\end{lemma}

\begin{proof}  This holds when $s$ is a stable constraint
system.
\end{proof}

\begin{lemma}[continuity]\guid{HDPLYGYv2}\label{lemma:compact-fan}
Let $s$ be a stable constraint system.  Then the function 
\[
\v\mapsto \tau^*(s,\v)
\]
is a continuous function on $\BB_s$.  Moreover, if $\BB_s$ is
nonempty, then the function attains a minimum.
\end{lemma}

\begin{proof} This follows easily from Lemma~\ref{lemma:compact-fan} and
Lemma~\ref{lemma:aug-compact}.
\end{proof}

\begin{lemma}\guid{JKQEWGV}\label{lemma:not-circular}
Let $s$ be a stable constraint system.  Let $\v\in \BB_s$.
Suppose that
\[
\sol(V_\v,E_\v,F_\v) \ge \pi.
\]
Then $\tau^*(s,\v)>0$.
In particular, if 
$\tau^*(s,\v)\le0$, then the azimuth angle of some dart of
$(V_\v,E_\v,F_\v)$ is 
less than $\pi$.  In particular, in this case, the local fan is not circular.
\end{lemma}

\begin{proof} By the definition of stable constraint system,
we have $d(s)\le 0.9$. The proof of Lemma~\ref{lemma:09} extends readily
to this context.
\end{proof}

A \newterm{diagonal} of a constraint system $s$ is a pair $\{p,q\}\subset I(s)$
such that $\{p,q\}$ is not a singleton and does not have the form $\{i,i+1\}$.

\begin{definition}[index,~$\iota$,~$\MM_s$]\guid{FNUEPJW}
Let $s$ be a stable constraint system.
Let $\MM_{s}\subset \BB_{s}$ be defined as follows.
Let $\BB'_{s}$ be the set of all $\v\in \BB_{s}$ such that
\begin{enumerate}
\item $\tau^*(s,\v)$ is equal to the minimum of $\tau^*(s,\wild)$ over $\BB_{s}$.
\item $\tau^*(s,\v)\le 0$.
\end{enumerate}
Define the \newterm{index} of $\v\in \BB_s$ to be the number of edges of $\v$
that attain its minimum bound $a_{i j}(s)$.  Let $\iota(s)$ be the minimum
of the index of $\v$ as $\v$ runs over $\BB'_s$.  
We let $\BB''_{s}$  be the set of $\v\in \BB'_{s}$ that attain
the smallest possible index $\iota(s)$,
and let $\MM_s$ be the subset of all $\w\in\BB''_s$ satisfying the additional
constraints.
\begin{enumerate}
\item If $i\in I_{str}$,  then $\v_i$ is straight.
\item If $i\in I_{lo}$,  then $\normo{\v_i}=2$.
%\item If $i\in I_{hi}$,  then $\normo{\v_i}=2h_0$.
\item $\alpha_{ij}(s)\le \norm{\w_i}{\w_j}\le \beta_{ij}(s)$ for all $i,j$.
\end{enumerate}
\end{definition}

%(Obviously, if $I_{hi}\cap I_{lo}\ne \emptyset$, then $\MM_s=\emptyset$.)


We have the following analogue of Lemma~\ref{lemma:esm}.

\begin{lemma}\guid{AYQJTMD}\label{lemma:init}
Let $s\in S_{init}$ be such that
\eqref{eqn:main:sv} fails to hold.
Then $\MM_s$ is nonempty.
\end{lemma}

\begin{proof}  Assume that \eqref{eqn:main:sv} fails to hold for
$s$.  By compactness and continuity, the set 
$\BB'_s$ of minimizers is nonempty.   
The subset $\BB''_s$ on which the index is as small as possible
is then also nonempty.  By our conventions in extending a constraint
system to a stable constraint system, $\BB''_s = \MM_s$.
\end{proof}


\begin{lemma}\guid{ODXLSTCv2}\label{lemma:odx2} 
Let $s$ be a stable constraint system,
and assume that $\w\in \MM_s$.  Let $i\in I(s)$.
Assume  that $\w_i$ is not the pole of a lunar local fan $(V_\w,E_\w,F_\w)$.
Assume that $4h_0 < b_{iq}(s)$ for every diagonal $\{i,q\}$ at $i$.
Then one of the following   constraints hold.
\begin{enumerate}
\item $\norm{\w_i}{\w_j}$ attains its lower bound $a_{i j}(s)$, for
  some $j\ne i$.
\item $\normo{\w_i}$ attains its lower bound $2$.
\item There exists $j$ adjacent to $i$ such that $\{i,j\}\in J(s)$.
\end{enumerate}
\end{lemma}

\begin{proof} 
We assume that the constraint on $J$ does not hold and
that $a_{p q}(s)<\norm{\v_p}{\v_q}$ for all diagonals, then continue as in the proof of
Lemma~\ref{lemma:odx}.  
By Lemma~\ref{lemma:not-circular}, we may assume  
that one of the following hold of the local fan $(V_\w,E_\w,F_\w)$.
\begin{enumerate}
\item The local fan is generic.
\item The local fan is lunar, the pole has azimuth
angle less than $\pi$, and $\w_i$ is not a pole.  
\end{enumerate}
The constraints on generic and local fans
allow us to use Lemmas~\ref{lemma:fan-open-lunar} and
\ref{lemma:fan-open-generic}, showing that fan conditions are preserved.
\end{proof}

\begin{lemma}\guid{IMJXPHRv2}\label{lemma:imj2}
Let $s$ be a stable constraint system.  Assume $i\in I_{str}(s)$ and
that $\w\in \MM_s$.  
Assume  that $\w_i$ is not the pole of a lunar local fan $(V_\w,E_\w,F_\w)$.
Assume that $4h_0 < b_{iq}(s)$ for every diagonal $\{i,q\}$ at $i$.
Then one of the following conditions holds.
\begin{enumerate}
\item $\norm{\w_i}{\w_{i+1}}$ attains its lower bound $a_{i,i+1}(s)$, and
 $\norm{\w_i}{\w_{i-1}}$ attains its lower bound $a_{i,i-1}(s)$.
\item $\normo{\w_i}$ attains its lower bound $2$.
\item There exists $j$ adjacent to $i$ such that $\{i,j\}\in J(s)$.
\item Some diagonal $\{i,q\}\subset I(s)$ at $i$ satisfies
$\norm{\w_i}{\w_q}=a_{i q}(s)$.
\end{enumerate}
\end{lemma}

\begin{proof} We assume that the constraint on $J$ does not hold and
that $a_{p q}(s)<\norm{\v_p}{\v_q}$ for all diagonals, then continue as in the proof of
Lemma~\ref{lemma:imj}.
By Lemma~\ref{lemma:not-circular}, we may assume that one of the
following
conditions hold.
\begin{enumerate}
\item The local fan is generic, or
\item The local fan is lunar, the pole has azimuth
angle less than $\pi$, and $\w_i$ is not a pole.  
\end{enumerate}
The constraints on generic and local fans
allow us to use Lemmas~\ref{lemma:fan-open-lunar} and
\ref{lemma:fan-open-generic}, showing that fan conditions are preserved.
\end{proof}

\begin{lemma}\guid{NUXCOEAv2}\label{lemma:nux2}
Let $s$ be a stable constraint system.  Assume $i\in I_{str}(s)$
and that $\w\in \MM_s$.   
Assume that $\norm{\w_i}{\w_j}=a_{ij}(s)$ for
some $j$ is adjacent to $i$.  
Assume  that $\w_i$ is not the pole of a lunar local fan $(V_\w,E_\w,F_\w)$.
Assume that $4h_0 < b_{iq}(s)$ for every diagonal $\{i,q\}$ at $i$.
Then
one of the following conditions holds.
\begin{enumerate}
\item We have $\norm{\w_i}{\w_\ell}=a_{ik}(s)$
(for both choices of $\ell\in I(s)$ adjacent to $i$).
\item There exists $j$ adjacent to $i$ such that $\{i,j\}\in J(s)$.
\item Some diagonal $\{i,q\}\subset I(s)$ at $i$ satisfies
$\norm{\w_i}{\w_q}=a_{i q}(s)$.
\end{enumerate}
\end{lemma}

\begin{proof} 
Let $s$ and $\w$ be as given.  The lemma is a special
case of the previous lemma, unless $\normo{\w_i}=2$, which we assume.
Let $\ell\ne j$ be the other index adjacent to $i$.
Assume that the final
two conditions do not hold.
Assume for a contradiction that 
\[
\norm{\w_i}{\w_\ell}>a_{i \ell}(s)
\]
We consider a curve $\v$ of the form \eqref{eqn:move1} that moves $\v_i$
in a circular arc with center $\orz$ through the point $\w_i$ and in
the fixed plane determined by $\{\orz,\w_i,\w_j,\w_{\ell}\}$.  
The function $\tau^*(s,\wild)$ is
constant along this curve.  We orient the curve to be increasing
in $\norm{\w_i}{\w_{j}}$.  For sufficiently, small $t$, we find that
$\v(t)\in\BB'_s$ has smaller index than $\w$.  This is contrary to the minimizing
properties of $\w\in\MM_s$.

As in the proofs of the previous lemmas, the constraints on generic and local fans
allow us to use Lemmas~\ref{lemma:fan-open-lunar} and
\ref{lemma:fan-open-generic}, showing that fan conditions are preserved.
\end{proof}

In the preceding three lemmas, we specifically allow the deformations
$\v(t)$ to occur within a lunar fan, moving a single node that is not a pole
of the lunar fan, as given by Lemma~\ref{lemma:fan-open-lunar}.
We are now ready to eliminate  lunar and circular
fans.

\begin{lemma}\guid{BJOQBJU}\label{lemma:bjo} 
Let $s$ be a stable constraint system.
Assume that $k(s)>3$.
If $k(s)=6$, then assume additionally that
$\stab\le a_{pq}(s)$, for every diagonal $\{p,q\}\subset I(s)$.
Then for every  $\w\in\MM_s$,  the local fan $(V_\w,E_\w,F_\w)$ is generic.
\end{lemma}

\begin{proof}
By the definition of stable constraint system, $4\le k(s)$ implies
that  for every edge $\{i,j\}\subset I$, we have
$b_{i j}(s) \le\stab$.

Let $\w\in \MM_s$.  By Lemma~\ref{lemma:09}, the local fan of $\w$  is not circular.
To show that the local fan is generic, it is enough to show that it is not lunar.
We assume for a contradiction that $\{\v_i,\v_j\}$ is the pole of a lunar fan.

We first treat the case  $4\le k(s)\le 5$,
and afterward we will return to the case $k(s)=6$. 
Assuming that  $k(s)\le 5$, the indices $i,j$ differ by at most two.
Also, the assumptions give $m\le 6-k(s)\le 2$, so there is a path from
$i$ to $j$ involving at most one edge with $b_{i j}(s)>2h_0$.
Then by the spherical triangle inequality (Lemma~\ref{lemma:sph-tri-ineq}),
\begin{equation}\label{eqn:pole}
\op{arc}_V(\orz,\v_i,\v_j) \le \op{arc}(2,2,2h_0) + \op{arc}(2,2,\stab) < \pi,
\end{equation}
showing that $\{\v_i,\v_j\}$ is not parallel and not a pole.

Now consider the case $k(s)=6$.  Here, $m=0$, so that $b_{\ell,\ell+1}(s)\le 2h_0$
and $a_{\ell,\ell+1}(s)=2$
for every edge. Equation~\ref{eqn:pole} shows that the indices $i$ and $j$
must be opposite in the hexagon: $j = i+3$.  By the structure of lunar fans
(Lemma~\ref{lemma:lunar}), $\w_\ell$ is straight for $\ell\ne i,j$.  Without
loss of generality, we may
pass to a stricter stable constraint system $s'$ such that
$\ell\in I_{str}(s')$, for $\ell\ne i,j$.  We then have $\w\in \MM_{s'}$.

Recall that $i$ is fixed at a pole. 
\claim{We claim that for any edge $\{i,\ell\}$, 
 if  $\norm{\w_{\ell}}{\w_i}>a_{ik}(s)=2$, then
$\normo{\w_\ell}=2$.}  Indeed, this follows directly from the preceding deformation
lemmas.  Thus, the tetrahedron with extreme points
$\{\orz,\w_{i-1},\w_i,\w_{i+1}\}$ has two edges
of length $2$ and a diagonal $\{\w_{i-1},\w_{i+1}\}$ of length at least
$\stab$, by hypothesis.  By a \cc{1117202051~4559601669}{}, 
the azimuth angle at the pole $i$ is at least $\pi/2$.
Hence the solid angle of the lune is at least $\pi$.  By Lemma~\ref{lemma:09},
we have $\tau^*(s',\w)>0$, which contradicts $\w\in \MM_{s'}$.
\end{proof}


\subsection{operations}

This section describes some operations on stable constraint
systems.

\begin{definition}[restriction]\guid{PFEOBSC}
Let $s$ be a stable constraint system.
We say that $s'$ is a restriction of $s$ if there are sets $A$, $B$ of
edges such that
\[
a_{ij}(s') = 
\begin{cases}
 \alpha_{ij}(s) & \{i,j\}\in A,\\
  a_{ij}(s) & \text{otherwise}
  \end{cases}
  \qquad\qquad
b_{ij}(s') = 
\begin{cases} \beta_{ij}(s) & \{i,j\}\in B,\\
  b_{ij}(s) & \text{otherwise}
\end{cases}
\]
and all other parameters $I_{lo}$, $I_{str}$, $J$, $\alpha$, $\beta$
are
the same for $s$ and $s'$.
\end{definition}


\begin{definition}[subdivision]\guid{YYKMEWW} 
Let $s$ be a stable constraint system,
and let $i,j\in  I(s)$, with $i\ne j$.  Select $c$ such that $a_{ij}(s)\le c\le b_{ij}(s)$.
The subdivision of $s$ is the pair $\{s_1,s_2\}$ 
of stable constraint systems, where all the parameters for $s_1$ and $s_2$
are the same as those of $s$, except that
\[
a_{ij}(s_1)=a_{ji}(s_1) = c, \text{ and } b_{ij}(s_2) = b_{ji}(s_2)=c.
\]
and
\[
\leftclosed\alpha_{ij}(s_\ell),\beta_{ij}(s_\ell)\rightclosed =
\leftclosed\alpha_{ij}(s),\beta_{ij}(s)\rightclosed\cap
\leftclosed a_{ij}(s_\ell),b_{ij}(s_\ell)\rightclosed.
\]
(If one one of these intersections is empty, then we set $s_1=s_2$ and
return the singleton $\{s_1\}$ consisting of the stable constraint
system with nonempty intersection.)
\end{definition}
The subdivision thus corresponds to splitting an interval 
$\leftclosed a_{ij},b_{ij}\rightclosed$ into
\[
\leftclosed a_{ij},c\rightclosed \cup \leftclosed c,b_{ij}\rightclosed.
\]



\begin{definition}\guid{LCTBALA}  
We say that a stable constraint system $s$
\newterm{transfers} to a stable constraint system $s'$ if
\begin{enumerate}
\item If $s$ is an ear, then $s=s'$.
\item $s'$ is unadorned.
\item $I(s) = I(s')$.
\item $d(s)\le d(s')$.
\item For all $i,j$, we have $a_{ij}(s')\le a_{ij}(s)\le b_{ij}(s)\le b_{ij}(s')$.
\item $J(s')\subset J(s)$.
\item $I_{str}(s')\subset I_{str}(s)$  and
$I_{lo}(s')\subset I_{lo}(s)$.
\end{enumerate}
\end{definition}



\begin{definition}\guid{TEQQCLX}
 The \newterm{opposite} $I'$ of a torsor $I$ is the torsor with the
same underlying set and the action is composed with the group automorphism
$\ring{Z}/k\ring{Z}\to\ring{Z}/k\ring{Z}$, sending $i\mapsto -i$.
Two torsors are \newterm{equivalent} if they are isomorphic or if 
one is isomorphic to the opposite of the other.
\end{definition}

If $s$ is a stable constraint system with torsor $I(s)$, and if
$I'$ is any equivalent torsor, then we can use the bijection between $I(s)$ and $I'$
to obtain a stable constraint system $s'$ with $I' = I(s')$.  An
stable constraint system $s'$ related in this way to $s$ is said
to be \newterm{equivalent} to $s$.  An \newterm{equi-transfer} of $s$
to $s'$ is a transfer from $s$ to a stable constraint system that
is equivalent to $s'$.

\begin{lemma}[]\guid{TECOXBM}\rating{0}\label{lemma:2hm-slice1}
\formalauthor{Hoang Le Truong}
Let $s$ be a stable constraint system, and let $\v\in \BB_s$.
Let $\u,\w\in V_\v$ satisfy $2\le\norm{\u}{\w}\le \stab$ where
$\{\u,\w\}\not\in E_\v$.  Then $\u$ and $\w$ are nonparallel.
Moreover,
$C^0\{\u,\w\}\subset \Wdarto(x)$ for all $x\in F$.
\end{lemma}


\begin{proof}  This is a repetition of Lemma~\ref{lemma:2hm-slice}.
\end{proof}


\begin{definition}[slice]\guid{CJMHFAT}
 Let $s$ be a stable constraint system and let $p,q\in I(s)$ where $p$ and
  $q$ are not adjacent.  (In particular, $k(s)=\card(I(s))>3$.)  We
  say that a pair $\{s',s''\}$ of stable constraint systems is an
\newterm{slice} of  the
  diagonal $\{p,q\}$ of $s$, if the following conditions hold.
\begin{enumerate}
\item $I(s')\cup  I(s'') = I(s),\quad I(s')\cap I(s'') = {p,q}$.
\item $d(s) \le d(s') + d(s'')$.
\item $d(s') < 0.9$, $d(s'') < 0.9$.
\item $J(t) \subset J(s) \cup \{\{p,q\}\}$, for $t = s',s''$.  
\item $\{p,q\}\in J(s')$ iff $s'$ or $s''$ is an ear.
\item $\{p,q\}\in J(s')$ iff $\{p,q\}\in J(s'')$.
\item 
\[
a_{ij}(t) = a_{ij}(s), \text{ and } b_{ij}(t) = a_{ij}(s),
\]
for all $t\in \{s',s''\}$ and all $i,j\in I(t)$ except $\{i,j\} = \{p,q\}$.
\item 
\[
a_{p q}(t)=\alpha_{p q}(s), \text{ and } b_{p q}(t)= \beta _{p q}(s),
\]
for all $t\in \{s',s''\}$.
\item
\[
\alpha_{i j}(s) = a_{i j}(t), \text{ and } \beta_{i j}(t) = b_{i j}(t),
\]
for all $t\in\{s',s''\}$ and all $i,j\in I(s)$.
%\item For $t=s',s''$,
%\[
%b_{pq}(s) \in \leftclosed a_{pq}(t),b_{pq}(t)\rightclosed.
%\]
%%!! Removed b_{pq}(s) in cover def on Jan 23, 2012. Restored here!
%% XX May 17, 2012 restored there.
\item
$I_{lo}(t)=I_{str}(t)=\emptyset$ for all  $t\in \{s',s''\}$.
\end{enumerate}
\end{definition}


The word {\it slice} is used for related operations on the indexing
set, the local fan, and the stable constraint system.  A slice of a
stable constraint system is used in parallel with the slice a fan of
cardinality $k$ into two smaller fans with cardinalities $k(s')$ and
$k(s'')$.  All of the edge length constraints are to be preserved
under slicing.  Note that the data $\alpha_{ij}$, $\beta_{ij}$,
$I_{lo}$, $I_{str}$ for $s'$ and $s''$ have been {\it reset} to
default values.

If $\{s',s''\}$ is an slice of a diagonal $\{p,q\}$, then we can use
the inclusions $I(s')\subset I(s)$ and $I(s'')\subset I(s)$ to
restrict an element $\v:I(s)\to\BB$ to $\v':I(s')\to \BB$ and
$\v'':I(s'')\to \BB$.  The inequality $d(s)\le d(s') + d(s'')$ of a
covered diagonal implies a related inequality.

\begin{lemma}\guid{QKNVMLB}\label{lemma:cover2}
Let $s$ be a stable constraint system with diagonal $\{p,q\}$ and slice
 $\{s',s''\}$. 
Let $\v\in \BB_s$ and let $\v'$ and $\v''$ be constructed from $\v$ as above.
Then  $\v'\in \BB_{s'}$ and $\v''\in \BB_{s''}$.
Moreover,
\begin{equation}
d(s,\v) \le d(s',\v') + d(s'',\v'')
\end{equation}
and
\begin{equation}
\tau^*(s,\v)\ge \tau^*(s,\v')+\tau^*(s,\v'').
\end{equation}
\end{lemma}

\begin{proof} See Lemma~\ref{lemma:cover}.
\end{proof}

\subsection{propagation}

The proof of the main lemma consists in showing that the nonemptiness
of $\MM_s$ propagates in an orderly way under the operations of
restriction, slicing, equi-transfer, and subdivision.

\begin{definition}[$\ra$~$\Ra$] \guid{AZGJNZO}
Let $\op{ACS}$ be the set of stable constraint systems
and let $P(\op{ACS})$ be the set of finite subsets of $\op{ACS}$. 
 We define a relation $(\ra)$ on $\op{ACS}\times P(\op{ACS})$.
We write $s\ra T$ if the following condition holds.
\begin{enumerate}
%\item $k=1$ and $s$ transfers to a stable constraint system equivalent to $s_1$.
%\item $k=2$ and $\{s_1,s_2\}$ is an slice along some diagonal of $s$.
%\item $k=2$ and $\{s_1,s_2\}$ is a subdivision of $s$.
\item If $\MM_s\ne\emptyset$, then 
$\MM_t\ne\emptyset$ for some $t\in T$.
\end{enumerate}
We define a binary relation $(\Ra)$ on $P(S)$ by
$S_1\Ra S_2$ if for every $s\in S_1$, we have $s\ra S_2$.
%$s\ra S''$.  We define a binary relation $\Ra^*$ on $P(S)$ as the reflexive
%transitive closure of $\Ra$.
\end{definition}

\begin{lemma}\guid{FZIOTEF}
The relation $(\Ra)$  is reflexive and transitive.
\end{lemma}

\begin{proof}  Clearly, $s\ra \{s\}$, and this implies reflexivity.
Transitivity is a simple matter.  Assume
$S_1\Ra S_2$ and $S_2\Ra S_3$.  Select any $s_1\in S_1$, then select
$s_2\in S_2$ such that $s_1\ra \{s_2\}$, and $s_3\in S_3$ such
that $s_2\ra \{s_3\}$.  Then $s_1\ra\{s_3\}$.  So $S_1\Ra S_3$.
\end{proof}

\begin{lemma}\guid{IEWZAVH}\label{lemma:propagate}
Let $S$  and $T$ be finite sets of stable constraint systems such that
$S\Ra T$.  Assume that $\MM_s\ne\emptyset$ for some $s\in S$.
Then $\MM_t\ne\emptyset$ for some $t\in T$.
\end{lemma}

\begin{proof} This follows directly from the definitions. 
\end{proof}

\begin{example}[restriction]\guid{EQTTNZI}
If $t$ is a restriction of $s$, then $\BB_t\subset \BB_s$.
A global minimizer over all of $\BB_s$ satisfying constraints $s$ is
a global minimizer over the subset $\BB_t$ and satisfies the
constraints $t$.  Hence $s\ra\{t\}$.
\end{example}

\begin{example}[subdivision]\guid{UAGHHBM}
Let $\{s_1,s_2\}$ be a subdivision of $s$.  We have $\BB_s =
\BB_{s_1}\cup \BB_{s_2}$.  A global minimizer in $\MM_s$ lies in
either $\BB_{s_1}$ or $\BB_{s_2}$.  Hence $s\ra\{s_1,s_2\}$.
\end{example}

\begin{example}[transfer, equi-transfer]\guid{YXIONXL}  Let $t$ be a transfer of
a stable constraint system $s$.  Assume that $\MM_s\ne\emptyset$.
Let $\w\in \MM_s$.
We have $w\in \BB_s\subset\BB_t$, and $\d(t,\w) \ge d(s,\w)$,
as well as $\tau^*(t,\w)\le\tau^*(t,\w)\le 0$.   By compactness, there
exists a global minimizer,
$\v\in\BB''_t\ne\emptyset$.  It satisfies
$\tau^*(t,\v)\le\tau^*(t,\w)\le 0$.  By the definition of transfer,
$t$
is unadorned, so that $\BB''_t=\MM_t$.  (Note that $\MM_s$ and
$\MM_t$
might be disjoint.)  This shows that
$\MM_t\ne\emptyset$.
Hence $s\ra\{t\}$.  Similarly, if $t$ is an equi-transfer of $s$,
then we again have $s\ra\{t\}$.
\end{example}

\begin{example}[slice]\guid{LKGRQUI}  Let $\{s',s''\}$ be a slice of
a stable constraint system $s$ along
a diagonal $\{p,q\}$.  Assume that $\w\in\MM_s$.
Let $\w'\in \BB_{s'}$ and $\w''\in \BB_{s''}$ be obtained
by restriction of parameters.  From Lemma~\ref{lemma:cover2},
we have $\tau^*(s',\w')\le 0$ or $\tau^*(s'',\w'')\le0$.  To be
concrete, say $\tau^*(s',\w')\le 0$.  The global minimizer
$\v'\in\BB_{s'}$
then also satisfies $\tau^*(s',\v')\le 0$.  By the definition of
slice, $s'$ is unadorned, so that $\BB''_{s'}=\MM_{s'}$ and
$\v'\in \MM_{s'}$.  Hence $s\ra\{s',s''\}$.
\end{example}

\begin{remark}[non-generic fans]\guid{KEOMGHK}
We have seen that a set $\MM_s$ does not contain any circular fans.
Assume that $\MM_s$ contains a lunar fan.
By a process of subdivision and Lemma~\ref{lemma:bjo}, we may assume
that it has some diagonal $\norm{\w_i}{\w_j}\le\stab$.  By slicing
along
the diagonal $\{i,j\}$, we obtain $\{s_1,s_2\}$ that are generic, with
$s\ra\{s_1,s_2\}$.  In this way, we may easily dispose of non-generic
fans.  In what follows, we may assume that every member of $\MM_s$ is
generic.
\end{remark}

The deformation lemmas can be expresed relational
arrows $s\ra T$.  The following example shows how this goes
for the first deformation lemma, Lemma~\ref{lemma:odx2}.  
In the example, we assume
that the members of $\MM_s$ are generic, to avoid the assumption
on poles of lunar fans in the lemma.

\begin{example}[deformation] \guid{KESHTYS} 
Let $s$ be a stable constraint
  system, with $\MM_s\ne\emptyset$.  Let $i\in I(s)$.
Assume that for both $j$ adjacent to $i$, we have $\{i,j\}\not\in
J(s)$.
Assume further that $4h_0 < b_{iq}(s)$ for every diagonal $\{i,q\}$ at
$i$.  Let $s'$ be a stable constraint system with all parameters
the same as $s$, except that $I_{lo}(s') = \{i\}\cup I_{lo}(s)$.
Keeping $i$ fixed, we
let $D=D_i\subset I(s)$ be the set of $j\ne i$ such that 
$\alpha_{ij}(s)=a_{ij}(s)$. 
Let $S_1$ be the set of stable constraint systems indexed by $j\in
D$,
obtained by
modifying $s$, setting $\beta_{ij}(s)=a_{ij}(s)$, with other
parameters unchanged.

We claim that there is an arrow $s\ra \{s'\}\cup S_1$.  Indeed, let
$\w\in \MM_s$.  For it to be minimal, there must be a constraint
that blocks the  deformation described in the proof of
Lemma~\ref{lemma:odx2}.
According to the lemma, this forces $\normo{\w_i}=2$ or
$\norm{\w_i}{\w_j}=a_{ij}(s)$ for some $j\ne i$.  This latter
condition
is incompatible with $\w\in \MM_s$, unless $j\in D$.  The result
follows.
\end{example}


In a similar way, the deformations of Lemmas~\ref{lemma:imj2} and \ref{lemma:nux2} and
the lateral deformations (Remark~\ref{rem:contract}) give arrows
\[
s\ra S.
\]

We have specified a set $S_{init}$ of initial stable constraints systems.
We specify a second set $S_{term}=\{s_{(6)},s_{(5)},\ldots\}$ 
of terminal stable constraint systems, where
 $s_{(k)}$, for $k=5,6$,  are given by
\[
I=\ring{Z}/k\ring{Z}, J=I_{lo}=I_{str}=\emptyset, d(s_{(5)})=0.616, d(s_{(6)})=0.712,
\]
and $a_{ij}=b_{ij}=2$ for all edges, 
and $a_{ij}=\stab$, $b_{ij}=6$ for all diagonals.

We do not list all of the elements of $S_{term}$.  There are about twenty
elements of $S_{term}$ with $k=3$ and four elements with $k=4$.  The elements
$s_{(5)}$ and $s_{(6)}$ are the only two with $k\ge 5$.
They are list explicitly
in the computer code (\verb!check_completeness.hl!).



\begin{remark}[Proof outline of main estimate]
The verification of 
Theorem~\ref{lemma:empty-d} and \ref{lemma:tau3} can now be
carried out as follows.  We assume for a contradiction that
one of the cases of these theorems is false.  By Lemma~\ref{lemma:init}, we have
$\MM_s\ne\emptyset$ for some $s\in S_{init}$.  

We prove by computer search
 that 
\begin{equation}\label{eqn:init-term}
S_{init}\Ra S_{term}.
\end{equation}
   By the definition of this
relation in terms of minimization problems over the compact sets
$\BB_s$,
it appears that some analysis might be required in the proof
of \eqref{eqn:init-term}.  However, this is not the case.  The
preceding
examples show how to construct many relational arrows $S_1\Ra S_2$
from equi-transfer, subdivision, slicing, and deformation.
We know that the relation is symmetric and transitive.   The computer
program makes a purely combinatorial search for the arrow
\eqref{eqn:init-term}
as a transitive composition of equi-transfers, subdivisions, slices,
and
deformations.  Hence no further nonlinear optimization is required
beyond
what has already been presented in this text.

There is a trivial amount of real arithmetic in the code that comes
from the triangle inequality.   (If
the constraints $\alpha_{ij},\beta_{ij}$ are such that the triangle
inequality cannot hold, we can conclude that $\MM_s=\emptyset$.)
Similarly, we use the spherical triangle inequality and the inequality
$\Delta\ge0$ in a few places to conlude that $\MM_s=\emptyset$.
But we insist that the proof of \eqref{eqn:init-term} is essentially
a combinatorial search.  The computer code makes no reference
to local fans, working consistently at the level of abstraction of stable
constraint
systems.


By Lemma~\ref{lemma:propagate}, and \eqref{eqn:init-term}, we have
$\MM_s\ne\emptyset$ for some $s\in S_{term}$.  When $s\in S_{term}$
with $k(s)\le 4$, by a \cc{various inequalities}{}, we show
that $\tau^*(s,\v) >0$ for all $\v\in \BB_s$.  This implies for such $s$
that $\MM_s=\emptyset$.  The two cases $s_{(5)}$ and $s_{(6)}$ remain.
They are treated in Secton~\ref{sec:computer-main}, where it is shown
that $\MM_s=\emptyset$ for $ s=s_{(5)}$ and $s_{(6)}$.  This
contradiction shows that $\MM_s=\emptyset$ for all $s\in S_{init}$,
so that $\tau^*(s,\v)>0$.  This proves the main estimate.

We remark that the verifications of  different cases in $S_{init}$ become
highly intertwined through the relation $(\Ra)$.  Each stable
constraint
system in $S_{term}$ contributes to the proof of many different cases
of the main estimate.
\end{remark}