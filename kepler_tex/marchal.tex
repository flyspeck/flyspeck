% file started March 22, 2009
% Marshal objective function

\chapter{Marchal's Objective Function}

C. Marshal has significantly simplified the objective function
for the non-linear optimization problem in the proof of the Kepler
Conjecture \cite{Mar08}, \cite{Marc07}.  His article claims to give
a {\it demonstration} of the Kepler conjecture.  However, the
mathematically rigorous part of the article only gives a reduction
of the problem to an optimization problem in a finite number of
variables.  The method of gradient descent is then used to explore
the local minima of the optimization problem in finitely many variables.

This chapter explains Marchal's method.  

We define the following constants and functions.
$$
\begin{array}{lll}
\alpha=\alpha_{cm} &= \arccos(1/3)\\
K=K_{cm} &= (3\alpha-\pi)\sqrt2/(12\pi - 30\alpha)\\
M = M_{cm} &= (18\alpha-7\pi)\sqrt2/(144\pi-360\alpha)\\
f(r) = f_{cm}(r) &=
\begin{cases}
 (\sqrt2-r) (r-1.3254) (9r^2 - 17 r + 3)/(1.627 (\sqrt2-1))& r\le\sqrt2\\
 0 & r >\sqrt2.
\end{cases}
\\
\end{array}
$$
We have 
\begin{equation}K - 12M = \sqrt{1/2}\end{equation}
and
\begin{equation}f(1) = 1,\quad f(\sqrt2) =0\end{equation}

\begin{conjecture}[Marchal]\label{conj:marchal} For every centered packing $(0,\Lambda)$,
we have
$$
\sum_{v\in\Lambda^*} f(|v|/2) \le 12.
$$
\end{conjecture}

\begin{theorem}[Marchal] 
Conjecture~\ref{conj:marchal} implies the Kepler Conjecture.
\end{theorem}

\begin{proof} Let $\T$ be the set of all triples
$\{u,v,w\}\subset\Lambda$ such that the circumradius of the triple is less than $\sqrt2$.
The sets $\op{conv}^0\{u,v,w\}$ for distinct triples are disjoint.

Let $\Q$ be the set of all simplices $Q=\{v_1,v_2,v_3,v_4\}$ with circumradius less than $\sqrt2$.
The faces of each of these simplices lie in $T$.  The sets $\op{conv}^0(Q)$ are disjoint
for distinct simplices in $Q$.

For each $T\in \T$ and each orientation $n$  of $T$, there is a unique point $v\in\ring{R}^3$
at distance $\sqrt2$ from each vertex in $T$, in the half-space determined by the oriented plane
of $T$.  Let $R(T,v) = \op{conv}^0(T\cup\{v\})$.  Let $\R$ be the set of all such
$(T,v)$ such that $R(T,v)$ is not in any $\op{conv}^0(S)$ for $Q\in\Q$.

Let $e=(v_1,v_2)$, $v_i\in\Lambda$ be a pair of vertices such that $|v_1-v_2|<\sqrt8$.  
Let $C(v_1,v_2)$ be the cone
$$
C(v_1,v_2) = \op{rcone}^0(v_1,v_2,XX).
$$
Let $S\{v_1,v_2\}$ be
the set of all vertices $v$ such that
\begin{itemize}
\item $\{v_1,v_2,v\}\in \T$, or
\item there exists $v_3$ such that $(\{v_1,v_2,v_3\},v)\in\R$.
\end{itemize}
Give $S$ the cyclic order around $(v_1,v_2)$ with azimuth cycle $\sigma$.
Let $S'\subset S$ be the subset of $v\in S$ such that the wedge
$W(v_1,v_2,v,\sigma v)$ contains a simplex of $\Q$ or of $\R$ along $\{v_1,v_2\}$.
Let 
$$
C'(v_1,v_2) = C(v_1,v_2)\setminus\left(\cup_{v\in S}\op{aff}(v_1,v_2,v)\cup\cup_{v\in S'} W(v_1,v_2,v,\sigma v)\right).
$$
It is an open part of $C(v_1,v_2)$ that avoids the sets in $\Q$ and $\R$.
let 
$$\dih(v_1,v_2) = \sum_{v\in S\setminus S'} \dih(v_1,v_2,v,\sigma v).
$$

For each $v\in\Lambda$, let 
$$B'(v) = B(v,\sqrt2) \setminus \ldots.$$
It is a radial set.


By construction, the sets are all disjoint from one another:
$$\begin{array}{lll}
  \op{conv}^0(Q) & Q\in \Q\\
  R(T,v) & (T,v)\in \R\\
  C'(v_1,v_2) & |v_1-v_2|<\sqrt8,\quad v_1,v_2\in\Lambda,\\
  B'(v) & v\in \Lambda
\end{array}
$$
Let $\rho>0$ be large.  Then the volume of $B=B(0,\rho)$ is at least
the volume of the constiuent pieces, up to a small error term from the boundary:
$$B \ge \sum_{Q\subset B}\op{vol}\op{conv}^0(Q) +
  \sum_{R(T,v)\subset B}\op{vol}(R(T,v)) +
  \sum_{v_1,v_2\in \Lambda_\rho}\op{vol}(C'(v_1,v_2)) +
   \sum_{v\in \Lambda_\rho} \op{vol}(B'(v)) + O(\rho^2).
$$
Marchal gives the following estimates on the terms:
$$
\begin{array}{lll}
\op{vol}B'(v) &\ge (2K/\pi)\op{sol}(B'(v)), \\
\op{vol}(C'(v_1,v_2)) &\ge (2K/\pi)\op{sol}(C'(v_1,v_2)) - (4 M/\pi) \dih(C'(v_1,v_2)) f(|v_1-v_2|/2)\\
\op{vol}(R(T,v)) &\ge (2K/\pi)\sum_{u\in T}\op{sol}(R(T,v),u) - (4M/\pi)\sum_{(u,u'):\{u,u'\}\subset T}
\dih(\{u,u'\},\{u'',u''''\}) f(|u-u'|/2)\\
\op{vol}(\op{conv}^0(Q)) &\ge (2K/\pi)\sum_{u\in Q}\op{sol}(Q,u) - (4M/\pi)\sum_{(u,u'):\{u,u'\}\subset Q} 
\dih(\{u,u'\},\{u'',u''''\}) f(|u-u'|/2)\\
\end{array}
$$
Here $\{u'',u''''\}$ are the remaining two vertices of $T\cup\{v\}$ or $Q$, as appropriate.

When we sum over all the pieces in $B(0,\rho)$ and regroup according to the lattice point, we get
(assuming the conjecture)
$$
\begin{array}{lll}
\op{vol}(B) &\ge \sum_{\Lambda_\rho} (2K/\pi) \op{sol}(B(v_1)) - (4 M/\pi) \sum_{v_2\in\Lambda,v\in S(v_1,v_2)}
\dih(v_1,v_2,v,\sigma(v)) f(|v_1-v_2|/2) + O(\rho^2)\\
   &= \sum_{v_1\in\Lambda_\rho} 8 K - 8 M\sum_{v_2} f(|v_1-v_2|/2) + O(\rho^2)\\
   &\ge\sum 8 (K - 12 M) + O(\rho^2)\\
   &= \sum_{v_1\in \Lambda_\rho} 4 \sqrt2 + O(\rho^2)\\
   &= (\frac{\pi}{\sqrt18})^{-1} \op{vol}(\cup_{v\in\Lambda_\rho}B(v,1))  + O(\rho^2)\\
   &=
\end{array}
$$
The result follows.


 
\end{proof}