% File added May 15, 2012

%\chapter{Supplementary Notes}\label{sec:supplement}

\section{Further notes on the main estimate}\label{sec:sup-local-fan}

This section was written in May 2012 in response to Hoang Le Truong's observation that the dependent type $\BB_s\subset \ring{R}^k$, with variable $k$, is unpleasant to deal with in HOL Light.  It also grew out of my efforts to write an Objective CAML program that checks the completeness of the arguments in Section~\ref{sec:weight}, at a informal level.

\begin{definition}
Let $s$ be a constraint system.  We say that $s$ is \newterm{tri-stable}, if $k(s)=3$
and if the
following conditions hold.
\begin{enumerate}
\item 
\[
2\le a_{ij} \text{ for all }  i,j\in I \text{ such that } i\ne j.
\]  
\item
  Also, 
\[0 = a_{ii}\text{ and }
  b_{i,i+1} < 4.
\]  
\item
If $\{i,j\}\in J$, then $\leftclosed
  a_{ij},b_{ij}\rightclosed=\leftclosed\sqrt{8},\stab\rightclosed$.
\end{enumerate}
(Note that these are the same as stability conditions, except that
the stability condition $a_{ij}\le \stab$ has been dropped, 
and $b_{i,i+1}\le \stab$ has been relaxed to $b_{i,i+1}<4$.)
\end{definition}

If $s$ is a tri-stable constraint system, we modify the definition
of $\BB_s$ by dropping the condition requiring $(V_\v,E_\v,F_\v)$ 
to be a local fan.  In other words, $\BB_s$ is the set of all functions
$\v:I(s)\to\BB$ that have the following property.
\begin{equation}\label{eqn:tri-stable}
a_{ij}(s)\le\norm{\v_i}{\v_j}\le b_{ij}(s), \text{ for all } i,j\in I(s).
\end{equation}


\begin{definition} An \newterm{augmented constraint system} is
a stable or tri-stable constraint system $s$ together with three subsets of $I(s)$:
$I_{hi}$, $I_{lo}$, and $I_{str}$.  
\end{definition}

Each stable or tri-stable constraint system may be
viewed as an augmented constraint system with $I_{hi}=I_{lo}=I_{str}=\emptyset$.
If $s$ is an augmented constraint
system, there is an associated constraint system that forgets the components
$I_{hi}$ $I_{lo}$, and $I_{str}$.

\begin{definition}  Let $s$ be an augmented constraint system, with
associated unaugmented constraint system $s'$.
We have defined $\BB_{s'}$. Let $\BB_{s}$ be the subset of all  $\v\in\BB_{s'}$ such that
\begin{enumerate}
\item If $i\in I_{str}$,  then $\v_i$ is straight.
\item If $i\in I_{lo}$,  then $\normo{\v_i}=2$.
\item If $i\in I_{hi}$,  then $\normo{\v_i}=2h_0$.
\end{enumerate}
\end{definition}



(Obviously, if $I_{hi}\cap I_{lo}\ne \emptyset$, then $\BB_s=\emptyset$.)

\begin{lemma} \label{lemma:aug-compact}
Let $s$ be an augmented constraint system. Then
$\BB_s$ is compact (as a subset of
$\BB^k \subset \ring{R}^{3k}$).
\end{lemma}

\begin{proof}  Assume that $s$ is an augmentation of a stable constraint 
system $s'$.
By Lemma~\ref{lemma:compact:bs},  then $\BB_{s'}$ is compact.  
The conditions defining $\BB_s\subset \BB_{s'}$ are clearly closed, so
$\BB_s$ is compact as well.

Assume that $s$ is tri-stable.  The set $\BB$ is compact, $I(s)$ is finite,
and the conditions \eqref{eqn:tri-stable} are closed.  Hence $\BB_s$ is
compact.
\end{proof}

 In the tri-stable case, we do not insist on local fans, but allow simplices that
may  degenerate to planar configurations.  By the constraint $b_{ij}(s)<4$, we have well-defined dihedral angles, so that $\tau^*(s,\v)$ is defined.

\begin{lemma}[continuity]\guid{HDPLYGY}\label{lemma:compact-fan}
Let $s$ be an augmented constraint system.  Then the function 
\[
\v\mapsto \tau^*(s,\v)
\]
is a continuous function on $\BB_s$.  Moreover, if $\BB_s$ is
nonempty, then the function attains a minimum.
\end{lemma}

\begin{proof} This follows easily from Lemma~\ref{lemma:compact-fan} and
Lemma~\ref{lemma:aug-compact}.
\end{proof}

\begin{lemma}
Let $s$ be an augmented constraint system.  Let $\v\in \BB_s$.
Suppose that
\[
d(s)\le 0.9 \text{ and } \sol(V_\v,E_\v,F_\v) \ge \pi.
\]
Then $\tau^*(s,\v)>0$.
In particular, if $d(s)\le 0.9$ and
$\tau^*(s,\v)\le0$, then $(V_\v,E_\v,F_\v)$ is not a circular local fan.
\end{lemma}

\begin{proof} The proof of Lemma~\ref{lemma:09} extends readily
to this context.
\end{proof}

A \newterm{diagonal} of a constraint system $s$ is a pair $\{p,q\}\subset I(s)$
such that $\{p,q\}$ is not a singleton and does not have the form $\{i,i+1\}$.


\begin{definition}[strict cover]
 Let $s$ be an augmented constraint system and let $p,q\in I(s)$ where $p$ and
  $q$ are not adjacent.  (In particular, $k(s)=\card(I(s))>3$, and $s$ is stable.)  We
  say that a pair $\{s',s''\}$ of augmented constraint systems 
\fullterm{strictly covers}{strict cover}  the
  diagonal $\{p,q\}$ of $s$, if the following conditions hold.
\begin{enumerate}
\item $\{s',s''\}$ covers the diagonal $\{p,q\}$ of $s$ in the sense of unaugmented
constraint systems.
\item For $t=s',s''$,
\[
b_{pq}(s) \in \leftclosed a_{pq}(t),b_{pq}(t)\rightclosed.
\]
%%!! Removed b_{pq}(s) in cover def on Jan 23, 2012. Restored here!
\item
$I_m(t) = I(t)\cap I_m(s)$, for all $m\in\{\op{str},\op{hi},\op{lo}\}$ and $t\in \{s',s''\}$.
\end{enumerate}
\end{definition}

A strict cover of a diagonal $\{p,q\}$ is used when we slice a fan of cardinality $k$ into
two smaller fans with cardinalities $k(s')$ and $k(s'')$.  All of the edge
length constraints are to be preserved under slicing, with a
mild compatibility condition on the new edge created by the slice.

If   $\{s',s''\}$ strictly covers a diagonal $\{p,q\}$, then we can use the inclusions
$I(s')\subset I(s)$ and $I(s'')\subset I(s)$ to restrict an element $\v:I(s)\to\BB$ to
$\v':I(s')\to \BB$ and $\v'':I(s'')\to \BB$.
The inequality $d(s)\le d(s') + d(s'')$ of a covered diagonal
implies a related inequality.

\begin{lemma}\label{lemma:cover2}
Let $s$ be an augmented constraint system with diagonal $\{p,q\}$ that is covered
by a pair of augmented constraint systems $\{s',s''\}$. 
Let $\v\in \BB_s$ and let $\v'$ and $\v''$ be constructed from $\v$ as above.
Then  $\v'\in \BB_{s'}$ and $\v''\in \BB_{s''}$.
Moreover,
\begin{equation}
d(s,\v) \le d(s',\v') + d(s'',\v'')
\end{equation}
and
\begin{equation}
\tau^*(s,\v)\ge \tau^*(s,\v')+\tau^*(s,\v'').
\end{equation}
\end{lemma}

\begin{proof} See Lemma~\ref{lemma:cover}.
\end{proof}

Departing from the presentation of Section~\ref{sec:weight}, we do not
use level functions and minimal counterexamples.  Rather, we proceed
as follows.


\begin{definition}[index,~$\iota$,~$\MM_s$] 
Let $s$ be an augmented constraint system.
Let $\MM_{s}\subset \BB_{s}$ be defined as follows.
Let $\BB'_{s}$ be the set of all $\v\in \BB_{s}$ such that
\begin{enumerate}
\item $\tau^*(s,\v)$ is equal to the infimum of $\tau^*(s,\wild)$ over $\BB_{s}$.
\item $\tau^*(s,\v)\le 0$.
\end{enumerate}
Define the \newterm{index} of $\v\in \BB_s$ to be the number of edges of $\v$
that attain its minimum bound $a_{i,j}(s)$.  Let $\iota(s)$ be the minimum
of the index of $\v$ as $\v$ runs over $\BB'_s$.  
We let $\MM_{s}$ to be the set of $\v\in \BB'_{s}$ that attain
the smallest possible index $\iota(s)$.
\end{definition}

\begin{definition}[$S_{init}$] 
For each of the cases of Theorem~\ref{lemma:empty-d} and \ref{lemma:tau3}, we fix a
stable constraint system that encodes its parameters, as described in Example~\ref{ex:main}. 
We extend each $s$ to an augmented constraint system with $I_{str}=I_{lo}=I_{hi}=\emptyset$.
Let $S_{init}$ be this set of  augmented constraint systems.
\end{definition}

We have the following analogue of Lemma~\ref{lemma:esm}.

\begin{lemma}
If \eqref{eqn:main:sv} fails to hold for some $s\in S_{init}$, 
then $\MM_s$ is nonempty.
\end{lemma}

\begin{proof}  Assume that \eqref{eqn:main:sv} fails to hold for
$s$.  By compactness and continuity, the set 
$\BB'_s$ of minimizers is nonempty.   
The subset $\MM_s$ on which the index is as small as possible
is then also nonempty.
\end{proof}

Lemma~\ref{lemma:min-crit} is not used directly in this supplement.
Instead, the criteria of the lemma will be presented somewhat differently
in the treatment that follows.

We have the following variant of Lemma~\ref{lemma:odx}.
It is not necessary to formalize that lemma.  We use this instead.

\begin{lemma} Let $s$ be an augmented constraint system,
and assume that $\w\in \MM_s$.  Let $i\in I(s)$.
Assume  that one of the following hold of the local fan $(V_\w,E_\w,F_\w)$.
\begin{enumerate}
\item The local fan is generic.
\item The local fan is lunar, the pole has azimuth
angle less than $\pi$, and $\w_i$ is not a pole.  
\end{enumerate}
Then one of the following   constraints hold.
\begin{enumerate}
\item $\norm{\w_i}{\w_{i+1}}$ attains its lower bound $a_{i,i+1}(s)$.
\item $\norm{\w_i}{\w_{i-1}}$ attains its lower bound $a_{i,i-1}(s)$.
\item $\normo{\w_i}$ attains its lower bound $2$.
\item There exists $j$ adjacent to $i$ such that $\{i,j\}\in J(s)$.
\item Some diagonal $\{p,q\}\subset I(s)$ satisfies
$\norm{\w_p}{\w_q}=a_{p,q}$.
\end{enumerate}
\end{lemma}

\begin{proof} 
We assume that the constraint on $J$ does not hold and
that $a_{p,q}(s)<\norm{\v_p}{\v_q}$ for all diagonals, then continue as in the proof of
Lemma~\ref{lemma:odx}.  The constraints on generic and local fans
allow us to use Lemmas~\ref{lemma:fan-open-lunar} and
\ref{lemma:fan-open-generic}, showing that fan conditions are preserved.
\end{proof}

We have the following variant of Lemma~\ref{lemma:imj}.
It is not necessary to formalize that lemma.  We use this instead.

\begin{lemma}
Let $s$ be an augmented constraint system.  Assume $i\in I_{str}(s)$ and
that $\w\in \MM_s$.  
Assume  that one of the following hold of the local fan $(V_\w,E_\w,F_\w)$.
\begin{enumerate}
\item The local fan is generic.
\item The local fan is lunar, the pole has azimuth
angle less than $\pi$, and $\w_i$ is not a pole.  
\end{enumerate}
Then one of the following conditions holds.
\begin{enumerate}
\item $\norm{\w_i}{\w_{i+1}}$ attains its lower bound $a_{i,i+1}(s)$, and
 $\norm{\w_i}{\w_{i-1}}$ attains its lower bound $a_{i,i-1}(s)$.
\item $\normo{\w_i}$ attains its lower bound $2$.
\item There exists $j$ adjacent to $i$ such that $\{i,j\}\in J(s)$.
\item Some diagonal $\{p,q\}\subset I(s)$ satisfies
$\norm{\w_p}{\w_q}=a_{p,q}(s)$.
\end{enumerate}
\end{lemma}

\begin{proof} We assume that the constaint on $J$ does not hold and
that $a_{p,q}(s)<\norm{\v_p}{\v_q}$ for all diagonals, then continue as in the proof of
Lemma~\ref{lemma:imj}.
The constraints on generic and local fans
allow us to use Lemmas~\ref{lemma:fan-open-lunar} and
\ref{lemma:fan-open-generic}, showing that fan conditions are preserved.
\end{proof}

We have the following variant of Lemma~\ref{lemma:nux}.
It is not necessary to formalize that lemma.  We use this instead.

\begin{lemma}
Let $s$ be an augmented constraint system.  Assume $i\in I_{str}(s)$
and that $\w\in \MM_s$.   Assume that $j$ is adjacent to $i$ and
that $\norm{\w_i}{\w_j}=a_{ij}(s)$.    
Assume  that one of the following hold of the local fan $(V_\w,E_\w,F_\w)$.
\begin{enumerate}
\item The local fan is generic.
\item The local fan is lunar, the pole has azimuth
angle less than $\pi$, and $\w_i$ is not a pole.  
\end{enumerate}
Then
one of the following conditions holds.
\begin{enumerate}
\item If $k\in I(s)$ is adjacent to $i$, then $\norm{\w_i}{\w_k}=a_{ik}(s)$
(for both choices of $k$ adjacent to $i$).
\item There exists $j$ adjacent to $i$ such that $\{i,j\}\in J(s)$.
\item Some diagonal $\{p,q\}\subset I(s)$ satisfies
$\norm{\w_p}{\w_q}=a_{p,q}(s)$.
\end{enumerate}
\end{lemma}

\begin{proof} Let $s$ and $\w$ be as given.  Assume that the final
two conditions do not hold.
Assume for a contradiction that 
\[
\norm{\w_i}{\w_k}>a_{i,k}(s)
\]
We consider a curve $\v$ of the form \eqref{eqn:move1} that moves $\v_i$
in a circular arc with center $\orz$ through the point $\w_i$ and in
the fixed plane determined by $\{\orz,\w_i,\w_j,\w_{k}\}$.  
The function $\tau^*(s,\wild)$ is
constant along this curve.  We orient the curve to be increasing
in $\norm{\w_i}{\w_{j}}$.  For sufficiently, small $t$, we find that
$\v(t)\in\BB'_s$ has smaller index than $\w$.  This is contrary to the minimizing
properties of $\w\in\MM_s$.

The constraints on generic and local fans
allow us to use Lemmas~\ref{lemma:fan-open-lunar} and
\ref{lemma:fan-open-generic}, showing that fan conditions are preserved.
\end{proof}

In the preceding three lemmas, we specifically allow the deformations
$\v(t)$ to occur within a lunar fan, moving a single node that is not a pole
of the lunar fan, as given by Lemma~\ref{lemma:fan-open-lunar}.
We are now ready to eliminate  lunar and circular
fans.

XX The constraints in the following lemma 
should be included in the definition of augmented constraint system.

\begin{lemma} Let $s$ be an augmented constraint system.
Assume that the following conditions hold.  
\begin{enumerate}
\item We have $4 \le k(s) \le 6$.
\item $d(s) \le 0.9$.
\item  $m+k(s)\le 6$, where 
$m$ is the number of edges $\{i,j\}\subset I(s)$ such that
$b_{i,j}(s)> 2h_0$ or $a_{i,j}(s)>2$.  
\end{enumerate}
Then for every  $\w\in\MM_s$, one of the following conditions hold.
\begin{enumerate}
\item 
 Some diagonal $\{p,q\}\subset I(s)$ satisfies
$\norm{\w_p}{\w_q}\le \stab$.
\item The local fan $(V_\w,E_\w,F_\w)$ is generic.
\end{enumerate}
\end{lemma}

\begin{proof}
Assume that $\norm{\w_p}{\w_q}> \stab$ for every diagonal $\{p,q\}\subset I(s)$.

By the definition of augmented constraint system, $4\le k(s)$ implies
that $s$ is stable, so that for every edge $\{i,j\}\in I$, we have
$b_{i,j}(s) \le\stab$.

Let $w\in \MM_s$.  By Lemma~\ref{lemma:09}, the local fan of $\w$  is not circular.
To show that the local fan is generic, it is enough to show that it is not lunar.
We assume for a contradiction that $\{\v_i,\v_j\}$ are poles of a lunar fan.

We first treat the case  $4\le k(s)\le 5$,
and afterward we will return to the case $k(s)=6$. 
Assuming that  $k(s)\le 5$, the indices $i,j$ differ by at most two.
Also, the assumptions give $m\le 6-k(s)\le 2$, so there is a path from
$i$ to $j$ involving at most one edge with $b_{i,j}(s)>2h_0$.
Then
\begin{equation}\label{eqn:pole}
\op{arc}_V(\orz,\v_i,\v_j) \le \op{arc}(2,2,2h_0) + \op{arc}(2,2,\stab) < \pi,
\end{equation}
showing that $\v_i$ and $\v_j$ are not parallel and not poles.

Now consider the case $k(s)=6$.  In this case, $m=0$, so that $b_{k,k+1}(s)\le 2h_0$
and $a_{k,k+1}(s)=2$
for every edge. Equation~\ref{eqn:pole} shows that the indices $i$ and $j$
must be opposite in the hexagon: $j = i+3$.  By the structure of lunar fans
(Lemma~\ref{lemma:lunar}), $\w_k$ is straight for $k\ne i,j$.  Without
loss of generality, we may
pass to a stricter augmented constraint system $s'$ such that
$k\in I_{str}(s')$, for $k\ne i,j$.  We then have $\w\in \MM_{s'}$.

Recall that $i$ is fixed at a pole. 
\claim{We claim that for any edge $\{i,k\}$, 
 if  $\norm{\w_{k}}{\w_i}>a_{ik}(s)=2$, then
$\normo{\w_i}=2$.}  Indeed, this follows directly from the preceding deformation
lemmas.  Thus, the tetrahedron with extreme points
$\{\orz,\w_{i-1},\w_i,\w_{i+1}\}$ has two edges
of length $2$ and a long edge $\{\w_{i-1},\w_{i+1}\}$ of length at least
$\stab$.  By a \cc{1117202051~4559601669}{}, 
the azimuth angle at the pole $i$ is at least $\pi/2$.
Hence the solid angle of the lune is at least $\pi$.  By Lemma~\ref{lemma:09},
we have $\tau^*(s',\w)>0$, which contradicts $\w\in \MM_{s'}$.
\end{proof}


\subsection{transformations}



\begin{definition} The \newterm{reverse} $I'$ of a torsor $I$ is the torsor with the
same underlying set and the action is composed with the group automorphism
$\ring{Z}/k\ring{Z}\to\ring{Z}/k\ring{Z}$, sending $i\mapsto -i$.
Two torsors are \newterm{equivalent} if they are isomorphic or if 
one is isomorphic to the reverse of the other.
\end{definition}


