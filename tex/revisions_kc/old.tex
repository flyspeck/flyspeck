A GCC mailing list entry from
1998\footnote{\url{http://gcc.gnu.org/ml/gcc/1998-02/msg00998.html}},
the year the Kepler project was completed, contains the following
comment (after describing fixes for a number of other 754-related
bugs): ``When you use fpu/cpu to do rounding, you have to mark the
variable volatile. Otherwise, the compiler may do something you don't
want.'' On the other hand, the \texttt{volatile} keyword is not
defined by the C language definition~\cite{Kernighan:1988:C}, meaning
that a C program using \texttt{volatile} is undefined by the language
definition, and is thus free to be dealt with by any compiler in any
way the compiler chooses.  The fact that GCC seems to rely upon
\texttt{volatile} use for the correctness of the compiler is
worrisome.  
%%% Local Variables: 
%%% mode: latex
%%% TeX-master: t
%%% End: 

The remainder of this section
gives some details about the status of the current implementation.

\paragraph{\emph{Tame Graphs Generation}} The graph generator
from~\cite{NipkowBS-IJCAR06} has been adapted into our interface (see
Section~\ref{sec:graph}).  Additionally we can generate an image of
any graph in the archive.  

\paragraph{\emph{Linear Programs}}

For each graph, we generate the corresponding linear programs using
the inequalities mentioned above.  In addition to generating linear
programs from graphs and lists of inequalities, we can bound them and
check the dual certificates using interval arithmetic as discussed in
Section~\ref{sec:lp}~\cite{Obua:2005:Thesis}.

\paragraph{\emph{Nonlinear Inequality Verification}}

The inequality verification code has been rewritten using the interval
arithmetic method of the original source, though independent of the
specifics of the interval implementation.  We can use a fast native
floating point implementation, or less efficient but more careful
alternatives such as MPFI.  The ability to use different
implementations gives us more confidence that our results are
independent of computer architecture.  (Though see Zumkeller's method
of Section~\ref{sec:zumkeller} for more rigorous results.)  Many of
the inequalities from the paper are organized using a unique
identifier, and thus can be executed from the user interface.  We are
still in the process of incorporating the thousands of such
inequalities.

We have also included the nonlinear optimization packages CFSQP
and Knitro to allow experimental testing of inequalities.  This has
allowed us to develop new inequalities for this revision, and to catch
a number of transcription errors in the computer code for the Flyspeck
project.
