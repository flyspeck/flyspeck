


\def\odpcvgh{
\tikzfig{trig}{\guid{ODPCVGH} Trigonometric and inverse trigonometric
functions}
%
%arctangent function on the domain \leftopen -4,4\rightopen\ 
%and the $\arccos$ function on $\leftclosed-1,1\rightclosed$.}}
{
[scale=0.5]
\draw (-2*1.57,0) sin (-1.57,-1) cos (0,0) sin (1.57,1) cos (3.14,0) sin (3*1.57,-1);
\draw   (-2*1.57,-1) cos (-1.57,0) sin  (0,1) cos (1.57,0) sin (2*1.57,-1) cos (3*1.57,0); 
\draw[help lines,<->] (-3.3,0) -- (3*1.57 + 0.2,0);
\draw[help lines,<->] (0,-1) -- (0,2.0);
\draw plot[smooth] file {tikz/tan.table};
\node at (-0.5,-1.8) {$\tan$};
\node at (2,0.5) {$\sin$};
\node at (1.1,-0.3) {$\cos$};
% GG need axis labels and ticks, base points of labels should be precisely aligned.
\begin{scope}[xshift=10cm]
\draw plot[smooth] file {tikz/arctan.table} node[above] {$\arctan$};
\draw plot[smooth] file {tikz/arccos.table} node[right] {$\arccos$};
%\draw[gray,->,very thin] (-1.2,0) -- (1.4,0);
\draw[help lines,<->] (0,-1.6) -- (0,3.2);
\draw[help lines,<->] (-4,0) -- (4,0);
\end{scope}
}}

%
\def\sample{
\tikzfig{circle}{\guid{HGMTQFG} Lemma~\ref{lemma:circle} as a special case of the Pythagorean theorem}
{
[scale=0.1]
\draw (0,0)  --(12,0) --  (12,5) --  cycle;
\draw[very thin] (11,0) -- (11,1) -- (12.0,1);
\path (5,-1.5) node {$\cos x$};
\path (16,2.5) node {$\sin x$};
\path (6,5)  node {$1$};
}
}

%
\def\sample{
\tikzfig{tan}{\guid{GQQAKYI} 
The tangent function on $\leftopen-\pi/2,\pi/2\rightopen$}
{
[scale=0.2]
\draw plot[smooth] file {tikz/tan.table};
\draw[help lines,->] (-1.57,0) -- (1.57,0);
\draw[help lines,->] (0,-6.0) -- (0,6.0);
}
}

\def\sample{
\tikzfig{arctrig}{\guid{RUJPPWJ} 
The arctangent function on the domain \leftopen -4,4\rightopen\ 
and the $\arccos$ function on $\leftclosed-1,1\rightclosed$.}
{
[scale=0.4]
\draw plot[smooth] file {tikz/arctan.table};
\draw plot[smooth] file {tikz/arccos.table};
%\draw[gray,->,very thin] (-1.2,0) -- (1.4,0);
\draw[help lines,->] (0,-0.2) -- (0,3.2);
\draw[help lines,->] (-4,0) -- (4,0);
}
}

\def\sample{
\tikzfig{atn-polar}{\guid{YOXQFUB} 
The function $\atn$ gives the polar angle $\theta$ of $(x,y)$.}
{
[scale=0.15]
\draw[gray,->,very thin] (-4,0) -- (14,0);
\draw[gray,->,very thin] (0,-2) -- (0,5);
\draw (0,0)  --(12,0) --  (12,5) --  cycle;
\draw[very thin] (11,0) -- (11,1) -- (12.0,1);
\path (6,-1.5) node {$x$};
\draw[very thin] (4,0) arc (0:22.62:4);
\path (14,2.5) node {$y$};
\path (2,3) node {$\theta$};
}
}

\def\tuligly{
\tikzfig{M}{\guid{TULIGLY} The quartic polynomial $M$}
{
[xscale=10.0,yscale=2.0]
\draw[help lines,<->] (1.0,0) -- (1.5,0);
\draw[help lines,<->] (1,-0.2) -- (1,1.2);
\draw plot[smooth] file {tikz/TULIGLY.table};
}}

\def\bjliekb{
\tikzfig{L}{\guid{BJLIEKB} Detail of the quartic $M$ and piecewise linear function $L$ on
the domain $\leftclosed1.2,1.35\rightclosed$}
{
[scale=12.0]
\draw plot[smooth] file {tikz/BJLIEKB.table};
\draw[help lines,<->] (1.18,0) -- (1.37,0);
\draw[help lines,<->] (1.2,-0.01) -- (1.2,0.25);
\draw[help lines] (1.23175,0.25) -- (1.23175,-0.01) node[anchor=north,black] {$h_-$};
\draw[help lines] (1.26,0.25) -- (1.26,-0.01) node[anchor=north,black] {$~~\hm$};
\draw[help lines] (1.3254,0.25) -- (1.3254,-0.01) node[anchor=north,black] {$h_+$};
\draw (1.2,0.230769 ) -- (1.26,0);  %
\draw (1.26,0) -- (1.35,0);
}}
%


\def\jxehxqy{
\tikzfig{fg}{\guid{JXEHXQY} The functions $g$ takes negative values,
but the function $f$ remains positive, as predicted by the cell cluster
inequality.  The nondifferentiability at $2h_0$ is inherited from the
nondifferentiability of $L$.}
{
[xscale=12.0,yscale=100.0]
\draw plot[smooth] file {tikz/jxehxqy1a.table};
\draw plot[smooth] file {tikz/jxehxqy1b.table} node[anchor=west,black] {$f$};
\draw plot[smooth] file {tikz/jxehxqy2a.table};
\draw plot[smooth] file {tikz/jxehxqy2b.table} node[anchor=west,black]{$g$};
\draw[help lines,<->] (2.46,0) -- (2.66,0);
%\draw[help lines] (2.4635,-0.004) -- (2.4635,0.01);
\draw[help lines] (2.4635,0.01) -- (2.4635,-0.004) node[anchor=north,black] {$2h_-$};
\draw[help lines] (2.6508,0.01) -- (2.4635,0.01) node[anchor=east,black] {$0.01$};
\draw[help lines] (2.6508,0.00) -- (2.4635,0.00) node[anchor=east,black] {$0.00$};
\draw[help lines] (2.52,0.01) -- (2.52,-0.004) node[anchor=north,black] {$2\hm$};
\draw[help lines] (2.6508,0.01) -- (2.6508,-0.004) node[anchor=north,black] {$2h_+$};
}}
%



\def\pqfexqn{
\tikzfig{fg1}{\guid{PQFEXQN} The functions $\beta_0$}
{
[xscale=20.0,yscale=150.0]
\draw plot[smooth] file {tikz/pqfexqn.table};
\draw[help lines,<->] (1.23,0) -- (1.33,0);
%\draw[help lines] (1.2318,-0.004) -- (1.2318,0.01);
\draw[help lines] (1.2318,0.005) -- (1.2318,-0.002) node[anchor=north,black] {$h_-$};
\draw[help lines] (1.3254,0.005) -- (1.2318,0.005) node[anchor=east,black] {$0.005$};
\draw[help lines] (1.3254,0.00) -- (1.2318,0.00) node[anchor=east,black] {$0.000$};
\draw[help lines] (1.26,0.005) -- (1.26,-0.002) node[anchor=north,black] {$\hm$};
\draw[help lines] (1.3254,0.005) -- (1.3254,-0.002) node[anchor=north,black] {$h_+$};
}}
%



\def\rrkrgpvjw#1#2{\shade[ball color=gray](#1,#2) circle (1);  }
\def\KRGPVJW{
\tikzfig{svdw}
{\guid{KRGPVJW} The Sch\"utte-van der Waerden contact graph and packing.  
Four edges that
belong to the standard graph but not the contact graph are shown in gray.  Twelve
balls in the packing are centered near the centers of the edges of a cube.}
{
{
\begin{scope}[scale=0.004]
%Set the coordinates of the points:
%\tikzstyle{every node}=[draw,shape=circle];
\path (45:400) coordinate (P0) ;
\path (135:400)  coordinate (P1) ;
\path (225:400) coordinate (P2) ;
\path (315:400) coordinate (P3) ;
\path (0:200) coordinate (P4) ;
\path (90:200) coordinate (P5) ;
\path (180:200) coordinate (P6) ;
\path (270:200) coordinate (P7) ;
\path(45:150) coordinate (P8) ;
\path (135:150) coordinate (P9) ;
\path (225:150) coordinate (P10) ;
\path (315:150) coordinate (P11) ; 
\path (0,0) coordinate (P12) ;
\foreach \i in {0,...,12}
{
  \fill (P\i) circle (15);
}
%Draw edges:
\draw
  (P12) -- (P8)
  (P12) -- (P9)
  (P12) -- (P10)
  (P12) -- (P11)
  (P8) -- (P4)
  (P4) -- (P11)
  (P11) -- (P7)
  (P7) -- (P10)
  (P10) -- (P6)
  (P6) -- (P9)
  (P9) -- (P5)
  (P5) -- (P8)
%
  (P0) -- (P1)
  (P1) -- (P2)
  (P2) -- (P3)
  (P3) -- (P0)
%
  (P0) -- (P5)
  (P5) -- (P1)
  (P1) -- (P6)
  (P6) -- (P2)
  (P2) -- (P7)
  (P7) -- (P3)
  (P3) -- (P4)
  (P4) -- (P0);
\draw[gray,very thin]
  (P8) -- (P9)
  (P9) -- (P10)
  (P10)--(P11)
  (P11)--(P8);
\end{scope}
%
\begin{scope}[scale=0.5,xshift=8cm]
\def\rr{\rrkrgpvjw}
\rr{-0.504725}{0.79793}
\rr{0.987379}{-0.530059}
\rr{-0.406371}{-1.76776}
\rr{-1.8337}{-0.370827}
\rr{1.68242}{1.01951}
\rr{0.}{2.0538}
\rr{1.35457}{-1.58937}
\rr{0.}{0.}
\rr{-1.68242}{1.20943}
\rr{-1.35457}{-1.43645}
\rr{1.8337}{0.000711695}
\rr{0.504725}{1.431}
\rr{0.406371}{-1.25805}
\rr{-0.987379}{0.159943}
\end{scope}
%\shade[ball color=blue] (2,2) circle (1); % color = gray
%\shade[ball color=blue] (2.5,2) circle (1); % color = gray
}
}}