% DCG Kepler Errata
% Author: Thomas C. Hales
% Affiliation: University of Pittsburgh
% email: hales@pitt.edu
%
% latex format

% History:
% May 1, 2007
% Dec 12, 2007


\documentclass[11pt]{amsart}
\usepackage{graphicx}
\usepackage{amsfonts}
\usepackage{amscd}
\usepackage{amssymb}
\usepackage{alltt}

% version
\def\ver{~Dec. 12, 2007}

% Math notation.
\def\op#1{{\text{#1}}}
\newcommand{\ring}[1]{\mathbb{#1}}
\def\to{{\quad\Longrightarrow\quad}}
\def\line{$\ell$}
\def\text{\hbox}

\parindent=0pt
\parskip=\baselineskip

%%%%%%%%%%%%%%%%%%%%%%%%%%%%%%%%%%

\begin{document}

\title{Errata for ``The Kepler Conjecture''}

\author{Thomas C. Hales}
\email{hales@pitt.edu}
\address{Math Department, University of Pittsburgh}


\maketitle

\section{Introduction}


\subsection{Relation between the Abridged and Unabridged Versions}

The abridged version of the Kepler conjecture
in the Annals \cite{A}
was generated by the same tex
files as the unabridged version in \cite{DCG}.


There are a few differences in wording that
were required between the two versions.
These are formatting
differences, different naming
conventions for the sections and subsections,
different conventions for references and citations,
and so forth.
The two articles also carry minor differences
in wording of transitional phrases that
accommodate the slightly different organizational
structure of the two documents.  A simple
tex macro was used to generate the occasional passage
that differs.

Because of the way these documents were produced
from the same tex files,
it seems that nearly every correction to
the abridged version will also be a correction to the unabridged version.
(So far, no errors have been found that are
unique to the abridged version.)
For that reason, we list the errata for the
unabridged version. The same list applies to corresponding 
passages in the abridged version.  




\subsection{Format}

Each correction gives its location in \cite{DCG}.
The location
\line+n counts down from the top of the page, or
if a section or lemma number is provided, it
counts from the top of that organizational unit.
The location \line-n counts up from the bottom
of the page. Footnotes are not included in the
count from the bottom.  Every line containing
text of any sort is included in the count,
including displayed equations, section headings,
and so forth.


\section{Errata}

[p.47,Lemma~5.16]
	$$
	Q \to F
	$$

[p.49,\line+2]
        $$
        \text{supposed} \to \text{suppose}
        $$
        
	
[p.63,Lemma~7.10]
	$$
	{\mathcal S}-system \to Q-system
	$$
	
[p.73][p.124] Theorem~8.4 is stated for entire
standard cluster, but is applied to a group of
three simplices in Lemma~11.27.   A similar
problem may exist with other applications of
Theorem~8.4.  FIX: restate Theorem~8.4 in
sufficient generality to cover all cases.
(The proof is based on a much finer decomposition
into pieces.  Because of the structure of
the proof, it the result holds with much
weaker hypotheses than what is stated.)

[p.75,Remark~8.11]
	$$
	\text{show} \to \text{shows}
	$$

[p.78,\line-7]
      $$
      \text{constraints} \to \text{constraint}
      $$

[p.86,\line+14]
        $$
        \text{Let $\{0,v\}$ be 
          the diagonal of an upright quarter in the $Q$-system}
        $$

        $$
        \to
        \text{Let $v$ be a vertex with $2t_0<|v|<\sqrt8$.}
          $$
	
          Remark: Section~9 assumes that the diagonal belongs to
          a quarter in the $Q$-system, but Lemma~10.14 uses these
          results when $\{0,v\}$ has $0$ or $1$ anchors.  To make
          this coherent, we should assume throughout Section~9 that
          we have the weaker condition that whenever $\{0,v\}$ has
          two or more anchors it belongs to a quarter in the $Q$-system.
          The proofs of Section~9 all go through in this context.
          (Lemma~9.7 is all that is relevant here.)

[p.87,Definition~9.3]
	In definition of $\Delta(v,W^e)$, we
	can have some $Q$ (as in Fig~9.1)
	with negative orientation.
	In this case, $E_v\cap E_i$ can clip
	the other side.  We want the object
	without clipping, so the definition must
	be modified slightly to reflect this.
	
[p.88,Definition~9.6]
	The definition is poorly worded.  First of
	all, it requires that the subscript to
	$\epsilon$ should be a vertex, but then in
	the displayed equation, it makes $w/2$ the
	subscript, which needn't be a vertex.  To
	define $\epsilon'$, move from $w/2$ along
	the ray through $x'$ until an edge of the
	Voronoi cell is encountered.  If $v,w,u$
	are the three vertices defining that edge,
	then set $\epsilon'_v(\Lambda,x)=u$.
	Degenerate cases, such as when two different
	edges are encountered at the same time,
	can be resolved in any reasonable fashion.
	
[p.88,Lemma~9.7,\line+2] 
	$$
	w\text{ and } v\to w \text{ and } u
	$$
	
[p.88,Lemma~9.7,Claim~1]
	$$
	\text{ with } |w - w'|\le 2t_0, \text{ and }
	\to \text{ with }
	$$
	
[p.88,Lemma~9.7,\line+5]
         Then: $\to$

        \narrower{Let
          $$
          R'_w = \{x\in R_w \cap(0,\{u,w\})\mid 
          \epsilon_0(x,\{u,w\}) = u.
          $$  
          Assume that $R'_w$ is not empty. Then:}


(This new hypothesis is satisfied
        in every application of Lemma~9.7.
        We note that this forces the orientation of $\{0,v,w'\}$ to
        be negative in $Q=\{0,v,w',u\}$, which in turn forces $Q$
        to be a quarter.)

[p88,Lemma~9.7,Claim~3]
        $$
        R_w \to R'_w
        $$

        (This weaker claim is all that is ever needed in applications
        of Lemma~9.7.)

[p.89,\line+2]
	$$
	\{w,v\}\to\{w,u\}
	$$

[p.92,\line+16]
        $$
        \max_j u_j \to \max_j |u_j|
        $$

[p.92,\line+21]
        $$
        \max_j u_j \to \max_j |u_j|
        $$
	
[p.93,\line-4]
	$$
	\text{obstructed from }w \to
	\text{obstructed from }w'
	$$
	
	
[p.93,\line-3] This is not an error, but we
	can cite one Lemma rather than three:
		$$
		\begin{array}{lll}
		\text{barrier} &\to \text{upright barrier}\\
		\text{Lemmas~9.8, 9.10, and 9.11} &\to
		\text{Lemma~9.11}
		\end{array}
		$$
		
[p.93,\line-2]
	$$
	\text{from some} \to \text{for some}
	$$

[p.99,\line+1]
        $$
        \text{start} \to \text{star}
        $$

[p.105,Lemma~10.14]  In the proof of the cases involving
   $0$ or $1$ anchor, a combination of the decompositions from
   Section~8.4 and Section~9 are used.  These decompositions haven't
   been shown to be compatible.  
   FIX: It is better to combine
   $\Delta(v,W)$ with $t_0$-truncation on the rest of the quad-cluster.
   With a $t_0$ truncation, we no longer have the non-positivity results
   from Section~8.  (The quoins give a positive contribution.) However,
   I have checked that
   the estimate on $\Delta(v,W)$ is sufficiently small that we still
   obtain a constant less than $-1.04\,\operatorname{pt}$.
   

[p.116][p.121] Definition~11.7 allows masked
flat in definition of $3$-unconfined.
Definition~11.24 requires no masked flats
in the same definition.  FIX: Use Definition~11.24 (no masked flats).  Where masked flats occur,
treat them with Lemma~11.23, parts (1) and (2).

[p.116,\line+1] 
	$$
	\text{Lemma}~4.16 \to \text{Lemma}~4.17
	$$
At any rate, the lemma is not being applied
precisely, nor was I trying to.

[p.117,before Lemma~11.9]
	$$
	\text{two others} \to \text{three others}
	$$
	
[p.117,Def~11.8]
    $$
    y1 \to y_1
    $$
    
 
	
	
[p.121] See p.116.

[p.121,\line-5]
	$$
	0.2274 \to 0.02274
	$$
	
[p.123. flat case (2)]  It is missing
isolated quarters cut from the side.
FIX: In condition 2(f), 
	$$
	\eta_{456}\ge\sqrt2 \to
	\eta_{456}\ge\sqrt2 \text{ or } \eta_{234}\ge\sqrt2.
	$$
	
[p.124] See p.73.
	
[p.126]  Theorem~12.1 needs to be stated in
a form that allows the application in pp.251-252
and Lemma~13.5.  In these applications, the
regions are smaller than standard regions.
Yet in the statement of the theorem, the regions
are standard regions.  This is not a problem
in practice, because the proof is at a much
finer level of decomposition than standard regions.
However, the wording needs to be changed so
that the theorem applies precisely.

[p.126] 
In Theorem~12.1, there is the restriction
	$5\le n\le 8$.
But in some applications of the theorem to
linear programming, the constraints on $n$
are   $$3\le n\le 8$$
with	$$\sigma_R(D)\le s_n$$
and with $s_3 = 1\,\op{pt}$ and $s_4=0$.
Similarly, when $n(R)=3$,
	$$\tau_R(D) \ge t_3 = 0.$$
(Thanks to S. Obua.)

[p.131]
In Section~12.7, the argument in the first two
paragraphs about reducing to a polygon is incomplete, because it doesn't show that the deformation can be done in such a way that the distances remain at least $2$.  (The Remark~12.7 about this issue only applies to the subsequent 
deformations.  This is a serious issue that requires an extended explanation.

If we only use distinguished edges of length
at most $2.91$, no problems arise.  This is
because geometric considerations give
  $$E(2,2,2,2.51,2,2.51,2,2,2) > 2.91.$$
However, this does not solve the problem, because
loops are allowed to have edges of length as
great as $3.2$.

This issue does not occur in the proof of the dodecahedral conjecture.  There we can start with edges of length up to length $2.91$, and make the deformations described in Section~12.7.  After deformations have made each subregion into a simple polygon, the edges can be extended out to $3.2$, and the argument continues as before.

The problem is with loops which have an anchored simplex with an edge in the range $[2.91,3.2]$.  To fix the problem, we need to be much more aggressive 
in expunging upright quarters.  Thus, we need to revisit earlier sections of the text that erase upright quarters.

Before we get into this too far, I want to amend a sloppiness of language in the published proof.  When the upright quarters do not mask any flat quarters, then we follow the terminology of the published proof for what it means to erase upright quarters.  When the anchored simplices around an upright diagonal mask flat quarters $Q_1,\ldots,Q_r$, then we say that we can expunge the anchored simplices $S_1,\ldots,S_k$ with penalty $\pi_0$, if 
  $$
  \sum_{i=0}^k \sigma^*(S_i) - 4\delta_{oct}\op{vol}(\delta_P(v))
  < \pi_0 + \sum_{i=0}^k \op{s-vor}_0(S_i)
   +\sum_{i=0}^r (\hat\sigma(Q_i) - \op{s-vor}_0(Q_i)).
  $$
We take $\sigma^*(S)$ to be $\sigma(S)$ when $S$ is an upright quarter; $\sigma(V_S(t_S))$ when it is a simplex of type $C$; and $\op{s-vor}_0(S)$ otherwise.
(When we merely erase rather than expunge, the sum over $r$ is absent. Compare p.112, Section~11.1.)
If there are no masked flat quarters,  there is no distinction between erasing and expunging.   In general, we wish to expunge rather than just erase.





[p.139,Lemma~12.18,proof,\line+3] 
	$$C_0(|v|,\pi) \to
	C_0^u(|v|,\pi)
	$$
	
[p.139,Lemma~12.18] The constant 2.2 of the
statement is not mentioned in the proof, except
in the first sentence of the proof.
	$$
	\tau_0(C_0^u(2t_0,\pi))-\pi_{\text{max}}\to
	\tau_0(C_0^u(2.2,\pi))-\pi_{\text{max}}
	$$
[This is what I believe it should be; I have
not rechecked the mathematical calculation.]

[p.144,\line+11,\line+17]
	$$2t_0^2 \to (2t_0)^2
	$$


[p.146] The last symbol $S_n^\pm$ in the
sentence before the start of 
Section~13.5 is undefined.
	$$\begin{array}{lll}
	S_n^\pm \to&\\
	\text{of 3-crowded, 3-undefined, and
	4-crowded combinations} 
	\end{array}
	$$
	
[p.148,Sect. 13.6]  This entire
section is misplaced.  It belongs with
Sections 25.5 and 25.6 much later.

[p.149,before 13.7]
	$$
	\text{the diagrams}\to
	\text{Figs~25.1--25.4}
	$$
	
[p.149] $\delta_{loop}$ is not defined.

[p.156] $\delta_{loop}$ is not defined.

[p.156,Lemma~13.5,\line+4]
	$$
	\begin{array}{lll}
	\text{respectively for }\tau_R(D)\to\\
	\text{respectively, for }\sigma_R(D) \text{ and }
	\tau_R(D),  
	\end{array}
	$$
	
[p.164,\line-1] 
	$$
	\begin{array}{lll}
	\text{This shows }\ldots \text{ occur.}\\
	\text{This completes the proof.}
	\end{array}
	$$
	
[p.182,Lemma~16.7]  I do not understand why
the bound holds on each half.  It seems that
the deomposition into the halves might not be
compatible with the geometry: cone or quoin
terms might ``cross over'' into the other half.
At any rate, it is not a direct consequence
of Theorem~8.4.

Proposed fix: Show by an interval arithmetic
calculation that each side separately satisfies
that bound $0.04\,\text{pt}$. (I'm making up
a number here.  This is just an educated guess.)
Then restrict to halves whose score exceeds
$-1\,\text{pt}$.

[p.241]  `Mixed' is defined so as to include
the pure analytic case.  In earlier papers,
`mixed' excludes the pure analytic.  
	$$
	\text{mixed}\to\text{mixed or pure}
	$$
	
[p.243,\line+13,\line+14,\line+15]
	Delete three sentences:
	`Let $v_{12}$ be $\ldots$  We let $\ldots$
	 Break the pentagon $\ldots$'
	
[p.248,last displayed formula]  
	$$
	= \to +
	$$
so that it reads
	$$
	\sum_i f_{R_i}(D) \le \hat\sigma(Q_i) +
	\op{vor}_{R',0}(D) + \pi_R
	$$

[p.252,Sec.~25.7,Cases~2 and 3]  `The flat quarter'
is mentioned, but there are no flat quarters
that have been introduced into the context.  
It seems that this passage
has been moved by a cut-and-paste edit to a
place it does not belong.

[p.254,\line+7]
	$$
	\text{to branch combine} \to \text{to combine}
	$$




\section{Index}

The index should have additional entries.
\begin{itemize}
	\item [p.128] distinguished edge
	\item [p.128] special
\end{itemize}

\section{Code}

\subsection{Mathematica Code}

The Mathematica code at the Annals website
(http://annals.math.princeton.edu/keplerconjecture/sphere.txt) needs to be updated.

[sphere.txt,global]
The function $\op{Norm}$ defined in sphere.txt
has become a Mathematica built-in function.
We need to rename our function: 
	$$\op{Norm} \to \op{norm}$$
	
[sphere.txt,\line+262]
The file {\it more.m} is not part of the distribution:
	$$\ll \op{more.m} \to \op{(* -- *)}$$



Please report further errors to
Thomas C. Hales.\footnote{Version: \ver}


\begin{thebibliography}{}

%% References
\bibitem{A} {T. Hales}, A proof of the Kepler
	conjecture, Annals of Mathematics,
	2006.
	
\bibitem{DCG} {S. Feguson and T. Hales},
	The Kepler Conjecture, Disc. and Comput.
	Geom. 36 (1), July 2006.


\end{thebibliography}







\end{document}
