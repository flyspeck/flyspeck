\chapter{Linear Assembly Problems}

\section{Linear Assembly Problems} \label{linear}

In this section we define a class of nonlinear optimization
problems that we call {\it linear assembly problems}.

Assume given a topological space $X$, and a finite collection of
topological spaces, called {\it local domains}.  For each local
domain $D$ there is a map $\pi_D:X\to D$.  There are functions
$u_i$, $i=1,\ldots,N$, each defined on some local domain $D_i =
\op{dom}(u_i)$, and we let $x_i$ denote the composite $x_i =
\pi_{D_i}\circ u_i$.

On each local domain $D$, the functions $u_i$ are related by a
finite set of nonlinear relations
\begin{equation}\phi(u_i : \op{dom}(u_i) = D) \ge0, \quad \phi \in \Phi_D.
    \label{phi}
\end{equation}

We use vector notation $x = (x_1,\ldots,x_N)$, with constant
vectors $c$, $b$, and matrix $A$ given.

The problem is to maximize $c\cdot x$ subject to the constraints
    \begin{equation}\label{Ax}A\, x \le b,
    \end{equation}
and to the nonlinear relations~\ref{phi}.  A problem of this form
is called a linear assembly problem.  (Intuitively, there are a
number of nonlinear objects $D$, that form the pieces of a jigsaw
puzzle that fit together according to the linear
conditions~\ref{Ax}.)

\begin{example}
Assume a single local domain $D$, and let $\pi_D:X=D$ be the
identity map.  The function $f = c\cdot x $ is nonlinear. The
problem is to maximize $f$ over $D$ subject to the nonlinear
relations $\Phi_D$.  This is a general constrained nonlinear
optimization problem.
\end{example}

\begin{example}  Assume that each $u_i$ has a distinct local domain $D_i = \ring{R}$.
Let $X = \ring{R}^N$, let $\pi_D$ be the projection onto the $i$th
coordinate, and let $x_i$ be the $i$th coordinate function on
$\ring{R}^n$. Assume that $\Phi_D$ is empty for each $D$.  The
problem becomes the general linear programming problem
    $$\max c\cdot x$$
such that $A x\le b$.
\end{example}

These two examples give the nonlinear and linear extremes in
linear assembly problems. The more interesting cases are the mixed
cases which combine nonlinear and linear programming.
Example~\ref{pr:third} gives one such case.

\begin{figure}[htb]
  \centering
  \myincludegraphics{\ps/vor.eps} %% CORRECT GRAPHIC?
  \caption{A truncated Voronoi cell and a subset of the cell lying in a sector}
  \label{voronoi}
\end{figure}

\begin{example} (2D Voronoi cell minimization). Take a packing of disks of
radius $1$ in the plane.  Let $\Lambda$ be the set of centers of
the disks.  Assume that the origin $0\in\Lambda$ is one of the
centers. The truncated Voronoi cell at $0$ is the set of all
$x\in\ring{R}^2$ such that $|x|\le t$, and $x$ is closer to the
origin than to any other center in $\Lambda$.  We assume
$t\in(1,\sqrt2)$.

Only the centers of distance at most $2t$ affect the shape and
area of the truncated Voronoi cell.  For each $n=0,1,2,\ldots$, we
have a topological space of all truncated Voronoi cells with $n$
nonzero disk centers $v_i$ at distance at most $2t$.  Fix $n$, and
let $X$ be the topological space.

Let $D=D_i$, $i=1,\ldots,n$,  be the sectors lying between
consecutive segments $(0,v_i)$.  Each sector is characterized by
its angle $\alpha$ and the lengths $y_a$ and $y_b$ of the two
segments $(0,v_i)$, $(0,v_j)$ between which the sector lies.  The
part $A$ in $D$ of the area of the truncated Voronoi cell is a
function of the variables $\alpha$, $y_a$, $y_b$.  A nonlinear
implicit equation $\phi=0$ relates $A$, $\alpha$, $y_a$, and $y_b$
on $D$. The variables $u_i$ of the linear assembly problem for the
local domain $D$ are $A$, $y_a$, $y_b$, $\alpha$.


We have a linear assembly problem.  The function $c\cdot x$ is the
area of the truncated Voronoi cell, viewed as a sum of variables
$A$, for each sector $D$ (or rather, their pullbacks to $X$ under
the natural projections $X\to D$).

The assembly constraints are all linear. One linear relation
imposes that the angles of the $n$ different sectors must sum to
$2\pi$. Other linear relations impose that the variable $y_a$ on
$D$ equals the variable $y_b$ on $D'$ if the two variables
represent the length of the same segment $(0,v_i)$ in $X$.
\end{example}


\subsection{solving linear assembly problems}

In this section we describe how various linear assembly problems
are solved in the proof of the Kepler conjecture in terms
sufficiently general to apply to other linear assembly problems as
well.

Let us introduce some general notation.  Let $x_D = (x_i:
\op{dom}(u_i)=D)$ be the vector of variables with local domain
$D$. Write $c\cdot x$ in the form $\sum_D c_D\cdot x_D$ and the
assembly conditions as
$$A \,x =\sum_D A_D x_D,$$
according to the local domain of the variable.


\subsubsection{Linear relaxation}
The first general technique is {\it linear relaxation}. We replace
the nonlinear relations $\phi(x_D)\ge0, \phi\in\Phi_D$ with a
collection of linear inequalities that are true whenever the
constraints $\Phi_D$ are satisfied: $A'_D x_D \le b_D$.  A linear
program is obtained by replacing the nonlinear constraints
$\Phi_D$ with the linear constraints. Its solution dominates the
nonlinear optimization problem.  In this way, the nonlinear
maximization problem can be bounded from above.

Let us review some constructions that insure rigor in linear
programming solutions. We assume general familiarity with the
basic theory and terminology of linear programming. It is
well-known that the primal has a feasible solution iff the dual is
bounded.  We will formulate our linear programs in such a way that
both the primal and the dual problems are feasible and bounded.

We use vector notation to formulate a primal problem as
    \begin{equation}
        \max\, c\cdot x
        \label{cx}
    \end{equation}
such that $A x \le b$, where $x$ is a column vector of free
variables (no positivity constraints), $A$ is a matrix, $c$ is a
row vector, and $b$ is a column vector.

We can insure that this primal problem is bounded by bounding each
of the variables $x_i$.  (This is easily achieved considering the
geometric origins of our problem, which provides interpretations
of variables as particular dihedral angles, edge lengths, and
volumes.) We assume that these bounds form part of the constraints
$A x\le b$.

The linear programs we consider have the property that if the
maximum is less than a constant $K$, the solution does not
interest us.  (For instance, in the dodecahedral conjecture,
Voronoi cell volumes are of interest only if the volume is less
than the volume of the regular dodecahedron.) This observation
allows us to replace the primal problem with one having an
additional variable $t$:
    %%


\subsection{nonlinear duality}
The second general technique is nonlinear duality.  Suppose that
we wish to show that the maximum of the primal problem~\ref{cx} is
at most $M$.

Let $x^* = (x^*_D)$ be a guess of the solution to the problem,
obtained for example, by numerical nonlinear optimization. We
relax the nonlinear optimization by dropping from the matrix $A$
and the vector $b$ those inequalities that are not binding at
$x^*$. With this modification, we may assume that $A\,x^*=b$.  Let
$m$ be the size of the vector $b$, that is, the number of binding
linear conditions. Let $d$ be the number of local domains $D$.

We introduce a linear dual problem with real variables $t$,
$r_\phi: \phi\in\Phi_D$, and $w\in\ring{R}^m$. The variables
$r_\phi$ and $w$ are constrained to be non-negative.

We consider the linear problem of maximizing $t$ such that
    \begin{equation}
        M + d\, t - c\cdot x^* \ge 0
        \label{Mx}
    \end{equation}
and such that for each $x_D$ in each $D$ the linear inequality
    \begin{equation}
        c_D\cdot (x_D-x^*_D) + \sum_{\Phi_D} r_\phi \phi(x) +
                    w A_D (x^*_D-x_D) + t
            < 0
        \label{xD}
    \end{equation}
is satisfied.

There is no guarantee that a feasible solution exists to this
system of inequalities.  However, any feasible solution gives an
upper bound $M$. Indeed, let $x=(x_D)$ be any feasible argument to
the primal, and let $t,r_\phi,w$ be a feasible solution to the
dual. Taking the sum of the linear inequalities~\ref{xD}, over $D$
at $x$, we have (recall $\phi\ge0$ and $A x\le b$):
$$
\begin{array}{lll}
M &\ge M + c\cdot (x-x^*) + \sum_D\sum_{\Phi_D} r_\phi \phi(x)
    + w A (x^*-x) + d\, t,\\
    &\ge c\cdot x + (M + d\, t - c\cdot x^*) + w (b-A x),\\
    &\ge c\cdot x.
\end{array}
$$

Since the dual problem has infinitely many constraints (because of
constraints for each $x\in D$), we solve the dual problem in two
stages. First, we approximate each $D$ by a finite set of test
points, and solve the finitely constrained linear programming
problem for $t, r_\phi$, and $w$.

We replace $t$ with $t_0 = (-M +c\cdot x^*)/d$ (to make the
constraint \ref{Mx} bind).  It follows from the feasibility of $t$
that $t\ge t_0$, and that $t_0,r_\phi,w$ is also feasible on the
finitely constrained problem. To show that $t_0,r_\phi,w$
satisfies all the inequalities~\ref{xD} (under the substitution
$t\mapsto t_0$), we use interval arithmetic to show that each of
these inequalities hold. (To make these interval arithmetic
verifications as easy as possible, we have chosen the solution
$t_0,r,w$ to make the closest inequality hold by as large a margin
$t-t_0$ as possible. This is the meaning of the maximization over
$t$ in the dual problem.)  The next section will give further
details about interval arithmetic verifications.

\subsection{branch and bound}
The third technique is branch and bound.  When no feasible
solution is found in step (2), it may still be possible to
partition $X$ into finitely many sets $X = \coprod X_i$, on which
feasible solutions to the dual may be found.  Although this is an
essential part of the solution, the rules for branching in the
Kepler conjecture follow the structure of that problem, and we do
not give a general branching algorithm.


\section{Definitions and Interpretations}

\begin{definition}
By $\optt{quarter}(\alpha)$ we mean that at dart $\alpha$, we have
an upright quarter.
\end{definition}

\begin{definition}
By $\optt{slice}(\alpha)$ we mean that at dart $\alpha$ we have an upright
diagonal and a slice: $\optt{azim}\slt \pi$ and
the opposite edge length is at most $3.2$ and at least $2$.
\end{definition}

\begin{definition}
By $\optt{gap}(\alpha)$ we mean that at dart $\alpha$, we have an
upright diagonal $\optt{azim}\slt \pi$ and the opposite edge length
is greater than $3.2$.
\end{definition}

\begin{definition}
By $\optt{upright}(\alpha)$ we mean that at $\alpha$ there is an upright
diagonal.
\end{definition}


If $\optt{f}$ is any of the
functions
    $$\optt{vor0},\optt{gamma}, \optt{nu},$$
we set $\optt{tau0}$, $\optt{tau\_gamma}$,
$\optt{tau\_nu}$, respectively,
to
    $$\optt{tau\_*} = -f(\alpha) +\optt{sol}(\alpha)\zeta\pt.$$
We set
    $$
    \optt{tau}(\alpha,t) = -
    \optt{sovo}(S,t,\lambda_{sq})+\optt{sol}(\alpha)\zeta\pt.
    $$
We say that $\alpha$ is compressed or decompressed 
according to the scoring of $\optt{mu}(\alpha)$.  (See
Section~\ref{sec:rules}.)

We  measure what is squandered by a flat quarter by $\hat\tau =
\sol\zeta\pt - \hat\sigma$.

XX Define width $y_4$ of a (geometric) dart,
types fitted crown, type C, enclosed masking dart, masking dart, etc. 


\section{Basic Relations}

\begin{lemma} If $\optt{upright}(\alpha)$ then $\optt{gap}(\alpha)$ or
$\optt{slice}(\alpha)$ or $\optt{azim}(\alpha) \sge \pi$.
\end{lemma}


\begin{lemma}  If $N$ is a node that is upright, then at least
one dart at $N$ is a quarter.
\end{lemma}

\section{Listing of Assembly Problems}

\subsection{Two darts at an upright node}


\begin{lemma}
Let $H$ be a geometric hypermap.  Let $N$ be an upright node
of cardinality $2$.  Then exactly one dart at $N$ is a quarter.
\end{lemma}

\begin{proof}
An upright node has at least one dart that is a quarter.
The dihedral angle of a quarter is
less than\footnote{\calc{971555266}} $\pi$, so it is
impossible for both darts to be quarters.
\end{proof}

\begin{lemma}\label{a:context21} %\label{eqn:4.9}
Let $H$ be a geometric hypermap.  Let $N$ be an upright node
of cardinality $2$.  Assume that exactly one dart $\alpha$ 
at $N$ is a quarter.  Let $\alpha'$ be the other dart at $N$.
Then
 $$
 \optt{mu}(\alpha) + \optt{kappa}(\alpha') < \optt{vor0}(\alpha).
 $$
\end{lemma}

\begin{proof}
This follows from
calculations\footnote{\calc{906566422}, \calc{703457064}, and
\calc{175514843}}.
\end{proof}

\subsection{Three darts at an upright node}
%\section{Three anchors} %DCG 11.3,p.114
    \oldlabel{3.4}


\begin{lemma}\dcg{Lemma~11.2}{114}
    \oldlabel{3.4.1}
Let $H$ be a geometric hypermap. Assume that $N$ is a node
of cardinality three and that a unique dart $\alpha_0$  of $N$
is an upright quarter.  Let $\alpha_1,\alpha_2$ be the other
two darts of $N$.
Let 
  $$\optt{s}(\alpha)=\begin{cases}
  \optt{kappa}(\alpha),&\text{if $\alpha$ fitted crown}\\
  \optt{vor\_anal}(\alpha)-\optt{vor0}(\alpha),&\text{if $\alpha$ type C}\\
  0,&\text{otherwise}
  \end{cases}
  $$
Then 
  $$
  s(\alpha_1) + s(\alpha_2) + \optt{nu}(\alpha_0) < \optt{vor0}(\alpha_0).
  $$
% The upright diagonal can be erased in the context $\x(3,2)$.
\end{lemma}


\begin{proof}
Let $v_1$ and $v_2$ be the two anchors of the upright diagonal $\{0,v\}$
along the quarter. Let the third anchor be $v_3$.

Assume first that $|v|\ge 2.696$. If $Q$ is compressed,
then\footnote{\calc{73974037}} %A10  by $\A_{10}$,
the score is dominated by the truncated
function $\svor_0$.  Assume $Q$ is decompressed. If $|v_1|$,
$|v_2|\le 2.45$, then a calculation\footnote{\calc{764978100}} %A11
gives the result. Take $|v_2|\ge
2.45$.  By symmetry, $|v-v_1|$ or $|v-v_2|\ge 2.45$. The case
$|v-v_1|\ge2.45$ is treated by another calculation.%
\footnote{\calc{764978100}} %A11
We take
$|v-v_2|\ge2.45$. Let $S=\{0,v,v_2,v_3\}$. If $S$ is of type $\SC$,
the result follows.\footnote{\calc{764978100}} %A11
$S$ is of type $\SC$, if and only if $y_4\le 2.77$, (because
by Lemma~\ref{tarski:eta245}, $\eta_{456}>\sqrt2$).
If $S$ is  not of type $\SC$, then by Lemma~\ref{tarski:eta696} and
Lemma~\ref{tarski:eta-rad},
we have $\rad(S) \ge\eta(|v|,2.45,24.5)\ge \eta_0(|v|/2)$.
This justifies the use of $\kappa$ (see Section~\ref{x-2.3}
Case (2)). That the truncated function dominates the score now
follows from a calculation.\footnote{\calc{618205535}} %A9

Now assume that $|v|\le 2.696$. If the simplices $\{0,v,v_1,v_3\}$
and $\{0,v,v_2,v_3\}$ are of type $\SC$, the bound follows from a
calculation.\footnote{\calc{73974037}} %A10
\footnote{\calc{764978100}} %A11
%$\A_{10},\A_{11}$.
If say  $S=\{0,v,v_2,v_3\}$ is not of type $\SC$,
then
    $$\rad(S)\ge\sqrt2>  \eta_0(2.696/2)\ge\eta_0(h),$$
justifying the use of $\kappa$. The bound follows from further
calculations.\footnote{\calc{618205535}} %A9
\footnote{\calc{73974037}} %A10
\footnote{\calc{764978100}} %A11
%    $\A_9,\A_{10},\A_{11}$.
($\Gamma+\kappa <\octavor_0$,
etc.)
\end{proof}


\begin{lemma}\dcg{Lemma~11.3}{115}
    \oldlabel{3.4.2}
    \label{lemma:unerased}
Let $H$ be a geometric hypermap.  Let $N$ be a node
of cardinality three.  Assume that exactly two darts at $N$
are quarters.  Assume that the width of the third dart $\alpha_0$
is at least $2\sqrt2$.  (XX By the rules of Definition~\ref{def:q-system},
this is equivalent to saying that the node is not enclosed over
a masked flat quarter.)
Then 
% The upright diagonal can be erased in the context $\x(3,1)$, provided
% the three anchors do not form a flat quarter at the origin.
\end{lemma}

\begin{proof}
In the absence of a flat quarter, truncate, score, and remove the
vertex $v$ as in the context $\x(3,1)$ of
Lemma~\ref{lemma:mixed-vor0}. 
\end{proof}

\subsection{Six darts at an upright node}
%\section{Six anchors} %DCG 11.4, p.115
    \oldlabel{3.5}

\begin{lemma}\dcg{Lemma~11.4}{115}  
Let $H$ be a geometric hypermap.
Let $N$ be a node of cardinality at least six.
Let the darts at $N$ that are quarters be
$\alpha_1,\ldots,\alpha_k$.
Then 
  $$
  \sum_{i=1}^k\optt{tau\_nu}(\alpha_i) > \squander.
  $$
%An upright diagonal has at most five anchors.
\end{lemma}

\begin{proof}
The proof relies on constants and inequalities from two
calculations.\footnote{\calc{729988292}} %A3
\footnote{\calc{83777706}} %A8
%$\A_3$ and $\A_8$.
If between two anchors there is a quarter, then the angle is
greater than $0.956$, but if there is not,  the angle is greater than
$1.23$.  So if there are $k$ quarters and at least six anchors, they
squander more than
    $$ k (1.01104) - [2\pi-(6-k)1.23]0.78701 > \squander,$$
for $k\ge0$.
\end{proof}

\subsection{Five darts at an upright node}
\label{sec:5updart}

\begin{lemma}\dcg{Sec~11.7,intro}{118}\label{a:5dart:concave}  
Let $H=(D,n,e,f)$ be a geometric
hypermap.  Let $N$ 
be a node of $H$ of cardinality 5.    Assume that for all $\alpha\in N$
  $$
  \optt{upright}(\alpha).
  $$
Then $\optt{azim}(\alpha)\slt \pi$ for all $\alpha\in N$.
Hence $\optt{gap}(\alpha)$ or $\optt{slice}(\alpha)$.
\end{lemma}

\begin{proof}  The angle is at most $2\pi - 4(0.956) < \pi$.
\end{proof}

\begin{lemma}\dcg{Sec~11.7,Rem~11.3}{118}  An upright diagonal with
five darts has at most $2$ gaps.  More precisely, let $H$ be a geometric
hypermap.  Let $N$ 
be a node of $H$ of cardinality 5.   Assume that $\optt{upright}(\alpha)$
for $\alpha\in N$.  There is a set $S\subset N$ of cardinality at most
$2$ such that $\optt{gap}(\alpha) \Rightarrow \alpha\in S$.
\end{lemma}

\begin{proof}
There are at most
two gaps by the calculation\footnote{\calc{83777706}} %A8
    $$3(1.65)+2(0.956)>2\pi.$$
\end{proof}

\begin{lemma}\label{a:5dart:3q}\dcg{Sec~11.7,in Lemma~11.14}{119}
Let $H$ be a geometric hypermap.
Let $N$ be a node of $H$ of  of cardinality $5$.  Assume that
two of the darts at $N$ are gaps.  Then the other three darts
at $N$ are quarters.
\end{lemma}

\begin{proof}
By a calculation,\footnote{\calc{83777706}} %A8 $\A_8$,
the slices are all quarters,
    $1.23+2(1.65)+2(0.956)>2\pi$.
\end{proof}

\begin{lemma}\dcg{Lemma~11.14}{119}  Let $H$ be a geometric
hypermap.  Let $N$ be a node of $H$ of cardinality $5$.  Assume
that $\optt{upright}(\alpha)$ for $\alpha\in N$.  Assume that there
is a set $S\subset N$ of cardinality $3$ such that
  $$\optt{quarter}(\alpha)\Leftrightarrow \alpha\in S.$$
Then 
  $$
  \sum_{\alpha\in S} \optt{tau\_nu}(\alpha) \sgt \squander.
  $$
\end{lemma}

\begin{proof}
The dihedral angle of the quarters combined is less than $2\pi-2(1.65)$.  
The linear programming
bound based on various inequalities\footnote{\calc{729988292}} %A3 $\A_3$
is greater than $0.859>\squander$.
\end{proof}



\begin{lemma}
    \oldlabel{3.8.3}\label{lemma:4-crowdedq}\dcg{Lemma~11.18}{119}
Let $H$ be a geometric hypergraph.
Suppose a node of cardinality five is an upright diagonal.
Suppose that exactly one of the five darts is a gap.
 If
any of the four slices is not an upright quarter then
the centered packing does not contravene.
\end{lemma}

\begin{proof}
We use a series of inequalities.\footnote{\calc{628964355}} %A5
\footnote{\calc{187932932}} %A7
\end{proof}

\begin{lemma}
    \oldlabel{3.8.3}\label{lemma:4-crowded}\dcg{Lemma~11.18}{119}
Let $H$ be a geometric hypermap.
Suppose a node of cardinality five is an upright diagonal.
Suppose that exactly one of the five darts is a gap.
The sum of $\optt{nu}$ over the four slices is at most
$-0.25$. The sum of $\optt{tau\_nu}$ over the
four slices is at least $0.4$.
\end{lemma}

\begin{proof}
A list of inequalities\footnote{\calc{815492935}} %A2 $\A_2$
together with\footnote{\calc{83777706}} %A8
$\dih>1.65$ give the bound $-0.25$.
Further inequalities \footnote{\calc{729988292}} %A3 $\A_3$
give the bound $0.4$.  
\end{proof}


\begin{lemma}\dcg{Cor~11.9}{120}  
Let $H$ be a geometric hypermap.
Suupose that the node $N$ is a $4$-crowded upright
diagonal.  Let $\alpha_1,\ldots,\alpha_4$ be the quarters
and let $\alpha_0$ be a gap at the node $N$.  Then
  $$
  \optt{kappa}(\alpha_0) + \sum_{i=1}^4 \optt{tau\_nu}(\alpha_i)
  > 0.42274.
  $$
\end{lemma}

\begin{proof}  The crown along the gap,
with the bound of Lemma~\ref{lemma:4-crowded}, 
gives\footnote{\calc{618205535}} %A9
    $0.4-\kappa \ge 0.4+0.02274$
squandered by the upright quarters around a $4$-crowded upright
diagonal.
\end{proof}


\begin{lemma} \label{a:min0-vor} 
Let $H$ be a geometric hypermap.
Let $N$ be a node of degree three.
Assume that darts $\alpha_1$ and $\alpha_2$ are quarters,
and that $\alpha_0$ is a dart with azimuth angle at most $\pi$ 
and width less than $2\sqrt2$.
Then $$\optt{vu}(\alpha_1) + \optt{nu}(\alpha_2) <
     \optt{vor0}(\alpha_1) + \optt{vor0}(\alpha_2).$$
\end{lemma}

\begin{proof}
By a calculation\footnote{\calc{855677395}}, if $|v|\ge 2.69$,
then the upright quarters satisfy
    $$\nu < \svor_0 + 0.01 (\pi/2-\dih)$$
so the upright quarters can be erased.  Thus we assume without
loss of generality that $|v|\le 2.69$.

By Lemma~\ref{tarski:E:part4:2}, we have $|v|>2.6$.
If $|v_1-v_2|\le 2.1$,  or $|v_1-v_3|\le 2.1$, then
Lemma~\ref{tarski:E:part4:3}, gives $|v|>2.72$, 
 contrary to assumption.  So take $|v_1-v_2|\ge 2.1$ and
$|v_1-v_3|\ge2.1$. Under these conditions we have the interval
calculation\footnote{\calc{148776243}} %A13
  $\nu(Q) < \svor_0(Q)$ where $Q$ is the upright quarter.
\end{proof}


\subsection{Four darts at an upright node}


\begin{lemma}\dcg{Remark~11.28}{125}
\label{remark:3rd-quarter} Let $H$ be a geometric hypermap.
Let $N$ be an upright node of degree four at which there are
exactly three darts $\alpha_1$, $\alpha_2$, $\alpha_3$
that are upright quarters.  Assume that the node is enclosed
over a masked flat quarter.  
Then
 $$
 \sum_{i=1}^3\optt{vu}(\alpha_i) <
     \sum_{i=1}^3\optt{vor0}(\alpha_i) + \xiV.
 $$
\end{lemma}

\begin{proof}
 If we have an upright diagonal enclosed
over a masked flat quarter in the context $\x(4,1)$, then there are
three upright quarters.  By the same argument as in Lemma~\ref{a:min0-vor}, 
the two quarters over the masked flat quarter score $\le\svor_0$. The
third quarter is dominated by $\svor_0 + \xiV$.
\end{proof}


\begin{lemma}[Erasing four darts, no masked]
\dcg{Lemma~11.21}{120}
\oldlabel{3.9.1}
Let $H$ be a geometric hypermap.  Let $N$ be a node
whose darts are upright.  Assume that $N$ has cardinality four.
Assume that there are at least as many non-slices as quarters at $N$.
Let $f(\alpha)$ be given as $\optt{nu}$ if $\alpha$ is an upright
quarter, $\optt{vor0}$ if it is another slice, and
$\optt{kappa}$ at darts that are not slices.  Then
  $$
  \sum_{\alpha\in N} f(\alpha) < \sum_{\alpha\in N} \optt{vor0}(\alpha).
  $$
\end{lemma}

\begin{proof}
By assumption, there are at least as non-slices as upright quarters. Each
non-slice drops us by $\xik$ and each quarter lifts us by at most%
\footnote{\calc{618205535}} %A9
\footnote{\calc{73974037}} %A10
\footnote{\calc{764978100}} %A11
$\xiG$. We have $\xikG<0$.
\end{proof}

\begin{lemma}[Erasing four darts, masked]\dcg{After Rem~11.22}{120}
Let $H$ be a geometric hypermap.  Let $N$ be a node
whose darts are upright.  Assume that $N$ has cardinality four.
Assume that ``$N$ is enclosed over a flat quarter'' at dart $\beta$.
Assume that there are at least as many non-slices as quarters at $N$.
Let $f(\alpha)$ be given as $\optt{nu}$ if $\alpha$ is an upright
quarter, $\optt{vor0}$ if it is another slice, and
$\optt{kappa}$ at darts that are not slices.  Then
  $$
  \sum_{\alpha\in N} f(\alpha) < -0.0114 + 
  \sum_{\alpha\in N} \optt{vor0}(\alpha).
  $$
\end{lemma}

\begin{proof}
The azimuth angle at a non-slice is $>1.65$. 
We have
$0.0114< -2\xikG$.
XX This proof seems incomplete.  Don't we need $\optt{nu}(\alpha) <
\optt{vor0}$ based on something like DCG-Remark~11.22?
\end{proof}

XX We remark that the preceding lemma is designed for use with 
the scoring of DCG-page123-2(c).




\begin{lemma}\dcg{Lemma~11.23}{121}
    \oldlabel{3.9.2}
    \label{lemma:0.008}
Let $H$ be a geometric hypermap.  Let
Let $N$ be a node that is an upright diagonal with four darts.  
Assume that one of the darts $\alpha_0$ is a gap and that the other
three $\alpha_1,\alpha_2,\alpha_3$ are slices.  
\begin{itemize}
\item If all of the slices are upright quarters, then
  $$
  \optt{kappa}(\alpha_0) + \sum_{i=1}^3 \optt{nu}(\alpha_i) <
  \sum_{i=1}^3 \optt{vor0}(\alpha_i) + 0.008.
  $$
%The slices can be erased with penalty  $\pi_0=0.008$. 
\item Assume that one of the slices is not an upright quarter.
Let 
$\optt{s}(\alpha)$ equal $\optt{nu}$ at slices that
are quarters and $\optt{vor0}$ at slices that are not.
Then
  $$
  \optt{kappa}(\alpha_0) + \sum_{i=1}^3 \optt{s}(\alpha) <
  \sum_{i=1}^3 \optt{vor0}(\alpha_i) + 0.00222.
  $$
% we can erase with penalty $\pi_0=0.00222$.
\end{itemize}
\end{lemma}


\begin{proof}
The constants and inequalities used in this proof can be found in
a series of calculations.%
\footnote{\calc{618205535}} %A9
\footnote{\calc{73974037}} %A10
\footnote{\calc{764978100}} %A11


First we establish the penalty $0.008$.   
By these inequalities, the result follows
if the diagonal satisfies $y_1\ge 2.57$.

Take $y_1\le 2.57$. If any of the upright quarters are decompressed, 
the result follows from $(\xikG+\xiG<0.008)$. If the edges
along the gap are less than $2.25$, the result follows from
$(-0.03883+3\xiG = 0.008)$. If all but one edge along the
gap are  less than 2.25, the result follows from $(-0.0325 + 2\xiG
+ 0.00928 = 0.008)$.

If there are at least two edges along the gap of length at
least $2.25$, we consider two cases according to whether they lie
on a common face of an upright quarter.  The same group of
inequalities gives the result. The bound $0.008$ is now fully
established.

\smallskip
Next we prove the bound involving $0.00222$, when one
of the slices is not a quarter.  If $|v|\ge2.57$, then
we use
    $$2\xiG + \xiV + \xik \le 0.00935+0.003521 -0.2274\le 0.$$
If $|v|\le2.57$, we use
    $$2(0.01561)-0.029 \le 0.00222.$$
\end{proof}


\begin{lemma}\dcg{Lemma~11.23}{121}
Let $H$ be a geometric hypermap.  Let
Let $N$ be a node that is an upright diagonal with four darts.  
Assume that one of the darts $\alpha_0$ is a gap and that the other
three $\alpha_1,\alpha_2,\alpha_3$ are slices.  
Let 
$\optt{s}(\alpha)$ equal $\optt{nu}$ at slices that
are quarters and $\optt{vor0}$ at slices that are not.
Assume 
some upright quarter along this diagonal masks a flat quarter.
Then (1) or (2) holds.
   \begin{enumerate}
    \item 
  $$
  \optt{kappa}(\alpha_0) + \sum_{i=1}^3 \optt{s}(\alpha) <
  \sum_{i=1}^3 \optt{vor0}(\alpha_i) -0.0063.
  $$
  The diagonal of the flat is at least $2.6$, and the edge
    opposite the diagonal is at least $2.2$.
    \item 
    $$
  \optt{kappa}(\alpha_0) + \sum_{i=1}^3 \optt{s}(\alpha) <
  \sum_{i=1}^3 \optt{vor0}(\alpha_i) -0.0114.
  $$
   The diagonal of the flat is at least $2.7$, and the edge
    opposite the diagonal is at most $2.2$.
    \end{enumerate}
\end{lemma}



\begin{proof}
\smallskip
Let $v_1\ldots,v_4$ be the consecutive anchors of
the upright diagonal $\{0,v\}$ with $\{v_1,v_4\}$ the gap.
Suppose $|v_1-v_3|\le 2\sqrt{2}$.

By Lemma~\ref{tarski:dcg-p122}, 
the upright diagonal $\{0,v\}$ is not enclosed over
$\{0,v_1,v_2,v_3\}$.   
Thus, $\op{conv}^0\{v_1,v_3\}$ meets $\op{conv}\{0,v,v_2\}$ so that the
simplices $\{0,v,v_1,v_2\}$
and $\{0,v,v_2,v_3\}$ are decompressed.

To complete the proof of the lemma, we show that when
some upright quarter along this diagonal masks a flat quarter, 
 either (1) or (2) holds.
Suppose we mask a flat quarter $Q'=\{0,v_1,v_2,v_3\}$.
We have established that $\op{conv}^0\{v_1,v_3\}$ meets 
$\op{conv}\{0,v,v_2\}$.
To establish (1) assume that $|v_2|\ge 2.2$.  Lemma~\ref{remark:2.6} 
gives
    $$|v_1-v_3|>2.6.$$
The bound $0.0063$ comes from
    $$\xikG + 2\xiV < -0.0063$$

To establish (2) assume that $|v_2|\le 2.2$. Lemma~\ref{remark:2.6} gives
    $$|v_1-v_3|>2.7.$$
  If the simplex
$\{0,v,v_3,v_4\}$ is decompressed, then $$\xik + 3\xiV  < -0.0114$$
Assume that $\{0,v,v_3,v_4\}$ is compressed. We have
    $$-0.004131 +\xikG + \xiV \le -0.0114.$$
\end{proof}

\subsection{Flat quarters} %

XX The following is a direct application of interval arithmetic.
Why repeat it here?


\begin{lemma}\dcg{Lemma~11.29}{125}
    \oldlabel{3.11.3}
$\mu < \svor_0 +0.0268$ for all flat quarters. If the central
vertex has height $\le2.17$, then $\mu<\svor_0+0.02$.
\end{lemma}

\begin{proof}
This is an interval calculation.\footnote{\calc{148776243}} %A13
\end{proof}




\begin{lemma}\dcg{Lemma~11.30}{125}\label{lemma:1.32}
    \oldlabel{3.11.4}
Let $H$ be a geometric hypermap.  Let $\alpha$ be a dart that
is standard (XX meaning not upright and edges of length $2$ to $2.51$).
Then the width of $\alpha$ is less than $2\sqrt2$.
\end{lemma}


\begin{proof} Let $S=S(y_1,\ldots,y_6)$ be the simplex inside the exceptional
cluster centered at $v$, with $y_1=|v|$. The inequality $\dih\le 1.32$
gives the interval calculation $y_4< 2\sqrt{2}$., so $S$ is a quarter.
\end{proof}


XX The following is a direct application of an interval
arithmetic calculation.  Why repeat it here?

\begin{lemma}
Let $v$ be a corner of a flat quarter at which the
dihedral angle is at most $1.32$. 
Then $\hat\tau(Q)>3.07\,\pt$. Moreover, if $\hat\sigma=\svor_0$ and if
$\eta_{456}\ge\sqrt2$, 
we may use the stronger constant
$\tau_0(Q)> 3.07\,\pt+\xi_V+2\xiG'$.
\end{lemma}


\begin{proof}
The result follows by
interval arithmetic.\footnote{\calc{148776243}} %A13
\end{proof}

\subsection{Tame Plane Graphs}


\begin{lemma}\label{a:6}\dcg{Lemma~21.4}{223} 
Formally contravening hypermaps satisfy Property
\ref{definition:tame:degree} of tameness: The cardinality of every
node is at least $2$ and at most $6$.
\end{lemma}

\begin{proof}
Let the type of the node be $(p,q,r)$.  If $r=0$, then the
impossibility of a node of cardinality $7$ or more is found in the
table entry $b(7,0)$ (Lemma~\ref{lemma:pq-types:bis}). If $r\ge1$,
then Lemma~\ref{lemma:0.8638} shows that the azimuth angles of the
darts at the node cannot sum to $2\pi$:
    $$6 (0.8638) + 1.153 > 2\pi.$$
\end{proof}




\subsection{Computer Calculations and Their Consequences}
\label{sec:ccc}

We have the following linear program. There are many different
choices of objective function and constraint {\it Csum} depending on
the particular constants $\sLP$, $\tauLP$, or $\tlp/\pt$ that need
to be computed.  In the linear program the constants $\pi$ and $\pt$
are replaced by numerical approximations.  Section~\ref{XX} explains
how the output from the numerical routines can be adjusted to yield
perfectly rigorous results.  The listing is in a format that can be
read by the program {\it LPSolve}.  See \cite{lpsolve}.

The origin of these inequalities is interval arithmetic.  They are
listed in nonlinear form at \cite{XX}.  The numeric labels of the
equations here is consistent with the labels in that archive.

The correspondence between linear program variables in the program
listing and the variables in use elsewhere in the book is the
following.  Here $F$ is a face, and $x$ is a dart in that face.
    $$
    \begin{array}{lll}
    \card(F)=3 &\Rightarrow\  \azim(x)=\azim_3,\ \tau(F)=\op{tau}_3,\ \sigma(F)=\op{sigma}_3\\
    \card(F)=4 &\Rightarrow\  \azim(x)=\azim_4,\ \tau(F)=\op{tau}_4,\ \sigma(F)=\op{sigma}_4\\
    \end{array}
    $$

{ \obeylines\tt
  \hbox{}\parindent=4pt

 /* Change  "min/max" and "Csum", according to the objective */
 \ \hbox{}
 /* This example computes b(2,2) */
 // min: 2 tau3\_s + 2 tau4\_s;
 // Csum: 2 azim3 + 2 azim4 - twopi = 0;
 \ \hbox{}
 /* This example computes tauLP(2,2,5.0) */
 //min: 2 tau3 + 2 tau4;
 //Csum: 2 azim3 + 2 tau4 <= 5.0;
 \ \hbox{}
 /* This example computes sigmaLP(5,0,2pi-1.153) */
 max: 5 sigma3 + 0 sigma4;
 Csum: 5 azim3 + 0 sigma4 - twopi <= -1.153;
 \ \hbox{}
 /* Variable bounds */
 twopi: twopi =  6.2831853071795862;
 \ \hbox{}
 // pt = 0.055373645668464144;
 Ctaup: 0.055373645668464144 tau3\_s - tau3 = 0;
 Ctauq: 0.055373645668464144 tau4\_s - tau4 = 0;
 \ \hbox{}
 /* assumed conditions: */
 /* triangle tau */
 J927432550: 0.3897 azim3 + tau3 > 0.4666;
 J221945658: 0.2993 azim3 + tau3 > 0.3683;
 J53415898:  tau3 > 0.0;
 J106537269: -0.1689 azim3 + tau3 > -0.208;
 J254527291: -0.2529 azim3 + tau3 > -0.3442;
 \ \hbox{}
 /* triangle sigma */
 J539256862: sigma3 - 0.37898 azim3 < -0.4111;
 J864218323: sigma3 + 0.142 azim3 < 0.23021;
 Jsigma\_1pt: -Infinity <= sigma3 <= 1.0;
 J776305271: sigma3 + 0.3302 azim3 < 0.5353;
 \ \hbox{}
 /* quad  tau */
 J539320075: 4.49461 azim4 + tau4 > 5.81446 ;
 J122375455: 2.1406 azim4 + tau4 > 2.955;
 J408478278: 0.316 azim4 + tau4 > 0.6438;
 J996268658: tau4 > 0.1317;
 J393682353: -0.2365 azim4 + tau4 > -0.3825;
 J775642319: -0.4747 azim4 + tau4 > -1.071;
 \ \hbox{}
 /* quad sigma */
 J310151857: sigma4 - 4.56766 azim4 < -5.7906;
 J655029773: sigma4 - 1.5094 azim4 < -2.0749;
 J\_73283761:  sigma4 - 0.5301 azim4 < -0.8341;
 JLemm14\_11: -Infinity <= sigma4 <= 0;
 J\_15141595:  sigma4 - 0.3878 azim4 < -0.6284;
 J574391221: sigma4 + 0.1897 azim4 < 0.4124;
 J396281725: sigma4 + 0.5905 azim4 < 1.5707;
 \ \hbox{}

 /* all vars have lower bound 0, except sigma3, sigma4  */


}

\bigskip

We let $\tauLP(p,q,\alpha)$ denote the solution to this linear
program with objective $$\min: p\, \tau_3 + q\, \tau_4$$ and
constraint
$$\op{Csum}: p\, \azim_3 + q\,\azim_4 \le d.$$

We let $\tlp(p,q)$ denote the solution to this linear program with
objective $$\min: p \tau_3 + q \tau_4$$ and constraint
$$\op{Csum}: p\, \azim_3 + q\,\azim_4 =2\pi.$$  The constants $b(p,q)$
are computed as lower bounds satisfying $\tlp(p,q) > b(p,q)\,\pt$,
which the exception of the constants $b(5,0)$ and $b(7,0)$, which
are slight improvements on the linear programs.

We let $\sLP(p,q,\alpha)$ denote the solution to this linear program
with objective $$\max: p\, \sigma_3 + q\, \sigma_4$$ and constraint
$$\op{Csum}: p\, \azim_3 + q\,\azim_4 \le d.$$



\begin{lemma} We have the following estimates:
    $$
    \begin{array}{lll}
    &s_5+\sLP(5,0,2\pi-1.153)< c(8)\,\pt\\
    &s_6+\sLP(5,0,2\pi-1.153) < s_9\\
    &s_5+\sLP(5,0,2\pi-1.153)<s_8\\
    &(9-2(0.48))\,\pt+s_5+\sLP(2,2,2\pi-1.153)<8\,\pt\\
    &2t_5+\tauLP(4,0,2\pi-2(1.153))>\squander\\
    \end{array}
    $$
\end{lemma}

\begin{proof} Run the linear programs and see what you get.
\end{proof}

These are just a few of a long list of inequalities such as these
that will appear in the pages that follow.  They all come from the
same basic linear program with varying objective function and angle
sum constraint.



\begin{lemma}  If $v$  is a node of an exceptional face,
and if there are $6$ faces meeting at $v$, then the exceptional face
is a pentagon and the other $5$ faces are triangles.  In particular,
the node has type $(5,0,1)$.
\end{lemma}

\begin{proof}  Let $(p,q,r)$ be the type of the node.  We consider
several cases, according to the value of $p$.

{\bf($p\le2$)} If there are at least four non-triangular regions at
the node, then the sum of azimuth angles around the node is at least
$4(1.153)+2(0.8638)>2\pi$, which is impossible.  (See
Lemma~\ref{lemma:0.8638}.)

{\bf($p=3$)} If there are three non-triangular regions at the node,
then $\tau^*(H)$ is at least
$2t_4+t_5+\tauLP(3,0,2\pi-3(1.153))>\squander$.

{\bf($p=4$)} If there are two exceptional regions at the node, then
$\tau^*(H)$ is at least $2t_5+\tauLP(4,0,2\pi-2(1.153))>\squander$.

If there are two non-triangular regions at the node, then
$\tau^*(H)$ is at least  $t_5+\tauLP(4,1,2\pi-1.153)>\squander$.

{\bf($p=5$)} We are left with the case of five triangles and one
exceptional face.

When there is an exceptional face at a node of cardinality six, we
claim that the exceptional face must be a pentagon. If the face is a
heptagon or more, then $\tau^*(H)$ is at least
$t_7+\tauLP(5,0,2\pi-1.153) > \squander$.

If the face is a hexagon, then $\tau^*(H)$ is at least $t_6 +
\tauLP(5,0,2\pi-1.153) > t_9$. Also, $s_6+\sLP(5,0,2\pi-1.153) <
s_9$. The contour loop around the six faces has at most $9$ face
steps. Lemma~\ref{lemma:s9-t9:bis} gives the bound of $8\,\pt$.
\end{proof}




\begin{lemma}
    \label{lemma:aggregate6}
    Let $(H,\azim,\flat,\sigma)$ be formally contravening.
    \begin{enumerate}
    \item The aggregate $F$ of the six faces at a node of type
    $(5,0,1)$ satisfies
            $$
            \begin{array}{lll}
            \sigma(F) < s_8,\\
            \tau(F) > t_8.
            \end{array}
            $$
    \item There are at most two nodes of type $(5,0,1)$.  If
        there are two, then they are non-adjacent vertices on a
        pentagon, as shown in Figure \ref{fig:doubledegree6}.  (The
        pentagon has a node of type $(1,0,1)$.)
    \end{enumerate}
\end{lemma}
\begin{figure}[htb]
  \centering
  \myincludegraphics{\ps/doubledegree6.eps}
  \caption{Non-adjacent nodes of cardinality $6$ on a pentagon}
  \label{fig:doubledegree6}
\end{figure}

\begin{proof}
We begin with the first part of the lemma. The sum  $\tau(F)$ over
these six standard regions is at least
    $$t_5+\tauLP(5,0,2\pi-1.153)> t_8.$$
Similarly,
    $$s_5+\sLP(5,0,2\pi-1.153)<s_8.$$
%
We note that there can be at most one exceptional face with a node
of cardinality six.  Indeed, if there are two, then they must both
be nodes of the same pentagon:
    $$t_8+t_5>\squander.$$
Such a second node on the octagonal aggregate leads to one of the
follow constants greater than $\squander$.  These same constants
show that such a second node on a hexagonal aggregate must share two
triangular faces with the first node of cardinality six.
$$\begin{array}{lll}
    t_8 &+\tauLP(4,0,2\pi-1.32-0.8638),\quad\text{or}\\
    t_8 &+1.47\,\pt+\tauLP(4,0,2\pi-1.153-0.8638),\quad\text{or}\\
    t_8 &+\tauLP(5,0,2\pi-1.153) .
\end{array}
$$
(The relevant constants are found at Lemma~\ref{lemma:1.47} and
Lemma~\ref{lemma:0.8638}.)
\end{proof}

\begin{lemma}\label{a:311}  % Used in separation props of tame graphs.
Assume the node has type $(3,1,1)$.
 Assume the azimuth angle of the dart on the exceptional face is
least $1.32$, then
    \begin{equation}
    \tauLP(3,1,2\pi-1.32)>1.4\,\pt + t_4.
    \label{eqn:tau1.32}
    \end{equation}
This gives the bound in the sense of Lemma~\ref{lemma:split} at such
a node. 
\end{lemma}

\begin{lemma}\label{a:no-ef}  Let $v$ be a node of type
$(p,q,r)=(4,0,1)$, $(3,1,1)$, or $(3,0,2)$.  Assume that at $v$ the
exceptional darts are not flat.  Then
    $$\tauLP(p,q,\alpha) > ( p d(3) + q d(4) + a(p))\,\pt.$$
\end{lemma}

\begin{proof}
By Lemma~\ref{lemma:1.32:bis},
the azimuth angles of the
exceptional regions at $v$ are at least $1.32$.   The conclusion%
\footnote{\calc{551665569}, \calc{824762926}, and
\calc{325738864}}
%% K.C.-2002-version: 17.20 Group 20, 17.21 Group 21. (page 49).
follows.
\end{proof}

