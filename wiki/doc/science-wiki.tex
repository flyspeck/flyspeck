\section{A Semantic Wiki for Science}
\label{sec:science}

\begin{wrapfigure}{r}{4.2cm}
  \centering
  \vspace{-.9cm}
  \begin{tikzpicture}
    \node (s) at (0,0) {\includegraphics[width=4cm]{images/creativity-spiral}};
    \node at (s.south) {\scriptsize (B.\ Buchberger, 1995)};
  \end{tikzpicture}
  \vspace{-1.2cm}
\end{wrapfigure}
Documents are the most important medium in science.
%% , if we assume a broad definition of
%% ``document'', including any materialized item of (scientific) knowledge.  
Scientific
communication consists mainly of exchanging documents: from informal drafts circulating
inside a working group to published, well-structured books.  A great deal of scientific
work consists of collaboratively authoring these documents. 
Common instances are taking down first hypotheses,
commenting on results of experiments or project steps, and structuring, annotating,
or re-organizing existing items of knowledge.  Tools that
``understand'' the knowledge contained in scientific documents are desirable for
editing such documents.  For example, we can  write scientific documents in a
\textit{semantic markup language} with an editor that knows the structures available in this
language. 

Besides generic approaches like SALT\cite{Groza:SALT07}, the most extensive work
in semantic markup has been in the domain of mathematics.  This is natural.
Mathematics has a ``long tradition in the pursuit of conceptual clarity and
representational rigor''\cite{Kohlhase:omdoc1.2}\ednote{FR: this quote doesn't seem quote-worthy}.  Languages like
MathML\cite{CarlisleEd:MathML07}, OpenMath\cite{BusCapCar:2oms04}, and
OMDoc\cite{Kohlhase:omdoc1.2} were developed to represent the clearly defined
and hierarchical structures of mathematics in a way that preserves the intricate
relationships.  In this tradition, OMDoc is a language that employs Content
MathML or OpenMath for structurally representing mathematical \emph{objects}
(symbols, numbers, equations, etc.).  OMDoc adds two layers.  Objects or
informal text can be annotated as mathematical \emph{statements} (symbol
declarations, definitions, axioms, theorems, proofs, examples, etc.), and
collections of interrelated statements can be grouped into \emph{theories}.

With SWiM, a semantic wiki for mathematical knowledge management (see
section~\ref{sec:swim}\ednote{FR:refer to paper instead of section}), we have investigated collaborative editing of OMDoc
documents.  It has turned out that a wiki is a suitable tool for supporting the
workflow of incremental formalization inherent to scientific writing.  But wikis
have not only shown to be appropriate for \emph{writing}, but there are also
many success stories from project management, e.\,g.\ in corporate
settings\cite{leuf01:wikiway,wikinomics}.  We are therefore interested in applying
our technologies to scientific knowledge engineering projects.

%%% Local Variables: 
%%% mode: latex
%%% TeX-master: "flyspeck-wiki-eswc08"
%%% End: 
