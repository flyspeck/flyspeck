% (c) Christoph Lange 2007
\documentclass{llncs}

% Draft?
\newif\ifdraft
\drafttrue
%\draftfalse

% \usepackage[english]{babel}
\usepackage[T1]{fontenc}
\usepackage[utf8]{inputenc}
%\usepackage{lmodern}
\usepackage{textcomp}

\ifdraft
\usepackage[show]{ed}
%\usepackage{pdfsync}
\else
\usepackage[hide]{ed}
%\usepackage{microtype}
\fi


% \usepackage{a4wide}
\usepackage{amsmath}
\usepackage{amsfonts}
\usepackage{amstext}
% \usepackage{array}
% \usepackage{graphicx}
% \usepackage{ifthen}
\usepackage{listings}
% \usepackage{lstpatch}
% \usepackage{lstomdoc}
% \usepackage{makeidx}
% \usepackage{scrpage2}
% \usepackage[binary,squaren]{SIunits}
% \usepackage{supertabular}
% \usepackage{tabularx}
% \usepackage{thm2e}
% \usepackage[normalem]{ulem}
\usepackage{wrapfig}
% \usepackage[svgnames]{xcolor}

\usepackage{tikz}

% % Symbol fonts
% \let\RealRightarrow=\Rightarrow
% \usepackage{marvosym}
% \renewcommand{\Rightarrow}{\RealRightarrow}
% \usepackage{wasysym}

% KWARC packages
\usepackage{acronyms,myindex,semantic-markup}
% \let\Realstex=\stex
% \usepackage{paths}
% \renewcommand{\stex}{\Realstex}

% ... and adjustments
\def\omdocni{{\sc OMDoc}} % non-indexed OMDoc
\def\swimni{{\sc SWiM}} % non-indexed SWiM

\hyphenation{name-space}
\hyphenation{Me-dia-Wi-ki}

% Local abbreviations
% \def\abSMW{\product{Semantic MediaWiki}}

% TikZ setup
\usetikzlibrary{arrows}
\tikzstyle{default}=[font=\sffamily,>=triangle 60]
\tikzstyle concept=[font=\sffamily\bfseries,draw,minimum height=3.5ex,rounded corners]

% % Page styles
% \pagestyle{scrheadings}
% \clearscrheadfoot
% \ohead{\headmark}
% \ofoot[\pagemark]{\pagemark}
% \setheadsepline{0.3pt}[\color{gray}]
% \setkomafont{pagehead}{\normalfont\small\sffamily\slshape}
% \setkomafont{pagenumber}{\normalfont\small\sffamily\slshape}
%
% % Listing styles
\lstset{float=htb,columns=flexible,frame=lines,basicstyle=\footnotesize\ttfamily,
        showstringspaces=false,basewidth=.5em}
%
% % Array setup
% \newcolumntype{v}[1]{>{\raggedright\arraybackslash\hspace{0pt}}p{#1}}

\def\thetitle{Flyspeck in a Semantic Wiki -- Collaborating on a Large Scale
Formalization of the Kepler Conjecture}

% load this last
% \definecolor{NavyBlue}{cmyk}{0.94,0.54,0,0.3}
% \usepackage[pdftex,pdfstartview=FitV,plainpages=false,pdfpagelabels,colorlinks=true,linkcolor=NavyBlue,citecolor=NavyBlue,urlcolor=NavyBlue,hypertexnames=true]{hyperref}
% \hypersetup{
%     pdfauthor = {Christoph Lange},
%     pdftitle = {\thetitle},
%     pdfkeywords = {Semantic Wiki OMDoc Ontology Services Science Mathematical
% Knowledge Management Mathematics}
% }
\usepackage{url}

% \hypersetup{bookmarksdepth=4}

\title{\thetitle}
\author{Christoph Lange\inst{1} \and Sean McLaughlin\inst{2} \and Florian Rabe\inst{3}}
\institute{Computer Science, Jacobs University Bremen\thanks{formerly
International University Bremen}, \email{\{ch.lange,f.rabe\}@jacobs-university.de} \and
School of Computer Science, Carnegie Mellon University, Pittsburgh, \email{seanmcl@gmail.com}}

\begin{document}

\maketitle

\begin{abstract}
  Semantic wikis have been successfully applied to many problems in knowledge management
  and collaborative authoring.  They are particularly appropriate for scientific and
  mathematical collaboration.  In previous work we described the \textit{OMDoc} semantic
  markup language, and an ontology for mathematical knowledge, and a semantic wiki based
  on both.  We are now evaluating these technologies in concrete application scenarios.

  In this paper we evaluate the applicability of our infrastructure to mathematical
  knowledge management by focusing on \textit{the Flyspeck project}\ednote{@Sean, there
    was a leftover cite{Flyspeck:definition}, which I'd like to move somewhere
    else. --CL}, a formalization of Thomas Hales' proof of the Kepler Conjecture.  The
  Flyspeck project is an ideal target for the semantic wiki framework: It is a relatively
  large project built upon a large number of smaller subprojects that must be organized
  and easily accessible by a diverse audience.

  After describing the Flyspeck project and its requirements in detail, we evaluate the
  applicability of a prototype based on Semantic MediaWiki and of our mathematics-specific
  semantic wiki SWiM to Flyspeck.  Finally, we establish a roadmap on how to improve SWiM
  to better meet the needs of the Flyspeck collaborators.
\end{abstract}

%---------------------------------  Intro: Wiki for Science  -------------------

\section{Scientific Communication and the Flyspeck Project}\ednote{ESWC: Bascially, you shouldn't presume what you hope to prove. I think lots of your requirements and proposed solutions are very interesting, but they are independent of the realizing technology. In particular, I don't see the choice of using a wiki is any more significant that the choice to use Ruby or Python to implement the system. There was no experiment, as far as I could tell, with using arbitrary users, so there's little evidence that participants in the flyspeck project will find this  system so very useful. In particular, it's unclear to me that it will recruit significantly more participants.}
\label{sec:science-flyspeck}

\begin{wrapfigure}{r}{4.7cm}
  \centering
  \vspace{-1.0cm}
  \begin{tikzpicture}
    \node (s) at (0,0) {\includegraphics[width=4.5cm]{images/creativity-spiral}};
    \node at (s.south) {\scriptsize (B.\ Buchberger, 1995)};
  \end{tikzpicture}
  \vspace{-1.5cm}
\end{wrapfigure}
%% Documents are the most important medium in science.
%% , if we assume a broad definition of
%% ``document'', including any materialized item of (scientific) knowledge.  
\begin{motivation}
\claim{Scientific communication consists mainly of exchanging documents, and a
%% : from informal drafts circulating inside a working group to published, well-structured books.
%% A
great deal of scientific work consists of collaboratively authoring them.}  Common instances are writing down first hypotheses, commenting on
results of experiments or project steps, and structuring, annotating, or
re-organizing existing items of knowledge, as depicted in Buchberger's figure on
the right.  \emph{Semantic markup languages} for
representing structures of scientific knowledge, and editing tools understanding
them, are a promising approach to supporting this work.\end{motivation}  \begin{background}Besides generic approaches like SALT~\cite{Groza:SALT07}, the most extensive
work in semantic markup has been in the domain of mathematics.  Mathematical
logic, depending on symbols and relationships between symbols, naturally lends
itself well to formal exposition.
%% Mathematics has a ``long tradition in
%% the pursuit of conceptual clarity and representational
%% rigor''~\cite{Kohlhase:omdoc1.2}
%% \ednote{FR: this quote doesn't seem
%%   quote-worthy}  
Languages like MathML~\cite{CarlisleEd:MathML07},
OpenMath~\cite{BusCapCar:2oms04}, and OMDoc~\cite{Kohlhase:omdoc1.2} were
developed to represent the clearly defined and hierarchical structures of
mathematics in a way that preserves the intricate relationships.  OMDoc employs
Content MathML or OpenMath for structurally representing mathematical
\emph{objects} (symbols, numbers, equations, etc.) and adds two layers on top:
Objects or informal text can be annotated as mathematical \emph{statements}
(symbol declarations, definitions, axioms, theorems, proofs, examples, etc.),
and collections of interrelated statements can be grouped into \emph{theories}.\end{background}

\begin{background}With SWiM, a semantic wiki for mathematical knowledge
management~\cite{lange:swim-demo08}, we have investigated collaborative editing
of OMDoc documents.  Additionally, we host a public knowledge base and
experimental ground about mathematical knowledge management on the web, powered
by Semantic MediaWiki\footnote{\url{http://mathweb.org/wiki/}}.\end{background}  \claim{It has become
evident that a wiki is a suitable tool for supporting the workflow of
incremental formalization inherent to scientific writing.}  \claim{Wikis have not only
shown to be appropriate for \emph{writing}, but are also effective for project
management, e.\,g.\ in corporate settings~\cite{leuf01:wikiway,wikinomics}.}  We
are therefore interested in applying our technologies to scientific knowledge
engineering projects.

%%% Local Variables: 
%%% mode: latex
%%% TeX-master: "flyspeck-wiki-eswc08"
%%% End: 


%---------------------------------  Flyspeck  ----------------------------------


\section{The Flyspeck Project}
\label{sec:flyspeck}
  The target of our case study is the Flyspeck Project, which seeks
to formally verify Thomas Hales' 2005 proof of the Kepler Conjecture.  
The Kepler Conjecture asserts that the density of a packing of unit spheres
is at most $\pi/(3\sqrt{2})$, the density of the face centered cubic and 
hexagonal close packings.  The conjecture, posed by Kepler in 1611, formed part
of Hilbert's 18th problem, and was recognized as one of the most famous unsolved
problems of mathematics.    

  Hales' proof, completed in 1995, was not accepted immediately by the mathematical community. 
Besides its considerable length,  the proof relies essentially on computer calculations.  
The 300 pages of text, and many thousands of lines of computer code, made
checking the proof for errors in the referee process especially difficult,
leading to a publication delay of nearly 10 years.  

In 2003, Hales proposed using computers to rigorously check the entire proof in
detail, including the computer code.  He dubbed this
effort the \textit{Flyspeck Project}.  
The software systems used in such formalizations are called \textit{theorem provers}
or \textit{proof assistants}.
Beginning with a set of axioms and inference rules, they can,
with adequate human guidance, verify that a purported proof follows from the axioms.  
Examples of proof assistants are Isabelle\cite{Isabelle}, Coq\cite{Coq}, 
and HOL Light\cite{HOLL}. (In the rest of this paper, we use ``formalize'' to mean
that the theorem, proof or definition has been expressed in one of these systems.) 

  Unfortunately, modern proof assistants are still far from being able to check
proofs at the level given in most journals and mathematics textbooks.  A rough
estimate is that it takes about a week to formalize a single page of mathematical
text.  Based on such estimates, Hales' expects that it will take 
around 20 man-years to complete the Flyspeck project.  

The first steps have already been taken.  Nipkow and
Bauer\cite{FlyspeckI:Nipkow} proved a fundamental algorithm in 
Isabelle\cite{Isabelle:definition}, and other parts of the code are
currently being investigated.  Hales is currently compiling a book\cite{FlyspeckBook}
of lemmas from different areas of mathematics that need to be formalized.    
The book is currently 450 pages.  
There is a project page\cite{GoogleCode:Flyspeck}
for Flyspeck hosted at GoogleCode\cite{GoogleCode}.  That page has a source repository
containing the book of lemmas and the definitions, in HOL Light, of some important functions
and inequalities.  Overall, though, the project is still in its infancy.

The Flyspeck project has garnered significant enthusiasm in the theorem proving
community: Hales has given talks about the project at a number of
prestigious conferences, and aspects of Flyspeck are currently the topic of at least
3 PhD dissertations.  Despite the interest, however, it is now somewhat difficult to 
determine the current state of the project or to get involved.  

%Flyspeck project\cite{hales:DSP:2006:432}

% So far on Google code (see section~\ref{sec:req})

\subsection{Requirements}
In this paper, we seek to remedy this situation.  We propose the use of a 
semantic wiki to organize and present details about the current state of Flyspeck.
An example scenario is as follows.  

% \begin{enumerate}  %
% \item A user wishes to contribute to Flyspeck. %
% \item She looks at our wiki main page, which shows her what still needs to be done. %
% \item Prefering analysis to geometry, she searches for open problems involoving analysis.  %
%   This returns a list of lemmas related to analysis from which she can choose one that %
%   seems possible given her time constraints %
% \item She downloads the relevant formal definitions and lemmas, along with the text %
%   of a paper proof culled from Hales' book. %
% \item She uses a proof assistant to begin formalizing the paper proof (or some variant thereof) %
% \item She needs clarification on some definition and additionally has an idea on how to generalize this %
%   lemma.  She thus asks for help and makes comments on the wiki forum.   %
% \item She completes her proof, and uploads the proof assistant file to the wiki. %
%   The wiki checks it for correctness and then adds it to the database.  That lemma will %
%   now not appear in the list of unfinished lemmas. %
% \end{enumerate}  %


% !%\ \ednote{@Christoph !%\ %
% !%\   Motivate this with the principles of wikinomics \url{http://www.wikinomics.com}: If a !%\ %
% !%\   smart but poor boy in Africa with his OLPC accesses our homepage\ldots !%\ %
% !%\   (Sean) not sure if this wikinomics stuff should go here or in the introduction or the conclusion...} !%\ %


% With this scenario in mind, we propose that the wiki should minimally offer: %

% \begin{enumerate}  %
% \item A database of the theory, constant, and lemma definitions %
% \item A way to browse the database by category, or search with keywords %
% \item A way to download the definitions and lemmas statements and proofs (when they exist).   %
% \item A way to upload new proofs %
% \item A forum to discuss issues involved in the formalizations %
% \end{enumerate}  %


%-----------------------------  Requirements/Workflow  ------------------------------


\section{Supporting Flyspeck in a Semantic Wiki}

Starting from a minimal set of requirements, we have evaluated the
applicability of two concrete semantic wikis to Flyspeck.  We used two
different frameworks: a prototype developed on top of Semantic
MediaWiki, and our own semantic wiki SWiM.  Based on the results of
this pre-study, we establish a roadmap for tailoring SWiM to
specifically meet the needs of the Flyspeck collaborators.

\subsection{Requirements}
\label{sec:req}

The Flyspeck project has garnered significant enthusiasm in the
theorem proving community: Hales has given talks about the project at
a number of prestigious conferences, and aspects of Flyspeck are
currently a motivatiion behind at least 3 PhD dissertations.  Despite
the interest, however, it is now somewhat difficult to determine the
current state of the project or to get involved.

An example usage scenario is as follows.  A user
wishes to contribute to Flyspeck.  She looks at our wiki main page,
which shows her what still needs to be done.  Prefering analysis to
geometry, she searches for open problems involoving analysis.  This
returns a list of lemmas related to analysis from which she can
choose one that seems possible given her time constraints She
downloads the relevant formal definitions and lemmas, along with the
text of a paper proof culled from Hales' book.  She uses a proof
assistant to begin formalizing the paper proof (or some variant
thereof) She needs clarification on some definition and additionally
has an idea on how to generalize this lemma.  She thus asks for help
and makes comments on the wiki forum.  She completes her proof, and
uploads the proof assistant file to the wiki.  The wiki checks it
for correctness and then adds it to the database.

With this scenario in mind, we propose that the wiki should minimally offer: 

\begin{itemize} 
\item A database of the theory, constant, and lemma definitions 
\item A way to browse the database by category, or search with keywords 
\item A way to download the definitions,lemmas statements and proofs (when they exist). 
\item A way to upload new proofs 
\item A forum to discuss issues involved in the formalizations 
\end{itemize} 


So far, the Flyspeck project has four core members who collaborate via
Google Code~\cite{website:GoogleCode}.  While the services offered by
Google Code (a Subversion repository, a mailing list, and others) were
found to be sufficient for the core development team, we were not
satisfied with the wiki integrated into the Google Code web interface.
Lacking support for mathematical formulae, it would not even allow for
presenting the theorems and lemmas to be formally proved in a
human-readable fashion.  Furthermore, GoogleCode offers very
little \emph{structuring} support, which we found essential for browsing and
querying Flyspeck's large knowledge collection.

\ednote{@Sean/Florian: One/two sentences about Twelf! (here or in
  some other section where it's appropriate, either workflow or flyspeck) What
  is it, why was it originally chosen? --CL}

Our focus in this work is on making the extent and structure of
Flyspeck comprehensible and on communicating where work needs to be
done.  For this the outline of the whole proof from the
book\cite{Hales:2007:FlyspeckBook} needs to be represented in the
wiki, where the mathematical statements (including definitions,
lemmas, and theorems) are available in a human-readable way (with
formulae in \LaTeX\ or presentational MathML) as well as a
machine-readable presentation suitable for downloading into a theorem
prover.  In order to obtain a well-structured network of knowledge
items, each mathematical statement should be presented on one wiki
page, which shows its human-readable representation from the book,
offers additional space for annotation, and allows for downloading a
formal representation.  We consider the following kinds of annotations
desirable:

\begin{description}
\item[Categorization by topic:] In the beginning, one would mirror the narrative structure
  of the book (e.\,g.\ ``ball'' being a subsection of ``primitive volumes'', which in turn
  is a section of the chapter ``volume calculations'').  Standardized ways of classifying
  mathematical topics, such as the Mathematical Subject Classification
  (MSC)\cite{AMS:MSC2000}, could be added later.
\item[Project-organization metadata] such as the information whether a lemma has already
  been proven formally, or in what theorem proving language there are proof objects
  available.\ednote{@Sean: more?}
\item[Dependency links:] These can be links from individual symbols in mathematical
  formulae to the place where they are declared, or from any page $p$ to other pages
  containing knowledge that is required for understanding $p$ --- either pages in the same
  wiki, or external resources like PlanetMath or Wikipedia articles.
\item[Discussion posts] should be strongly tied to the topic being discussed, and they
  should be classified into categories like question, answer, explanation,
  etc.
\end{description}

To the visitor and potential collaborator, an impression of the extent and structure of
the project --- its enourmous size and its specialization into diverse fields of
mathematics --- must be given, as well as tools for browsing and querying the knowledge.
The topical structure as well as the dependencies must be browsable via links.  Not only
should it be possible to query knowledge items by their annotations, but important query
results must also be available as dynamically generated lists.  Examples for queries are:

\begin{enumerate}
\item\label{item:proven-lemma} ``Which lemmas about composite regions have not been
  formally proven so far?''
\item ``What do I need to read in order to understand Jordan's curve
  theorem?''\ednote{internal dependency graph, tutorial, planetmath}
\item ``What lemmas are difficult to prove?''
  \begin{enumerate}
  \item \ldots in the sense that many invalid proofs have already been submitted
  \item\label{item:question-count} \ldots in the sense that many people have asked
    questions in the related discussion
  \end{enumerate}
\item \ldots\ednote{@Christoph: Check Matthias' BSc thesis for further ideas. --CL}
\end{enumerate}

A volunteer who is willing to work out and contribute a formal proof for some lemma should
be able to download a self-contained formal representation of this lemma and everything it
depends on.  Different notions of dependency could make sense: The strongest one is that a
lemma depends on the declarations and definitions of all symbols it uses and on the
transitive closure of all symbols used by the latter---i.\,e.\ the minimum set of
knowledge required to \emph{understand} the respective lemma.  This would, however,
require a collaborator to develop his new proof from scratch.  We anticipate that it will
be more desirable for a user to download other, previously proven
lemmas\ednote{@Sean/Florian: and their proofs? Does one also need proof objects for that?}
as well and treat them like axioms.  Assuming that the Flyspeck book\ednote{rename} is
written in a reasonable order, this would be all lemmas \emph{before} the current one, in
the narrative order of the book.  But it should still be possible to download \emph{all}
theorems, as a creative proof might involve lemmas one would not think of
first.\ednote{@Florian: Do we want to convert between theorem proving languages?  Is this
  feasible with OMDoc?}.

For the integration of the contributed proofs, we currently envisage that a small group of
maintainers will integrate submitted formal proofs into one central ``master'' proof
script outside of the wiki and have this script checked semi-automatically.  Metadata
\emph{about} the progress of the proof, e.\,g.\ whether a submission by a certain user was
valid, and a human-readable outline of the proof\ednote{@Sean: This as well?  Is it
  technically possible to explain a machine proof to a human.} will be added to the wiki.
In a later phase, it has to be investigated whether formal proofs can also be
collaboratively developed and validated in the wiki (cf.\ section~\ref{sec:wiki-pa}).

\newcommand{\wikipage}[5]{\node[draw,text width=5cm,font=\tiny\sffamily] (#1) at #2 {
    {\footnotesize\bfseries #3}\\
    #4
    ~\\[1em]
    [Download Isabelle representation]\\
    #5
  };}
\begin{figure}
  \centering
  \begin{tikzpicture}[set style={{default}+=[scale=1.5,font=\sffamily]},default,xscale=.8]
    \wikipage{lemma}{(0,0)}{Lemma 1.3}{The cosine is an even function.  The sine is an odd function.\\
      $\cos(-x)=\cos(x),\qquad\sin(-x)=-\sin(x)$}{Page type: Lemma\\
      Topic: Trigonometry\\
      Proven: no (3 unsuccessful attempts)}
    \wikipage{cos}{(6,0)}{Cosine}{$cos\colon\mathbb{R}\to\mathbb{R},x\mapsto\ldots$}{
      Page type: Definition\\
      Topic: Trigonometry
    }
    \wikipage{todo}{(0,3)}{To do}{Unproven lemmas:
      \begin{itemize}
      \item Lemma 1.3
      \item \ldots
      \end{itemize}
    }{
      Page type: Overview
    }
    \draw[->] (lemma) -- node[above] {usesSymbol} (cos);
    \draw[->] (todo) -- node[left] {references} (lemma);
  \end{tikzpicture}
  \caption{Page structure}
  \label{fig:pagestructure}
\end{figure}

\ednote{To do: put the following into the right place --CL} Currently we are investigating
whether Twelf is actually appropriate for formalizing Flyspeck.  We are also investigating
Isabelle, but we consider the design of the wiki support to be largely independent of that
decision.  After ???\ednote{@Sean: insert number} years, we consider the basic narrative
structure of the book a sufficiently stable for modeling the structure of the wiki after
it.  During the formalization of the knowledge, we anticipate that mainly its axiomatic
parts (i.\,e.\ the way concepts are defined) will undergo major refactoring in order to
facilitate the actual development of the proofs, as certain areas of mathematics the
Kepler proof heavily relies on, such as geometry, are underrepresented in the existing
formal mathematical libraries of theorem provers\ednote{@Sean: this is in a nutshell what
  you once told me about this; is it sufficient?}.  Additional refactoring support by the
wiki would thus be of advantage.

Secondly, we have not yet committed to a definitive workflow for formalizing the book.
Given the size of the book, which is much smaller than the size of the final
proofs\ednote{@Sean: right?}, the current Flyspeck core team may be able to create formal
representations of the symbol declarations, definitions, and lemmas manually and upload
them to the wiki.  On the other hand, wikis support a workflow where different groups of
authors, such as domain experts or proof-readers, collaborate in formalizing knowledge
step by step.  This potential could also be unleashed for Flyspeck.



%-----------------------------  Case Studies  ------------------------------

\section{Case Studies and Evaluation}

In the following two sections, we evaluate a prototype based on Semantic
MediaWiki and SWiM 0.2 for their applicability to Flyspeck with regard to their
support for annotations, browsing, and querying, as specified in
section~\ref{sec:req}.  For the case study, we had the {\TeX} sources of the
Flyspeck book and a Twelf formalization of the first chapter (Trigonometry) at
our disposal.\ednote{@Sean/Florian: One/two sentences about Twelf! (here or in
  some other section where it's appropriate, either workflow or flyspeck) What
  is it, why was it originally chosen? --CL}

Both systems are semantic wikis, where one resource (in the RDF sense)---e.\,g.\
one mathematical theorem---is represented by one wiki page and relations between
resources by links between pages.  Both pages and links can be typed with terms
from ontologies\cite{OrDeMoVoHa06:annotation-navigation-semwiki}, which are
either preloaded into the wiki or modelled
ad-hoc\cite{KrSchVr:semwiki-reasoning07}.  This is the prevalent approach of
adding semantics to wikis, although other ways have been
investigated\cite{semwiki06}.  Semantic wikis offer enhanced navigation
capabilities, at least by displaying a summary of all typed links, grouped by
type, with each page.  Most of them allow to search for pages by type or by them
being subject or object of any RDF triple (= typed link), while it depends on
the reasoner used by the wiki whether only explicit RDF triples or also inferred
ones are considered\cite{KrSchVr:semwiki-reasoning07}.  Such queries can usually
be executed interactively via a special search form, or in an automated way as
\emph{inline} queries embedded into the content of a page.  Except for
interactive triple search not yet being supported by SWiM, both systems support
this basic set of semantic wiki features.

\subsection{Semantic MediaWiki 1.0}
\label{sec:smw-study}

Semantic MediaWiki\cite{KrSchVr:semwiki-reasoning07} is a semantic web extension
to MediaWiki, the system driving Wikipedia.  Plain MediaWiki supports
mathematical formulae written in {\LaTeX} and allows for categorizing pages.
Semantic MediaWiki interprets category membership as an instance-of relationship
and adds the possibility to type links and to create and edit link types, called
properties.  External ontologies can be referenced from the wiki, but at most
sites powered by Semantic MediaWiki, site-specific ontologies are developed in
an ad-hoc manner.  We found this useful while \emph{prototyping} the annotations
that might be required for Flyspeck, e.\,g.\ project-related metadata like the
information whether a lemma has already been proven, or categorization by topic.
It was less useful in places where ontologies already existed; for structures of
mathematical documents, it was just possible to reference \emph{vocabulary} from
the OMDoc document ontology (see section~\ref{sec:swim}), but not to apply
further inference rules given there to items of mathematical knowledge, as
Semantic MediaWiki does not support a full \emph{import} of external ontologies.
Moreover, Semantic MediaWiki does not understand the semantics of mathematical
formulae, as the {\LaTeX} formulae cannot be annotated.

In Semantic MediaWiki, we imported the Twelf master source of Flyspeck via a
customly implemented special page that uses the well-documented extension API of
MediaWiki.  The Twelf file was first enhanced by special comment lines marking
beginning and end of a declaration\ednote{@Sean/Florian: What's the general term
  for \texttt{id : decl = def}?} with information about topical categorization.
The Twelf upload extension breaks an uploaded file down into declarations and
creates two wiki pages for each Twelf declaration: one page that just contains
the Twelf listing, categorized in its respective category of the OMDoc document
ontology (e.\,g.\ \textit{Lemma}), and one container page that includes the
Twelf page via MediaWiki's template inclusion mechanism, but also allows for
including a {\LaTeX} representation and leaving space for free-form annotations
made by the contributors step by step.  The Twelf pages are overwritten on every
import from the master source, whereas existing container pages remain
untouched.

As a first step, we encoded the types of knowledge items in the identifier of
the declarations; the identifier of a lemma would start with ``lemma-''.  We
could also have employed Twelf's inference to determine the
type\ednote{@Florian: Reword this correctly! Do you get the idea; is this
  possible?}, as our Twelf-to-OMDoc converter does.  Additionally, the upload
extension recognizes all previously imported symbols in Twelf expressions and
turns them into links of the type \textit{\ldots--uses--Symbol}.

\begin{figure}
  \centering
  \includegraphics[width=\textwidth]{smw-lemma}
  \caption[A Flyspeck lemma in Semantic MediaWiki]{A Flyspeck lemma in Semantic
    MediaWiki\protect\footnotemark}
  \label{fig:smw-lemma}
\end{figure}
\addtocounter{footnote}{-1}
\stepcounter{footnote}\footnotetext{See \url{http://mathweb.org/wiki/Flyspeck}}

The annotations generated that way can be used for browsing, either via the
``fact box'' (the summary of all typed links), or by the special ``browse''
page.  For querying, Semantic MediaWiki offers a simple triple search, as well
as inline queries.  The latter are intuitive to write but not as powerful as
required.  The query language corresponds to the description logic
$\mathcal{EL}^{++}$\cite{KrSchVr:semwiki-reasoning07}, which, for example, does
not support negation.  A query for unproven lemmas about a certain topic could
only be performed if the ``unprovenness'' were explicitly annotated.  The
following queries additionally asks for lemmas available in a Twelf formalization:

\begin{lstlisting}
<ask>[[Category:Unproven]] [[Category:Lemma]]
     [[Category:Trigonometry]] [[written in::Twelf]]</ask>
\end{lstlisting}

Here, most annotations are modelled by categorization, i.\,e.\ instantiation of
classes---certainly not the most formal way of structuring knowledge in view of
many classes just corresponding to narrative sections of the book, but the one
that is supported best by Semantic MediaWiki.  More complex reasoning tasks like
inference of dependencies are not possible in Semantic MediaWiki; in this
domain-specific setting one could realize them by hard-coded extension
functions.

Exporting formal representations of knowledge items is not yet supported
conveniently.  The Twelf listings can be viewed on their own pages, but due to
the auto-generated symbol links in the source code, these are not suitable for
download.  One would either have to implement a special Twelf download page that
cleans these sources again, or one would have to implement the symbol linking as
an extension of the rendering process.

\subsection{SWiM 0.2}
\label{sec:swim}

SWiM is a semantic wiki for mathematical knowledge management.  Based on the
general-purpose semantic wiki IkeWiki\cite{KrSchVr:semwiki-reasoning07}, it adds support
for browsing, editing, rendering, importing and exporting mathematical documents written
in OMDoc.  The semantics of mathematical knowledge is mainly captured in a \emph{document
  ontology}: Whenever a wiki page containing OMDoc fragments is stored, its type and its
(typed) relations to other items of mathematical knowledge in the wiki are extracted from
the OMDoc XML markup and explicitly represented as RDF triples in terms of the OMDoc
document ontology\cite{OMDocDocOnto:web}.  This ontology models those aspects of the
three layers of mathematical knowledge supported by OMDoc to the extent supported by the
expressivity of OWL-DL.  Modeling all modules of the OMDoc specification in this ontology
is work in progress; so far, most mathematical statements as well as key aspects of
theories have been implemented.  Relevant classes for Flyspeck would be
\textit{Lemma}/\textit{Theorem}/\textit{Corollary}/\ldots (all being subclasses of
\textit{Assertion}), \textit{Proof}, \textit{Symbol} (a symbol declaration),
\textit{Definition}, and the properties \textit{Proof--proves--Assertion} and
\textit{Symbol--hasDefinition--Definition}.  Dependencies can partly be inferred by a DL
reasoner, but for a complete support of OMDoc's notion of dependency, an OMDoc-specific
calculus will have to be applied, which is currently in development.

\begin{figure}
  \centering
  \begin{tikzpicture}[set style={{default}+=[scale=.47,font=\normalsize\sffamily]},default]
    \tikzstyle{every path}=[font=\small\sffamily];
    \node[concept] (s) at (0,0) {\itshape Statement};
    \node[concept] (d) at (-7.5,-3) {Definition};
    \node[concept] (y) at (-2.5,-3) {Symbol};
    \node[concept] (a) at (+2.5,-3) {Assertion};
    \node[concept] (p) at (+7.5,-3) {Proof};

    \node[concept] (l) at (-1.5,-6) {Lemma};
    \node[concept] (c) at (+2.5,-6) {Corollary};
    \node[concept] (t) at (+6.5,-6) {Theorem};

    \draw[-open triangle 60] (y) -- (s);
    \draw[-open triangle 60] (d) -- (s);
    \draw[-open triangle 60] (a) -- (s);
    \draw[-open triangle 60] (p) -- node[right=1ex] {$\sqsubseteq$} (s);

    \draw[-open triangle 60] (l) -- (a);
    \draw[-open triangle 60] (c) -- (a);
    \draw[-open triangle 60] (t) -- (a);

    \draw[->] (d) -- node[below] {uses} (y);
    \draw[->] (a) -- node[below] {uses} (y);
    \draw[->] (p) -- node[below] {proves} (a);

    \draw[->] (s.0) .. controls +(0:2cm) and +(60:2cm)
    .. node[right=1pt,text width=2cm,text centered] (dep)
    {\itshape depends on} (s.60);

    \draw[->] (y.-120) .. controls +(-120:1cm) and +(-60:1cm) .. node[below] {hasDefinition} (d.-60);
  \end{tikzpicture}
  \caption{A relevant subset of the OMDoc document ontology}
  \label{fig:doconto}
\end{figure}

In the current version 0.2 of SWiM, the browsing of mathematical documents is powered by
the document ontology: Whenever RDF triples having the current page as subject or object
are available, most of them using terms from the OMDoc document ontology if the current
page is a mathematical document, they are displayed as navigation links (see
figure~\ref{fig:swim-lemma}).  Adding more ontology-powered services, particularly ones
that facilitate editing documents, is planned for version
0.3\cite{swim-roadmap,Lange:SWiMSciColl07}.  Documents are presented as XHTML+MathML,
generated by the \textit{mmlkit} renderer\cite{mmlkit:web}, with mathematical symbols
linked to their declarations.

\begin{figure}
  \centering
  \includegraphics[width=.7\textwidth]{swim-lemma}
  \caption{A Flyspeck lemma in SWiM}
  \label{fig:swim-lemma}
\end{figure}

We manually recreated one Flyspeck lemma in SWiM and judged about the further annotation
capabilities from our previous experience with implementing SWiM and modeling other
document collections (as the OpenMath content dictionaries) in the system.\ednote{This is
  honest and really sufficient IMHO, but maybe sounds a bit weak -- what do you think?
  --CL} For testing a larger subset of Flyspeck, it would, however, have been possible to
convert the Twelf master source to OMDoc with our already existing converter, and to
import the generated OMDoc documents into SWiM using the built-in import functionality.
OMDoc would offer excellent support for the alternative workflow of stepwise formalization
as well\cite[chap.\ 4]{Kohlhase:omdoc1.2}.  One could either start with converting the
Flyspeck book from {\LaTeX} to HTML with Presentation MathML and step by step formalize
the presentation markup into content markup, or one could start the formalization on the
{\TeX} side, using s\TeX{}, a content-oriented {\TeX} notation for OMDoc which can then be
converted to OMDoc\cite{Kohlhase:albwo06}.

We found SWiM to be suitable for making Flyspeck browsable, as it supports breaking down
mathematical knowledge to the statement level and recognizes all required link types.  As
every SWiM page has an associated discussion page and discussion posts are semantically
represented using the SIOC ontology\cite{SIOC:web}, one could also support the
coordination of the project by queries like query~\ref{item:question-count} from
section~\ref{sec:req}.  OMDoc markup and mathematical formulae in discussion pages are not
yet implemented, though.  Pages and non-OMDoc links can be annotated with types from
ontologies loaded into the wiki\footnote{Types of OMDoc links are automatically extracted
  from the markup; see above.}.  That means that the annotations required by Flyspeck can
be made, but not in an ad-hoc way, which we would have found useful in the prototyping
phase.  Instead, one would have to import an existing ontology into the wiki, or create it
using IkeWiki's ontology editor---SWiM supports both---, and then one would be able to
annotate documents using terms from that ontology.

Searching for arbitrary RDF triples is not yet supported by the user interface,
but authors can embed inline SPARQL queries into wiki pages.  Note that not all desirable
queries are easy to express in SPARQL; consider query~\ref{item:proven-lemma}:

\begin{lstlisting}
SELECT ?l WHERE { ?l rdf:type odo:Lemma .
                  ?l swrc:isAbout <Composite_Regions> .
                  OPTIONAL { ?p rdf:type odo:Proof .
                             ?p odo:proves ?l . }
                  FILTER ( ! bound(?p) ) }
\end{lstlisting}

This query assumes a SPARQL semantics with negation as failure\cite{Polleres:SPARQL-Rules07}.  Alternatively, the query
can be made more intuitive by enhancing the ontology by the following concept:

\[
\mbox{LemmaWithoutProof}\equiv\mbox{Lemma}\sqcap\neg(\exists\mbox{proves}^{-1}.\mbox{Proof})
\]

The downloading part of the Flyspeck workflow is not yet natively supported by SWiM.
Currently, the only mathematical export format supported by SWiM is OMDoc, which could
then be converted to theorem proving languages by client-side software\cite[chap.\
25.2]{Kohlhase:omdoc1.2}.


%%% Local Variables: 
%%% mode: latex
%%% TeX-master: "flyspeck-wiki-eswc08"
%%% End: 


% ---------------------  Related Work  --------------------

\section{Related Work (all)}
\label{sec:related}


\subsection{Management of Mathematical Knowledge}
\label{sec:mkm}

Projects that manage mathematical knowledge can be classified into three groups.  First, there are projects that do not systematically employ computer support for their management needs.  An example for such a project is the classification of the finite simple groups~\cite{Gorenstein-Lyons-Salomon:1994}.  This group could be dubbed the ``informal'' group because the produced knowledge, while formal to a large degree, does not exist in a machine-readable form.

Projects in the second, fully formal group use computer systems to manipulate the mathematical knowledge.  The most important examples for such systems are automated and interactive reasoning tools that permit to construct and search for mathematical proofs. Several such systems are in use, such as Mizar~\cite{mizarmanual}, Coq~\cite{Coq}, or Isabelle~\cite{Isabelle:definition}.  For these tools, large libraries, usually with a central repository and a repository viewer, have been developed.  For example, the standard Isabelle library covers elementary number theory, analysis, algebra, and set theory\ednote{@FR: But how about geometry and other areas relevant for Flyspeck?  They are   not adequately represented, are they?}.  An example or such a project is the fully formal proof of the four color theorem~\cite{Gonthier:FourColor}.

Finally, semi-formal projects try to combine the advantages of the other two groups by formalizing only parts of or only certain aspects of the mathematical knowledge.  For example, OMDoc and SWiM permit to combine formal mathematical content with natural language in a way that keeps the structure and interfaces of the knowledge machine-readable.  This is particularly suited for projects where an informal document is incrementally transformed into a formal one, such as in Flyspeck.

\subsection{Wikis for Mathematics}
\label{sec:math-wiki}

\subsubsection{Informal Knowledge Collections}
\label{sec:math-knowledge-collections}

Current collaborative projects for managing \emph{informal} mathematical knowledge range
from comprehensive encyclopediæ like the mathematical sections of \product{Wikipedia}\ednote{reference} or
the courseware repository and content management system \product{Connexions}\ednote{reference} to projects
specially focused on mathematics like \product{PlanetMath}\ednote{reference}\footnote{See
  \url{http://www.wikipedia.org}, \url{http://cnx.org} or \url{http://www.planetmath.org},
  respectively.}, which is powered by a highly customized wiki-like system.  The pages in
these systems are categorized and searchable in full-text, with additional metadata
records in the case of \product{PlanetMath}.  Neither of these systems is a
\emph{semantic} wiki, and thus they fail to solve the following two problems, which are
essential for MKM:

\begin{enumerate}
\item\label{item:formula-search-usecase} In \product{Wikipedia} and \product{PlanetMath},
  formulæ are given in presentation-oriented {\LaTeX}.  Imagine a wiki page about the
  Pythagorean Theorem, stated as $a^2 + b^2 = c^2$, and a user searching for the
  equivalent formula $x^2 + y^2 = z^2$ (or even $c=\sqrt{a^2+b^2}$!) --- The system would
  not find the theorem.
\item Neither could ``all theorems about triangles for which a
  proof exists'' be searched for, as the link from a proof to the theorem it proves is not
  typed.
\end{enumerate}

\product{Connexions}, on the other hand, could in principle cope with these two problems,
but in practice it does not: Formulæ are written in the content-oriented sublanguage of
{\mathml}~\cite{CarlisleEd:MathML07}, and the CNXML markup language used for larger
structures allows for annotating texts as mathematical statements like lemmas, but this
structural information is not yet \emph{used} by the system.  Moreover, none of the
systems mentioned so far supports an easy navigation from the occurrence of a mathematical
symbol in a formula to the declaration or definition of this symbol, if it is defined in
some other place of the wiki; instead, the author has to provide links he considers
relevant in the text surrounding the formula.

\begin{oldpart}{Move this somewhere else}
  \subsubsection{Domain-Specific Semantics}
  \label{sec:domain-semantics}

  Note that general-purpose semantic wikis do not support the above-mentioned use case   (\ref{item:formula-search-usecase}) either, as they neither have a sufficient notion of equality   nor understand mathematical content markup.  If we assume ``semantic'' not just to mean RDF or   description logics, but any kind of (higher-order) logic required for specific   domains\ednote{@Florian: This is quite superficial, can we write it in a more sophisticated     way?} and employ domain-specific ways of knowledge representation we can imagine semantic   wikis specifically supporting mathematics.  For use case (\ref{item:formula-search-usecase}), we   could have the wiki pages crawled by a formula search engine like   MathWebSearch~\cite{KohSuc:asemf06}, which applies substitution tree indexing to mathematical   formulae.  Even more formal approaches integrate automated theorem provers into wikis.  Two of   these systems are discussed in section~\ref{sec:wiki-pa}.
\end{oldpart}

\subsubsection{Wikis with Integrated Proof Assistants}
\label{sec:wiki-pa}

Recently, there is a growing interest in integrating automated theorem provers or proof
assistants with wikis\footnote{See \url{http://homepages.inf.ed.ac.uk/da/mathwiki/} for
  relevant activities.}.  Both Logiweb and ProofWiki are wiki-like systems that support
checking or interactive development of proofs.  Both are ``semantic'' in the sense that
the integrated proof checker can utilize the mathematical knowledge in the wiki pages.
But the semantics is not utilized for purposes other than that, such as facilitating
browsing or editing, or connecting to services on the semantic web.  Developing and
verifying formal proofs in the wiki is not yet the focus of Flyspeck in this early stage,
but it may be required later if the central maintainer approach does not turn out to work.

\paragraph{Logiweb} is a distributed system for publishing machine checked mathematics in
high-quality PDF~\cite{Grue:Logiweb07}.  While the author does not call it a ``wiki'', it
shares part of the key wiki principles: Anybody can contribute to a Logiweb site and edit
new pages in a simple text syntax with a browser.  On the other hand, Logiweb does not
offer other features that would be essential for Flyspeck: browsing by traversing links is
supported neither in the editor nor in the generated PDF, and Logiweb does not offer a
built-in search or query facility.  Logiweb does not allow for \emph{exchanging} knowledge
as required for Flyspeck: Documents can be exported in presentational formats like PDF or
\TeX{}, and their internal, low-level data structures can be exported as XML or Lisp
S-expressions, but currently there is no easy way to convert these representations to
other languages for mathematical markup or theorem proving.  Finally, the way Logiweb
checks proofs is not compatible with other theorem provers, as all calculi and proof
tactics need to be defined in the Logiweb system itself.

\paragraph{ProofWiki} is an integration of ProofWeb, a web frontend to the Coq proof
assistant, into MediaWiki~\cite{CorKal:CoopReposFormalProofs07}.  Coq's converting tools
are used to generate human-readable and browsable HTML or {\LaTeX} presentation from the proof
scripts.  In the HTML generated that way, symbols are linked to their declaration.  Index
pages, such as lists of all definitions or all theorems, are generated, but their
generation cannot be influenced or customized through the wiki
interface\ednote{@Christoph: check!}.  So far, there is just text search, and dependencies
among knowledge items are only computed for exporting proof scripts but not used for
browsing inside the system.  Another disadvantage of ProofWiki in the context of Flyspeck
is that pages can either be formal proof scripts (with restricted possibilities to include
informal comments), or informal wiki pages.  Semi-formal documents or stepwise formalizing
of knowledge is not supported.  Importing and exporting Coq proof scripts to and from the
wiki is possible, but other formats are not yet supported.  While the authors do provide
instructions on how to integrate other theorem provers, doing so would be a lot of work,
as there is no abstraction layer or metalanguage for exchanging or converting data.

\ednote{@Christoph: Find out whether such systems have already been used in Flyspeck-like
  scenarios, i.\,e.\ collaboratively proving something.}

%%% Local Variables: 
%%% mode: latex
%%% fill-column: 90
%%% TeX-master: "flyspeck-wiki-eswc08"
%%% End: 


% ---------------------  Conclusion  ----------------------


\section{Conclusion and Further Work}
\label{sec:conc}

\subsection{Towards Full Flyspeck Support in SWiM}
\label{sec:flyspeck-swim}

The experiments with Semantic MediaWiki and SWiM led to the conclusion to
integrate further support for Flyspeck in SWiM, using its rich semantic web and
OMDoc infrastructure.  Features that rely on \emph{structures}, like the linking
of symbols, could only be realized in a very prototypical way for the text-based
page format of MediaWiki, using regular expressions.  Relying on the XML
infrastructure of OMDoc, these features are easier to develop, or already
available.  However, rapidly \emph{prototyping} our first ideas about the wiki
support required for Flyspeck was easier in Semantic MediaWiki due to its
ability to design ad-hoc ontologies and its implementation in the interpreted
language PHP.

Outline (following the workflow):
\begin{itemize}
\item import formal/informal knowledge, formalizing/structuring (automating
  this, LaTeXML)
\item annotate (categorize, project metadata, dependency links, discussion)
\item browse (idea: use narrative structure)
\item search/query
\item download
\item upload/reintegrate
\end{itemize}

\subsubsection{Snippets/Thoughts}

OMDoc offers excellent support for the alternative workflow of stepwise
formalization as well\cite[chap.\ 4]{Kohlhase:omdoc1.2}.  One could either start
with converting the Flyspeck book from {\LaTeX} to HTML with Presentation MathML
and step by step formalize the presentation markup into content markup, or one
could start the formalization on the {\TeX} side, using s\TeX{}, a
content-oriented {\TeX} notation for OMDoc which can then be converted to
OMDoc\cite{Kohlhase:albwo06}.

use OMDoc as universal exchange format between ATP languages; every ATP system so far is
an island; OMDoc makes ATP scale to the web.  At least definitions.  However, we're not
yet sure\ednote{@Florian: right?}, whether we can actually rely on OMDoc, as translation
of \emph{proofs} is not that trivial.\ednote{@Sean/Florian: why?}

\ednote{@Florian: \textbf{Write sth. about OMDoc-1.2-style theories and imports for refactoring!}}

\ednote{@Christoph: Needs to support both preloaded and ad-hoc ontologies: The OMDoc
  document ontology is preloaded, while other annotations can be added at will. (Make SWiM
  more ``wiki''!)}

\begin{todo}{@Christoph: Elaborate on this discussion with Florian}
  OMDoc is agnostic towards logics -- that could be a benefit as long as we do
  not yet have a proof object.
\end{todo}

\ednote{@Christoph: narrative structure: do not use ad-hoc categories, but OMDoc's native
  NarCons\cite{KohMueMue:dfncimk07}; need to be added to document ontology.}

\begin{oldpart}{Use this idea somehow}
  In \product{Wikipedia} and \product{PlanetMath}, formulæ are given in
  presentational {\LaTeX}.  If the Pythagorean Theorem were represented as $a^2
  + b^2 = c^2$, and a user searched for the equivalent formula $x^2 + y^2 = z^2$
  (or even $c=\sqrt{a^2+b^2}$!), the system would not find the theorem.

  Note that general-purpose semantic wikis do not support the above-mentioned
  use case either, as they neither have a sufficient notion of equality nor
  understand mathematical content markup.  If we assume ``semantic'' not just to
  mean RDF or description logics, but any kind of (higher-order) logic required
  for specific domains\ednote{@Florian: This is quite superficial, can we write
    it in a more sophisticated way?} and employ domain-specific ways of
  knowledge representation we can imagine semantic wikis specifically supporting
  mathematics.  For the Pythagoras use case, we could have the wiki pages
  crawled by a formula search engine like MathWebSearch\cite{KohSuc:asemf06},
  which applies substitution tree indexing to mathematical formulae.
\end{oldpart}

\issue{@Florian, do you think we could make use of MathWebSearch for certain services?
  (@Sean, that's our semantic math formula search engine.)  Now that I've mentioned the
  Pythagoras example, we could think about it. --CL}

%%% Local Variables: 
%%% mode: latex
%%% TeX-master: "flyspeck-wiki-eswc08"
%%% End: 


\paragraph{Acknowledgments}
\label{sec:ack}

\begin{itemize}
\item Michael Kohlhase
\item Immanuel Normann
\item Stefan Decker
\end{itemize}

\bibliographystyle{abbrv}
% load crossrefs last when using modular bib files
\bibliography{flyspeck-wiki,seanbib/all}

% \printindex

\ednotemessage
\end{document}

% vim:tw=90:autoindent: 
%%% Local Variables: 
%%% mode: latex
%%% fill-column: 90
%%% End: 
