
\section{Conclusion and Further Work}
\label{sec:conc}

\subsection{Towards Full Flyspeck Support in SWiM}
\label{sec:flyspeck-swim}

The experiments with Semantic MediaWiki and SWiM led to the conclusion to integrate
further support for Flyspeck in SWiM, using its rich semantic web and OMDoc
infrastructure.  Features that rely on \emph{structures}, like the linking of symbols,
could only be realized in a very prototypical way for the text-based page format of
MediaWiki, using regular expressions.  Relying on the XML infrastructure of OMDoc, these
features are easier to develop, or already available.  However, rapidly \emph{prototyping}
our first ideas about the wiki support required for Flyspeck was easier in Semantic
MediaWiki due to its ability to design ad-hoc ontologies and its implementation in the
interpreted language PHP.

Outline (following the workflow):
\begin{itemize}
\item import formal/informal knowledge, formalizing/structuring
\item annotate (categorize, project metadata, dependency links, discussion)
\item browse (idea: treemaps, narrative structure)
\item search/query
\item download
\item upload/reintegrate
\end{itemize}

\subsubsection{Snippets/Thoughts}

use OMDoc as universal exchange format between ATP languages; every ATP system so far is
an island; OMDoc makes ATP scale to the web.  At least definitions.  However, we're not
yet sure\ednote{@Florian: right?}, whether we can actually rely on OMDoc, as translation
of \emph{proofs} is not that trivial.\ednote{@Sean/Florian: why?}

\ednote{@Florian: Would your work about theory translation somehow help?}

\ednote{@Christoph: Needs to support both preloaded and ad-hoc ontologies: The OMDoc
  document ontology is preloaded, while other annotations can be added at will. (Make SWiM
  more ``wiki''!)}

\begin{todo}{@Christoph: Elaborate on this discussion with Florian}
  OMDoc is agnostic towards logics -- that could be a benefit as long as we do not yet
  have a proof object. Work in progress, nodes without content, loosely coupled: Ideal
  setting for a wiki!  Wiki as a means of information for the collaborators about the
  progress of the \emph{whole} project.  Make a proof that's \emph{so} complex
  comprehensible in some way, show open issues/problems to potential collaborators.
  There's a whole book containing the informal outline of the proof; link the wiki to the
  book.  Probably also import the book via sTeX to the wiki.\\
  Make the whole project publicly viewable (e.g. for a progress report), make it
  manageable.\\
  Analogy: from software documentation to literate programming
\end{todo}

\ednote{@Christoph: narrative structure: do not use ad-hoc categories, but OMDoc's native
  NarCons~\cite{KohMueMue:dfncimk07}; need to be added to document ontology.}

\issue{@Florian, do you think we could make use of MathWebSearch for certain services?
  (@Sean, that's our semantic math formula search engine.)  Now that I've mentioned the
  Pythagoras example, we could think about it. --CL}

%%% Local Variables: 
%%% mode: latex
%%% TeX-master: "flyspeck-wiki-eswc08"
%%% End: 
