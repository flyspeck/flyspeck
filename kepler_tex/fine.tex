\chapter{Partitioning Space}%DCG "The S-System" Sec.9, p85
    \label{sec:fine}
    \oldlabel{2}

\section{Interaction of $V$-cells with the $Q$-system}

We study the structure of one $V$-cell, which we take to be the
$V$-cell at the origin $v=0$.  Let $\CalQ$ be the set of simplices
in the $Q$-system.  For $v\in\Lambda$, let $\CalQ_v$ be the subset
of those with a vertex at $v$.\index{Qv@$\CalQ_v$}



\begin{lemma} \label{lemma:voronoi-truncation-over-Q}
If $x$ lies in the \index{Voronoi cell} Voronoi cell at the
origin, but not in the $V$-cell at the origin, then there exists a
simplex $Q\in\CalQ_0$, such that $x$ lies in the cone (at $0$)
over $Q$. Moreover, $x$ does not lie in the interior of $Q$.
\end{lemma}

\begin{proof}
By
the definition of $V$-cell, there is a barrier $\{v_1,v_2,v_3\}$
that meets $\op{conv}^0(0,x)$.  
By Lemma~\ref{tarski:vor-bar-tet}
and Lemma~\ref{tarski:vor-bar-quad},
the simplex $Q=\{0,v_1,v_2,v_3\}$ is a quasi-regular tetrahedron
or a flat quarter.  If it is a flat quarter then it shares a diagonal
with a barrier coming from the $Q$-system.  Thus, $Q$ is in
the $Q$-system too.  By Lemma~\ref{tarski:pass-cone}, the
point $x$ lies in the cone over $Q$.

If $x\in\op{conv}^0(Q)$, then the barrier, which is a face
of $Q$, does not separate
$x$ from $v$.  The rest is clear.
\end{proof}



\begin{lemma}\label{lemma:VC-Omega}
Inside the ball of radius $t_0$ at the origin, the $V$-cell and
Voronoi cell coincide:
   $$B(0,t_0)\cap \op{VC}(0) \equiv B(0,t_0)\cap \Omega(0).$$
That is, they are equal up to a null set.
\end{lemma}

\begin{proof} Let $x\in B(0,t_0)\cap \op{VC}(0)\cap\Omega(v)$, where
$v\ne0$.  
% By Lemma~\ref{lemma:unobstr-t0}, 
The origin is
unobstructed  at $x$ by Lemma~\ref{lemma:unobstr-t0}.  
Thus, $|x-v|< |x|\le t_0$.  By
Lemma~\ref{lemma:unobstr-t0} again, $v$ is unobstructed at $x$, so
that $x\in \op{VC}(v)$, contrary to the assumption
$x\in\op{VC}(0)$.  Thus $B(0,t_0)\cap\op{VC}(0)\subset\Omega(0)$.
Similarly, if $x\in B(0,t_0)\cap \Omega(0)$, then $x$ is
unobstructed at the origin, and $x\in \op{VC}(0)$.
\end{proof}

\bigskip

\begin{remark} The next lemma helps to determine which $V$-cell
a given point $x$ belongs to.  If $x$ lies in the open cone over a
simplex $Q_0$ in $\CalQ$, then Lemma~\ref{lemma:Q-divide}
describes the $V$-cell decomposition inside $Q$;  beyond $Q$ the
origin is obstructed by a face of $Q$, so that such $x$ do not lie
in the $V$-cell at $0$. If $x$ does not lie in the open cone over
a simplex in $\CalQ$, but lies in the open cone over a standard
region $R$, then Lemma~\ref{lemma:V-cell-local} describes the
$V$-cell.  It states in particular, that for unobstructed $x$, it
can be determined whether $x$ belongs to the $V$-cell at the
origin by considering only the vertices $w$ that lie in the closed
cone over $R$ (the standard region containing the radial
projection of $x$). In this sense, the intersection of a $V$-cell
with the open cone over $R$ is {\it local\/} to the cone over $R$.
\end{remark}



Let $\CalB_0'$ be the set of triangles $T$ such that at least one
of the following holds:
\begin{itemize}
    \item $T$ is a barrier at the origin, or
    \item $T=\{0,v,w\}$ consists of a diagonal of a quarter in the
    $Q$-system together with one of its anchors.
\end{itemize}

%DCG Lemma 5.29, page 50.
\begin{lemma} [Decoupling Lemma]\label{lemma:V-cell-local}
%Let $x\in I_0$, the cube of side $4$ centered at the origin
%parallel to coordinate axes.  
Assume that the closed segment
$\{x,w\}$ intersects the closed $2$-dimensional cone with center
$0$ over $F=\{0,v_1,v_2\}$, where $F\in\CalB'_0$. Assume that the
origin is not obstructed at $x$. Assume that $x$ is closer to the
origin than to both $v_1$ and $v_2$. Then $x\not\in\op{VC}(w)$.
\end{lemma}
\index{decoupling lemma}

\begin{remark}  The Decoupling Lemma is a crucial result.  It
permits estimates of the scoring function in
\Chap~\ref{sec:scoring} to be made separately for each standard
region.  The estimates for separate standard regions are far
easier to come by than estimates for the score of the full
centered packing.  Eventually, the separate estimates for each
standard will be reassembled with linear programming techniques in
\Chap~\ref{XX}.
\end{remark}

\begin{proof}
Assume for a contradiction that $x$ lies in $\op{VC}(w)$. In
particular, we assume that $w$ is not obstructed at $x$.  Since
the origin is not obstructed at $x$, $w$ must be closer to $x$
than $x$ is to the origin.

By Lemma~\ref{tarski:decouple}, 
   $|w|,|w-v_1|,|w-v_2|\le 2t_0$.  Thus, $Q=\{0,w,v_1,v_2\}$ is
a quarter or a quasi-regular tetrahedron.  By the definition of
$\CalB'_0$, the face $F$ must lie in the $Q$-system.  Thus,
$F$ is a barrier.  Again, by Lemma~\ref{tarski:decouple},
$\op{conv}(F)$ meets the segment from $x$ to $w$, so $x$ is obstructed
at $w$.  Thus, $x$ does not lie in $\op{VC}(w)$.
\end{proof}

%\section{Local Optimality}%DCG Sec. 8, p72  
%\label{sec:local-opt}
%Moved after the main estimate.




\section{Overview}%DCG 9.1, p95
    \label{sec:fine-overview}
    \oldlabel{2.1}



In this \chap, we define a decomposition of a $V$-cell. Let
$\op{VC}$ be the $V$-cell at the origin.  For any $t > 0$, let
$V(t)$ be the intersection of $\op{VC}$ with the ball $B(0,t)$ at
the origin of radius $t$. We write $\op{VC}$ as the disjoint union
of $V(t_0)$ and its complement $\delta$.

Assume that there is an upright 
diagonal $\{0,v\}$. 
We will define $\delta_i(v)\subset\delta$, where $i$ runs
over some finite indexing set.
The sets $\delta_i(v)$ will be defined so as not to overlap
one another. 

We will define a set $\CalS$ of simplices, each having a vertex at
the origin. (The letter `$\CalS$' is for simplex.)
The vertices of the simplices will be vertices of the
packing, and their edges will have length at most $2\sqrt{2}$. The
sets $\op{cone}^0(0,S)$, for distinct $S\in\CalS$, will not overlap. Over a
simplex $S\in\CalS$, the $V$-cell will be truncated at a radius
$t_S\ge t_0$. After defining the constants $t_S$, we will set
    $$V_S(t_S) = \op{cone}^0(0,S) \cap V(t_S) =\op{cone}^0(S)\cap B(t_S)\cap \op{VC}(0).$$
That is, $V_S(t_S)$ is the part of the $V$-cell at the origin,
contained in the cone over $S$ and in the ball of radius $t_S$.
If
    $\op{VC}(0) \cap \op{cone}^0(S)\subset B(t_S)\subset B(t'_S)$,
then
    $V_S(t_S)=V_S(t'_S)$.

Since $t_S\ge t_0$, the sets $V_S(t_S)$ and $\delta$ may meet at
interior points. Nevertheless, we will show that $V_S(t_S)$ does
not meet the interior of any $\delta(v)$.  Let $\tildeV(t_0)$ be
the set of points in $V(t_0)$ that do not lie in $\op{cone}^0(S)$,
$S\in\CalS$. We will derive an explicit formula for the volume of
$\tildeV(t_0)$.

In $\op{VC}(0)$, there are nonoverlapping sets
$$\delta(v),\quad   V_S(t_S),\quad \tildeV(t_0).$$
Let $\delta'$ be the complement in $\op{VC}(0)$ of the union of
these sets. These sets give a decomposition of $\op{VC}(0)$.
Corresponding to this decomposition is a formula for $\sigma(v,\Lambda)$
of the form
    $$
    \sigma(v,\Lambda) =
        \op{sovo}(\tildeV(t_0),\lambda_{oct})
        + \sum_{\CalS}\op{sovo}(V_S(t_S),\lambda_{oct})
        -\sum_{v,i} 4\doct\op{vol}(\delta_i(v))
        -4\doct\op{vol}(\delta').
    $$
Since $\op{vol}(\delta')\ge0$, we obtain an upper bound on
$\sigma(v,\Lambda)$ by dropping the rightmost term.



\section{The set $\delta(v)$}%DCG 9.2, p86
    \label{sec:deltaP}
    \oldlabel{2.3}



\begin{definition}\label{def:eta0}
Set $\eta_0(h)=\eta(2h,2,2t_0)$.
\index{zzeta@$\eta_0$}
\end{definition}

%By Lemma~\ref{tarski:1453}, 
%if $h\le\sqrt2$, then $\eta_0(h)\le \eta_0(\sqrt2) <
%1.453$.\index{ZZZZ1.453@1.453}



Let $v$ be a vertex with $2t_0 < |v| <\sqrt8$.
Let $D_0 = \op{rcone}^0(0,v,|v|/(2\eta_0(|v|/2)))$.
Let $v_1,\ldots,v_k$ be the anchors around $\{0,v\}$ indexed
cyclically. The half planes $A_i=\op{aff}_+(\{0,v\},v_i)$
slice $\ring{R}^3$ into $k$ open wedges
$W_i$, between
    $A_i$ and $A_j$,
where $j\equiv i+1\mod k$, so that
    $D\ring{R}^3\setminus (A_1\cup\cdots\cup A_k) =\cup W_i$.

\begin{definition}\label{def:wedge}
Let $\CalW=\CalW(0,v)$ be the set of wedges $W=W_i$ along $(0,v)$ 
such that either
\begin{enumerate}
    \item The azimuth angle of $W$ (along $\{0,v\}$) is at least $\pi$.
    \item The azimuth angle of $W$ is less than $\pi$, 
 $|v_i-v_j|\ge 2.77$,
    $\rad(0,v,v_i,v_j)\ge\eta_0(|v|/2)$, and the
    circumradius of $\{0,v_i,v_j\}$ or $\{v,v_i,v_j\}$ is
    $\ge\sqrt2$.
    \label{enum:wedge2}
\end{enumerate}
\index{wedge}
\index{W@$\CalW$}
\end{definition}

(If $(0,v)$ has only one or two anchors, then it is understood
 that $W$ is all of $\ring{R}^3$ or all of $\ring{R}^3$, except a half-plane.)
Fix $i,j$, with $j\equiv i+1\mod k$. If $W = W_i$ is a wedge in
$\CalW$, let $\{0,v_i,v\}^\perp$ be the plane through the origin and
the circumcenter of $\{0,v_i,v\}$, perpendicular to $\{0,v_i,v\}$.
Skip the following step if the circumradius of $\{0,v_i,v\}$ is
greater than $\eta_0(|v|/2)$, but if the circumradius is at most
this bound, the plane
of $\{0,v_i,v\}^\perp$
intersects the right circular cone boundary of $D_0$ along two rays
emanating from $0$.  Let $c_i$ be a point on the ray (selected on
the $W$-side of $\{0,v_i,v\}$).  Simlarly, we construct the point
$c_i'$ for $\{0,v,v_j\}$ (again on the $W$-side).

Define $\theta=\theta(v)$ by $\cos\theta = |v|/(2\eta_0(|v|/2))$.
If Condition~2 holds, we let $c$ be the 
circumcenter of $\{0,v_i,v_j,v\}$.  The angle
at $0$ between $c$ and $v$ is
$\theta'$, where
$$\cos\theta' = |v|/(2\rad)\le |v|/(2\eta_0) = \cos\theta.$$
We conclude that $\theta'\ge\theta$ and $c$ does not lie in $D_0$.
Thus, the half-planes
   $$
   A_i,\quad B_i=\op{aff}_+(\{0,v\},c_i),\quad 
   B'_i = \op{aff}_+(\{0,v\},c'_i), \quad
   A_j
   $$
are ordered cyclically around $\{0,v\}$. (Set $B_i=A_i$
or $B'_i=A_j$, if the corresponding circumradius is greater than
$\eta_0(|v|/2)$.)
Let $W'=W'_i$ be the open wedge of $D_0$ between $B_i$ and $B'_i$.
Let
    $$E_w = \{x : 2 x\cdot w \le w\cdot w\},$$
for $w = v,v_i,v_j$. These are half-spaces bounding the Voronoi
cell. Set $E_\ell = E_{v_\ell}$.

\begin{definition} \label{def:delta-e}
In both cases (Conditions~1 and~2), let $W$ be the wedge between
$v_i$ and $v_j$ along $(0,v)$, $W'$ the smaller wedge, 
and let $c=\eta_0(|v|/2)$ in
    $$
    \begin{array}{lll}
      \BigD'(v,W) &= [E_v\cap W'\cap D_0]\cup \op{rog}^0(0,v,v_i,p_i,c)
      \cup \op{rog}^0(0,v,v_j,p_j,c)
      \\
    \BigD(v,W) &= \BigD'(v,W) \cup 
    \op{rog}^0(0,v_i,v,p_i,c)
    \cup \op{rog}^0(0,v_j,v,p_j,c)\\
    \bigd(v,W) &= \{x\in\BigD(v,W)\mid |x|>t_0\},
    \end{array}
    $$
where $p_i,p_j$ are selected so that the simplices $\op{rog}^0$ lie
in the wedge $W$.
\index{zzdelta@$\bigd(v,W)$}
\index{zzDelta@$\BigD(v,W)$}
\end{definition}

\begin{remark} We note that the union is actually a disjoint union,
and that each of the pieces is one of the primitive regions, so
the volume of $\BigD$ is immediate.
\end{remark}

\begin{figure}[htb]
  \centering
  \myincludegraphics{\ps/diag46.ps}
  \caption{$\BigD(v,W)$ lies in a cone. The intersection of
    that cone with the unit sphere is the shaded region.}
  \label{fig:anchor-quarter}
\end{figure}

\begin{remark}
Recall that in the definition of Rogers simplex (Definition~\ref{def:rog}, 
it is defined to be the
empty set 
if the corresponding parameters are not coherent.  It is to
be understood that the all discussion regarding this set
can (and should) be disregarded in the case that it is empty.
\end{remark}

We present a series of lemmas that explore the geometry of the sets
$\BigD(v,W)$.



\begin{lemma}\label{lemma:new-anchor}
  Let $S=\{0,v,w,u\}$ be a simplex.  Assume that $\{0,v\}$ is an
upright diagonal, that $w$ and $u$
are anchors of $\{0,v\}$, and that $\rad(S)< \eta_0(|v|/2)$.
Assume there is a wedge $W$ of $\CalW$ along the face $\{0,v,w\}$
(on the same side of the face as $u$).  
Then there exists an anchor $w'$ of $\{0,v\}$ between $u$ and $w$
(that is, in $\op{aff}_+^0(\{0,v\},\{u,w\})$) with
    $|w'-w|\ge2.77$ and
    $\rad\{0,w,w',v\} \ge \eta_0(|v|/2)$.
\end{lemma}

\begin{proof} The conditions on $S$ are incompatible with the
conditions of Definition~\ref{def:wedge} defining wedges.
Therefore, $w$ and $u$ cannot be consecutive anchors around
$\{0,v\}$.  Let $w'$ be the anchor such that $S'=\{0,v,w,w'\}$
determine the wedge $W$.  Since $w$ and $w'$ are consecutive anchors,
we see that $w'$ must be between $u$ and $w$.  It must have the
second type in Definition~\ref{def:wedge}.  The conclusion follows.
\end{proof}






In the following lemmas, we adopt a uniform notation.
$\{0,v\}$ is always be an upright diagonal: $|v|<\sqrt8$.
$W$ is  a wedge along $(0,v)$
between two anchors $w$ and $w'$.  Set $R_w=\op{rog}^0(0,w,v,p,\eta_0(|v|/2))$,
where $p$ is selected so that $R_w$ lies in $W$.


The following definition will be used briefly, then discarded.
Its purpose is to group together a few separate cases in the next
few lemmas.

\begin{definition}  Let $(0,v,u_1,u_2)$ be a four-tuple of vertices
in $\ring{R}^3$.  We say that it is {\it normal} if $2t_0<|v|<\sqrt8$
and one of the
following holds:
\begin{enumerate}
  \item (qrt) $\{0,u_1,u_2\}$ is a quasi-regular triangle.
  \item (upright) $\{0,u_1,u_2\}$ is an upright triangle; that is, say,
    $2t_0 < |u_1| < \sqrt8$, and $u_2$ is an anchor of $u_1$.
  \item (flat)
   $2t_0<|u_1-u_2|<\sqrt8$; and if $u_1,u_2$ are both anchors $(0,v)$, 
   then
    no further anchor of $\{0,v\}$
   lies in the lune $\op{aff}^0_+(\{0,v\},\{u_1,u_2\})$.
\end{enumerate}
\end{definition}

\begin{lemma}\label{lemma:BigD-}  Let $S=(0,v,u_1,u_2)$ be normal.
Then
$\BigD'(v,W)$ does not meet $F=\op{cone}(0,\{u_1,u_2\})$.
\end{lemma}

\begin{proof}  By elementary geometry\tarf{tarski:eps-bigd-}, 
the hypotheses imply
that $S$ is normal of the flat or qrt variety, and 
that $u_1$ and $u_2$ are anchors.  The lemma also
gives that 
$\rad(S)<\eta_0(|v|/2)$.    
In the qrt case, $S$ forms
an upright quarter.  By elementary geometry\tarf{tarski:consec-anchors}, $u_1$ and $u_2$
are consecutive anchors. In the flat case as well, 
$u_1$ and $u_2$ are consective
anchors around $\{0,v\}$.

If the wedge $W$ is not between
the anchors $u_1$ and $u_2$, then $\BigD$ lies outside the 
lune $\op{aff}_+^0(\{0,v_0\},\{u_1,u_2\})$, but $F$ lies inside it.  Thus,
the wedge must run between $u_1$ and $u_2$.   This contradicts 
the rules for forming wedges.  (See
Lemma~\ref{lemma:new-anchor}, which implies that
anchors $u_1$ and $u_2$ cannot be consecutive.)
\end{proof}

\begin{lemma}\label{lemma:fine-Rw}
Let $S=(0,v,w,u_1)$ be normal.
Suppose that $w$ is an anchor of $\{0,v\}$.
Then
$R_w$
does not meet $F=\op{cone}(0,\{u_1,u_2\})$.
\end{lemma}

\begin{proof}  We can use the same proof as in Lemma~\ref{lemma:BigD-},
except with one lemma\tarf{tarski:eps:fine:Rw} substituted for 
another\tarf{tarski:eps-bigd-}.
\end{proof}

\begin{lemma}\label{lemma:fine:barrier}  
Let $S=\{0,v,w,u\}$ be a set of four distinct vertices
in the packing.  Assume that $\{0,v\}$ is an upright diagonal of
a quarter in the $Q$-system and that $w$ is an anchor of $\{0,v\}$.
Let $R_w$ be as above; that is, a Rogers simplex attached to a wedge
$W\in \CalW$ around the diagonal $\{0,v\}$.
Assume $x\in R_w$ satisfies $\epsilon_0(x,\{v,w,u\})= u$.
Then there exists a vertex $w'$ that is an anchor of $\{0,v\}$ such
$b=\{v,w',0\}$ is a barrier and $x$ is obstructed from $u$
by $b$.
\end{lemma}

\begin{proof} Assume for a contradiction that $\epsilon_0(x)=u$.
We separate the proof into two cases, depending on whether
$x$ and $u$ lie on the same side of $A=\op{aff}(v,w,0)$.

Assume that $x$ and $u$ lie on opposite sides of $A$.
For any nonzero vertices $v',v''$,
Let $L(v',v'')$ be the line of points equidistant from $\{0,v',v''\}$.
The three lines $L(u,v)$, $L(v,w)$, $L(u,w)$ meet at the circumcenter
$c$ of $S$.  The rays $L^+(u,v)$, $L^+(v,w)$, $L^+(u,w)$ demarcate
the regions between $\epsilon_0=w,u,v$.  (Pick the direction
of ray so that it runs through the circumcenter of the face $\{0,v',v''\}$
if the remaining vertex has positive orientation, and so that it
runs in the opposite direction otherwise.)

If $u$ has positive orientation in $S$, then $L^+(v,w)$ runs through
the circumcenter of $\{0,v,w\}$ and along an edge of $R_w$.
The point $w/2$ in the closure of  $R_w$ also has $\epsilon_0=w$.
It follows that $\epsilon_0$ has value $w$ on $R_w$, which is contrary
to our assumption.

Thus, $u$ has negative orientation in $S$.  
This implies that $|u-v|,|u-w|,|u|\le 2.51$.  In particular,
$S$ is a quarter.  Since it has the same diagonal as a quarter in
the $Q$-system, we have that $S$ is in the $Q$-system, so that
$\{0,v,w\}$ is a barrier.  By Lemma~\ref{tarski:tip-cone}, we have
that $x\in \op{cone}(u,\{v,w,0\})$.  In particular, $x$ is obstructed
from $u$ by the barrier $\{v,w,0\}$.  Take $w'=w$ in this case.

Now assume that $x$ and $u$ lie on the same side of $A$.
First consider the special case where
we also have that $\rad_V(S)\ge \eta_0(|v|/2)$ and that that
the orientation of $u$ is non-positive in $S$.  In this case, it
follows that $S$ is a quarter in the $Q$-system.  By the rule
for constructing wedges $W\in\CalW$, there is no $R_w$ along $\{0,v,w\}$
in this case.  Next consider the case where
$\rad_V(S)\ge \eta_0(|v|/2)$ and the orientation of $u$ is positive
in $S$.  In this case, the ray $L^+(v,w)$ runs along the edge of 
$R_w$ as before, and we see that $\epsilon_0=w$ on $R_w$.

Finally consider the case where $\rad_V(S)<\eta_0(|v|/2)$.
It follows that $u$ is an anchor of $S$.
By Lemma~\ref{lemma:new-anchor}, there exists a further anchor
$w'$ between $u$ and $w$.  We are in the situation of Lemma~\ref{lemma:prev}.
We have that $\op{conv}\{x,u\}$ meets $\op{conv}\{0,v,w'\}$ and
that $|u-w'|\le 2.51$.  In particular $S'=\{0,v,w',u\}$ is
an upright quarter and $\{0,v,w'\}$ is a barrier.
\end{proof}



\begin{lemma}\label{lemma:fine-Rw:5}
Let $\{0,v,w,u_1,u_2\}$ be a set of five distinct vertices in the
packing.  Assume that $\{0,v\}$ is an upright diagonal of a quarter
in the $Q$-system and that $w$ is an anchor of $\{0,v\}$.
Assume that $S=(0,v,u_1,u_2)$ be normal.
Let $R_w$ be as above; that is, a Rogers simplex attached to a wedge
$W\in \CalW$ around the diagonal $\{0,v\}$.
Then $R_w$ does not meet $F=\op{cone}(0,\{u_1,u_2\})$.
\end{lemma}

\begin{proof}
For a contradiction, assume these sets meet at $x\in R_w\cap F$.
We have $\epsilon_0=\epsilon_0(x,\{w,u_1,u_2\})\in\{w,u_1,u_2\}$.  We consider
two cases depending on whether $\epsilon_0=w$.

Assume that $\epsilon_0=w$.  By Lemma~\ref{tarski:fine:Rw:5},
we have $|w-u_1|\le 2.51$ and $|w-u_2|\le 2.51$.  By the same
lemma, 
there are three possibilities.  The first is that
for either $u=u_1$ or $u=u_2$, we have that $(0,v,w,u)$ is normal
and that $R_w$ meets $\op{cone}(0,\{w,u\})$.  This is contrary to
Lemma~\ref{lemma:fine-Rw:5}.
The second possibility is that
$2.51<|u_1-u_2|<\sqrt8$ and that $v\in\op{cone}^0(0,\{u_1,u_2,w\})$
with $u_1,u_2,w$ all anchors of $\{0,v\}$.  By the normality
hypothesis, these are the only anchors of $\{0,v\}$.  None of the
corresponding wedges $W$ satisfy the conditions to belong to
a wedge of $\CalW$ around $\{0,v\}$.  Thus, this case does not
occur. The third possibility is that $w\in \op{aff}_+(\{0,v\},\{u_1,u_2\})$
and $2.51<|u_1-u_2|$.  This is contrary to the normality
condition on $S$.

In the remaining case, we have $\epsilon_0=u\in\{u_1,u_2\}$.  
Let $S'=\{0,v,w,u\}$.  By Lemma~\ref{lemma:fine:barrier}, there
exist a barrier $b=\{0,v,w'\}$ such that $x$ is obstructed from
$u$ by $b$.  However, if $x\in\op{cone}(0,\{u_1,u_2\})$, then
there exists no such obstruction.  Thus, the intersection
is empty.
\end{proof}

\begin{lemma}\label{lemma:delta-tri}
Let $F=\{0,u_1,u_2\}$ be a quasi-regular triangle.  Let $\{0,v\}$ be
an diagonal. Assume that there
exists a quarter in the $Q$-system along $\{0,v\}$.  Then
$\BigD(v,W)$ does not meet $C=\op{cone}(0,\{u_1,u_2\})$.
\end{lemma}

\begin{proof}
We have separated $\BigD^-(v,W)$ and $R_w$ from $C$.
Together they form $\BigD(v,W)$.
\end{proof}

XX MOve to a section on standard regions.

\begin{corollary}
Each $\BigD(v,W)$ lies entirely in the cone over the standard
region that contains $\{0,v\}$.
\end{corollary}

\begin{proof}
The cone over a standard region is bounded by the cones  over the
quasi-regular triangles.
\end{proof}


\begin{lemma}\label{lemma:delta-flat}
Let $F=\{0,u_1,u_2\}$ be a triangle.  Assume that $|u_1|\le 2t_0$,
$|u_2|\le 2t_0$, and $2t_0\le|u_1-u_2|\le\sqrt8$.  Let $\{0,v\}$
be the diagonal of an upright quarter in the $Q$-system.  Assume
that if $u_1$ and $u_2$ are both anchors of $v$, then they are
consecutive anchors around $v$. Under these conditions, the set
$\BigD(v,W)$ does not overlap the cone at $0$ over the triangle
$F$.
\end{lemma}

\begin{proof} The proof is identical to that of
Lemma~\ref{lemma:delta-tri}. 
\end{proof}

\begin{lemma}\label{lemma:delta-upright}
Let $F=\{0,u_1,u_2\}$ be a triangle.  Assume that $2t_0\le|u_1|\le
\sqrt8$, $2\le|u_2|\le 2t_0$, and $2\le|u_1-u_2|\le2t_0$.  Let
$\{0,v\}$ be the diagonal of an upright quarter in the $Q$-system.
Under these conditions, the set $\BigD(v,W)$ does not overlap
the cone at $0$ over the triangle $F$.
\end{lemma}

\begin{proof}
The proof is identical to that of Lemma~\ref{lemma:delta-tri}.
\end{proof}


\begin{lemma}
Let $\{0,v\}$ be an upright diagonal of a quarter in the
$Q$-system.   If $x$ lies in the interior of $\BigD(v,W)$,
then $x$ is unobstructed at $0$.
\end{lemma}

\begin{proof} For a contradiction, assume that $x$ is obstructed
at $0$ by barrier $T =\{u_1,u_2,u_3\}$.


The convex hull of $T$ can be partitioned into three sets $T(i)$
depending on which vertex of $T$ is closest to a given point in
the convex hull. (Ties can be resolved in any consistent manner.)
Let $y\in \BigD(v,W)$ be the point in the convex hull of $T$ on
the segment from $0$ to $x$.  Fix $i$ so that $y\in T(i)$. If
$v=u_i$, then each point $y$ of $T(i)$ is closer to $v$ than to
$0$.  But each point of $\BigD(v,W)$ is closer to $0$ than to
$v$.  So $x$ is not obstructed by $T$ at $0$.

We may now assume that $v\ne u_i$.

Partition $\ring{R}^3$ geometrically into three sets $V(u_i)$,
$V(0)$, $V(v)$ according to which of $\{u_i,0,v\}$ a point
$z\in\ring{R}^3$ is closest to.  (Again resolve ties in any
consistent manner.)

Assume further that $\max_j u_j \ge 2t_0$. This implies that $y\in
T(i) \subset V(v) \cup V(u_i)$.  On the other hand, we have by
construction that $y\in \BigD(v,W) \subset V(0)$.  (There are
two cases involved in this conclusion, depending on whether $u_i$
is an anchor of $\{0,v\}$.)  However, the sets $V(\cdot)$ are
disjoint; and we reach a contradiction.  Thus, under these
assumptions, $x$ is unobstructed at $0$.

Next assume that $\max_j u_j < 2t_0$.  Let $S=\{0,u_1,u_2,u_3\}$.
Since $T$ is a barrier, $S\in\CalQ_0$.  By assumption, $\{0,v\}$
is a diagonal of an upright quarter in $\CalQ_0$.  By the fact
that the interiors of quarters in $\CalQ_0$ do not meet, we see
that $v$ is not enclosed over $S$.  The set $\BigD(v,W)$ has
a star convexity with respect to the ray from $0$ through $v$.
Thus, if $\BigD(v,W)$ intersects the convex hull of
$T$ at $y$, then $\BigD(v,W)$ intersects the cone over a face
$\{0,u_1,u_2\}$ of $S$ at $y'$. (We can take $y'/|y'|$ to lie on
the cone generated by the arc running from $v/|v|$ to $y/|y|$.
This is impossible by Lemmas~\ref{lemma:delta-tri} and
\ref{lemma:delta-flat}.
\end{proof}

\begin{lemma}  Let $\{0,v\}$ be the upright diagonal of a quarter
in the $\CalQ_0$-system.  Then $\BigD(v,W)$ is
a subset of $\op{VC}(0)$.
\end{lemma}

\begin{proof}
We begin by showing that $\BigD^-(v,W)\subset\op{VC}(0)$.
Suppose to the contrary, that a point $x$ in the interior of
$\BigD^-$ lies in $\op{VC}(w)$, with $w\ne0$.  Then $x$ is closer
to $w$ than to  $0$.  Thus, $\eta(0,v,w)<\eta_0(|v|/2)$, and $w$
is an anchor of $\{0,v\}$.  The face $E_w$ in the construction
$\BigD^-(v,W)$ prevents this from happening.

Now consider a point $x$ of $R_w$, which we assume to lie in
$\op{VC}(u)$, with $u\ne0$.  To avoid a trivial case, we may
assume that $w\ne u$.

Assume that the orientation of $S=\{0,v,w,u\}$ is negative along
the face $\{0,v,w\}$.  Then $S$ must be an upright quarter.  By
the construction of wedges $W\in\CalW$, we have that $R_w$ must
lie on the opposite side of the plane $\{0,v,w\}$ from $u$ (for
there is no wedge between the anchors of an upright quarter).  The
result now follows from Lemma~\ref{lemma:back}.

If $\rad(S) <\eta_0(|v|/2)$, then $u$ and $w$ are anchors.  In
this case, the result follows from Lemma~\ref{lemma:prev}.

Finally if the orientation is positive and if $\rad(S)\ge
\eta(|v|/2)$, then a point of $R_w$ cannot be closer to $u$ than
to $0$.
\end{proof}


\section{Overlap}%DCG 9.3, p93
    \label{sec:overlap}
    \oldlabel{2.4}


\begin{lemma}  The sets $\BigD(v,W)$ do not overlap one another.
\end{lemma}

\begin{proof}
This is clear for two sets around the same vertex $v$.  Consider
the sets $\BigD(u,W(u))$ and $\BigD(v,W(v))$ at $u$ and $v$.

To treat the points in $\BigD^-(u,W(u))$ and $\BigD^-(v,W(v))$, we
may contract $\{u,v\}$ until $|u-v|=2$.  By the constraints on the
edges of $\{0,u,v\}$, the circumcenter $c$ of this triangle lies
in the convex hull of the triangle.  We have $\eta(0,u,v)\ge
\eta_0(|v|/2)$ and $\eta(0,u,v)\ge\eta_0(|u|/2)$.  So the plane
through $\{0,c\}$ perpendicular to the plane $\{0,u,v\}$ separates
$\BigD^-(u,W(u))$ from $\BigD^-(v,W(v))$.

Next we separate points in $\BigD^-(u,W(u))$ from points of
$R_w^{(v)}$, where $w$ is an anchor of $v$ and $u\ne v$.  Let
$S=\{0,u,v,w\}$. The orientation of $S$ along $\{0,v,w\}$ is
positive.  The circumradius of $S$ satisfies
    $$
    \rad(S) \ge \eta(0,u,v)>\eta_0(|v|/2).
    $$
Thus, $\epsilon_0(S,\cdot)$ takes different values on
$\BigD^-(u,W(u))$ and $R_w^{(v)}$, so that the sets are disjoint.

Next we separate points of $R_w^{(v)}$ from $R_w^{(u)}$.  (Notice
that we assume that the anchor is the same for the two Rogers
simplices.) Let $S=\{0,u,v,w\}$.   As above, we have
    $$
    \rad(S) \ge \eta_0(|v|/2), \quad \eta_0(|w|/2).
    $$
The simplex $S$ has positive orientation along the faces
$\{0,u,w\}$ and $\{0,v,w\}$.  Let $c_u$ be the circumcenter of
$\{0,u,w\}$, let $c_v$ be the circumcenter of $\{0,v,w\}$, and let
$c$ be the circumcenter of $S$.  Then $R_w^{(v)}$ lies in the
convex hull of $\{0,w,c_v,c\}$, but $R_w^{(u)}$ lies in the convex
hull of $\{0,w,c_u,c\}$.  Thus, the sets are disjoint.

Finally, we separate points of $R_w^{(u)}$ from points of
$R_{w'}^{(v)}$, where $w\ne w'$ and $u\ne v$.  If the function
$\epsilon_0(\{0,w,w'\},\cdot)$ separates the sets, we are done.
Otherwise, we may assume say that $\epsilon_0(\{0,w,w'\},x) = w'$
from some $x\in R_w^{(u)}$.  Let $S=\{0,u,w,w'\}$.

If $w'$ is not an anchor of $u$, then $\rad(S) \ge\eta_0(|u|/2)$
and the orientation of $S$ along $\{0,w,u\}$ is positive.  In this
case, we have $\epsilon_0 = w$ on $R_w^{(u)}$, which is contrary
to assumption. Thus, we may assume that $w'$ is an anchor of $u$.

If the orientation of $\{0,u,w,w'\}$ is negative along $\{0,w,u\}$,
then $\{0,u,w,w'\}$ is a quarter, contrary to the existence of $W\in
\CalW$.  So the orientation is positive.  If $\rad(\{0,u,w,w'\}) <
\eta_0(|u|/2)$, then Lemma~\ref{lemma:prev} implies that each point
of $R_w$ is obstructed from $w'$.  But no point of $R_{w'}^{(v)}$ is
obstructed from $w$. (In fact, a barrier that crosses
$\BigD(v,W(u))$ is inconsistent with Lemmas~\ref{lemma:delta-tri},
\ref{lemma:delta-flat}, \ref{lemma:delta-upright}.) So
$\rad(\{0,u,w,w'\}) \ge \eta_0(|u|/2)$.  This is contrary to
$\epsilon_0(\{0,w,w'\},x) = w'$ from some $x\in R_w^{(u)}$.
\end{proof}



\section{The $\CalS$-system defined}%DCG 9.4, p94
    \oldlabel{2.5}

We consider three types of simplices $A$, $B$, $C$.  Each type has
its vertices at vertices of the packing.  The edge lengths of
these simplices are at most $2\sqrt{2}$.

$A$.  This family consists of simplices $S(y_1,\ldots,y_6)$ whose
edge lengths satisfy
    $$
    y_1,y_2,y_3\in[2,2t_0],\quad
    y_4,y_5\in[2t_0,2.77],
    \quad
    y_6\in[2,2t_0],\quad \text{and }
    \eta(y_4,y_5,y_6)<\sqrt{2}.
    $$
(These conditions imply $y_4,y_5<2.697$, because
$\eta(2.697,2t_0,2)>\sqrt2$.)

$\SB$.  This family consists of certain flat quarters that are
part of an isolated pair of flat quarters. It consists of those
satisfying $y_2,y_3\le 2.23$, $y_4\in[2t_0,2\sqrt{2}]$.

$\SC$.  This family consists of certain simplices
$S(y_1,\ldots,y_6)$ with edge lengths satisfying
    $y_1,y_4\in[2t_0,2\sqrt{2}]$, $y_2,y_3,y_5,y_6\in[2,2t_0]$.
We impose the condition that the first edge is the diagonal of
some upright quarter in the $Q$-system, and that the upper
endpoints of the second and third edges (that is, the second and
third vertices of the simplex) are consecutive anchors of this
diagonal. We also assume that $y_4< 2.77$, or that both face
circumradii of $S$ along the fourth edge are less than $\sqrt{2}$.

\begin{lemma}
    \label{lemma:2.77}
If a vertex $w$ is enclosed over a simplex $S$ of type $A$, $\SB$,
or $\SC$, then its height is greater than $2.77$.  Also, $\{0,w\}$
is not the diagonal of an upright quarter in the $Q$-system.
\end{lemma}

\begin{proof}
In case $A$, $\eta(y_4,y_5,y_6)<\sqrt{2}$, so an enclosed vertex
must have height greater than $2\sqrt{2}$.  It is too long to be
the diagonal of a quarter.

In case $\SB$, we use the fact that the isolated quarter does not
meet in the interior with any quarter in the $Q$-system. 
By Lemma~\ref{tarski:enclosed-v}, an
enclosed vertex has length at least $2.77$.
By the symmetry of isolated quarters, this means that the diagonal
of a flat quarter must also be at least $2.77$.

In case $\SC$, the same calculation gives that the enclosed vertex
$w$ has height at least $2.77$.  Let the simplex $S$ be given by
$\{0,v,v_1,v_2\}$, where $\{0,v\}$ is the upright diagonal. By
Lemma~\ref{lemma:pass-anchor}, $v_1$ and $v_2$ are anchors of
$\{0,w\}$. The edge between $w$ and its anchor cannot cross
$\{v,v_i\}$ by Lemma~\ref{lemma:2t0-doesnt-pass-through}. (Recall
that two sets are said to {\it cross\/} if their radial
projections overlap.) The distance between $w$ and $v$ is at most
$2t_0$ by Lemma~\ref{lemma:double-face}. If $\{0,w\}$ is the
diagonal of an upright quarter, the quarter takes the form
$\{0,w,v_1,v_3\}$, or $\{0,w,v_2,v_3\}$ for some $v_3$, by
Lemma~\ref{lemma:double-face}. If both of these are quarters, then
the diagonal $\{v_1,v_2\}$ has four anchors $v$, $w$, $0$, and
$v_3$. The selection rules for the $Q$-system place the quarters
around this diagonal in the $Q$-system. So neither $\{0,w,v_1,v_3\}$
nor $\{0,w,v_2,v_3\}$ is in the $Q$-system. Suppose that
$\{0,w,v_1,v_3\}$ is a quarter, but that $\{0,w,v_2,v_3\}$ is not.
Then $\{0,w,v_1,v_3\}$ forms an isolated pair with $\{v_1,v_2,v,w\}$.
In either case, the quarters along $\{0,w\}$ are not in the
$Q$-system.
\end{proof}

\begin{remark}  The proof of this lemma does not make use of all the hypotheses
on $\SC$.  The conclusion holds for any simplex
$S(y_1,\ldots,y_6)$, with $y_1,y_4\in[2t_0,2\sqrt{2}]$,
$y_2,y_3,y_5,y_6\in[2,2t_0]$.
\end{remark}

\section{Disjointness}%DCG 9.5, p95
    \oldlabel{2.6}

Let $S=\{0,v_1,v_2,v_3\}$ be a simplex of type $A$, $\SB$, or
$\SC$. An edge $\{v_4,v_5\}$ of length at most $2\sqrt{2}$ such
that $|v_4|,|v_5|\le 2t_0$ cannot cross two of the edges
$\{v_i,v_j\}$ of $S$.  In fact, it cannot cross any edge $\{v_i,v_j\}$
with $|v_i|,|v_j|\le 2t_0$ by Lemma~\ref{lemma:skew-quad}.  The
only possibility is that the edge $\{v_4,v_5\}$ crosses the two
edges with endpoint $v_1$, with $|v_1|\ge2t_0$ in case $\SC$.  But
this too is impossible by Lemma~\ref{lemma:double-face}.

Similar arguments show that the same conclusion holds for an edge
$\{v_4,v_5\}$ of length at most $2t_0$ such that $|v_4|\le2t_0$,
$v_5\le2\sqrt{2}$.  The only additional fact that is needed is
that $\{v_4,v_5\}$ cannot cross the edge between the vertex $v$ of
an upright diagonal $\{0,v\}$ and an anchor
(Lemma~\ref{lemma:2t0-doesnt-pass-through}).





\begin{lemma}
    \label{lemma:no-overlap}
    Consider two simplices $S$, $S'$, each of  type $A$, $\SB$, $\SC$,
or a quarter in the $Q$-system.
    Assume that $S$ and $S'$ do not lie
    in the cone over a quadrilateral region.  Then the interiors
    of
    $S$ and $S'$ do not meet.
\end{lemma}

\begin{proof}
By hypothesis, the standard region is not a quadrilateral, and we
thus exclude the case of conflicting diagonals in a quad cluster.
We claim that no vertex $w$ of $S$ is enclosed over $S'$.
Otherwise, $w$ must have height at least $2t_0$, so that $\{0,w\}$
is the diagonal of an upright in the $Q$-system, and this is
contrary to Lemma~\ref{lemma:2.77}. Similarly, no vertex of $S'$
is enclosed over $S$.

Let $\{v_1,v_2\}$ be an edge of $S$ crossing an edge $\{v_3,v_4\}$ of
$S'$. By the preceding remarks, neither of these edges can cross
two edges of the other simplex. The endpoints of the edges are not
enclosed over the other simplex. This means that one endpoint of
each edge $\{v_1,v_2\}$ and $\{v_3,v_4\}$ is a vertex of the other
simplex.  This forces $S$ and $S'$ to have three vertices in
common, say $0$, $v_2$, and $v_3$.  We have $S=\{0,v_1,v_3,v_2\}$
and $S'=\{0,v_3,v_2,v_4\}$. If
    $|v_2|\in[2t_0,2\sqrt{2}]$,
then we see that the anchors $v_3$, $v_4$ of $\{0,v_2\}$ are not
consecutive.  This is impossible for simplices of type $\SC$ and
upright quarters.  Thus, $v_2$ and $v_3$ have height at most
$2t_0$.  We conclude, without loss of generality, that
    $|v_4|\in[2t_0,2\sqrt{2}]$
and $|v_1-v_2|\ge 2t_0$.

The heights of the vertices of $S$ are at most $2t_0$, so it has
type $A$ or $\SB$, or it is a flat quarter in the $Q$-system. If
$S'$ is an upright quarter in the $Q$-system, then it does not
overlap an isolated quarter or a flat quarter in the $Q$-system,
so $S$ has type $\SA$. By Lemma~\ref{tarski:277}, we have
$|v_1-v_2|>2.77$.  This imposes the contradictory constraints
on $\SA$
    $$
    2.77\ge |v_1-v_2|>2.77.
    $$
Thus $S'$ has type $\SC$.  This forces $S$ to have type $\SA$.  We
reach the same contradiction  $2.77 > 2.77$.
\end{proof}

\section{Separation of simplices of type $\SA$}%DCG 9.6, p96
    \label{sec:separation}
    \oldlabel{2.7}

Let $S = \{0,v_1,v_2,v_3\}$.
Let $\op{cone}^0(S) = \op{cone}^0(0,\{v_1,v_2,v_3\}$.
Let $V_S = \op{VC}(0)\cap \op{cone}^0(S)$, for a simplex $S$ of type $\SA$,
$\SB$, or $\SC$. 
We truncate $V_S$ to $V_S(t_S)$ by intersecting
$V_S$ with a ball of radius $t_S$.  The parameters $t_S$ depend on
the type of $S$.

If $S$ has type $\SA$, we use $t_S=+\infty$ (no truncation).

\begin{lemma} Let $S=\{0,v_1,v_2,v_3\}$ be a simplex of type $\SA$.
There is a null set $E$, such that
we have  $ \Omega(0,S) \cap \op{cone}^0(S) \subset V_S \cup E$.
\end{lemma}

\begin{proof} 
We use the fact that if $b$ is a barrier, then $\op{conv}$ does
not meet $\op{conv}^0(S)$ by Lemma~\ref{XX}.  


Excluding a null set, we may assume 
for a contradiction that
$x\in \Omega(0,S) \cap \op{cone}^0(S) \cap \op{VC}(v)$,
for some $v\ne 0$.  

% ...
By Lemma~\ref{tarski:vor-bar-sqrt2}, $x$ and $0$ lie on the
same side of $\op{aff}\{v_1,v_2,v_3\}$.  Thus, $x$ is in
$\op{conv}^0(S)$.  
Thus, every vertex of $S$ is unobstructed at $x$.  Thus, $x$
is closer to $v$ than to any vertex of $S$.

By Lemma~\ref{tarski:vor-bar-sqrt2}, $\op{conv}\{v_1,v_2,v_3\}$ 
separates
$\Omega(0,S)\cap \op{cone}^0(S)$ from $\Omega(v,\{v,v_1,v_2,v_3\})$ when
$v$ is enclosed over $S=\{0,v_1,v_2,v_3\}$.  This is contrary
to the assumption that $x$ lies in the intersection of these
two sets.

If $\Omega(v,\{0,v_1,v_2\})$ meets $\op{conv}^0(S)$, then
$S'=\{v,0,v_1,v_2\}$ must be a quarter or quasi-regular tetrahedron.
If $x$ is a barrier, then $x\not\in\op{VC}(v)$.  This implies
that $S'$ is a quarter that is not in the $Q$-system.
It
cannot be an isolated quarter because of the edge length
constraint $2.77$ on simplices of type $\SA$.
There must be a
conflicting diagonal $\{0,w\}$, where $w$ is enclosed over $Q$. ($w$
cannot be enclosed over $S$ by results of
Lemma~\ref{lemma:no-overlap}.) This shields the $V$-cell at $v$
from $\op{cone}^0(S)$ by the two barriers $\{0,w,v_1\}$ and $\{0,w,v_2\}$ of
quarters in the $Q$-system.
\end{proof}

\begin{lemma} Let $S=\{0,v_1,v_2,v_3\}$ be a simplex of type $A$.
  $V_S$ is disjoint from all of the set $\BigD(v,W)$.
\end{lemma}

\begin{proof}
This is evident from
Lemmas~\ref{lemma:delta-tri} and \ref{lemma:delta-flat}.
\end{proof}


Our justification that $V_S(t_S)$ can be treated as an
independently scored entity is now complete.

\section{Separation of simplices of type $\SB$}%DCG 9.7, p96
    \oldlabel{2.8}

If $S(y_1,\ldots,y_6)$ has type $\SB$, we label vertices so that
the diagonal is the fourth edge, with length $y_4$. We set
$t_S=1.385$. The calculation in Lemma~\ref{lemma:2.77}
shows that any enclosed vertex over $S$ has height at least
$2.77=2t_S$.

\begin{lemma} Let $S=\{0,v_1,v_2,v_3\}$ be a simplex of type $\SB$.
There is a null set $E$, such that
we have  $ \Omega(0,S) \cap \op{cone}^0(S) \cap B(0,1.385) 
\subset V_S \cup E$.
\end{lemma}

\begin{proof}  As above, assume for a contradiction that there
is a point in 
 $$\Omega(0,S)\cap \op{cone}^0(S) \cap B(0,1.385)\cap \op{VC}(v'),$$
with $v'\ne 0$.
Vertices outside $\op{cone}^0(S)$ cannot reach inside $S$ this way.  In
fact, such a vertex $v'$ would have to form a quarter or
quasi-regular tetrahedron with a face of $S$.  The $V$-cell at
$v'$ cannot meet $\op{cone}^0(S)$ unless it is a quarter that is not in the
$Q$-system. But by definition, an isolated quarter is not adjacent
(along a face along the diagonal) to any other quarters.
\end{proof}

To separate the scoring of $V_S(t_S)$ from the rest of the
standard cluster, we also show that the terms of
Formula~\ref{eqn:3.5}  for $V_S(t_S)$ are represented
geometrically by solids that lie in the cone $\op{cone}^0(S)$.   This
is the purpose of the following lemma.

\begin{lemma} Let $S=\{0,v_1,v_2,v_3\}$ be a simplex of type $\SB$.
The cone $\op{rcone}^0(0,v_1,|v|/2.77)$ does not meet the
cone $\op{cone}(0,\{v_2,v_3\}$.
\end{lemma}

\begin{proof} This is Lemma~\ref{tarski:beta:B}.
\end{proof}

\begin{lemma} $\Omega(0,S) \cap \op{cone}^0(S) \cap B(0,1.385)$
does not meet the sets $\delta(v)$.
\end{lemma}

\begin{proof}
The reasons given in Section~\ref{sec:separation} for the
disjointness of $\delta(v)$ and $V_S(t_S)$ apply to this
situation as well.
\end{proof}


This completes the justification that
$V_S(t_S)$ is an object that can be treated in separation from the
rest of the local $V$-cell.

\section{Separation of simplices of type $\SC$}%DCG 9.8, p97
    \oldlabel{2.9}

If $S(y_1,\ldots,y_6)$ is of type $\SC$, we label vertices so that
the upright diagonal is the first edge.  We use $t_S =+\infty$ (no
truncation).   

\begin{lemma} Let $S=\{0,v_1,v_2,v_3\}$ be a simplex of type $C$.
There is a null set $E$, such that
we have  $ \Omega(0,S) \cap \op{cone}^0(S) \subset V_S \cup E$.
\end{lemma}

\begin{proof}  %% XX Rewrite this proof.
Vertices outside $S$ cannot affect the shape of $V_S(t_S)$.  Any
vertex $v'$ would have to form a quarter along a face of $S$.  If
the shared face lies along the first edge, it is a quarter $Q$ in
the $Q$-system, because one and hence all quarters along this edge
are in the $Q$-system.  The faces of this quarter are then
barriers. If the shared face lies along the fourth edge, then its
length is at most $2.77$, so that the quarter cannot be part of an
isolated pair. If it is not in the $Q$-system, there must be a
conflicting diagonal. The two faces along this conflicting
diagonal of the adjacent pair in the $Q$-system (that is, the pair
taking precedence over $Q$ in the $Q$-system) are barriers that
shield the $V$-cell at $v'$ from $S$.
\end{proof}

The reasons given in Section~\ref{sec:separation} for the
disjointness of $\delta(v)$ and $V_S(t_S)$ apply to simplices of
type $\SC$ as well. This completes the justification that
$V_S(t_S)$ is an object that can be treated in separation from the
rest of the local $V$-cell.

\section{Simplices of type $\SCp $}%DCG 9.9, p97
    \oldlabel{2.10}

We introduce a small variation on simplices of type $\SC$, called
type $\SCp $.  We define a simplex $\{0,v,v_1,v_2\}$ of type $\SCp $
to be one satisfying the following conditions.
    \begin{enumerate}
    \item The edge $\{0,v\}$ is an upright diagonal of an upright quarter
        in the $Q$-system.
    \item $|v_2|\in[2.45,2t_0]$.
    \item $v_1$ and $v_2$ are anchors of $v$.
    \item $|v-v_2|\in [2.45,2t_0]$.
    \item The edge $\{v_1,v_2\}$
    is a diagonal of a flat quarter with face $\{0,v_1,v_2\}$.
    \end{enumerate}

It follows that $v_1$ and $v_2$ are consecutive anchors of
$\{0,v\}$.

On simplices $S$ of type $\SCp $, we label vertices so that the
upright diagonal is the first edge.  We use $t_S=+\infty$ (no
truncation).  

Simplices of type $\SCp $ are separated from quarters in the
$Q$-system and simplices of types $\SA$ and $\SB$ by procedures
similar to those described for type $\SC$.  The following lemma is
helpful in this regard.


\begin{lemma}\label{lemma:C'Q}
 The flat quarter along the face $\{0,v_1,v_2\}$ is
in the $Q$-system.
\end{lemma}

\begin{proof}
By Lemma~\ref{tarski:245}, there cannot be an enclosed vertex
of height at most $\sqrt2$. 
So nothing is enclosed over the flat quarter.
By Lemma~\ref{tarski:245bis}, there cannot be an edge of length
at most $2\sqrt2$ that crosses inside the slice.
(XX slice has not yet been defined.) 
This implies that the flat quarter does not have
a conflicting diagonal and is not part of an isolated pair.
\end{proof}


\begin{lemma}
Suppose that $v'$ is enclosed over $S$.  Then $\op{VC}(v')$ does
not meet $\op{conv}^0(S)$.
\end{lemma}

\begin{proof} If there is a point $x$ of intersection, then
$x$ is closer to $v'$ than to any point of $S$. 
By Lemma~\ref{tarski:vor-bar-quad}, this implies that 
$S'=\{v',v,v_1,v_2\}$ is a quarter.  By Lemma~\ref{lemma:C'Q},
$S'$ is in the $Q$-system.  Thus, $\{v,v_1,v_2\}$ is a barrier,
and $x$ is obstructed from $v'$.
\end{proof}


Unlike the other cases, there can in fact be overlap between
$\BigD(v,W)$ and simplex of type $\SCp$, when the upright
diagonal of the simplex is $\{0,v\}$.  This is because the
conditions defining a wedge $W\in\CalW$ are not incompatible with
the conditions defining type $\SCp$.  Nevertheless, except in the
obvious case where the simplex of type $\SCp$ and the wedge are both
constructed between the same consecutive anchors of $\{0,v\}$, there
can be no overlap of a $\BigD(v,W)$ with a simplex of type
$\SCp$.





The construction of the decomposition of the $V$-cell $\op{VC}(0)$
is now complete. It consists of the pieces

    \begin{itemize}
    \item $\delta(v)=\delta_i(v)$,
         for each diagonal $\{0,v\}$ of an upright quarter
        in the $Q$-system, and $i$ as in Definition~\ref{def:wedge}.
    \item truncations of Voronoi pieces $V_S(t_S)$ for simplices of type
        $\SA$, $\SB$, or $\SC$ (and on rare occasion $\SCp$),
    \item $\tildeV(t_0)$, the truncation at $t_0$ of all parts of
        $\op{VC}(0)$ that do not lie in any of the cones $\op{cone}^0(S)$ over
        simplices
        of type $\SA$, $\SB$ or $\SC$,
    \item $\delta'$, the part not lying in any of the preceding.
    \end{itemize}





\chapter{From Sphere Packings to Hypermaps}
%\chapter{Basic Properties of Standard Regions}%DCG Sec.10, p99
    \label{sec:intro}
    \oldlabel{1}
\label{chapter:VQ}




\section{Standard Regions}

%% XX Bring in hypermap stuff here to make it rigorous.

\begin{definition} \label{def:arc} For every pair of vertices
$v_1$, $v_2$ such that $\{0,v_1,v_2\}$ is a quasi-regular
triangle,  draw a geodesic arc on the unit sphere with endpoints
at the radial projections of $v_1$ and $v_2$. These arcs break the
unit sphere into regions called {\it standard regions}, as
follows. Take the complement of the union of arcs inside the unit
sphere.  The closure of a connected component of this complement
is a \index{standard region} standard region. We say that the
standard region is triangular \index{triangular!standard region}
if it is bounded by three geodesic arcs, and say that it is
non-triangular otherwise.
%
\end{definition}

\begin{definition}
A standard region is said to be {\it exceptional\/} if it is not a
triangle or a quadrilateral.  The pair $(D,R)$ consisting of a
centered packing and an exceptional standard region is said to be
an {\it exceptional cluster}.  The vertices of the packing of
height at most $2t_0$ that are contained in the closed cone over
the standard region are called its {\it corners}.
\end{definition}



If $R$ is a standard region, we write $V_R(t)$ for the
intersection of the local $V$-cell $V_R = \op{VC}(0)\cap C(R)$
with a ball $B(t)$, centered at the origin, of radius $t$.  We
usually take $t=t_0$. If $\{0,v\}$, of length between $2t_0$ and
$2\sqrt{2}$, is not the diagonal of an upright quarter in the
$Q$-system, then $v$ does not affect the truncated cell $V_R(t_0)$
and may be disregarded. For this reason we confine our attention
to upright diagonals that lie along an upright quarter in the
$Q$-system.


\begin{lemma}
\label{lemma:Q-in-region} Each simplex in the $Q$-system with a
vertex at the origin lies entirely in the closed cone over some
standard region $R$.
\end{lemma}

\begin{proof}  Assume for a contradiction that $Q=\{0,v_1,v_2,v_3\}$
with $v_1$ in the open cone over $R_1$ and with $v_2$ in the open
cone over $R_2$.  Then $\{0,v_1,v_2\}$ and $\{0,w_1,w_2\}$ (a wall
between $R_1$ and $R_2$) cross;  this is contrary to
Lemma~\ref{lemma:barrier-no-overlap}.
\end{proof}


\section{Functions on standard regions}

\subsection{Formula for VC}


Let $S$ be a simplex and let $v$ be a vertex of that simplex. Let
$\op{VC}(S,v)$ be the subset of $|S|$ consisting of points closer
to $v$ than to any other vertex of $S$. By
Lemma~\ref{lemma:Q-divide}, if $S\in\CalQ_0(v,\Lambda)$, then
$$\op{VC}(S,0) = \op{VC}(v,\Lambda)\cap |S|.$$
Under the assumption that $S$ contains its circumcenter and that
every one of its faces contains its circumcenter, an explicit
formula for the volume $\op{vol}(\op{VC}(S,v))$ has been
calculated in \cite[Section~8.6.3]{part1}. This volume formula is
an algebraic function of the edge lengths of $S$, and may be
analytically continued to give a function of $S$ with chosen
vertex $v$:
  $$\op{vol}\,\op{VC}^\op{an}(S,v).$$



The following appeared as Claim~\ref{claim:volan}:
\begin{lemma}\tlabel{lemma:volan}  %%Cf. claim:volan
Let $S=\{v_1,v_2,v_3,v_4\}$ be in the $\CalQ$-system. Then
    $$
    \sum_{i=1}^4 \op{volan}(S,v_i) = \sum_{i=1}^4
    \op{vol}(\op{VC}(S,v_i)) = \op{vol}(|S|).
    $$
\end{lemma}



 Let $S=\{v_0,v_1,v_2,v_3\}$ be a simplex. Fix $t$ in the range
$t_0\le t\le\sqrt2$.  Assume that $t$ is at most the circumradius
of $S$. Assume that it is at least the circumradius of each of the
faces of $S$.  Let $\op{VC}_t(S,v_0)$ be the intersection of
$\op{VC}(S,v_0)$ with the ball $B(v_0,t)$. Under the assumption
that $S$ contains its circumcenter and that every one of its faces
contains it circumcenter, an explicit formula for the volume
$$\op{vol}(\op{VC}_t(S,v_0))$$ is calculated by means of
Lemma~\ref{XX} from the six quoins that form $\op{VC}(S,v_0)$.
This leads to the
following formula. Let $h_i = |v_i|/2$ and
$\eta_{ij}=\eta(0,v_i,v_j)$, and let $S_3$ be the group of
permutations of $\{1,2,3\}$ in
\begin{equation}
   \op{vol}\,\op{VC}_t(S,v_0) =
   \sol(S)/3 - \sum_{i=1}^3 \frac{\dih(S,v_i)}{2\pi}\op{vol}\,\op{cap_i}
   +\sum_{(i,j,k)\in S_3} \quo(R(h_i,\eta_{ij},t)).
   \tlabel{eqn:vol-theta-0}
\end{equation}


We extend Formula~\ref{eqn:vol-theta-0} by setting
    $$\quo(R(a,b,c)) = 0,$$
if the constraint $a < b < c$ fails to hold.  Similarly, set
$\op{vol}\,\op{cap}_i=0$ if $|v_i|\ge 2t$.  With these
conventions,  Formula~\ref{eqn:vol-theta-0} extends to all
simplices.  We write the extension of $\op{vol}\,\op{VC}_t(S,v)$
as
$$\op{vol}\,{\op{VC}^+_t}(S,v).$$


\subsection{Scores}
\tlabel{sec:ssc}

The last section introduced a function $\sigma$ called the score.
We show that the function $\sigma$ can be expressed as a sum over
terms attached to each of the standard regions.

\begin{definition} \tlabel{def:standard-cluster}
A {\it standard cluster\/} is a pair $(R,D)$ where $(v,\Lambda)$ is a
centered packing and $R$ is one of its standard regions.  A {\it
quad cluster\/} is the standard cluster obtained when the standard
region is a quadrilateral.
\end{definition}
%
 \index{cluster!standard}
 \index{cluster!quad}
 \index{quad cluster}

%Recall $|S|$ is the convex hull of a set $S\subset
%\ring{R}^3$.

We break $\sigma$ into a sum
   \begin{equation}
   \sigma(v,\Lambda) = \sum_R\,\sigma_R(v,\Lambda),
   \end{equation}
indexed by the standard clusters $(R,D)$.  Let
   $$
   \op{VC}_R(v,\Lambda) = \op{VC}(v,\Lambda)\cap \op{cone}(R),
   $$
whenever $R$ is a measurable subset of the unit sphere.  Let
   $$
   \CalQ_0(R,D) = \{Q\in \CalQ_0(v,\Lambda) : Q\subset \op{cone}(R)\}.
   $$
By Lemma~\ref{lemma:Q-in-region},
 each $Q$ is entirely contained in the cone over a single
standard region.

\begin{definition} \tlabel{def:score-std-region}
   Let $R$ be a measurable subset of the unit sphere.  Set
      $$
      \vor_R(v,\Lambda) =4\left(-\doct \op{vol}\,\op{VC}_R(v,\Lambda)  +
      \sol(R)/3\right).
      $$
      Let $R$ be a standard region. Set
      $$
      \sigma_R(v,\Lambda) = \vor_R(v,\Lambda) - 4\doct
         \sum_{Q\in\CalQ_0(R,D)} A_1(Q,c(Q,D),0).
      $$
\index{vzorR@$\vor_R$} \index{zzsigmaR@$\sigma_R$}
\end{definition}

\begin{lemma} $\sigma(v,\Lambda) = \sum_R\sigma_R(v,\Lambda)$, where the sum runs
over all standard regions $R$.
\end{lemma}

\begin{proof}
   $$
   \begin{array}{lll}
      \sigma(v,\Lambda)
      &= -4\doct (\op{vol}\,\Omega(v,\Lambda) + A_0(v,\Lambda))+16\pi/3\\
      &= -4\doct (\op{vol}\,\op{VC}(v,\Lambda)+\sum_{Q\in\CalQ_0(v,\Lambda)}
         A_1(Q,c(Q,D),0)) + (4) (4\pi/3)\\
      &= \sum_R 4\left (-\doct \op{vol}\,\op{VC}_R(v,\Lambda) -\doct
         \sum_{Q\in\CalQ_0(R,D)} A_1(Q,c(Q,D),0) +
         \sol(R)/3\right).
   \end{array}
   $$
\end{proof}

Also, we have
    \begin{equation}
    \op{vor}(v,\Lambda)=\sum_{R\in \CalR(v,\Lambda)}
    \op{vor}_R(v,\Lambda).
    \tlabel{eqn:vorD}
    \end{equation}

\begin{lemma}\tlabel{lemma:R'}
Let $R'\subset R$ be the part of a standard region that does not
lie in any cone over any $Q\in Q_0(R,D)$.  Then
   $$
   \sigma_R(v,\Lambda) = \vor_{R'}(v,\Lambda)
      + \sum_{Q\in\CalQ_0(R,D)} \sigma(Q,c(Q,D),0).
   $$
\end{lemma}

\begin{proof} Substitute the definition of $A_1$
(Equation~\ref{eqn:a1-sigma}) into the definition of
$\sigma_R(v,\Lambda)$, noting that $\op{VC}(Q,0) = \op{VC}_{R''}(v,\Lambda)$,
where $R''$ is the intersection of $Q$ with the unit sphere.
\end{proof}

\begin{remark}   Lemma~\ref{lemma:R'} explains why we have chosen
the same symbol $\sigma$ for the functions $\sigma_R(v,\Lambda)$ and
$\sigma(Q,c,v)$.  We can view Lemma~\ref{lemma:R'} as asserting a
linear relation in the functions $\sigma$:
   $$\sigma_R(v,\Lambda) = \sigma_{R'}(v,\Lambda) + \sum \sigma(Q,c,0).$$
The sum runs over $Q\in\CalQ_0$ that lie in the cone over $R$.
\end{remark}

\subsection{Scoring}

By the results of Sections~\ref{x-2.7}, \ref{x-2.8}, \ref{x-2.9},
$\sigma(v,\Lambda)$ can be broken into a corresponding sum,
    $$
    \begin{array}{lll}
    \sigma_R(v,\Lambda) &= \sum_Q \sigma(Q) + \sigma(V_P),
                \hbox{ for quarters $Q$ in the $Q$-system, where}\\
    \sigma(V_P) &= \op{sovo}(\tildeV_P(t_0),\lambda_{oct})+  \sum_{\SA,\SB,\SC} \op{sovo}(V_S(t_S),\lambda_{oct})
        - \sum_v 4\doct\op{vol}(\delta_P(v)) - 4\doct\op{vol}(\delta'_P).\\
    \end{array}
    $$

By dropping the final term, $4\doct\op{vol}(\delta'_P)$, we obtain
an upper bound on $\sigma(V_P)$.  Because of the separation
results of Sections~\ref{x-2.7}--\ref{x-2.8},  we may score
$\tildeV_P(t_0)$ by Formula~\ref{eqn:3.7}. Bounds on the score of
simplices of type $\SB$ appear in \calc{193836552}.



\subsection{Truncated Tetrahedron}


We set
    \begin{equation}
    \begin{array}{lll}
    \svor(S,t) &=
    \sol(S)\phi(t,t)
    +\sum_{i=1,\ h_i\le t}^3 d_i (1-h_i/t) (\phi(h_i,t)-
    \phi(t,t)) \\
    &-\sum_{(i,j,k)\in S_3}
    4\doct
    \quo(R(h_i,\eta(y_i,y_j,y_{k+3}),t)).
    \tlabel{eqn:3.5}
    \end{array}
    \end{equation}
In the definition, we adopt the convention that $\quo(R)=0$, if
$R=R(a,b,c)$ does not exist (that is, if the condition
    $0< a\le b\le c$
is violated). In the second sum, $S_3$ is the set of permutations
on three letters. This definition is compatible with
Definition~\ref{def:svor}.

Similarly, we define $\vor_P(D,t)$ for arbitrary standard clusters
$(P,D)$.  (We shift notation from $R$ to $P$ for a standard region
to avoid conflict with Rogers simplices $R$ in the following
definition.)  Extending the notation in an obvious way, we have
    \begin{equation}
    \begin{array}{lll}
    \vor_P(D,t) &=
    \sol(P)\phi(t,t)
    +\sum_{|v_i|\le 2t} d_i (1-|v_i|/(2t)) (\phi(|v_i|/2,t)-
    \phi(t,t)) \\
    &-\sum_{R} 4\doct \quo(R).
    \tlabel{eqn:3.7}
    \end{array}
    \end{equation}
The first sum runs over vertices in $P$ of height at most $2t$.
The second sum runs over Rogers simplices $R(|v_i|/2,\eta(F),t)$
in $P$, where $F=\{0,v_1,v_2\}$ is a face of circumradius
$\eta(F)$ at most $t$, formed by vertices in $P$.  The constant
$d_i$ is the total dihedral angle along $\{0,v_i\}$ of the
standard cluster. The truncations $t=t_0=1.255$ and $t=\sqrt2$
will be of particular importance.
    Set $A(h) = (1-h/t_0) (\phi(h,t_0)-\phi(t_0,t_0))$.\index{A}

\begin{remark}  We have introduced both untruncated and truncated
versions of functions $\vor$ and $\sigma$.  The truncated versions
are used to give upper bounds on the untruncated versions.  For
example,  in the function $\sigma(v,\Lambda)$, the $V$-cell contributes
through its volume.  The volume
appears with a negative coefficient 
$\lambda_v=-4\doct$.  Thus, we obtain an
upper bound on $\sigma(v,\Lambda)$ by discarding bits of volume from the
$V$-cell.   This suggests that we might try to give upper bounds
on the score $\sigma(v,\Lambda)$ by truncating the $V$-cell in various
ways. This is the reason for the truncated versions of these
functions.
\end{remark}




\subsection{The function $\tau$}% DCG 10.1, p99
    %\heads{3. Functions}

%Set $\zeta^{-1}:=\sol(S(2,2,2,2,2,2))=2\arctan(\sqrt{2}/5)$. The
%constant $\zeta$ is related to the other fundamental constants by the
%relations $\pt= 2/\zeta-\pi/3$ and $\doct=(\pi-2/\zeta)/\sqrt{8}$.
%Rogers's bound is $\sqrt{2}/\zeta\approx 0.7796$.

% Plus Formula 7 on scores.

We consider the functions
    $\sigma_R(v,\Lambda)-\lambda\zeta\sol(R)\,\pt$,
for $\lambda=0$, $1$, or $3.2$, where $R$ is a standard cluster.
%The constant $3.2$ was determined experimentally.
We write
    $$
    \tau_R(v,\Lambda) = \sol(R)\zeta\,\pt -
    \sigma_R(v,\Lambda).
    $$
We will see that $\tau_R(v,\Lambda)$ has a simple interpretation.  If $(v,\Lambda)$
is a centered packing with standard clusters $\{R\}$, set $\tau(v,\Lambda)
= \sum_{R}\tau_R(v,\Lambda)$.
\smallskip



\begin{lemma}\label{lemma:sigma-tau}
    %\proclaim{Lemma 3.2}
    $$\sigma(v,\Lambda) = {4\pi \zeta\,\pt} - \tau(v,\Lambda).$$
\end{lemma}

\begin{proof} Let $\{R\}$ be the standard clusters in $(v,\Lambda)$. Then
    $$
    \sigma(v,\Lambda) = \sum_R\sigma_{R_i}(v,\Lambda) +
        (4\pi-\sum_R\sol(R_i))\zeta\,\pt = 4\pi \zeta\,\pt - \sum_R\tau_{R_i}(v,\Lambda).
    $$
\end{proof}


\begin{lemma}
If there are standard clusters $R_1,\ldots,R_k$ such that
$$\sum_{i=1}^k \tau_{R_i}(v,\Lambda)> \squander,$$
then $(v,\Lambda)$ does not contravene.
\end{lemma}

\begin{proof}
$$\sigma(v,\Lambda) = 4\pi\zeta,\pt -\sum_R{R_i}(v,\Lambda) < 8\,\pt.$$
\end{proof}

We note that $14.8\,\pt > \squander$.  We sometimes use this
approximation.


The function $\tau_R(v,\Lambda)$ gives the amount {\it squandered\/} by a
particular standard cluster $R$.  If nothing is squandered, then
$\tau_{R_i}(v,\Lambda)=0$ for every standard cluster, and the upper bound
on $\sigma(v,\Lambda)$ is
    $4\pi\zeta\,\pt\approx 22.8\,\pt$.



%% XX Deleting mentioned \calc{629256313},
%% \calc{917032944}, \calc{738318844}, and \calc{587618947}.
%% Perhaps they are no longer needed in the proof!

%% Major Deletion: SVN:16 has proof of local optimality. Gone in SVN:23.


\section{Aggregated Regions}
%\section{Contravening Plane Graphs defined}
\label{sec:stargraph}

A hypermap $G$ is attached to every contravening centered packing
as follows.  From the centered packing $(v,\Lambda)$, it is possible to
determine the coordinates of the set $U(v,\Lambda)$ of vertices at
distance at most $2t_0 $ from the origin.

If we draw a geodesic arc on the unit sphere at the origin with
endpoints at the radial projections of $v_1$ and $v_2$ for every
pair of vertices $v_1$, $v_2\in U(v,\Lambda)$ such that $|v_1|, |v_2|,
|v_1-v_2|\le 2t_0 $, we obtain a hypermap that breaks the unit
sphere into standard regions. (The arcs do not meet except at
endpoints by Lemma~\ref{lemma:2t0-doesnt-pass-through}.)
%Each
%standard region is defined as the closure in the unit sphere of a
%connected component of the unit sphere with all arcs removed.

For a given standard region, we consider the arcs forming its
boundary together with the arcs that are internal to the standard
region.  We consider the points on the unit sphere formed by the
endpoints of the arcs, together with the radial projections to the
unit sphere of vertices in $U$ whose radial projection lies in the
interior of the region.


\begin{remark} \label{remark:tri-pent}
Observe that one case of Figure~\ref{XX} is bounded by a
triangle and a pentagon, and that the others are bounded by a
polygon. Replacing the triangle-pentagon arrangement with the
bounding pentagon and replacing the others with the bounding
polygon, we obtain a partition of the sphere into simple polygons.
Each of these polygons is a single standard region, except in the
triangle-pentagon case (Figure~\ref{fig:tri-pent}), which is a union
of two standard regions (a triangle and a eight-sided region).
\end{remark}
\begin{figure}[htb]
  \centering
  \myincludegraphics{\ps/tripent.eps}
  \caption{An aggregate forming a pentagon}
  \label{fig:tri-pent}
\end{figure}

\begin{remark}\label{remark:degree6}
To simplify further, if we have an arrangement of six standard
regions around a vertex formed from five triangles and one
pentagon, we replace it with the bounding octagon (or hexagon).
See Figure~\ref{fig:degree6}.  (It will be shown in
Lemma~\ref{lemma:deg5} that there is at most one such
configuration in the standard decomposition of a contravening
centered packing, so we will not worry here about how to treat the
case of two overlapping configurations of this sort.)
\end{remark}
\begin{figure}[htb]
  \centering
  \myincludegraphics{\ps/degree6.eps}
  \caption{Degree $6$ aggregates}
  \label{fig:degree6}
\end{figure}

In summary, we have a hypermap that is approximately that given by
the standard regions of the centered packing, but simplified to a
bounding polygon when one of the configurations of Remarks
\ref{remark:tri-pent} and \ref{remark:degree6} occur.  We refer to
the combination of standard regions into a single face of the
hypermap as {\it aggregation}.  We call it the hypermap $G = G(v,\Lambda)$
attached to a contravening centered packing $(v,\Lambda)$.
%
 \index{aggregation}

Lemma~\ref{prop:nonempty} will show the vertex set $U$ is non-empty
and that the hypermap $G(v,\Lambda)$ is non-empty.

When we refer to the hypermap in this manner, we mean the
combinatorial hypermap as opposed to the embedded metric graph on
the unit sphere formed from the system of geodesic arcs.  Given a
node $v$ in $G(v,\Lambda)$, there is a uniquely determined vertex $v(v,\Lambda)$ of
$U(v,\Lambda)$ whose radial projection to the unit sphere determines $v$. We
call $v(v,\Lambda)$ the {\it corner} in $U(v,\Lambda)$ over $v$.
%
 \index{corner}

By construction, the hypermaps associated with a centered packing
do not have loops or multiple joins.  In fact, the edges of $G(v,\Lambda)$
are defined by triangles whose sides vary between lengths $2$ and
$2t_0 $. The angles of such a triangle are strictly less than
$\pi$. This implies that the edges of the metric graph on the unit
sphere always have arc-length strictly less than $\pi$. In
particular, the endpoints are never antipodal.  A loop on the
combinatorial hypermap corresponds to a edge on the metric graph
that is a closed geodesic. A multiple join on the hypermap
corresponds on the metric graph to a pair of points joined by
multiple minimal geodesics, that is, a pair of antipodal points on
the sphere.  By the arc-length constraints on edges in the metric
graph, there are no loops or multiple joins in the combinatorial
hypermap $G(v,\Lambda)$.




\section{Refinements}



\section{Special Structures}

\subsection{Classification of Quadrilateral Regions}
\label{sec:quad-class}


\begin{lemma}\label{quad:classify}
Let $C=(0,(v_1,v_2,v_3,v_4))$ be a quad cluster.  Then it has
exactly one of the following forms.
\begin{itemize}
  \item (flat) $\{0,v_1,v_2,v_3\}$ and $\{0,v_1,v_3,v_4\}$ are flat
    quarters in the $Q$-system.
   \item (flat) $\{0,v_2,v_4,v_1\}$ and $\{0,v_2,v_4,v_3\}$ are flat
     quarters in the $Q$-system.
     \item (octahedron) There exists an enclosed vertex $w$, such that
       $(0,w,v_1,v_2,v_3,v_4)$ is a quartered octahedron with
       four upright quarters in the $Q$-system.
   \item (pure) We have $|v_1-v_3|>\sqrt8$, $|v_2-v_4|>\sqrt8$ and
     there is no enclosed vertex $w$ with $|w|<\sqrt8$.
    \item (mixed) We have $|v_1-v_3|>\sqrt8$, $|v_2-v_4|>\sqrt8$, there
      is no enclosed vertex forming a quartered octahedron, but
      there exists some enclosed vertex $w$ with $|w|<\sqrt8$.
      \index{pure}\index{mixed}
\end{itemize}
\end{lemma}

\begin{proof}
If the quad cluster has a diagonal of length at most $\sqr8$
between two corners, there are three possible decompositions. (1)
The two quarters formed by the diagonal lie in the $Q$-system so
that the scoring rules for the $Q$-system are used.  (2) There is
a second diagonal of length at most $\sqr8$, and we use the two
quarters from the second diagonal for the scoring. (3) There is an
enclosed vertex that makes the quad cluster into a quartered
octahedron and the four upright quarters are in the $Q$-system.

Now suppose that neither diagonal is less than $\sqr8$ and the
quad cluster is not a quartered octahedron. If there is no
enclosed vertex of length at most $\sqr8$, the quad cluster
contains no quarters. An upper bound on the score of the quad
cluster $(P,D)$ is $\vor_P(D,\sqr2)$. The remaining cases are
called {\it mixed\/} quad clusters. Mixed quad clusters enclose a
vertex of height at most $\sqr8$ and do not contain flat quarters.
\end{proof}



\begin{definition}
Define the {\it central vertex\/} $v$ of a flat quarter to be the
vertex for which $\{0,v\}$ is the edge opposite the diagonal.
\end{definition}





\section{Deleting Nodes from a Hypermap}

\subsection{Masked Flat Quarters}

\section{Deforming Geometric Hypermaps}



%%%%%%%%%%%%%%%%  LOCAL BOUNDS %%%%%%%%%%%%%%%%


\chapter{Local Bounds}

\subsection{Positivity}%DCG 9.10, p98 %% Delta(v) stuff
    \oldlabel{2.11}



\begin{lemma}
    \label{lemma:tau-positive}
    Let $R$ be a standard region that is not a triangle in a
    centered packing $(v,\Lambda)$.
    $\tau_{0,R}(v,\Lambda)\ge 0$.
\end{lemma}

\begin{proof}
Everything truncated at $t_0$ can be broken into three types of
pieces: Rogers simplices $R(a,b,t_0)$, wedges of $t_0$-cones, and
spherical regions. (See Figure~\ref{fig:doct}.) The wedges of
$t_0$-cones and spherical regions can be considered as the
degenerate cases $b=t_0$ and $a=b=t_0$ of Rogers simplices, so it
is enough to show that $\tau(R(a,b,t_0))\ge 0$.
This follows from Lemma~\ref{lemma:rog-tet}, because $t_0\ge\sqrt{3/2}$
and $b\ge \eta(2,2,2)=2/\sqrt3$.
\end{proof}

\begin{lemma}\label{lemma:roger0}
    %proclaim{Lemma 3.1}
    %\oldlabel{part3.3.1}
    $\tau_R(v,\Lambda)\ge 0$, for all standard clusters $R$.
\end{lemma}

\begin{proof}
If $R$ is not a quasi-regular tetrahedron, then $\sigma_R(v,\Lambda)\le0$
by Theorem~\ref{lemma:quad0} and $\sol(R)> 0$, so that the result
is immediate. If $R$ is a quasi-regular tetrahedron, the result
appears in the archive of inequalities \calc{53415898}.
\end{proof}




\section{Triangles}


\begin{lemma}
        \label{lemma:no-enclosed-tri}
        A triangular standard region does not contain any enclosed
        vertices.
\end{lemma}

\begin{proof}
    This fact is proved in Lemma~\ref{lemma:2t0-doesnt-pass-through}.
\end{proof}



\begin{lemma} \label{lemma:1pt}
%\proclaim{Lemma 3.13}
A quasi-regular tetrahedron $S$ satisfies $\sigma(S)\le 1\,\pt$.
Equality occurs if and only if the quasi-regular tetrahedron is
regular of edge length $2$.\index{quasi-regular!tetrahedron}
%
\end{lemma}

\begin{proof}
This is \calc{586468779}.
\end{proof}



\section{Types}\label{sec:types}%DCG 10.2, p100

Let $v$ be a vertex of height at most $2t_0$.  We say that $v$ has
{\it type\/} $(p,q)$ if every standard region with a vertex at $\bar
v$ (the radial projection of $v$) is a triangle or a quadrilateral,
and if there are exactly $p$ triangular faces and $q$ quadrilateral
faces that meet at $\bar v$.  We write $(p_v,q_v)$ for the type of
$v$.

%If more than $\squander$ are squandered at a vertex of a given type,
%then that type of vertex cannot be part of a centered packing scoring
%more than $8\,\pt$.  These relations between scores and vertex types
%will allow us to reduce the feasible planar maps to an explicit finite
%list. For each of the planar maps on this list, we calculate a second,
%more refined linear programming bound on the score. Often, the refined
%linear programming bound is less than $8\,\pt$.

This section derives the bounds on the scores of the clusters
around a given vertex as a function of the type of the vertex.
Define constants $\tlp(p,q)/\pt$ by Table~\ref{eqn:old5.1}.  The
entries marked with an asterisk will not be needed.

\bigskip
% Table eqn:old5.1 of constants.


% page 246 of TeXBook
%\def\pt{\hbox{\it pt}}

\begin{equation}
\vbox{\offinterlineskip \hrule
\halign{&\vrule#&\strut\ \hfil#\hfil\ \cr   % "\ " was quad
height 7pt&\omit&&\omit&&\omit&&\omit&&\omit&&\omit&&\omit&\cr
&\hfil $\tlp(p,q)/\pt$\hfil
        &&\hfil $q=0$\hfil
        &&\hfil1\hfil
        &&\hfil2\hfil
        &&\hfil3\hfil
        &&\hfil4\hfil
        &&\hfil5\hfil&
\cr height 7pt&\omit&&\omit&&\omit&&\omit&&\omit&&\omit&&\omit&\cr
\noalign{\hrule}
height7pt&\omit&&\omit&&\omit&&\omit&&\omit&&\omit&&\omit&\cr
&$p=0$&& *&& *&& 15.18&& 7.135&& 10.6497&& 22.27&\cr &1&&    *&&
*&&  6.95&& 7.135&&17.62  && 32.3&\cr &2&&    *&&
8.5&&4.756&&12.9814&&*&&*&\cr &3&& *&&
3.6426&&8.334&&20.9&&*&&*&\cr
&4&&4.1396&&3.7812&&16.11&&*&&*&&*&\cr
&5&&0.55&&11.22&&*&&*&&*&&*&\cr &6&&6.339&&*&&*&&*&&*&&*&\cr
&7&&14.76&&*&&*&&*&&*&&*&\cr
height7pt&\omit&&\omit&&\omit&&\omit&&\omit&&\omit&&\omit&\cr}
\hrule }
    %oldtag 5.1
    \label{eqn:old5.1}
\end{equation}
% based on sp in more.m



\begin{lemma}
    \label{lemma:pq}
    %{Proposition 5.2}
Let $S_1,\ldots,S_p$ and $R_1,\ldots,R_q$ be the tetrahedra and quad
clusters around a vertex of type $(p,q)$. Consider the constants of
Table~\ref{eqn:old5.1}.               Now,
    $$
    \begin{array}{lll}
    &\sum^p\tau(S_i) + \sum^q\tau_{R_i}(v,\Lambda) \ge \tlp(p,q),\\
    \end{array}
    $$
\end{lemma}

\begin{proof} Set
    $$
    (d_i^0,t_i^0)=(\dih(S_i),\tau(S_i)),\qquad
    (d_i^1,t_i^1)=(\dih(R_i),\tau(R_i)).
    $$
The linear combination $\sum^p\tau(S_i)+\sum^q\tau_{R_i}(v,\Lambda)$ is at
least the minimum of $\sum^p t_i^0+\sum^q t_i^1$ subject to
$\sum^p d_i^0+\sum^q d_i^1 = 2\pi$ and to the system of linear
inequalities \calc{830854305} and the system of linear
inequalities \calc{940884472} (obtained by replacing $\tau$ and
dihedral angles by $t_i^j$ and $d_i^j$). The constant $\tlp(p,q)$
was chosen to be slightly smaller than the actual minimum of this
linear programming problem.

The entry $\tlp(5,0)$ is based on Lemma~\ref{lemma:0.55}, $k=1$.
\end{proof}


\begin{lemma}
    \label{lemma:0.55}
    %proclaim{Lemma 5.3}
Let $v_1,\ldots, v_k$, for some $k\le 4$, be distinct vertices of
a centered packing of type $(5,0)$.  Let $S_1,\ldots, S_r$ be
quasi-regular tetrahedra around the edges $\{0,v_i\}$, for $i\le
k$. Then
    $$\sum_{i=1}^r \tau(S_i)> 0.55k\,\pt,$$
and
    $$\sum_{i=1}^r \sigma(S_i) < r\,\pt - 0.48k\,\pt.$$
\end{lemma}


\begin{proof}
We have $\tau(S)\ge 0$, for any quasi-regular tetrahedron $S$.  We
refer to the edges $y_4,y_5,y_6$ of a simplex $S(y_1,\ldots,y_6)$
as its top edges. Set $\xi=2.1773$.

The proof of the first inequalities relies on seven
calculations\footnote{\calc{636208429}}. Throughout the proof, we
will refer to these inequalities simply as Inequality~$i$, for
$i=1,\ldots,7$.

We claim (Claim~1) that if $S_1,\ldots,S_5$ are quasi-regular
tetrahedra around an edge $\{0,v\}$ and if $S_1=S(y_1,\ldots,y_6)$,
where $y_5\ge\xi$ is the length of a top edge $e$ on $S_1$ shared
with $S_2$, then $\sum_1^5\tau(S_i) > 3(0.55)\,\pt$.  This claim
follows from Inequalities~1 and~2 if some other top edge in this
group of quasi-regular tetrahedra has length greater than $\xi$.
Assuming all the top edges other than $e$ have length at most
$\xi$, the estimate follows from $\sum_1^5\dih(S_i)=2\pi$ and
Inequalities~3, ~4.

Now let $S_1,\ldots,S_8$ be the eight quasi-regular tetrahedra
around two edges $\{0,v_1\}$, $\{0,v_2\}$ of type $(5,0)$. Let $S_1$
and $S_2$ be the simplices along the face $\{0,v_1,v_2\}$. Suppose
that the top edge $\{v_1,v_2\}$ has length at least $\xi$. We claim
(Claim 2) that $\sum_1^8\tau(S_i)> 4(0.55)\,\pt$.  If there is a
top edge of length at least $\xi$ that does not lie on $S_1$ or
$S_2$, then this claim reduces to Inequality~1 and Claim 1. If any
of the top edges of $S_1$ or $S_2$ other than $\{v_1,v_2\}$ has
length at least $\xi$, then the claim follows from Inequalities~1
and ~2. We assume all top edges other than $\{v_1,v_2\}$ have length
at most $\xi$. The claim now follows from Inequalities~3 and ~5,
since the dihedral angles around each vertex sum to $2\pi$.

We prove the bounds for $\tau$.  The proof for $\sigma$ is
entirely similar, but uses the constant $\xi=2.177303$ and seven
new calculations\footnote{\calc{129662166}} rather than the seven
given above. Claims analogous to Claims~1 and 2 hold for the
$\sigma$ bound by this new group of seven inequalities.


Consider $\tau$ for $k=1$.  If a top edge has length at least
$\xi$, this is Inequality~1.  If all top edges have length less
than $\xi$, this is Inequality~3, since dihedral angles sum to
$2\pi$.

We say that a top edge lies around a vertex $v$ if it is an edge
of a quasi-regular tetrahedron with vertex $v$. We do not require
$v$ to be the endpoint of the edge.

Take $k=2$. If there is an edge of length at least $\xi$ that lies
around only one of $v_1$ and $v_2$, then Inequality~1 reduces us
to the case $k=1$.  Any other edge of length at least $\xi$ is
covered by Claim 1.  So we may assume that all top edges have
length less than $\xi$.  And then the result follows easily from
Inequalities~3 and ~6.

Take $k=3$. If there is an edge of length at least $\xi$ lying
around only one of the $v_i$, then Inequality~1 reduces us to the
case $k=2$. If an edge of length at least $\xi$ lies around
exactly two of the $v_i$, then it is an edge of two of the
quasi-regular tetrahedra. These quasi-regular tetrahedra give
$2(0.55)\,\pt$, and the quasi-regular tetrahedra around the third
vertex $v_i$ give $0.55\,\pt$ more. If a top edge of length at
least $\xi$ lies around all three vertices, then one of the
endpoints of the edge lies in $\{v_1,v_2,v_3\}$, so the result
follows from Claim 1. Finally, if all top edges have length at
most $\xi$, we use Inequalities~3, ~6, ~7.

Take $k=4$.  Suppose there is a top edge $e$ of length at least
$\xi$. If $e$ lies around only one of the $v_i$, we reduce to the
case $k=3$. If it lies around two of them, then the two
quasi-regular tetrahedra along this edge give $2(0.55)\,\pt$ and
the quasi-regular tetrahedra around the other two vertices $v_i$
give another $2(0.55)\,\pt$.  If both endpoints of $e$ are among
the vertices $v_i$, the result follows from Claim 2.  This happens
in particular if $e$ lies around four vertices.  If $e$ lies
around only three vertices, one of its endpoints is one of the
vertices $v_i$, say $v_1$.  Assume $e$ is not around $v_2$. If
$v_2$ is not adjacent to $v_1$, then Claim 1 gives the result. So
taking $v_1$ adjacent to $v_2$, we adapt Claim 1, by using all
seven Inequalities, to show that the eight quasi-regular
tetrahedra around $v_1$ and $v_2$ give $4(0.55)\,\pt$. Finally, if
all top edges have length at most $\xi$, we use Inequalities~3,
~6, ~7.
\end{proof}

In a special case, the constant of Lemma~\ref{lemma:0.55} can be
improved by a small amount.

\begin{lemma}
    \label{lemma:0.55A}
    %proclaim{Lemma 5.3}
Let $v$ be a vertex of a centered packing of type $(5,0)$.  Let
$S_1,\ldots, S_5$ be quasi-regular tetrahedra around the edge
$\{0,v\}$. Then
    $$\sum_{i=1}^5 \sigma(S_i) < 4.52\,\pt - 10^{-8}.$$
\end{lemma}

\begin{proof}
If any of the top edges has length greater than $\xi$, we use a
slightly improved calculation\footnote{\calc{241241504-1}} that
yields this constant. Otherwise, the same
calculation\footnote{\calc{82950290}} that was used in the previous
lemma gives the desired estimate
  $$
  \sum\sigma < 5(0.31023815) - 2\pi(0.207045) < 4.52\,\pt - 10^{-8}
  $$
\end{proof}


\section{Limitations on Types}%DCG 10.3, p103
    %\heads{6. Limitations on types}

Recall that a vertex of a planar map has type $(p,q)$ if it is the
vertex of exactly $p$ triangles and $q$ quadrilaterals. This
section restricts the possible types that appear in a centered
packing.

Let $t_4$ denote the constant $0.1317\approx 2.37838774\,\pt$.

\begin{lemma}\label{lemma:0.1317} If $R$ is a quad cluster, then
   $$\tau_R(v,\Lambda) \ge t_4.$$
\end{lemma}

\begin{proof}
A calculation\footnote{\calc{996268658}} asserts precisely this.
\end{proof}

\begin{lemma} \label{lemma:pq-impossible}
    %\proclaim{Lemma 6.1}
The following eight types $(p,q)$ are impossible:
    (1)  $p\ge 8$,
    (2)  $p\ge 6$ and $q\ge 1$,
    (3)  $p \ge 5$ and $q\ge 2$,
    (4)  $p \ge 4$ and $q\ge 3$,
    (5)  $p \ge 2$ and $q\ge 4$,
    (6)  $p \ge 0$ and $q\ge 6$,
    (7)  $p \le 3$ and $q=0$,
    (8) $p \le 1$ and $q=1$.
\end{lemma}

\begin{proof}
Calculations\footnote{\calc{657406669}, \calc{208809199},
\calc{984463800}, and \calc{277330628}} give a lower bound on the
dihedral angle of $p$ simplices and $q$ quadrilaterals at
$0.8638p+1.153 q$ and an upper bound of $1.874445 p + 3.247 q$. If
the type exists, these constants must straddle $2\pi$. One readily
verifies in Cases 1--8 that these constants do not straddle
$2\pi$.
\end{proof}

\begin{lemma}
    \label{lemma:pq-types}
    %\proclaim{Lemma 6.2}
If the type of any vertex of a centered packing is one of
$(4,2)$, $(3,3)$, $(1,4)$, $(1,5)$, $(0,5)$, $(0,2)$, %$(7,0)$,
then the centered packing does not contravene.
\end{lemma}


\begin{proof}  According to Table~\ref{eqn:old5.1},
we have $\tlp(p,q)> \squander$, for $(p,q) = (4,2)$, $(3,3)$,
$(1,4)$, $(1,5)$, $(0,5)$, or $(0,2)$. By
Lemma~\ref{lemma:sigma-tau}, the result follows in these cases.
\end{proof}



\begin{remark} \label{rem:pq-list}
In summary of the preceding two lemmas, we find that we may
restrict our attention to the following types of vertices.
    $$
    \begin{matrix}
   (7,0)&      &       &       &       \\
   (6,0)&      &       &       &       \\
   (5,0)&(5,1) &       &       &       \\
   (4,0)&(4,1) &       &       &       \\
        &(3,1) &(3,2)  &       &       \\
        &(2,1) &(2,2)  &(2,3)  &       \\
        &      &(1,2)  &(1,3)  &       \\
        &      &       &(0,3)  &(0,4)  \\
    \end{matrix}
    $$
It will be shown in Lemma~\ref{lemma:70}, that the type $(7,0)$
does not occur in a contravening centered packing.
\end{remark}


\section{Bounds in Quadrilateral Regions}%DCG 10.4, p104
    \label{sec:bounds}



\subsection{pure bound}%DCG 8.2, p73



\begin{lemma} \label{lemma:wedge} Consider the wedge of a cone
    $$
    W =W(\alpha,z_0) =
    \{ t\, x : 0\le t \le 1, x\in P(\alpha,z_0)\}\subset\ring{R}^3,
    $$
where $P(\alpha,z_0)$ has the form
    $$
    P = \{(x_1,x_2,x_3) :
    x_3 = z_0,\   x_1^2+x_2^2+x_3^2\le 2,\ 0\le x_2\le \alpha x_1\},
    $$
with $z_0\ge1$.  Let $A$ be the volume of the intersection of the
wedge with $B(0,1)$. Then
    $$A\le\doct\,\op{vol}(W).$$
Equality is attained if and only if $W$ has zero volume.\index{cone}
\end{lemma}

\begin{proof} This is calculated in \cite[Sec. 4]{part2}.  See the
second frame of Figure~\ref{fig:doct}.
\end{proof}

\begin{lemma} \label{lemma:cone}
Let $C$ be the cone at the origin over a set $P$, where $P$ is
measurable and every point of $P$ has distance at least $1.18$
from the origin.  Let $A$ be the volume of the intersection of $C$
with $B(0,1)$. Then
    $$A\le\doct\,\op{vol}(C).$$
Equality is attained if and only if $C$ has zero volume.
\end{lemma}

\begin{proof} The ratio $A/\op{vol}(C)$ is at most $1/1.18^3 < \doct$.   See the
first frame of Figure~\ref{fig:doct}.
\end{proof}

\begin{figure}[htb]
  \centering
  \myincludegraphics{\ps/haII42.ps}
  \caption{Some sets of low density.}
  \label{fig:doct}
\end{figure}

\begin{lemma}\label{lemma:pure0}
Let $(R,D)$ be a pure quad cluster.  Then
  $\sigma_R(v,\Lambda)\le 0$.
\end{lemma}

\begin{proof}  The pure quad cluster breaks into the types
of regions of low density described by Lemmas~\ref{lemma:cone},
\ref{lemma:wedge}, and \ref{lemma:rog-doct}.  XX FILL IN DETAILS XX.
\end{proof}




\section{Quad Cluster Bound}

XX Eventually much of this proof can be moved to {\it Scissors and Volumes}.

\begin{lemma} \label{lemma:quarter0}
Let $Q$ be a quarter in the $Q$-system (either flat or upright).
Then $\sigma(Q)\le 0$. 
\end{lemma}

\begin{proof}  
We make use
of the definition of $\sigma$ on quarters from
Definition~\ref{def:sigma}. The general context (that is, contexts
other than $(2,1)$ and $(4,0)$) of upright quarters is established
by the inequalities\footnote{\calc{522528841} and
\calc{892806084}} that hold for all upright quarters $Q$ with
distinguished vertex $v$:
    $$
    \begin{array}{lll}
    &2\Gamma(Q) + \svor_0(Q,v) - \svor_0(Q,\hat v) \le 0\\
    &\svor(Q,v) + \svor(Q,\hat v) +\svor_0(Q,v) - \svor_0(Q,\hat v)\le0.
    \end{array}
    $$
For the remaining upright quarters (that is, contexts $(2,1)$ and $(4,0)$)
and for all flat quarters,
it is enough to show that $\Gamma(Q)\le0$, if $\eta^+\le\sqrt2$ and
$\svor(Q,v)\le0$, if $\eta^+\ge\sqrt2$.

Consider the case $\eta^+\le\sqrt2$.  If $Q$ is a quarter such that
every face has circumradius at most $\sqrt2$,
then\footnote{\calc{346093004}} $\Gamma(Q)\le0$.  
Because of this, we may assume that the circumradius of $Q$ is
greater than $\sqr2$. 
Since
(Definition~\ref{def:svor})
    $$4\Gamma(Q)=\sum_{i=1}^4 \svor(Q,v_i),$$
it is enough to show that $\svor(Q)<0$.  Since $\eta^+\le\sqrt2$ 
and the circumradius is greater than
$\sqrt2$, $\svor(Q,\sqrt2)$ is a strict truncation of the $V$-cell
in $Q$, so that
    $$\svor(Q)<\svor(Q,\sqrt2).$$
We show the right hand side is nonpositive.  Let $v$ be the
distinguished vertex of $Q$.  Let $A$ be $1/3$ the solid angle of
$Q$ at $v$ . By the definition of $\svor(Q,\sqrt2)$, it is
nonpositive if and only if
    \begin{equation}
        A\le \doct \,\op{vol}(\op{VC}(Q,v)\cap B(v,\sqrt2)).
        \label{eqn:Adoct}
    \end{equation}
($\op{VC}(Q,0)$ is defined in Section~\ref{sec:rules}.) The
intersection $\op{VC}(Q,v)\cap B(v,\sqrt2)$ consists of six Rogers
simplices $R(a,b,\sqrt2)$, three conic wedges (extending out to
$\sqrt2$), and the intersection of $B(v,\sqrt2)$ with a cone over
$v$. By Lemmas~\ref{lemma:rogers-app}, \ref{lemma:wedge}, and
\ref{lemma:cone}, these three types of solids give inequalities like
that of Equation~\ref{eqn:Adoct}. Summing the inequalities from
these lemmas, we get Equation~\ref{eqn:Adoct}.

Consider the case $\eta^+\ge\sqrt2$ and $\sigma=\svor$. If the
quarter is upright, then\footnote{\calc{40003553}} $\svor(Q)\le0$.
Thus, we may assume the quarter is flat.  
The
analytic continuation defining $\svor(Q)$ is the same
\footnote{This claim is justified by \calc{5901405}, which
shows that $\svor(Q)\le0$ when the two functions differ.} as
    $$4(-\doct\op{vol}(X) + \sol(X)/3),$$
where $X$ is the subset of the cone at $v$ over $Q$ consisting of
points in that cone closer to $v$ than to any other vertex of $Q$.
The extreme point of $X$ has distance at least $\sqrt2$ from $v$
(since $\eta^+$ and hence the circumradius of $Q$ are at least
$\sqrt2$).  Thus,
    $$\svor(Q) \le \svor(Q,\sqrt2).$$
We have $\svor(Q,\sqrt2)\le0$ as in the previous paragraph, by
Lemma~\ref{lemma:rogers-app}, \ref{lemma:wedge}, and
\ref{lemma:cone}.
\end{proof}



The following theorem is also one of the main results of this
\chap. It is a key part of the proof of local optimality.


\begin{theorem}\label{lemma:quad0} Let $(R,D)$ be a quad cluster.
Then $\sigma_R(v,\Lambda)\le 0$.
\end{theorem}\index{cluster!quad}

\begin{proof}
The types of quad clusers have been classified in Lemma~\ref{lemma:quad-class}.
We prove the bound for each type.
If it consists of two flat quarters, then the result is
Lemma~\ref{lemma:quarter0}.  If it is a quartered octahedron with
four upright quarters, then the result again follows from
Lemma~\ref{lemma:quarter0}.  If it is a mixed quad cluster,
the result follows from Lemma~\ref{lemma:1.04}.  Finally,
if it is a pure quad cluster, then an upper bound on the score
is given by $\sigma_R(D,\sqrt2)$ by Lemma~\ref{lemma:pure0}.  
\end{proof}







\section{Local Optimality}%DCG 8.1, p72
\label{sec:local-opt}

\begin{lemma}  %=claim\label{claim-F}
Contravening centered packings $(v,\Lambda)$ exist such that
$\sigma(v,\Lambda)=8\pt$. If $(v,\Lambda)$ is a contravening centered packing, and
if the hypermap of $(v,\Lambda)$ is isomorphic to $G_{fcc}$ or $G_{hcp}$,
then $\sigma(v,\Lambda) \le 8\,\pt$.
\end{lemma} %\label{lemma:local-optimality} in local_opt.tex

\begin{proof}
In each of these two hypermaps there are $8$ triangles and
$6$ quadrilaterals.  In the corresponding centered packings,
there are  eight quasi-regular tetrahedra and six quad clusters.
In each triangular region $\sigma_R(v,\Lambda)\le 1,\pt$ by Lemma~\ref{lemma:1pt}.
In each quad cluser $\sigma_R(v,\Lambda)\le 0$ by Lemma~\ref{lemma:quad0}.  
Thus, the total is
at most $8\,\pt$.
\end{proof}











\section{A Mixed Quad Bound}%DCG 10.5, p107

In Definition~\ref{def:delta-e}, we found a region $\delta(v)$
that lies outside the ball of radius $t_0$ at $0$ but inside
$\op{VC}(0)$.  A formula for its volume is developed
in Section~\ref{sec:anc}.  It introduces two functions
$\cro$ and $\anc$.


\smallskip
If $(P,D)$ is a mixed quad cluster, let $(P,D')$ be the new quad
cluster obtained by removing all the enclosed vertices.  We define
a $V$-cell $V(P,D')$ of $(P,D')$ and the truncation of $V(P,D')$
at $t_0$. We take its score $\op{vor}_{0,P}(D')$  as we do for
standard clusters.  $(P,D')$ does not contain any quarters.

\begin{lemma} \label{lemma:mixed-vor0}
%\proclaim{Proposition 4.7}
If $(P,D)$ is a mixed quad cluster, $\sigma_P(D') <
\vor_{0,P}(v,\Lambda)$.  
% Moreover, we can erase any number of the enclosed 
% vertices over the mixed quad cluster.
\end{lemma}

\begin{proof}
%
Suppose there exists an enclosed vertex that has context
$\x(2,1)$; that is, there is a single upright quarter
$Q=S(y_1,y_2,\ldots,y_6)$ and no additional anchors.  In this
context $\sigma(Q)=\mu(Q)$. Let $v$ be the enclosed vertex.  To
compare $\sigma_P(v,\Lambda)$ with $\vor_{0,P}(D')$, consider the $V$-cell
near $Q$. The quarter $Q$ cuts a wedge of angle $\dih(Q)$ from the
crown at $v$. There is an anchor term for the two anchors of $v$
along the faces of $Q$. Let $V_P^v$ be the truncation at height
$t_0$ of $V_P$ near $v$ and under the four Rogers simplices
stemming from the two anchors.
(Figure~\ref{fig:anchor-quarter:bis} shades the truncated parts of
the quad cluster.) As a consequence
\smallskip
    \begin{equation}
        \op{sovo}(V_P,\lambda_{oct}) <(1-\dih(Q)/(2\pi))\cro(y_1/2)+\anc(y_1,y_2,y_6)
        +\anc(y_1,y_3,y_5) +\op{sovo}(V_P^v,\lambda_{oct}).
    \label{eqn:4.8}
    \end{equation}
Combining this inequality with Lemma~\ref{a:contex21}, we get the
result.

Now suppose there is an enclosed vertex $v$ with context
$\x(3,1)$. Let the quad cluster have corners $v_1$, $v_2$, $v_3$,
$v_4$, ordered consecutively.  Suppose the two quarters along $v$
are $Q_1=\{0,v,v_1,v_2\}$ and $Q_2=\{0,v,v_2,v_3\}$.  We consider
two cases.

\noindent Case 1:  $\dih(Q_1)+\dih(Q_2)<\pi$ or
$\rad(0,v,v_1,v_3)\ge\eta(|v|,2,2t_0)$. In this case, the use of
correction terms to the crown are legitimate as in
Definition~\ref{def:wedge}. Proceeding as in context $\x(2,1)$, we
find that
\smallskip
    \begin{equation}
    \op{sovo}(V_P,\lambda_{oct}) < (1-(\dih(Q_1)+\dih(Q_2))/(2\pi))\cro(|v|/2)
    +\anc(F_1) +\anc(F_2) +\op{sovo}(V_P^v,\lambda_{oct}).
    \label{eqn:4.10}
    \end{equation}
Here $V_P^v$ is defined by the truncation at height $t_0$ under the
$V$-face determined by $v$ and under the Rogers simplices stemming
from the side of $F_i$ that occur in the definition of $\anc$. Also,
$\anc(F_i)=\anc(y_i,y_j,y_k)$ for a face $F_i$ with edges $y_i$
along an upright quarter. By a
calculation\footnote{\calc{554253147}} applied to both $Q_1$ and
$Q_2$, we have
    \begin{equation}
    \op{sovo}(V_P,\lambda_{oct}) +\sum_{i=1}^2\sigma(Q_i)
    < \op{sovo}(V_P^v,\lambda_{oct}) + \sum_{i=1}^2 \svor_0(Q_i).
    \label{eqn:4.11}
    \end{equation}
That is, by truncating near $v$, and changing the scoring of the
quarters to $\svor_0$, we obtain an upper bound on the score.

\noindent Case 2:  $\dih(Q_1)+\dih(Q_2)\ge\pi$ and
    $\rad(0,v,v_1,v_3)\le \eta_0(|v|/2)$.
 In the mixed case,
$\sqr8<|v_1-v_3|$, so
$$\sqr2<{\frac{1}{2}}|v_1-v_3|\le\rad \le \eta_0(|v|/2),$$
and this implies $|v|\ge 2.696$. We
have\footnote{\calc{855677395}}
$$\sum_{i=1}^2 \sigma(Q_i) < \sum_{i=1}^2 \svor_0(Q_i) +
\sum_{i=1}^2 0.01(\pi/2-\dih(Q_i))< \sum_{i=1}^2 \svor_0(Q_i).$$
Inequality~\ref{eqn:4.11} holds, for $V_P^v=V_P$.

In the general case, we run over all enclosed vertices $v$ and
truncate around each vertex.  For each vertex we obtain
Inequality~\ref{a:context21} or \ref{eqn:4.11}. These inequalities can
be coherently combined over multiple enclosed vertices because the
$V$-faces were associated with different vertices $v$ and none of
the Rogers simplices used in the terms $\anc()$ overlap. More
precisely, if $Z$ is a set of enclosed vertices, set $V_P^Z =
\cap_{v\in Z} V_P^v$, and $V_P^{v,Z} = V_P^Z\cap V_P^v$. Coherence
means that we obtain valid inequalities by adding the superscript
$Z$ to $V_P$ and $V_P^v$ in Inequalities~\ref{context21} and
\ref{eqn:4.11}, if $v\not\in Z$. In sum,
    $\sigma_P(v,\Lambda) < \vor_{0,P}(v,\Lambda)$.
%
\end{proof}


%% INTRO SPIV

\section{Quarters} %DCG 11.
    \label{sec:upright}
    \oldlabel{3}


%\section{Erasing Upright Quarters} %DCG 11.1,p.112 (deleted)
    \oldlabel{3.1}

% Fix an exceptional cluster $R$. 

%\section{Truncation} (deleted)
    \oldlabel{3.2}

%% XX Clean this up. What if something is masked?
%% In fact, just get rid of "erasing" It is so messy.


\subsection{Contexts} %DCG 11.2, p.113
    \oldlabel{3.3}

The context $\x(3,0)$ is to be regarded as two
quasi-regular tetrahedra sharing a face rather than as three
quarters along a diagonal.  In particular, by
Definition~\ref{def:q-system}, the upright quarters do not belong
to the $Q$-system.

\subsection{Slices} %DCG 11.5,p.115
    \oldlabel{3.6}
    \label{sec:slice}  % was sec:anchored-simplex

Let $\{0,v\}$ be an upright diagonal, and let
$v_1,v_2,\ldots,v_k=v_1$ be its anchors, ordered cyclically around
$\{0,v\}$.  This cyclic order gives dihedral angles between
consecutive anchors around the upright diagonal. We define the
dihedral angles so that their sum is $2\pi$, even though this will
lead us to depart from our usual conventions by assigning a
dihedral angle greater than $\pi$ when all the anchors are
concentrated in some half-space bounded by a plane through
$\{0,v\}$. When the dihedral angle of $S=\{0,v,v_i,v_{i+1}\}$ is at
most $\pi$, we say that $S$ is a {\it slice\/} if
$|v_i-v_{i+1}|\le3.2$. (The constant $3.2$ appears throughout this
\chap.) All upright quarters are slices. If an upright
diagonal is completely surrounded by slices, the
upright diagonal is sometimes called a {\it loop}. If
$|v_i-v_{i+1}|>3.2$ and the angle is less than $\pi$, we say there
is a {\it gap\/} around $\{0,v\}$ between $v_i$ and $v_{i+1}$.

To understand how the interiors of slices meet, we
need a bound satisfied by vertices enclosed over a slice.


\begin{lemma}
    \label{lemma:anc-simplex-not-enc}
A vertex $w$ of height between 2 and $2\sqrt{2}$, enclosed in the cone
over a slice $\{0,v,v_1,v_2\}$ with diagonal $\{0,v\}$ satisfies
$|w-v|\le 2t_0$. In particular, if $|w|\le 2t_0$, then $w$ is an anchor.
\end{lemma}

\begin{proof}
This appears as Lemma~\ref{tarski:anc-simplex-not-enc}.
\end{proof}


\begin{corollary}
A vertex of height at most $2t_0$ is never enclosed over a slice.
\end{corollary}

\begin{proof}  If so, it would be an anchor to the upright diagonal, contrary to
the assumption that the slice is formed by consecutive
anchors.
\end{proof}


\subsection{Anchored simplices do not overlap} %DCG 11.6, p.116
    \oldlabel{3.7}



\begin{definition}\index{unconfined@3-unconfined}
 \index{crowded4@3-crowded}\index{crowded3@4-crowded}
% Def'n copied from linprog.tex
Consider an upright diagonal that is not a loop. Let $R$ be the
standard region that contains the upright diagonal and its
surrounding quarters.  Assume we are in the context $(4,1)$ or
$(5,1)$.  In the context $(4,1)$, suppose that there does not exist
a plane through the upright diagonal such that all three quarters
lie in the same half-space bounded by the plane. Then we say that
the context is {\it $3$-unconfined}. If such a plane exists, we say
that the context is $3$-crowded. We call the context $(5,1)$ a
$4$-crowded upright diagonal. Sections~\ref{x-3.4} and \ref{x-3.5}
reduce everything to contexts with four or five anchors around each
vertex.  If there are $5$ darts, 
Remark~\ref{rem:5dart} shows that we can assume at most one
gap. This gives contexts $(5,0)$ and $(5,1)$.  If there are four
anchors, then Lemma~\ref{x-3.9.1} will dismiss all contexts except
$(4,0)$ and $(4,1)$. Thus, every upright diagonal is exactly one of
the following: a loop, $3$-unconfined, $3$-crowded, or $4$-crowded.
%\def\Sfour{{{\cal\mathbf S}_4^+}}  --> $4$-crowded upright diagonal
%\def\Sminus{{{\cal\mathbf S}_3^-}} --> $3$-crowded upright diagonal
%\def\Splus{{{\cal\mathbf S}_3^+}}  --> $3$-unconfined upright diagonal
\end{definition}


This lemma is a consequence of the two others that follow. The
context of the lemma is the set of slices that have
not been erased by previous reductions.

\begin{lemma}
    \label{lemma:anchor-no-overlap}
The interiors of slices do not meet.
\end{lemma}

The remaining contexts have four or  five anchors. Let $w$ and the
slice $S=\{0,v,v_1,v_2\}$ be as in Section~\ref{x-3.6}.
Our object is to describe the local geometry when an upright
diagonal is enclosed over a slice. If $|v_1-v_2|\le
2\sqrt{2}$, we have seen in Lemma~\ref{lemma:double-face} that
there can be no enclosed upright diagonal with $\ge 4$ anchors
over the slice $S$.

Assume  $|v_1-v_2|>2\sqrt{2}$. Let $w_1,\ldots,w_k$, $k\ge4$, be the
anchors of $\{0,w\}$, indexed consecutively. The anchors of $\{0,w\}$ do not
lie in $C(S)$, and the triangles $\{0,w,w_i\}$ and $\{0,v,v_j\}$ do not
overlap. Thus, the plane $\{0,v_1,v_2\}$ separates $w$ from
$\{w_1,\ldots,w_k\}$. Set $S_i=\{0,w,w_i,w_{i+1}\}$.
By a calculation\footnote{\calc{83777706}} %A8
%$\A_8$,
    $$\pi\ge \dih(S_1)+\cdots+\dih(S_{k-1})\ge (k-1)0.956.$$

Thus, $k=4$. The common upright diagonal  of the three simplices
$\{S_i\}$ is {\it $3$-crowded}.  We claim that
$\{v_1,v_2\}=\{w_1,w_4\}$. Suppose to the contrary that, after
reindexing as necessary, $S_0=\{0,w,w_1,v_1\}$ is a simplex, with
$v_1\ne w_1$, that does not overlap $S_1,\ldots,S_3$. Then $\pi\ge
\dih(S_0)+\cdots+\dih(S_3)$. So
    $0.28\ge \pi-3(0.956)\ge \dih(S_0)$.
A calculation\footnote{\calc{83777706}} %A8
now implies that $|w-v_1|\ge 2\sqrt{2}$.

By Lemma~\ref{tarski:336}, the four vertices
$\{0,w,v_1,v_2\}$ cannot be coplanar.
We have that $2\sqrt{2}\ge|w|$ and by Lemma~\ref{tarski:E:part4:1},
we also have $|w|>2\sqrt2$.
This contradiction establishes that $v_1=w_1$.

\begin{lemma}
Around a $3$-crowded upright diagonal, all of the slices
are quarters.
\end{lemma}

\begin{proof}  The proof makes use of constants and inequalities from
several different calculations.\footnote{\calc{815492935}} %A2
\footnote{\calc{83777706}} %A8
\footnote{\calc{855294746}} %A12
%$\A_2$, $\A_8$, and $\A_{12}$.
The dihedral angles are at most $\pi-
2(0.956) < 1.23$. This forces $y_4\le 2t_0$, for each simplex $S$.
So they are all quarters.
\end{proof}

\begin{lemma}
    \oldlabel{3.7.1}\label{lemma:3-crowded}
If there is $3$-crowded upright diagonal, then the three 
slices squander more than $0.5606$ and score at most $-0.4339$.
\end{lemma}


\begin{proof}  The proof makes use of constants and inequalities from
several different calculations.\footnote{\calc{815492935}} %A2
\footnote{\calc{83777706}} %A8
\footnote{\calc{855294746}} %A12
%$\A_2$, $\A_8$, and $\A_{12}$.
The three slices squander at
least
    $$
    3 (1.01104) - \pi (0.78701) > 0.5606.
    $$
The bound on score follows similarly from $\nu<-0.9871+0.80449\dih$.
\end{proof}

\begin{lemma}
    \oldlabel{3.7.2}
If a simplex at a $3$-crowded upright diagonal meets at an
interior point with a slice, the centered packing does
not contravene.
\end{lemma}

\begin{proof}
Suppose that $\{0,v,v_1,v_2\}$ is a slice that another
slice overlaps, with $\{0,v\}$ the upright diagonal.  Let
$\{0,w\}$ be a $3$-crowded upright diagonal. We score the two
simplices $S'_i = \{0,v,w,v_i\}$ by truncation at $\sqrt{2}$.
Truncation at $\sqrt{2}$ is justified by Lemma~\ref{tarski:old372}.
A calculation\footnote{\calc{855294746}} %A12
gives
%$\A_{12}$,
    $$\tau_V(S'_1,\sqrt{2})+\tau_V(S'_2,\sqrt{2})\ge 2(0.13) +
        0.2(\dih(S'_1)+\dih(S'_2)-\pi) > 0.26.
    $$
Together with the three simplices around the $3$-crowded upright
diagonal that squander at least $0.5606$, we obtain the stated
bound.
\end{proof}



\subsection{Four and Five Darts} %DCG 11.7, p.118 % Was "Five Anchors"
    \oldlabel{3.8}
    \label{sec:five-anchors}
    %\section{Four darts} %DCG 11.8, p. 120 % Was "Four anchors"
    \oldlabel{3.9}



\begin{remark}\label{rem:5dart}
The situation of five darts at an upright diagonal is
described in Section~\ref{sec:5updart}.
%    \oldlabel{3.8.1}
\end{remark}

\begin{definition}
Let a $3$-unconfined node be a node that is an upright diagonal 
with four darts and one gap in a situation where none of
the quarters along this upright diagonal masks a flat quarter.
\end{definition}

\begin{lemma}\dcg{Cor~11.25}{122}
If there are four anchors and if the upright diagonal is enclosed over a
flat quarter, then there are four slices and at least three
quarters around the upright diagonal.
\end{lemma}

\begin{proof}
This follows by Lemma~\ref{tarski:dcg-p122}.
\end{proof}


\subsection{penalty constants}

\begin{definition}\index{zzxiG@$\xiG$}\index{zzxiV@$\xiV$}
We set $\xiG = 0.01561$, $\xiV = 0.003521$, $\xiG'=0.00935$,
$\xik=-0.029$, $\xikG = \xik+\xiG = -0.01339$.
\end{definition}

The first two constants appear in calculations%
%$\A_{10}$ and $\A_{11}$ as
\footnote{\calc{73974037}} %A10
\footnote{\calc{764978100}} %A11
as penalties for erasing upright quarters that are compressed, and
decompressed, respectively. $\xiG'$ is an improved bound on the
penalty for erasing when the upright diagonal is at least $2.57$.
Also, $\xik$ is an upper bound\footnote{\calc{618205535}} %A9
 on $\kappa$, when the
upright diagonal is at most $2.57$.  If the upright diagonal is at
least $2.57$, then we still obtain the bound%
\footnote{\calc{618205535}} %A9
$\xikG =-0.02274+\xiG'$ on the sum of $\kappa$ with the
penalty from erasing an upright quarter.

Recall that $\xiV=0.003521$, $\xiG=0.01561$, $\xiG'=0.00935$. They are
the penalties that result from erasing a 
decompressed upright quarter, a comprssed upright quarter, 
and a comprssed upright quarter
with diagonal $\ge2.57$. (See calculations.%
\footnote{\calc{73974037}} %A10
\footnote{\calc{764978100}} %A11)


%
%
%\section{Summary} %DCG 11.9 p 122
%    \oldlabel{3.10}
%    \label{sec:upright-summary}
%
%The following index summarizes the cases of upright quarters that have
%been treated in Section~\ref{sec:upright}. If the number of anchors is
%the number of slices (no gaps), the results appear in
%Section~\ref{x-5.11}. Every other possibility has been treated.
%
%    \begin{itemize}
%    \item 0,1,2 anchors\hfill Sec.~\ref{x-3.3}
%    \item $3$ anchors \hfill Sec.~\ref{x-3.4}
%        \begin{itemize}
%        \item context $\x(3,0)$
%        \item context $\x(3,1)$
%        \item context $\x(3,2)$
%        \item context $\x(3,3)$
%        \end{itemize}
%    \item $4$ anchors \hfill Sec.~\ref{x-3.9}
%        \begin{itemize}
%        \item $0$ gaps (Section~\ref{x-5.11})
%        \item $1$ gap
%        \item $2$ or more gaps
%        \end{itemize}
%    \item $5$ anchors \hfill Sec.~\ref{x-3.8}
%        \begin{itemize}
%        \item $0$ gaps (Section~\ref{x-5.11})
%        \item $1$ gap ($4$-crowded)
%        \item $2$ or more gaps
%        \end{itemize}
%    \item $6$ or more anchors \hfill Sec.~\ref{x-3.5}
%    \end{itemize}
%
%
%\smallskip
%By truncation and various comparison lemmas, we have entirely eliminated
%upright diagonals except when there are between three and five anchors.
%We may assume that there is at most one gap around the upright
%diagonal.
%
%\smallskip
%1.  Consider a slice $Q$ around a remaining upright
%diagonal. The score of is $\nu(Q)$ if $Q$ is a quarter, the
%analytic function $\svor(Q)$ if the simplex is of type $\SC$
%(Section~\ref{x-2.5}), and the truncated function $\svor_0(Q)$
%otherwise.
%
%\smallskip
%2.  Consider a flat quarter $Q$ in an exceptional cluster. An
%upper bound on the score is obtained by taking the maximum of all
%of the following functions that satisfy the stated conditions on
%$Q$.  Let $y_4$ denote the length of the diagonal and $y_1$ be the
%length of the opposite edge.
%
%(a)  The function $\mu(Q)$.
%
%(b)  $\svor_0(Q) - 0.0063$, if $y_4\ge 2.6$ and $y_1\ge
%2.2$.\hfill
%    (Lemma~\ref{lemma:0.008})
%
%(c)  $\svor_0(Q) - 0.0114$, if $y_4\ge 2.7$ and $y_1\le 2.2$.
%    \hfill (Lemma~\ref{lemma:0.008})
%
%(d)  $\nu(Q_1)+\nu(Q_2)+\svor_x(S)$, if there is an enclosed
%vertex
%    $v$ over $Q$ of height between $2t_0$ and $2\sqrt{2}$ that
%    partitions the convex hull of $(Q,v)$ into two upright quarters
%    $Q_1$, $Q_2$ and a third simplex $S$. Here $\svor_x=\svor$
%    if $S$ is of type $\SC$, and $\svor_x=\svor_0$ otherwise.
%    \hfill (Lemma~\ref{lemma:unerased})
%
%(e)  $\svor(Q,1.385)$ if the simplex is of type $\SB$
%(Section~\ref{x-2.5}).
%
%(f) $\svor_0(Q)$ if the simplex is an isolated quarter with
%    $\max(y_2,y_3)\ge2.23$, $y_4\ge2.77$,
%    and $\eta_{456}\ge\sqrt2$.
%
%\smallskip
%3.   If $S$ is a simplex is of type $\SA$, its score is
%$\svor(S)$. (Section~\ref{x-2.5}.)
%
%\smallskip
%
%4.  Everything else is scored by the truncation $\vor_0$.
%    Formula~\ref{eqn:3.7} is used on these remaining pieces.
%    On top of what is obtained for the standard cluster by summing all
%these terms, there is a penalty $\pi_0=0.008$ each time a
%$3$-unconfined upright diagonal is erased.
%
%\smallskip
%5.  The remaining upright diagonals that are not completely
%surrounded by slices are $3$-unconfined, $3$-crowded,
%or $4$-crowded from Section~\ref{x-3.7}, \ref{x-3.8},  and
%\ref{x-3.9}.
%


\subsection{Some flat quarters} %DCG 11.10, p124
    \oldlabel{3.11}
    \label{sec:some-flat}




In the next lemma, we score a flat quarter by any of the functions
on the given domains
     $$\hat\sigma=
        \begin{cases}
            \Gamma,& \eta_{234},\eta_{456}\le\sqrt2,\\
             \svor, &\eta_{234}\ge\sqrt2,\\
            \svor_0, & y_4\ge 2.6, y_1\ge2.2,\\
            \svor_0, & y_4\ge 2.7,\\
            \svor_0,& \eta_{456}\ge\sqrt2.
        \end{cases}
    $$

\begin{lemma}
    \oldlabel{3.11.1}
    \label{lemma:hatsigma}
$\hat\sigma$ is an upper bound on the functions in
Section~\ref{x-3.10}.2(a)--(f). That is, each function in
Section~\ref{x-3.10}.2 is dominated by some choice of $\hat\sigma$.
\end{lemma}

\begin{proof}  The only case in doubt is the function of 3.10(d):
$$\nu(Q_1)+\nu(Q_2)+\svor_x(S).$$ This is established by the
following lemma.
\end{proof}


We consider the context $\x(3,1)$ that occurs when two upright
quarters in the $Q$-system lie over a flat quarter. Let $\{0,v\}$ be
the upright diagonal, and assume that $\{0,v_1,v_2,v_3\}$ is the
flat quarter, with diagonal $\{v_2,v_3\}$. Let $\sigma$ denote the
score of the upright quarters and other slice lying
over the flat quarter.

\begin{lemma}\label{lemma:min0-svor}
    \oldlabel{3.11.2}
    $\sigma\le \min(0,\svor_0)$.
\end{lemma}

\begin{proof}
The bound of $0$ is established in Theorem~\ref{lemma:quad0}.
The bound of $\svor_0$ is established in Lemma~\ref{a:min0-svor}.
\end{proof}



%\chapter{Further Bounds}%DCG Sec. 14, p. 157
    \oldlabel{5.12}
    \label{sec:fb}



\section{Small dihedral angles} %DCG 14.1, p157
\label{sec:small-dih}

Recall that Section~\ref{sec:the-main-theorem} defines an integer $n(R)$
that is equal to the number of sides if the region is a polygon.  Recall
that if the dihedral angle along an edge of a standard cluster is at
most $1.32$, then there is a flat quarter along that edge
(Lemma~\ref{x-3.11.4}).

\begin{lemma}
    \oldlabel{5.12.1}\dcg{Lemma~14.1}{157}
Let $R$ be an exceptional cluster with a dihedral angle
$\le1.32$ at a vertex $v$. Then $R$ squanders $>t_n+1.47\,\pt$, where
$n=n(R)$.
\end{lemma}

\begin{proof}
In most cases we establish the stronger bound $t_n+1.5\,\pt$. In the
proof of Theorem~\ref{thm:the-main-theorem}, we erase all upright
diagonals, except those completely surrounded by slices. The
contribution to $t_n$ from the flat quarter $Q$ at $v$ in that proof is
$D(3,1)$ (Sections~\ref{x-4.5} and Inequalities~\ref{eqn:tau>D(n,k)}).
Note that $\epsilon_\tau(Q)=0$ here because there are no deformations.
If we replace $D(3,1)$ with $3.07\,\pt$ from Lemma~\ref{x-3.11.4}, then
we obtain the bound. Now suppose the upright diagonal is completely
surrounded by slices. Analyzing the constants of
Section~\ref{x-5.11}, we see that $\DLP(n,k)-D(n,k)>1.5\,\pt$ except
when $(n,k)=(4,1)$.

Here we have four slices around an upright diagonal. Three
of them are quarters.  We erase and take a penalty. Two possibilities
arise.  If the upright diagonal is enclosed over the flat quarter, its
height is $\ge2.6$ by Lemma~\ref{tarski:last:E} and the top face of the
flat quarter has circumradius at least $\sqrt2$.  The penalty is
$2\xiG'+\xiV$, so the bound holds by the last statement of
Lemma~\ref{x-3.11.4}.

If, on the other hand, the upright diagonal is not enclosed over the
flat diagonal, the penalty is $\xiG+2\xiV$.  In this case, we obtain the
weaker bound $1.47\,\pt+t_n$:
    $$3.07\,\pt > D(3,1) + 1.47\,\pt +\xiG+2\xiV.$$
\end{proof}

\begin{remark} \label{remark:1.47}
If there are $r$ nonadjacent vertices with dihedral angles
$\le1.32$, we find that $R$ squanders $t_n+r(1.47)\,\pt$.
\end{remark}

In fact, in the proof of the lemma, each $D(3,1)$ is replaced with
$3.07\,\pt$ from Lemma~\ref{x-3.11.4}.  The only questionable case
occurs when two or more of the vertices are anchors of the same upright
diagonal (a loop). Referring to Section~\ref{x-5.11}, we have the
following observations about various contexts.

\begin{itemize}
    \item $(4,1)$ can mask only one flat quarter and it is treated in the
lemma.
    \item $(4,2)$ can mask only one flat quarter and
    $\DLP(4,2)-D(4,2)>1.47\,\pt$.
    \item $(5,0)$ can mask two flat quarters.  Erase the five upright quarters,
        and take a penalty $4\xiV+\xiG$.  We get
    $$D(3,2)+2(3.07)\,\pt > t_5+4\xiV+\xiG+2(1.47)\,\pt.$$
    \item $(5,1)$ can mask two flat quarters, and $\DLP(5,1)-D(5,1)>2(1.47)\,\pt$.
\end{itemize}




\section{A particular $4$-circuit} %DCG 14.2, p158

This subsection bounds the score of a particular $4$-circuit on a
contravening planar hypermap.  The interior of the circuit
consists of two faces: a triangle and a pentagon.  The circuit and
its enclosed vertex are show in Figure \ref{fig:no4circuit} with
vertices marked $p_1,\ldots,p_5$.  The vertex $p_1$ is the
enclosed vertex, the triangle is $(p_1,p_2,p_5)$ and the pentagon
is $(p_1,\ldots,p_5)$.

\begin{figure}[htb]
  \centering
  \myincludegraphics{\ps/no4circuit.eps}
  \caption{A $4$-circuit}
  \label{fig:no4circuit}
\end{figure}

Suppose that $(v,\Lambda)$ is a centered packing whose associated hypermap
contains such triangular and pentagonal standard regions. Recall
that $(v,\Lambda)$ determines a set $U(v,\Lambda)$ of vertices in Euclidean
$3$-space of distance at most $2t_0$ from the origin, and that
each vertex $p_i$ can be realized geometrically as a point on the
unit sphere at the origin, obtained as the radial projection of
some $v_i\in U(v,\Lambda)$.

\begin{lemma}\dcg{Lemma~14.3}{158}  
One of the edges $\{v_1,v_3\}$, $\{v_1,v_4\}$ has
length less than $2\sqrt{2}$.  Both of the them have lengths less
than $3.02$. Also, $|v_1|\ge2.3$.
\end{lemma}

\begin{proof} This is Lemma~\ref{tarski:4circuit}.
\end{proof}


There are restrictive bounds on the dihedral angles of the
simplices $\{0,v_1,v_i,v_j\}$ along the edge $\{0,v_1\}$. The
quasi-regular tetrahedron has a dihedral angle of at most%
\footnote{\calc{984463800}} $1.875$.  The dihedral angles of the
simplices $\{0,v_1,v_2,v_3\}$, $\{0,v_1,v_5,v_4\}$
adjacent to it are at most%
\footnote{\calc{821707685}}  $1.63$. The dihedral angle of the
remaining simplex $\{0,v_1,v_3,v_4\}$ is at most%
\footnote{\calc{115383627}} $1.51$.   This leads to lower bounds
as well. The quasi-regular tetrahedron has a dihedral angle that
is at least $2\pi - 2(1.63)-1.51 > 1.51$.  The dihedral angles
adjacent to the quasi-regular tetrahedron is at least $2\pi-
1.63-1.51-1.875> 1.26$. The remaining dihedral angle is at least
$2\pi-1.875-2(1.63) > 1.14$.

A centered packing $(v,\Lambda)$ determines a set of vertices $U(v,\Lambda)$ that
are of distance at most $2t_0$ from $v$.  Three
consecutive vertices $p_1$, $p_2$, and $p_3$ of a standard region
are determined as the projections to the unit sphere of three
corners $v_1$, $v_2$, and $v_3$, respectively in $U(v,\Lambda)$. By
Lemma~\ref{lemma:1.32}, if the interior angle of the standard
region is less than $1.32$, then $|v_1-v_3|\le\sqrt{8}$.

\begin{lemma}\dcg{Lemma~14.4}{159} \label{lemma:11.16}
These two standard regions $F=\{R_1,R_2\}$ give
    $\tau_F(v,\Lambda) \ge 11.16\,\pt$.
\end{lemma}

\begin{proof}
Let $\dih$ denote the dihedral angle of a simplex along a given
edge. Let $S_{ij}$ be the simplex $\{0,v_1,v_i,v_j\}$, for
$(i,j)=(2,3),(3,4), (4,5),(2,5)$. We have $\sum_{(4)}\dih(S_{ij})
= 2\pi$. Suppose one of the edges $\{v_1,v_3\}$ or $\{v_1,v_4\}$ has
length $\ge2\sqrt2$. Say $\{v_1,v_3\}$.

We have\footnote{\calc{572068135}, \calc{723700608},
\calc{560470084}, and \calc{535502975}}
    $$
    \begin{array}{lll}
    \tau(S_{25}) &- 0.2529\dih(S_{25}) > -0.3442,\\
    \tau_0(S_{23}) &- 0.2529\dih(S_{23}) > -0.1787,\\
    \hat\tau(S_{45}) &- 0.2529\dih(S_{45}) > -0.2137,\\
    \tau_0(S_{34}) &- 0.2529\dih(S_{34}) > -0.1371.\\
    \end{array}
    $$
We have a penalty $\xiG$ for erasing, so that
    $$
    \begin{array}{lll}
        \tau(v,\Lambda) &\ge \sum_{(4)}\tau_x(S_{ij}) - 5\xiG\\
                &>2\pi(0.2529)-0.3442\\
                &\qquad -0.1787-0.2137-0.1371-5\xiG\\
                &>11.16\,\pt,
    \end{array}
    $$
where $\tau_x=\tau,\hat\tau,\tau_0$ as appropriate.

Now suppose $\{v_1,v_3\}$ and $\{v_1,v_4\}$ have length $\le2\sqrt2$.
If there is an upright diagonal that is not enclosed over either
flat quarter, the penalty is at most $3\xiG+2\xiV$. Otherwise, the
penalty is smaller: $4\xiG'+\xiV$. We have
    $$
    \begin{array}{lll}
    \tau(v,\Lambda)
    &\ge \sum_{(4)}\tau(S_{ij})-(3\xiG+2\xiV)\\
    &>2\pi(0.2529)-0.3442\\
    &\qquad -2(0.2137)-0.1371 -(3\xiG+2\xiV)\\
    &>11.16\,\pt.\\
    \end{array}
    $$
\end{proof}



\section{mixed bound} %DCG 10.14, p105 (moved -1.04 bound)
% Rewritten.

\begin{lemma}\dcg{Lemma~10.14}{105} \label{lemma:1.04}
%\proclaim{Proposition 4.1}
The score of a mixed quad cluster is less than $-1.04\,\pt$.
\end{lemma}

\begin{proof}
In a mixed quad cluster there is at least one enclosed vertex.
Any enclosed vertex in a quad cluster has length at least $2t_0$
by Lemma~\ref{lemma:enclosed}. In particular, the anchors of an
enclosed vertex are corners of the quad cluster. There are no flat
quarters.

We erase all of the enclosed vertices except for one, which
we can do with the estimate of Lemma~\ref{lemma:mixed-vor0}.
The enclosed vertex has zero, one, or two anchors.  The upright
quarters around that vertex are scored with the function appropriate
to its context.
The rest of the quad cluster is estimated by the function $\vor_0$.

If the enclosed vertex has zero anchors, then the entire quad
cluster $(R,D)$ satisfies $$\sigma_R(v,\Lambda)\le \vor_{0,R}(v,\Lambda).$$
The right-hand side of this equation is independent of the enclosed
vertex $v$.  In particular, we can move it until 
two consecutive  corners of the quad cluster are anchors of $\{0,v\}$
and one of the distances $|v-v_i|=2.51$ for one of those two corners.

A calculation shows%
\footnote{\calc{XX}. This is a new interval calculation that
needs to be verified: If $(R,D)$ is any quad cluster with both
diagonals greater than $\sqrt8$, and some distance $|v_i-v_{i+1}|>2.38$,
then $\vor_{0,R}(v,\Lambda) < -1.04\,\pt$.} 
that if $\sigma_R(v,\Lambda) \ge -1.04\,\pt$, then 
$|v_i-v_{i+1}|\le 2.38$ for $i=1,2,3,4$.  We assume these
constraints.

We may use the deformation of Lemma~\ref{x-4.9.2}
at each vertex $v_i$ that is not an anchor of $\{0,v\}$ so either
it becomes an anchor (with $|v-v_i|=2t_0$), or it satisfies
  $$|v_i-v_{i+1}|=|v_i-v_{i-1}|=2;\quad |v_i|\in\{2,2t_0\}.$$ 

In the course of deformation (say of corner $v_1$),  
the diagonal $\{v_1,v_3\}$ may reach length $\sqrt8$.  
In this case, we stop
the deformation at that corner and continue with another
corner (say $v_2$, if it exists)
that is not an anchor until its deformation
is complete, and then return to the complete the deformation
at $v_1$.  We cannot have both diagonals $\{v_1,v_3\}$ and $\{v_2,v_4\}$
drop to $\sqrt8$, by Lemma~\ref{XX:tarski}, unless $\{0,v\}$ has four
anchors.  The result, after the deformations are complete, may
have a diagonal of length less than $\sqrt8$.  At this point,
that constraint is no longer needed.  Its only purpose was to
fulfill a hypothesis of Lemma~\ref{x-4.9.2}.

We claim that if corner $v_i$ is not an anchor of $\{0,v\}$,
then $2t_0 < |v-v_i| \le 2.95$.  In fact, if $|v-v_i| > 2.96$,
then the sum of the dihedral angles around $\{0,v\}$ satisfies%
\footnote{\calc{XX}.  These are new interval calculations 
  that need to be verified.  For an upright quarter in
  $[2.51,\sqrt8][2,2.51]^2[2,2.38][2,2.51]^2$, $\dih < 2.01$.
For a simplex in $[2.51,\sqrt8][2,2.51]^2 [2] [2.96,++][2,++]$,
$\dih < 1.13$.}
 $$
 2\pi = \sum_{i=1}^4 \dih_V(\{0,v\},\{v_i,v_{i+1}\}) < 2(2.01)+2(1.13) < 2\pi.
 $$

We write the upper bound on $\sigma_R(v,\Lambda)$ 
as a sum of contributions from the four simplices
$\{0,v,v_i,v_{i+1}\}$.    That contribution is the score of the
upright quarter, if the simplex is an upright quarter in the $Q$-system.
Otherwise, we use the upper bound $\vor_0$.

If $\{0,v\}$ has just two anchors (say at adjacent corners $v_1,v_4$),
then the contribution from $\{0,v,v_1,v_4\}$ is at most $0$.  In
fact, if the simplex is an upright quarter in the $Q$-system, this
follows from Lemma~\ref{XX}.  If the simplex is scored by $\vor_0$,
then (say) $|v-v_1|=2t_0$ and the result is a calculation.%
\footnote{\calc{XX}.  This is a new calculation.  
It needs to be verified.  If an upright
simplex satisfies $[2.51,\sqrt8][2,2.51]^2[2,2.38][2.51][2,2.51]$,
then $\vor_0(S) < 0$.}  Moreover, the contributions from the other
three simplices give%
\footnote{\calc{XX}. These are new calculations.  They need
to be verified.  On
$[2.51,\sqrt8]\{2,2.51\}^2[2][2.51,2.96]^2$, $\vor_0 < -0.0475$; and
on $[2.51,\sqrt8]\{2,2.51\}[2,2.51][2][2.51,2.96][2,2.51]$,
   $\vor_0 < -0.0055$.
}
 $$\sum_{i=1}^3\vor_{0,R,V}(0,\{v,v_i,v_{i+1}\}) <
   -0.0055 - 0.0475 - 0.0055 < -1.04\,\pt.$$

If $\{0,v\}$ has three anchors (say $v_1,v_2,v_4$), then
writing $\sigma'_i$ for the upper bound on the contribution
from $\{0,v,v_i,v_{i+1}\}$, we have%
\footnote{\calc{XX}.  These are new calculations that need to
be verified.  For an upright quad with $y_4\in[2,2.38]$, we
have $\sigma(Q) < \epsilon_1 -0.08(\dih(Q)-\pi/2)$.  I added
this small $\epsilon_1$ to make it more likely that it will follow
by a linear program for the other inequalities for $\sigma(Q)$.
Note that there are various cases, according to the context; we
haven't erased anything here.  We have the similar
  $\vor_0 < \epsilon_1 -0.08(\dih(Q)-\pi/2)$,
when $[2.51,\sqrt8][2,2.51]^2[2,2.38][2,2.51][2.51]$.  Then we
have $\vor_0 < \epsilon_2  -0.08(\dih(Q)-\pi/2)$,
when $[2.51,\sqrt8][2,2.51]^2[2][2,2.51][2.51,2.96]$.
}
  $$
  \sum_{i=1}^4\sigma'_i < 
  \sum (\epsilon_i -0.08 (\dih_V(\{0,v\},\{v_i,v_{i+1}\})-\pi/2))
  = \sum\epsilon_i = -1.04\,\pt.
  $$
where $\epsilon_2=\epsilon_3 = -0.54\,\pt$ and $\epsilon_1=\epsilon_4 =
0.02\,\pt$.
\end{proof}

%% OLD PROOF. 
%
%We generally truncate the $V$-cell at $\sqr2$ as in the proof of
%Theorem~\ref{lemma:quad0}.  By that lemma, it breaks the $V$-cell
%into pieces whose score is nonpositive. Thus, if we identify
%certain pieces that score less than $-1.04\,\pt$, the result
%follows. Nevertheless, a few simplices will be left untruncated in
%the following argument. We will leave a simplex untruncated only
%if we are certain that this is justified.
%%% Avoid mention of orien-tation.
%% Each of its faces has positive orien-tation
%% and that the simplices sharing a face $F$ with $S$ either lie in
%% the $Q$-system or have positive orien-tation along $F$.  
%If so, we may use\footnote{\calc{185703487},
%\calc{69785808}, and \calc{104677697}} the function $\svor$ on $S$
%rather than truncation $\svor_0$.
%
%In this proof, by enclosed vertex, we mean one of height at most
%$2\sqrt2$. Let $v$ be an enclosed vertex with the fewest anchors.
%If there are no anchors, the right circular cone $C(h,\eta_0(h))$
%(aligned along $\{0,v\}$; see Definition~\ref{def:cone}) belongs
%to $\op{VC}(0)$, where $\eta_0(h)=\eta(2h,2,2t_0)$ as in
%Definition~\ref{def:eta0} and $|v|=2h$. In fact, if such a point
%lies in $\op{VC}(u)$, with $u \ne v$, then $u$ must be a corner of
%the quad cluster or an enclosed vertex of height at least $2t_0$.
%In either case, the right circular cone belongs to $\op{VC} (0)$.
%By Formula~\ref{lemma:sovoFR}, the score of this cone is
%$2\pi(1-h/\eta_0(h))\phi(h,\eta_0(h))$. An optimization in one
%variable gives an upper bound of $-4.52\,\pt$, for $t_0\le h\le
%\sqr2$.   This gives the bound of $-1.04\,\pt$ in this case.
%
%If there is one anchor,  we cut the cone in half along the plane
%through $\{0,v\}$ perpendicular to the plane containing the anchor
%and $\{0,v\}$. The half of the cone on the far side of the anchor
%lies under the face at $v$ of the $V$-cell.  We get a bound of
%$-4.52\,\pt/2 < -1.04\,\pt$.
%
%In the remaining cases, each enclosed vertex has at least two
%anchors.  Each anchor is a corner of the quad cluster.  Fix an
%enclosed vertex $v$. Suppose that $v_1$, a corner, is an anchor of
%$v$. Assume that the face $\{0,v,v_1\}$ bounds at most one upright
%quarter. We sweep around the edge $\{0,v_1\}$, away from the
%upright quarter if there is one,  until we come to another
%enclosed vertex $v'$ such that $\{0,v_1,v'\}$ has circumradius
%less than $\sqr2$ or such that $v_1$ is an anchor of $\{0,v'\}$.
%If such a vertex $v'$ does not exist, we sweep all the way to
%$v_2$ a corner of the quad cluster adjacent to $v_1$.
%
%Section~\ref{sec:K} defines a function $K$ that we use in
%this proof.
%
%If $v'$ exists, then various
%calculations\footnote{\calc{104677697}, \calc{69785808},
%\calc{586706757}, and \calc{87690094}} give the bound
%$-1.04\,\pt$, depending on the size of the circumradius of
%$\{0,v,v'\}$. This allows us to assume that we do not encounter
%such an enclosed vertex $v'$ whenever we sweep away, as above,
%from the face formed by an anchor.
%
%Now consider the simplex $S=\{0,v_1,v_2,v\}$, where $v_1$ is an
%anchor of $\{0,v\}$.  We assume that it is not an upright quarter.
%There are three alternatives. The first is that $S$ decreases the
%score of the quarter by at least $0.52\,\pt$.
%Calculations\footnote{\calc{185703487} and \calc{441195992}} show
%that this occurs if the circumradius of the face $\{0,v,v_2\}$ is
%less than $\sqr2$, or if the circumradius of the face is greater
%than $\sqr2$, provided that the length of $\{v,v_1\}$ is at most
%$2.2$. The second alternative\footnote{\calc{848147403},
%\calc{969320489}, and \calc{975496332}.} is that the face
%$\{0,v,v_1\}$ of $S$ is shared with a quarter $Q$ and that $S$ and
%$Q$ taken together bring the score down by $0.52\,\pt$. In fact,
%if there are two such simplices $S$ and $S'$ along $Q$, then the
%three simplices $Q$, $S$, and $S'$ pull the
%score\footnote{\calc{766771911}} below $-1.04\,\pt$. The third
%alternative is that there is a simplex $S'=\{0,v,v,v_3\}$ sharing
%the face $\{0,v,v_1\}$, which, like $S$, scores less than
%$-0.31\,\pt$.  In each case, $S$ and the adjacent simplex through
%$\{0,v,v_1\}$ score less than $-0.52\,\pt$. Since $v$ has at least
%two anchors, the quad cluster scores less than $2(-0.52)\,\pt
%=-1.04\,\pt$.
%%
%
%



\section{A particular $5$-circuit} %DCG 14.3, p160

\begin{lemma}\dcg{Lemma~14.5}{160}\label{lemma:6079}  
Assume that $R$ is a pentagonal standard region
    with an enclosed vertex $v$ of height at most $2t_0$.
    %(See Figure~\ref{fig:pent-tri1}.)
    Assume further that
    \begin{itemize}
        \item $|v_i|\le 2.168$ for each of the five corners.
        \item Each interior angle of the pentagon is at most
        $2.89$.
        \item If $v_1$, $v_2$, $v_3$ are consecutive corners over
        the pentagonal region, then $$|v_1-v_2|+|v_2-v_3|<4.804.$$
        \item $\sum_5 |v_i-v_{i+1}|\le 11.407.$
    \end{itemize}
    Then $\sigma_R(v,\Lambda)< -0.2345$ or $\tau_R(v,\Lambda) > 0.6079.$
\end{lemma}

\begin{proof}
Since $-0.4339$ is less than this the lower bound, a $3$-crowded
upright diagonal does not occur. Similarly, since $-0.25$ is less
than the lower bound, a $4$-crowded upright diagonal does not
occur (Lemma~\ref{lemma:4-crowded} and Lemma~\ref{x-3.8}).

Suppose that there is a loop in context $(n,k)=(4,2)$. Again by
Lemma~\ref{lemma:loop} (with $n(R)=7$),
$$\sigma_R(v,\Lambda)  < -0.2345.$$
%The constants come from
%Table~\ref{x-5.11} and  Theorem~\ref{thm:the-main-theorem}.

%If we branch and bound on the triangular faces, this LP-derived
%inequality can be improved to
%    $$\tau[F] < 0.6079.$$

%If there is a loop other than $(4,2)$ and $(4,1)$, the linear
%program becomes infeasible:
%    $$\tau[F] < 0.644 < t_7 + \dloop(n,k) < \tau[F].$$
We conclude that all loops have context $(n,k)=(4,1)$.


{\bf Case 1.}  {\it The vertex $v=v_{12}$ has distance at least
$2t_0$ from the five corners of $U(v,\Lambda)$ over the pentagon.}

%The interval calculations relevant in Case 1 appear in
%~\ref{A.3.8}.

The penalty to switch the pentagon to a pure $\vor_0$ score is at
most $5\xiG$ (see Section~\ref{sec:prep-cluster}).  There cannot
be two flat quarters because then Lemma~\ref{tarski:E:part4:5} gives
$$|v_{12}|>2t_0.$$


{\bf (Case 1-a)} Suppose there is one flat quarter,
$|v_1-v_4|\le2\sqrt2$. There is a lower bound of 1.2 on the
dihedral angles of the simplices $\{0,v_{12},v_i,v_{i+1}\}$.  This
is obtained as follows.  The proof relies on the convexity of the
quadrilateral region.  We leave it to the reader to verify that
the following pivots can be made to preserve convexity.  Disregard
all vertices except $v_1,v_2,v_3,v_4,v_{12}$.  We give the
argument that $\dih(0,v_{12},v_1,v_4)>1.2$.  The others are
similar. Disregard the length $|v_1-v_4|$.  We show that
    $$
    \begin{array}{lll}
        sd &:=\dih(0,v_{12},v_1,v_2)+\dih(0,v_{12},v_2,v_3)\\
           &+\dih(0,v_{12},v_3,v_4) < 2\pi-1.2.
    \end{array}
    $$
Lift $v_{12}$ so $|v_{12}|=2t_0$. Maximize $sd$ by taking
$|v_1-v_2|=|v_2-v_3|=|v_3-v_4|=2t_0$.  Fixing $v_3$ and $v_4$,
pivot $v_1$ around $\{0,v_{12}\}$ toward $v_4$, dragging $v_2$
toward $v_{12}$ until $|v_2-v_{12}|=2t_0$.  Similarly, we obtain
$|v_3-v_{12}|=2t_0$. We now have $sd\le 3(1.63)< 2\pi-1.2$, by a
calculation.\footnote{\calc{821707685}}

Return to the original figure and move $v_{12}$ without increasing
$|v_{12}|$ until each simplex $\{0,v_{12},v_i,v_{i+1}\}$ has an edge
$(v_{12},v_j)$ of length $2t_0$. Interval
calculations\footnote{\calc{467530297} and \calc{135427691}} show
that the four simplices around $v_{12}$ squander
    $$2\pi(0.2529)-3(0.1376)-0.12 > \squander + 5\xiG.$$

{\bf (Case 1-b)} Assume there are no flat quarters. By hypothesis,
the perimeter satisfies $$\sum|v_i-v_{i+1}|\le 11.407.$$ We have
$\arc(2,2,x)'' = 2x/(16-x^2)^{3/2} >0$. The arclength of the
perimeter is therefore at most
$$2\arc(2,2,2t_0) + 2\arc(2,2,2) + \arc(2,2,2.387) <  2\pi.$$
There is a well-defined interior of the spherical pentagon, a
component of area $<2\pi$.  If we deform by decreasing the
perimeter, the component of area $<2\pi$ does not get swapped with
the other component.

Disregard all vertices but $v_1,\ldots,v_5,v_{12}$.  If a vertex
$v_i$ satisfies  $|v_i-v_{12}|>2t_0$, deform $v_i$ as in
Section~\ref{x-4.9} until $|v_{i-1}-v_{i}|=|v_i-v_{i+1}|=2$, or
$|v_i-v_{12}|=2t_0$. If at any time, four of the edges realize the
bound $|v_i-v_{i+1}|=2$, we have reached an impossible situation,
because it leads to the contradiction\footnote{\calc{115383627}
and \calc{603145528}}
    $$2\pi = \sum^{(5)}\dih < 1.51 + 4 (1.16) < 2\pi.$$
(This inequality relies on the observation, which we leave to the
reader, that in any such assembly, pivots can by applied to bring
$|v_{12}-v_i|=2t_0$ for at least one edge of each of the five
simplices.)



The vertex $v_{12}$ may be moved without increasing $|v_{12}|$ so
that eventually by these deformations (and reindexing if
necessary) we have $|v_{12}-v_i|=2t_0$, $i=1,3,4$. (If we have
$i=1,2,3$, the two dihedral angles along $\{0,v_2\}$
satisfy\footnote{\calc{115383627}} $<2(1.51)<\pi$, so the
deformations can continue.)



There are two cases. In both cases $|v_i-v_{12}|=2t_0$, for
$i=1,3,4$.
$$
\begin{array}{lll}
(i)\quad &|v_{12}-v_2|=|v_{12}-v_5|=2t_0,\\
(ii)\quad &|v_{12}-v_2|=2t_0,\quad |v_4-v_5|=|v_5-v_1|=2,\\
\end{array}
$$
Case (i) follows from interval
calculations\footnote{\calc{312132053}}
$$
\sum\tau_0 \ge 2\pi(0.2529) - 5 (0.1453) > 0.644+7\xiG.
$$
In case (ii), we have again
    $$2\pi(0.2529)-5 (0.1453).$$
In this interval calculation we have assumed that
$|v_{12}-v_5|<3.488$. Otherwise, setting $S=(v_{12},v_4,v_5,v_1)$, Lemma~\ref{tarski:3488}
shows the simplex does not exist.
($|v_4-v_1|\ge2\sqrt2$ because
there are no flat quarters.)
This completes Case 1.

\medskip

{\bf Case 2.} {\it The vertex $v_{12}$ has distance at most $2t_0$
from the vertex $v_1$ and distance at least $2t_0$ from the
others.}

Let $\{0,v_{13}\}$ be the upright diagonal of a loop $(4,1)$.  The
vertices of the loop are not $\{v_2,v_3,v_4,v_5\}$ with $v_{12}$
enclosed over $\{0,v_2,v_5,v_{13}\}$ by
Lemma~\ref{lemma:anc-simplex-not-enc}. The vertices of the loop
are not $\{v_2,v_3,v_4,v_5\}$ with $v_{12}$ enclosed over
$\{0,v_1,v_2,v_5\}$ because this and Lemmas~\ref{tarski:E:part4:6}
and \ref{tarski:E:part4:7} would lead to a contradiction
$y_{12}>2t_0$. 
We get a contradiction for the same reasons
 unless $\{v_1,v_{12}\}$ is an edge of some
upright quarter of every loop of type $(4,1)$.

We consider two cases.  (2-a) There is a flat quarter along an
edge other than $\{v_1,v_{12}\}$.  That is, the central vertex is
$v_2$, $v_3$, $v_4$, or $v_5$.  (Recall that the {\it central
vertex} of a flat quarter is the vertex other than the origin that
is not an endpoint of the diagonal.) (2-b) Every flat quarter has
central vertex $v_1$.

{\bf Case 2-a.}  We erase all upright quarters including those in
loops, taking penalties as required. There cannot be two flat
quarters because then Lemmas~\ref{tarski:E:part4:8} and
\ref{tarski:E:part4:9} would imply $|v_{12}|>2t_0$.

The penalty is at most $7\xiG$.  We show that the region (with
upright quarters erased) squanders $>7\xiG+0.644$.  We assume that
the central vertex is $v_2$ (case 2-a-i) or $v_3$ (case 2-a-ii).
In case 2-a-i, we have three types of simplices around $v_{12}$,
characterized by the bounds on their edge lengths.  Let
$\{0,v_{12},v_1,v_5\}$ have type A, $\{0,v_{12},v_5,v_4\}$ and
$\{0,v_{12},v_4,v_3\}$ have type B, and let $\{0,v_{12},v_3,v_1\}$
have type C.  In case 2-a-ii there are also three types.  Let
$\{0,v_{12},v_1,v_2\}$ and $\{0,v_{12},v_1,v_5\}$ have type A,
$\{0,v_{12},v_5,v_4\}$ type B, and $\{0,v_{12},v_2,v_4\}$ type D.
(There is no relation here between these types and the types of
simplices $A$, $B$, $C$ defined in \Chap~\ref{sec:fine}.) Upper
bounds on the dihedral angles along the edge $\{0,v_{12}\}$ are
given as calculations\footnote{\calc{821707685}, \calc{115383627},
\calc{576221766}, and \calc{122081309}}. These upper bounds come
as a result of a pivot argument similar to that establishing the
bound 1.2 in Case 1-a.

These upper bounds imply the following lower bounds.  In case
2-a-i,
$$
\begin{array}{lll}
\dih &> 1.33 \quad(A),\\
\dih &> 1.21 \quad(B),\\
\dih &> 1.63 \quad(C),\\
\end{array}
$$
and in case 2-a-ii,
$$
\begin{array}{lll}
\dih &> 1.37 \quad(A),\\
\dih &> 1.25 \quad(B),\\
\dih &> 1.51 \quad(v,\Lambda),\\
\end{array}
$$
In every case the dihedral angle is at least $1.21$. In case
2-a-i, the inequalities give a lower bound on what is squandered
by the four simplices around $\{0,v_{12}\}$. Again, we move $v_{12}$
without decreasing the score until each simplex
$\{0,v_{12},v_i,v_{i+1}\}$ has an edge satisfying
$|v_{12}-v_j|\le2t_0$. Interval
calculations\footnote{\calc{644534985}, \calc{467530297}, and
\calc{603910880}} give
    $$
    \begin{array}{lll}
    \sum_{(4)}\tau_0 &> 2\pi (0.2529) - 0.2391-2(0.1376)-0.266\\
        &>0.808.
    \end{array}
    $$
In case 2-a-ii, we have\footnote{\calc{135427691}}
    $$
    \begin{array}{lll}
    \sum_{(4)}\tau_0 &> 2\pi (0.2529) - 2(0.2391)-0.1376-0.12\\
        &>0.853.
    \end{array}
    $$
So we squander more than $7\xiG+0.644$, as claimed.

{\bf Case 2-b.}  We now assume that there are no flat quarters
with central vertex $v_2,\ldots,v_5$. We claim
 that $v_{12}$ is not enclosed over $\{0,v_1,v_2,v_3\}$ or
$\{0,v_1,v_5,v_4\}$. In fact, if $v_{12}$ is enclosed over
$\{0,v_1,v_2,v_3\}$, then we reach the
contradiction\footnote{\calc{821707685} and \calc{115383627}}
    $$
    \begin{array}{lll}
    \pi&<\dih(0,v_{12},v_1,v_2)+\dih(0,v_{12},v_2,v_3)\\
        &< 1.63+1.51 < \pi.
    \end{array}
    $$

We claim
 that $v_{12}$ is not enclosed over $\{0,v_5,v_1,v_2\}$.
Let $S_1=\{0,v_{12},v_1,v_2\}$, and $S_2=\{0,v_{12},v_1,v_5\}$.  We
have by hypothesis,
$$y_4(S_1)+y_4(S_2) = |v_1-v_2|+|v_1-v_5|< 4.804.$$
An interval calculation\footnote{\calc{69064028}} gives
    $$
    \begin{array}{lll}
    \sum_{(2)}\dih(S_i) &\le \sum_{(2)}
    \left(\dih(S_i)+0.5(0.4804/2-y_4(S_i))\right)\\
    &<\pi.
    \end{array}
    $$
So $v_{12}$ is not enclosed over $\{0,v_1,v_2,v_5\}$.

Erase all upright quarters, taking penalties as required.  Replace
all flat quarters with $\svor_0$-scoring taking penalties as
required. (Any flat quarter has $v_1$ as its central vertex.) We
move $v_{12}$ keeping $|v_{12}|$ fixed and not decreasing
$|v_{12}-v_1|$.  The only effect this has on the score comes
through the quoins along $\{0,v_1,v_{12}\}$. Stretching
$|v_{12}-v_1|$ shrinks the quoins and increases the score. (The
sign of the derivative of the quoin with respect to the top edge
is computed in the proof of Lemma~\ref{x-4.9.1}.)

If we stretch $|v_{12}-v_1|$ to length $2t_0$, we are done by case
1 and case 2-a. (If deformations produce a flat quarter, use case
2-a, otherwise use case 1.) By the claims, we can eventually
arrange (reindexing if necessary) so that
$$
\begin{array}{lll}
(i)&\quad |v_{12}-v_3|=|v_{12}-v_4|=2t_0,\quad\text{or}\\
(ii)&\quad |v_{12}-v_3|=|v_{12}-v_5|=2t_0.
\end{array}
$$
We combine this with the deformations of Section~\ref{x-4.9} so
that in case (i) we may also assume that if $|v_5-v_{12}|>2t_0$,
then $|v_4-v_5|=|v_5-v_1|=2$ and that if $|v_2-v_{12}|>2t_0$, then
$|v_1-v_2|=|v_2-v_3|=2$. In case (ii) we may also assume that if
$|v_4-v_{12}|>2t_0$, then $|v_3-v_4|=|v_4-v_5|=2$ and that if
$|v_2-v_{12}|>2t_0$, then $|v_1-v_2|=|v_2-v_3|=2$.

Break the pentagon into subregions by cutting along the edges
$(v_{12},v_i)$ that satisfy $|v_{12}-v_i|\le2t_0$. So for example
in case (i), we cut along $(v_{12},v_3)$, $(v_{12},v_4)$,
$(v_{12},v_1)$, and possibly along $(v_{12},v_2)$ and
$(v_{12},v_5)$.  This breaks the pentagon into triangular and
quadrilateral regions.

In case (ii), if $|v_4-v_{12}|>2t_0$, then the argument used in
Case 1 to show that $|v_4-v_{12}|<3.488$ applies here as well.
%% Keep comment: DCG p164.  I commented out to avoid mention of Delta.
%% But it is used eventually in the argument.
%% Deelta doesn't explicitly get mentioned in the two footnoted calcs,
%% so what I'm commented out isn't that essential.
% In
%case (i) or (ii), if $|v_{12}-v_2|>2t_0$, then for similar
%reasons, we may assume
%    $$\Deelta(|v_{12}-v_2|^2,4,4,8,(2t_0)^2,|v_{12}-v_1|^2)\ge0.$$
We use
calculations\footnote{\calc{312132053} and \calc{644534985}} to
conclude that
    $$\sum\tau_0 \ge 2\pi (0.2529) -3 (0.1453) -2 (0.2391) > 0.6749.$$
If the penalty is less than $0.067=0.6749-0.6079$, we are done.

We have ruled out the existence of all loops except $(4,1)$. Note
that a flat quarter with central vertex $v_1$ gives penalty at
most $0.02$ by Lemma~\ref{x-3.11.3}.
  If there is at most one
such a flat quarter and at most one loop, we are done:
$$3\xiG + 0.02 < 0.067.$$
Assume there are two loops of context $(n,k)=(4,1)$.  They both
lie along the edge $\{v_1,v_{12}\}$, which precludes any unmasked
flat quarters. If one of the upright diagonals has height
$\ge2.696$, then the penalty is at most $3\xiG+3\xiV< 0.067$.
Assume both heights are at most $2.696$. The total interior angle
of the exceptional face at $v_1$ is at least four times the
dihedral angle of one of the flat quarters along $\{0,v_1\}$, or
$4(0.74)$ by an interval calculation\footnote{\calc{751442360}}. This is
contrary to the hypothesis of an interior angle $<2.89$.   This
completes Case 2. This shows that heptagons with pentagonal hulls
do not occur.
\end{proof}

\begin{lemma}\dcg{Lemma~14.6}{165}\label{lemma:excess-1}
Let $R$ be an exceptional standard region.  Let $V$
be a set of vertices of $R$.  If $v\in V$, let $p_v$ be the number
of triangular regions at $v$ and let $q_v$ be the number of
quadrilateral regions at $v$.  Assume that $V$ has the following
properties:
    \begin{enumerate}
        \item No two
        vertices in $V$ are adjacent.
        \item No two vertices
        in $V$ lie on a common quadrilateral.
        \item If $v\in V$, then there are five standard regions at
        $v$.
        \item If $v\in V$, then the corner over $v$ is a central
        vertex of a flat quarter in the cone over $R$.
        \item If $v\in V$, then $p_v\ge 3$.  That is, at least
        three of the five standard regions at $v$ are triangular.
        \item If $R'\ne R$ is an exceptional region at $v$, and if $R$
        has interior angle at least $1.32$ at $v$, then $R'$ also has interior
        angle at least $1.32$ at $v$.
        \item If $(p_v,q_v)=(3,1)$, then the internal angle at $v$ of the exceptional
        region is at most $1.32$.
    \end{enumerate}
  Define $a:\N\to \R$ by
  $$a(n) = \begin{cases}
    14.8 &n=0,1,2,\\
    1.4 & n=3,\\
    1.5 & n=4,\\
    0 & \text{otherwise.}
  \end{cases}
  \index{aZ@$a(n)$}
  $$
Let $\{F\}$ be the union of $\{R\}$ with the set of triangular and
quadrilateral regions that have a vertex at some $v\in V$. Then
    $$\sum_F\tau_F(v,\Lambda) > \sum_{v\in V} (p_v d(3) + q_v d(4) + a
    (p_v))\,\pt.$$
\end{lemma}

\begin{proof}   We erase all upright diagonals in the
$Q$-system, except for those that carry a penalty: loops,
$3$-unconfined, $3$-crowded, and $4$-crowded diagonals.

We assume that if $(p_v,q_v)=(3,1)$, then the internal angle is at
most $1.32$. Because of this, if we score the flat quarter by
$\vor_0$, then the flat quarter $Q$ satisfies
(Lemma~\ref{lemma:1.32})
   \begin{equation}
   \vor_0(Q) > 3.07\,\pt > 1.4\,\pt + D(3,1) + 2\xiV + \xiG.
   \label{eqn:307}
   \end{equation}



Every flat quarter that is masked by a remaining upright quarter
in the $Q$-system has $y_4\ge2.6$.  Moreover, $y_1\ge2.2$ or
$y_4\ge2.7$.  Let $\pi_v = 2\xiV + \xiG$ if the flat quarter is
masked, and $\pi_v = 0$ otherwise.

We claim that the flat quarter (scored by $\vor_0$) together with
the triangles and quadrilaterals at a given vertex $v$ squander at
least
   \begin{equation}
   (p_v d(3) + q_v d(4) + a(p_v))\,\pt + D(3,1) + \pi_v
   \label{eqn:one-v}
   \end{equation}
If $p_v=4$, this is \calc{314974315}.  If $p_v=3$, we may assume
by the preceding remarks that there are two exceptional regions at
$v$.  If the internal angle of $R$ at $v$ is at most $1.32$, then
we use Inequality~\ref{eqn:307}.  If the angle is at least $1.32$,
then by hypothesis, the angle $R'$ at $v$ is at least $1.32$.  We
then appeal to the calculations \calc{675785884} and
\calc{193592217}.

To complete the proof of the lemma, it is enough to show that we
can erase the upright quarters masking a flat quarter at $v$
without incurring a penalty greater than $\pi_v$.  For then, by
summing the Inequality~\ref{eqn:one-v} over $v$, we obtain the
result.

If the upright diagonal is enclosed over the masked flat quarter,
then the upright quarters can be erased with penalty at most
$\xiV$ (by Remark~\ref{remark:3rd-quarter}). Assume the upright
diagonal is not enclosed over the masked flat quarter.

If there are at most three upright quarters, the penalty is at
most $2\xiV + \xiG$.  Assume four or more upright quarters.  If
the upright diagonal is not a loop, then it must be $4$-crowded.
This can be erased with penalty
   $$2\xiV + 2\xiG - \kappa < 2\xiV + \xiG.$$

Finally, assume that the upright quarter is a loop with four or
more upright quarters.  Lemma~\ref{lemma:loop} limits the
possibilities to parameters $(5,0)$ or $(5,1)$.  In the case of a
loop $(5,1)$, there is no need to erase because $|V|\le3$ and by
Lemma~\ref{lemma:loop}, the hexagonal standard region squanders at
least
   $$t_6 + 3 a(p_v)\,\pt$$
as required by the lemma.  In the case of a loop $(5,0)$ in a
pentagonal region, if $|V|=1$ then there is no need to erase
(again we appeal to Lemma~\ref{lemma:loop}).  If $|V| =2$, then
the two vertices share a penalty of $4\xiV + \xiG$, with each
receiving
   $$2\xiV + \xiG/2 < 2\xiV +\xiG.$$
\end{proof}
