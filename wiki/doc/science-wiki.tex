\section{A Semantic Wiki for Science}

\begin{wrapfigure}{r}{5.5cm}
  \centering
  \vspace{-.5cm}
  \includegraphics[width=5cm]{creativity-spiral}
  \vspace{-.5cm}
  \caption{The Math/Science Creativity Spiral (after Buchberger, 1995)}
  \label{fig:creativity-spiral}
\end{wrapfigure}

Documents are the most important medium in science, if we assume a broad definition of
``document'', including any materialized item of (scientific) knowledge.  Scientific
communication mainly consists of exchanging documents---from informal drafts circulating
inside a working group to published, well-structured books.  A great deal of scientific
work consists of collaboratively authoring these documents---taking down first hypotheses,
commenting on results of experiments or project steps, as well as structuring, annotating,
and re-organizing existing items of knowledge\ednote{OK? --CL}.  Tools that
\emph{understand} the knowledge contained in scientific documents are desirable for
editing such documents.  One approach towards this is writing scientific documents in a
semantic markup language with an editor that knows the structures available in this
language.

Besides generic approaches like SALT~\cite{Groza:SALT07}, semantic markup has been most
deeply investigated in the specific domain of mathematics with ``long tradition in the
pursuit of conceptual clarity and representational rigor''~\cite{Kohlhase:omdoc1.2},
resulting in languages like MathML~\cite{CarlisleEd:MathML07},
OpenMath~\cite{BusCapCar:2oms04}, and OMDoc~\cite{Kohlhase:omdoc1.2}.  OMDoc is a language
that employs Content MathML\footnote{MathML comes in two flavors: Presentation MathML
  expresses the way a formula is rendered, whereas Content MathML models its logical
  structure.  Formal mathematical software like a Computer Algebra System would export
  formulae in Content MathML, but for publishing, they would be converted to Presentation
  MathML.} or OpenMath for structurally representing mathematical \emph{objects} (symbols,
numbers, equations, etc.) and adds two layers on top of that: Objects or informal text can
be annotated as mathematical \emph{statements} (symbol declarations, definitions, axioms,
theorems, proofs, examples, etc.), and collections of interrelated statements are grouped
into \emph{theories}.

With SWiM, a semantic wiki for mathematical knowledge management (see
section~\ref{sec:swim}), we have investigated collaborative editing of OMDoc documents.
It has turned out that a wiki is a suitable tool for supporting the workflow of
incremental formalization inherent to scientific writing.  But wikis have not only shown
to be appropriate for \emph{writing}, but also for project management, e.\,g.\ in
corporate settings~\cite{leuf01:wikiway}.  Thus, we are interested in applying our
technologies to scientific knowledge engineering projects.  The Flyspeck project
introduced in section~\ref{sec:flyspeck} was particularly appealing as a use case: It
involves both highly formal and semi-formal mathematical knowledge, as well as informal
text.  Its large extent and the required manpower suggest ``crowdsourcing'' the workload
and supporting the project organization by social software.  We are going to investigate
whether the three conditions for successful peer production stated by Tapscott and
Williams~\cite{wikinomics} will hold or can be satisfied by software support:\ednote{This
  is now more sound. --CL}

\begin{enumerate}
\item The object of production is information, which keeps the cost of participation low.
\item We show how to break tasks down into small, independent pieces.\ednote{Check whether
    we actually show this in this paper! --CL}
\item We need to investigate whether the cost of integrating the individual contributions
  is low.
\end{enumerate}

%%% Local Variables: 
%%% mode: latex
%%% fill-column: 90
%%% TeX-master: "flyspeck-wiki-eswc08"
%%% End: 
